%!Mode:: "TeX:UTF-8"
%!TEX TS-program = xelatex
\documentclass{ctexart}
\newif\ifpreface
%\prefacetrue
\usepackage{fontspec}
\usepackage{bbm}
\usepackage{tikz}
\usepackage{amsmath,amssymb,amsthm,color,mathrsfs}
\usepackage{fixdif}
\usepackage{hyperref}
\usepackage{cleveref}
\usepackage{enumitem}%
\usepackage{expl3}
\usepackage{lipsum}
\usepackage[margin=0pt]{geometry}
\usepackage{listings}
\definecolor{mGreen}{rgb}{0,0.6,0}
\definecolor{mGray}{rgb}{0.5,0.5,0.5}
\definecolor{mPurple}{rgb}{0.58,0,0.82}
\definecolor{backgroundColour}{rgb}{0.95,0.95,0.92}

\lstdefinestyle{CStyle}{
  backgroundcolor=\color{backgroundColour},
  commentstyle=\color{mGreen},
  keywordstyle=\color{magenta},
  numberstyle=\tiny\color{mGray},
  stringstyle=\color{mPurple},
  basicstyle=\footnotesize,
  breakatwhitespace=false,
  breaklines=true,
  captionpos=b,
  keepspaces=true,
  numbers=left,
  numbersep=5pt,
  showspaces=false,
  showstringspaces=false,
  showtabs=false,
  tabsize=2,
  language=C
}
\usetikzlibrary{calc}
\theoremstyle{remark}
\newtheorem{lemma}{Lemma}
\usepackage{fontawesome5}
\usepackage{xcolor}
\newcounter{problem}
\newcommand{\Problem}{\begin{tikzpicture}[baseline]%
    \node at (-0.02em,0.3em) {$\mathbb{P}$};%
    \node[scale=0.7] at (0.2em,-0.0em) {R};%
    \node[scale=0.7] at (0.6em,0.4em) {O};%
    \node[scale=0.8] at (1.05em,0.25em) {B};%
    \node at (1.55em,0.3em) {L};%
    \node[scale=0.7] at (1.75em,0.45em) {E};%
    \node at (2.35em,0.3em) {M};%
  \end{tikzpicture}%
}
\renewcommand{\theproblem}{\Roman{problem}}
\newenvironment{problem}{\refstepcounter{problem}\noindent\color{blue}\Problem\theproblem}{}

\crefname{problem}{\protect\Problem}{Problem}
\newcommand\Solution{\begin{tikzpicture}[baseline]%
    \node at (-0.04em,0.3em) {$\mathbb{S}$};%
    \node[scale=0.7] at (0.35em,0.4em) {O};%
    \node at (0.7em,0.3em) {\textit{L}};%
    \node[scale=0.7] at (0.95em,0.4em) {U};%
    \node[scale=1.1] at (1.19em,0.32em){T};%
    \node[scale=0.85] at (1.4em,0.24em){I};%
    \node at (1.9em,0.32em){$\mathcal{O}$};%
    \node[scale=0.75] at (2.3em,0.21em){\texttt{N}};%
  \end{tikzpicture}}
\newenvironment{solution}{\begin{proof}[\Solution]}{\end{proof}}
\title{\input{../../.subject}\input{../.number}}
\makeatletter
\newcommand\email[1]{\def\@email{#1}\def\@refemail{mailto:#1}}
\newcommand\schoolid[1]{\def\@schoolid{#1}}
\ifpreface
  \def\@maketitle{
  \raggedright
  {\Huge \bfseries \sffamily \@title }\\[1cm]
  {\Huge  \bfseries \sffamily\heiti\@author}\\[1cm]
  {\Huge \@schoolid}\\[1cm]
  {\Huge\href\@refemail\@email}\\[0.5cm]
  \Huge\@date\\[1cm]}
\else
  \def\@maketitle{
    \raggedright
    \begin{center}
      {\Huge \bfseries \sffamily \@title }\\[4ex]
      {\Large  \@author}\\[4ex]
      {\large \@schoolid}\\[4ex]
      {\href\@refemail\@email}\\[4ex]
      \@date\\[8ex]
    \end{center}}
\fi
\makeatother
\ifpreface
  \usepackage[placement=bottom,scale=1,opacity=1]{background}
\fi

\author{白永乐}
\schoolid{25110180002}
\email{ylbai25@m.fudan.edu.cn}

\def\to{\rightarrow}
\newcommand{\xor}{\vee}
\newcommand{\AND}{\wedge}
\newcommand{\OR}{\vee}
\newcommand{\bor}{\bigvee}
\newcommand{\band}{\bigwedge}
\newcommand{\xand}{\wedge}
\newcommand{\minus}{\mathbin{\backslash}}
\newcommand{\mi}[1]{\mathscr{P}(#1)}
\newcommand{\card}{\mathrm{card}}
\newcommand{\oto}{\leftrightarrow}
\newcommand{\hin}{\hat{\in}}
\newcommand{\gl}{\mathrm{GL}}
\newcommand{\im}{\mathrm{Im}}
\newcommand{\re }{\mathrm{Re }}
\newcommand{\rank}{\mathrm{rank}}
\newcommand{\tra}{\mathop{\mathrm{tr}}}
\renewcommand{\char}{\mathop{\mathrm{char}}}
\DeclareMathOperator{\ot}{ordertype}
\DeclareMathOperator{\dom}{dom}
\DeclareMathOperator{\ran}{ran}

\begin{document}
\large
\setlength{\baselineskip}{1.2em}
\ifpreface
  \input{../../../global/preface}
\else
  \newgeometry{left=2cm,right=2cm,top=2cm,bottom=2cm}
  \title{Logic 2}
  \maketitle
\fi
%from_here_to_type
\begin{problem}\label{pro:1}
  设 $p$ 的真值为真, $q$ 的真值为假, 求 $\neg p , p \wedge q , p \vee q , p \rightarrow q , p \leftrightarrow q$的真值。
\end{problem}
\begin{solution}
  显然有如下真值表:
  \[
    \begin{matrix}
      \neg p               & F \\
      p \AND q             & F \\
      p \OR q              & T \\
      p \to q              & F \\
      p \leftrightarrow  q & F \\
    \end{matrix}
  \]
\end{solution}

\begin{problem}\label{pro:2}
  设 $p$ 为 $T$ (真) $q$ 为 $T, r$ 为 $F$ (假), 下列公式中哪些公式取值为 $T$ ?
  \begin{enumerate}
    \item \label{it:2.1} $q \wedge r$
    \item \label{it:2.2}$\neg p \wedge \neg r$
    \item \label{it:2.3}$p \leftrightarrow \neg q \vee r$
    \item \label{it:2.4}$q \vee \neg r \rightarrow p$
    \item \label{it:2.5}$(q \rightarrow p) \rightarrow((p \rightarrow \neg r) \rightarrow(\neg r \rightarrow q))$
  \end{enumerate}
\end{problem}
\begin{solution}
  \begin{enumerate}
    \item 显然\(q \AND r\)为假。
    \item 显然有\(\neg p\)为假,故\(\neg p \AND \neg r\)为假。
    \item 由\(\neg q\)和\(r\)均为假可知\(\neg q \OR r\)为假,但\(p\)为真,故\(p \leftrightarrow \neg q \OR r\)为假。
    \item 由于\(p\)为真,故\(q \OR \neg r \to p\)为真。
    \item 易于得到以下真值表:
      \[
        \begin{matrix}
          p & q & r & \neg r & p \to \neg r & \neg r \to q & q \to p & (p \to \neg r) \to (\neg r \to q) & (q \to p)\to ((p \to \neg r) \to (\neg r \to q)) \\
          T & T & F & T      & T            & T            & T       & T                                 & T                                                \\
        \end{matrix}
      \]
  \end{enumerate}
  故\ref{it:2.4},\ref{it:2.5}为真。
\end{solution}
\begin{problem}\label{pro:3}
  用真值表判定下列各组公式哪些表示相同的真值函项 (即哪些是等值的)?
  \begin{enumerate}
    \item $p \rightarrow q, \neg p \vee q$
    \item $\neg p \vee \neg q, \neg(p \vee q)$
  \end{enumerate}

\end{problem}
\begin{solution}
  \begin{enumerate}
    \item 显然有如下真值表:
      \[
        \begin{matrix}
          p & q & \neg p & \neg p \OR q & p \to q \\
          T & T & F      & T            & T       \\
          T & F & T      & F            & F       \\
          F & T & F      & T            & T       \\
          F & F & T      & T            & T       \\
        \end{matrix}
      \]
      故逻辑等值。
    \item 显然有如下真值表:
      \[
        \begin{matrix}
          p & q & \neg p & \neg q & p \OR q & \neg(p \OR q) & \neg p \OR \neg q \\
          T & T & F      & F      & T       & F             & F                 \\
          T & F & F      & T      & T       & F             & T                 \\
          F & T & T      & F      & T       & F             & T                 \\
          F & F & T      & T      & F       & T             & T                 \\
        \end{matrix}
      \],故不逻辑等值。
  \end{enumerate}
\end{solution}
\begin{problem}\label{pro:4}
  列出下列公式的真值表, 并指出它们分别为重言式、矛盾式或协调式。
  \begin{enumerate}
    \item $p \leftrightarrow p \vee p$
    \item $p \wedge(q \wedge \neg q)$
  \end{enumerate}
\end{problem}
\begin{solution}
  \begin{enumerate}
    \item 显然有如下真值表:
      \[
        \begin{matrix}
          p & p \OR p & p \leftrightarrow p \OR p \\
          T & T       & T                         \\
          F & F       & T                         \\
        \end{matrix}
      \],故为重言式。
    \item 显然有如下真值表:
      \[
        \begin{matrix}
          p & q & \neg q & q \AND \neg q & p \AND q \AND \neg q \\
          T & T & F      & F             & F                    \\
          T & F & T      & F             & F                    \\
          F & T & F      & F             & F                    \\
          F & F & T      & F             & F                    \\
        \end{matrix}
      \],故为矛盾式。
  \end{enumerate}
\end{solution}
\begin{problem}\label{pro:5}
  用命题的自然推理, 证明下列公式是否为有效式 (为系统中之定理)?
  \begin{enumerate}
    \item $p \wedge p \rightarrow p$
    \item $(q \rightarrow r) \rightarrow(p \vee q \rightarrow p \vee r)$
  \end{enumerate}
\end{problem}
\begin{solution}
  \begin{enumerate}
    \item \[
        \begin{matrix}
          [1] &                & p \AND p & hyp           \\
          [2] &                & p        & \AND E:[1]    \\
          [3] & p \AND p \to p &          & \to I:[1]-[2] \\
        \end{matrix}
      \]
    \item \[
        \begin{matrix}
          [1]  &                                   & q \to r             &         &         & hyp           \\
          [2]  &                                   &                     & p \OR q &         & hyp           \\
          [3]  &                                   &                     &         & p       & hyp           \\
          [4]  &                                   &                     &         & p \OR r & \OR I:[3]     \\
          [5]  &                                   &                     &         & q       & hyp           \\
          [6]  &                                   &                     &         & r       & \to E:[5][1]  \\
          [7]  &                                   &                     &         & p \OR r & \OR I:[6]     \\
          [8]  &                                   &                     & p \OR r &         & \OR E:[3]-[7] \\
          [9]  &                                   & p \OR q \to p \OR r &         &         & \to I:[2]-[8] \\
          [10] & (q \to r)\to(p \OR q \to p \OR r) &                     &         &         & \to I:[1]-[9] \\
        \end{matrix}
      \]
  \end{enumerate}

\end{solution}
\begin{problem}\label{pro:6}
  请用简化真值表方法判定以下命题是否为重言式。
  \begin{enumerate}
    \item $((p \to q) \wedge(r \to s)) \vee(p \vee r) \to q \vee s$
    \item $((p \to q) \wedge(r \to s)) \wedge(p \wedge r) \to q \wedge s$
    \item $(p \wedge(q \vee j)) \wedge(p \to(q \to k \wedge t)) \wedge(p \wedge j \to \neg(k \vee t)) \to(k \wedge t) \vee(\neg k \wedge \neg t)$
  \end{enumerate}
\end{problem}
\begin{solution}
  \begin{enumerate}
    \item 反设\(((p \to q) \wedge(r \to s)) \vee(p \vee r) \to q \vee s\)不成立,
      则得到\(((p \to q)\AND (r \to s))\OR(p \OR r)\)和\(\neg q\)以及\(\neg s\)。
      观察发现当\(p,r\)为真,\(q,s\)为假时上述条件成立,原表达式为假。
      故不是重言式。
    \item 反设\(((p \to q) \wedge(r \to s)) \wedge(p \wedge r) \to q \wedge s\)不成立,
      则得到\(((p \to q) \wedge(r \to s)) \wedge(p \wedge r) \) 和\(\neg q \OR \neg s\)。
      从而有\(p \to q,r \to s,p,r,\neg q \OR \neg s\)。
      由\(p,p \to q\)得到\(q\),同样由\(r,r \to s\)得到\(s\),从而得到\(q \AND s\),与\(\neg q \OR \neg s\)矛盾!
      故原式是重言式。
    \item 反设\((p \wedge(q \vee j)) \wedge(p \to(q \to k \wedge t)) \wedge(p \wedge j \to \neg(k \vee t)) \to(k \wedge t) \vee(\neg k \wedge \neg t)\) 为假,
      则有\(p,q \OR j,p \to (q \to k \AND t),p \AND j \to \neg (k \OR t),\neg(k \AND t),\neg(\neg k \AND \neg t)\)。
      由\(p,p \to (q \to k \AND t)\)可得\(q \to k \AND t\),再由\(\neg(k \AND t)\)得到\(\neg q\),进一步
      由\(q \OR j\)得到\(j\),从而得到\(p \AND j\),再结合\(p \AND j \to \neg(k \OR t)\)得到
      \(\neg (k \OR t)\),从而得到\(\neg k \AND \neg t\)。
      但是我们已经由假设得到了\(\neg(\neg k \AND \neg t)\),故产生矛盾。
      从而原式是重言式。
  \end{enumerate}
\end{solution}

\end{document}
