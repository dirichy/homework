%!Mode:: "TeX:UTF-8"
%!TEX TS-program = xelatex
\documentclass{ctexart}
\newif\ifpreface
% \prefacetrue
\usepackage{fontspec}
\usepackage{bbm}
\usepackage{tikz}
\usepackage{amsmath,amssymb,amsthm,color,mathrsfs}
\usepackage{fixdif}
\usepackage{hyperref}
\usepackage{cleveref}
\usepackage{enumitem}%
\usepackage{expl3}
\usepackage{lipsum}
\usepackage[margin=0pt]{geometry}
\usepackage{listings}
\definecolor{mGreen}{rgb}{0,0.6,0}
\definecolor{mGray}{rgb}{0.5,0.5,0.5}
\definecolor{mPurple}{rgb}{0.58,0,0.82}
\definecolor{backgroundColour}{rgb}{0.95,0.95,0.92}

\lstdefinestyle{CStyle}{
  backgroundcolor=\color{backgroundColour},
  commentstyle=\color{mGreen},
  keywordstyle=\color{magenta},
  numberstyle=\tiny\color{mGray},
  stringstyle=\color{mPurple},
  basicstyle=\footnotesize,
  breakatwhitespace=false,
  breaklines=true,
  captionpos=b,
  keepspaces=true,
  numbers=left,
  numbersep=5pt,
  showspaces=false,
  showstringspaces=false,
  showtabs=false,
  tabsize=2,
  language=C
}
\usetikzlibrary{calc}
\theoremstyle{remark}
\newtheorem{lemma}{Lemma}
\usepackage{fontawesome5}
\usepackage{xcolor}
\newcounter{problem}
\newcommand{\Problem}{\begin{tikzpicture}[baseline]%
    \node at (-0.02em,0.3em) {$\mathbb{P}$};%
    \node[scale=0.7] at (0.2em,-0.0em) {R};%
    \node[scale=0.7] at (0.6em,0.4em) {O};%
    \node[scale=0.8] at (1.05em,0.25em) {B};%
    \node at (1.55em,0.3em) {L};%
    \node[scale=0.7] at (1.75em,0.45em) {E};%
    \node at (2.35em,0.3em) {M};%
  \end{tikzpicture}%
}
\renewcommand{\theproblem}{\Roman{problem}}
\newenvironment{problem}{\refstepcounter{problem}\noindent\color{blue}\Problem\theproblem}{}

\crefname{problem}{\protect\Problem}{Problem}
\newcommand\Solution{\begin{tikzpicture}[baseline]%
    \node at (-0.04em,0.3em) {$\mathbb{S}$};%
    \node[scale=0.7] at (0.35em,0.4em) {O};%
    \node at (0.7em,0.3em) {\textit{L}};%
    \node[scale=0.7] at (0.95em,0.4em) {U};%
    \node[scale=1.1] at (1.19em,0.32em){T};%
    \node[scale=0.85] at (1.4em,0.24em){I};%
    \node at (1.9em,0.32em){$\mathcal{O}$};%
    \node[scale=0.75] at (2.3em,0.21em){\texttt{N}};%
  \end{tikzpicture}}
\newenvironment{solution}{\begin{proof}[\Solution]}{\end{proof}}
\title{\input{../../.subject}\input{../.number}}
\makeatletter
\newcommand\email[1]{\def\@email{#1}\def\@refemail{mailto:#1}}
\newcommand\schoolid[1]{\def\@schoolid{#1}}
\ifpreface
  \def\@maketitle{
  \raggedright
  {\Huge \bfseries \sffamily \@title }\\[1cm]
  {\Huge  \bfseries \sffamily\heiti\@author}\\[1cm]
  {\Huge \@schoolid}\\[1cm]
  {\Huge\href\@refemail\@email}\\[0.5cm]
  \Huge\@date\\[1cm]}
\else
  \def\@maketitle{
    \raggedright
    \begin{center}
      {\Huge \bfseries \sffamily \@title }\\[4ex]
      {\Large  \@author}\\[4ex]
      {\large \@schoolid}\\[4ex]
      {\href\@refemail\@email}\\[4ex]
      \@date\\[8ex]
    \end{center}}
\fi
\makeatother
\ifpreface
  \usepackage[placement=bottom,scale=1,opacity=1]{background}
\fi

\author{白永乐}
\schoolid{25110180002}
\email{ylbai25@m.fudan.edu.cn}

\def\to{\rightarrow}
\newcommand{\xor}{\vee}
\newcommand{\AND}{\wedge}
\newcommand{\OR}{\vee}
\newcommand{\bor}{\bigvee}
\newcommand{\band}{\bigwedge}
\newcommand{\xand}{\wedge}
\newcommand{\minus}{\mathbin{\backslash}}
\newcommand{\mi}[1]{\mathscr{P}(#1)}
\newcommand{\card}{\mathrm{card}}
\newcommand{\oto}{\leftrightarrow}
\newcommand{\hin}{\hat{\in}}
\newcommand{\gl}{\mathrm{GL}}
\newcommand{\im}{\mathrm{Im}}
\newcommand{\re }{\mathrm{Re }}
\newcommand{\rank}{\mathrm{rank}}
\newcommand{\tra}{\mathop{\mathrm{tr}}}
\renewcommand{\char}{\mathop{\mathrm{char}}}
\DeclareMathOperator{\ot}{ordertype}
\DeclareMathOperator{\dom}{dom}
\DeclareMathOperator{\ran}{ran}

\begin{document}
\large
\setlength{\baselineskip}{1.2em}
\ifpreface
  \input{../../../global/preface}
\else
  \newgeometry{left=2cm,right=2cm,top=2cm,bottom=2cm}
  \title{Logic 4}
  \maketitle
\fi
%from_here_to_type
\begin{problem}\label{pro:1}
  从对称性的角度下列命题中划有横线的关系各属于何种关系?
  \begin{enumerate}
    \item 甲概念\underline{包含于}乙概念。
    \item 甲队\underline{战胜}乙队。
    \item 田路和王州\underline{同岁}。
  \end{enumerate}
\end{problem}
\begin{solution}
  \begin{enumerate}
    \item 理解为非严格包含的话为非对称关系。如果理解为严格对称关系的话为反对称关系。
    \item 战胜应当是一个动词而非关系,我觉得应该改为``在XX比赛中战胜了'',或者``战胜过''。
      按照第一种理解应当是反对称关系。按照第二种理解应当是非对称关系。
    \item 对称关系。
  \end{enumerate}
\end{solution}
\begin{problem}\label{pro:2}
  下列关系推理是否正确?为什么?
  \begin{enumerate}
    \item W足球队败给了S足球队,S足球队又败给了K足球队,所以,这次足球球赛的名次是冠军为K足球队,亚军是S足球队,W足球队获得第三名。
    \item 吕珍佩服于烨,于是,于烨也佩服吕珍。
    \item 美国在加拿大以南,巴西在美国以南,所以,巴西在加拿大以南。
    \item 甲组有的同学批评了小王,张红是甲组的同学,所以张红也批评了小王。
  \end{enumerate}
\end{problem}
\begin{solution}
  \begin{enumerate}
    \item 信息不足无法判断。首先,如果是积分制的比赛,名次是不能用胜败关系得出的。
      其次,没有明确是否只有W,S,K三支队伍。
      如果是淘汰赛并且只有这三支队伍的话,推理正确。
    \item 不正确。``佩服''是非对称关系。
    \item 推理正确。``在……以南''是传递关系。
    \item 推理不正确。媒概念在关系命题中必须周延。
  \end{enumerate}
\end{solution}
\begin{problem}\label{pro:3}
  下面有若干组模态命题,已知每一组的第一个命题为真,请指出其余命题的真假。
  \begin{enumerate}
    \item
      \begin{enumerate}
        \item 竺红必然不能取得100米决赛的冠军。
        \item 竺红必然取得100米决赛的冠军。
        \item 竺红可能取得100米决赛的冠军。
        \item 竺红可能不会取得100米决赛的冠军。
      \end{enumerate}
    \item
      \begin{enumerate}
        \item 市第一百货商店今天可能没有彩电出售。
        \item 市第一百货商店今天必然有彩电出售。
        \item 市第一百货商店今天必然没有彩电出售。
        \item 市第一百货商店今天可能有彩电出售。
      \end{enumerate}
  \end{enumerate}
\end{problem}
\begin{solution}
  \begin{enumerate}
    \item
      \begin{enumerate}
        \item \label{it:1.1} 依题意为真。
        \item 根据\ref{it:1.1},依照反对关系可知为假。
        \item 根据\ref{it:1.1},依照矛盾关系可知为假。
        \item 根据\ref{it:1.1},依照差等关系可知为真。
      \end{enumerate}
    \item
      \begin{enumerate}
        \item \label{it:2.1} 依题意为真。
        \item 根据\ref{it:2.1},依照矛盾关系可知为假。
        \item 根据\ref{it:2.1},依照差等关系可知真假不定。
        \item 根据\ref{it:2.1},依照下反对关系可知真假不定。
      \end{enumerate}
  \end{enumerate}
\end{solution}
\begin{problem}\label{pro:4}
  请指出下列模态推理的形势,并说明它是否正确,为什么?
  \begin{enumerate}
    \item 周师傅今年不可能回家,所以,周师傅今年春节不回家是必然的。
    \item 老王可能不住在五层楼上,所以,老王不可能住在五层楼上。
    \item 北方人冬天到南方来对气候可能不适应,所以,北方人冬天到南方来对气候不必然不适应。
  \end{enumerate}
\end{problem}
\begin{solution}
  \begin{enumerate}
    \item 推理形式:\(\neg \lozenge p \vdash \square \neg p\) 。正确。属于矛盾关系推理。
    \item 推理形式:\(\lozenge \neg p \vdash \neg \lozenge p\)。不正确。
      根据矛盾关系可知,\(\neg \lozenge p \iff \square \neg p\),根据差等关系知\(\lozenge \neg p \not \vdash \square \neg p\)。
      故为无效推理。
    \item 推理形式:\(\lozenge \neg p \vdash \neg \square \neg p \)。不正确。根据差等关系可知其为无效推理。
  \end{enumerate}
\end{solution}
\begin{problem}\label{pro:5}
  下面有若干组规范命题,已知每一组的第一个命题是正确的,请根据对当关系,指出其他三个命题的正确与否。
  \begin{enumerate}
    \item
      \begin{enumerate}
        \item 进出C厂厂门必须佩戴厂徽。
        \item 进出C厂厂门必须不佩戴厂徽。
        \item 进出C厂厂门允许佩戴厂徽。
        \item 进出C厂厂门允许不佩戴厂徽。
      \end{enumerate}
    \item
      \begin{enumerate}
        \item M国的新产品允许不交所得税。
        \item M国的新产品必须交所得税。
        \item M国的新产品必须不交所得税。
        \item M国的新产品允许交所得税。
      \end{enumerate}
  \end{enumerate}
  \begin{solution}
    \begin{enumerate}
      \item
        \begin{enumerate}
          \item \label{it:3.1}依题意为真。
          \item 根据\ref{it:3.1},依照反对关系知为假。
          \item 根据\ref{it:3.1},依照差等关系知为真。
          \item 根据\ref{it:3.1},依照矛盾关系知为假。
        \end{enumerate}
      \item
        \begin{enumerate}
          \item \label{it:4.1}依题意为真。
          \item 根据\ref{it:4.1},依照矛盾关系知为假。
          \item 根据\ref{it:4.1},依照差等关系知真假不定。
          \item 根据\ref{it:4.1},依照下反对关系知真假不定。
        \end{enumerate}
    \end{enumerate}
  \end{solution}

\end{problem}

\end{document}
