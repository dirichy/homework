%arara: xelatex
\documentclass[UTF8]{ctexart}

\usepackage{geometry}
\geometry{a4paper, margin=2.5cm}

\usepackage{setspace}
\setstretch{1.5}

\usepackage{parskip}
\setlength{\parindent}{2em}

\usepackage{titling}
\pretitle{
\begin{center}\Huge\bfseries}
  \posttitle{\par
\end{center}\vskip 2em}
\preauthor{
\begin{center}\Large}
  \postauthor{\par
\end{center}}
\predate{
\begin{center}\large}
  \postdate{\par
\end{center}}

% 封面信息
\title{参加就业指导集市活动感想}
\author{姓名:白永乐 \\ 学号:202011150087 \\ 学院:数学科学学院}

\begin{document}

% 封面页
\maketitle
\thispagestyle{empty}
\newpage

% 正文
最近,我有幸参加了一场以“就业指导集市”为主题的活动,
这次活动设置了生涯体验互动、就业权益、生涯指导、简历指导、
就业政策五个主要板块。作为一名即将踏入社会的大学生,
我深知就业对于每一个人来说都是一个重要而又充满挑战的课题。
通过这次活动,我不仅获得了许多实用的信息和技巧,
更对自己的未来有了更加清晰的认识。
以下是我对这五个活动板块的具体感想。

首先,生涯体验互动给我留下了很深的印象。
这个环节通过一系列趣味化、互动式的小游戏和测试,
让我们了解自己的性格类型、兴趣取向以及与之匹配的职业类型。
以前我对自己的认知比较模糊,只是大致知道自己对某些领域感兴趣,
但没有系统地分析过。
通过这次互动,我参加了MBTI性格测试以及职业兴趣匹配,
得出的结果让我发现自己其实更适合某些之前没有考虑过的岗位,
比如涉及创意策划、跨领域沟通的工作。
这让我意识到,职业规划不仅仅是盲目跟风或听从家人建议,
而是需要结合自我特质去做出理性选择。

接着,就业权益板块让我学到了许多以前不太了解的法律知识。
在过去的印象中,求职和就业更像是一种单方面的选择,
即用人单位挑选我们,而我们只是被动接受。
但在这次活动中,讲解老师详细介绍了劳动合同、试用期、
工资支付、加班费、社保等方面的权利和义务,
并通过真实案例让我们明白,
作为求职者和未来的员工,
我们是有权利去保障自身合法权益的。
比如,如果用人单位在试用期内随意辞退员工、
或是随意延长试用期、克扣工资等,
都是违反《劳动合同法》的行为,
是可以通过法律途径维护的。
这些知识让我在未来面对职场问题时多了一份底气,
也提醒我在签署任何合同之前要仔细阅读,避免吃亏。

第三个环节是生涯指导。
在这个板块中,有专业的生涯规划老师和我们进行了一对一
或者小组的交流,帮助我们理清生涯发展的思路。
老师引导我们回顾大学期间的成长、
分析所学专业的优势和不足、
结合个人兴趣和价值观,去思考未来的职业方向。
特别令我感动的是,老师不仅仅给出了“就业”的建议,
还鼓励我们放眼更长远的生涯发展,
比如继续深造、出国留学、创业等多种选择。
以前我总觉得就业就是大学毕业后的唯一出路,
但通过这次交流,我发现生涯的选择其实非常多元,
没有唯一的标准答案。
重要的是,我们要认识自己、找到最适合自己的道路,
而不是人云亦云。

简历指导板块是整个集市中最实用的环节之一。
活动现场有专门的老师帮我们修改简历,
指出了很多细节上的问题,
比如简历格式是否规范、语言是否简洁有力、
经历描述是否突出亮点、是否使用了量化成果等。
我之前写的简历比较简单,
基本就是把自己做过的事情罗列出来,没有太多重点。
通过老师的指导,我学到了如何用“STAR法则”去描述经历,
即情境(Situation)、任务(Task)、行动(Action)、
结果(Result),这样才能让用人单位看到我们在特定环境下的
具体表现和成效。
老师还强调了简历内容要与应聘岗位高度相关,
不要把所有经历一股脑堆上去,
而是要有选择地展现最能匹配岗位需求的部分。
这个环节让我受益匪浅,
我也打算回去以后好好修改和完善自己的简历。

最后一个环节是就业政策解读。
这个板块内容丰富、信息量大,
包括国家和地方的最新就业政策、
毕业生就业优惠政策、基层就业、参军入伍、研究生扩招等内容。
讲解老师还详细介绍了国家针对应届毕业生推出的一些就业补贴、
见习岗位、创业扶持等措施,这些以前我基本都没有了解过。
通过这次解读,我发现国家和地方政府其实在就业方面投入了
大量资源,只是很多时候我们信息不对称,错过了许多机会。
老师提醒我们要多关注学校的就业信息网、
各级人社部门发布的政策,
积极申请各种补贴和岗位,这对我们顺利就业有很大帮助。

总的来说,这次就业指导集市是一场干货满满、收获颇丰的活动。
不仅让我学到了简历修改、法律维权等实用技巧,
也让我更全面地认识了自己、看清了就业和生涯发展的多种可能。
我认为这样的活动对大学生来说非常重要,
它不仅帮助我们提升就业竞争力,
也让我们少走弯路、少踩坑。
我希望未来还能多参加类似的活动,
让自己在步入社会前做好更充分的准备。

通过这次活动,我也深刻感受到:
就业不是一场单纯的“找工作”行动,
而是一个系统的、生涯性的规划过程。
我们需要综合考虑个人兴趣、能力、价值观、
社会需求、政策环境等因素,
做出既理性又符合内心选择的决定。
我相信,只要我们愿意学习、愿意探索、愿意努力,
未来一定是充满希望和机会的。

\end{document}
