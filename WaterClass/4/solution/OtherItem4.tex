%!Mode:: "TeX:UTF-8"
%!TEX encoding = UTF-8 Unicode
%!TEX TS-program = xelatex
\documentclass{ctexart}
\newif\ifpreface
%\prefacetrue
\usepackage{fontspec}
\usepackage{bbm}
\usepackage{tikz}
\usepackage{amsmath,amssymb,amsthm,color,mathrsfs}
\usepackage{fixdif}
\usepackage{hyperref}
\usepackage{cleveref}
\usepackage{enumitem}%
\usepackage{expl3}
\usepackage{lipsum}
\usepackage[margin=0pt]{geometry}
\usepackage{listings}
\definecolor{mGreen}{rgb}{0,0.6,0}
\definecolor{mGray}{rgb}{0.5,0.5,0.5}
\definecolor{mPurple}{rgb}{0.58,0,0.82}
\definecolor{backgroundColour}{rgb}{0.95,0.95,0.92}

\lstdefinestyle{CStyle}{
  backgroundcolor=\color{backgroundColour},
  commentstyle=\color{mGreen},
  keywordstyle=\color{magenta},
  numberstyle=\tiny\color{mGray},
  stringstyle=\color{mPurple},
  basicstyle=\footnotesize,
  breakatwhitespace=false,
  breaklines=true,
  captionpos=b,
  keepspaces=true,
  numbers=left,
  numbersep=5pt,
  showspaces=false,
  showstringspaces=false,
  showtabs=false,
  tabsize=2,
  language=C
}
\usetikzlibrary{calc}
\theoremstyle{remark}
\newtheorem{lemma}{Lemma}
\usepackage{fontawesome5}
\usepackage{xcolor}
\newcounter{problem}
\newcommand{\Problem}{\begin{tikzpicture}[baseline]%
    \node at (-0.02em,0.3em) {$\mathbb{P}$};%
    \node[scale=0.7] at (0.2em,-0.0em) {R};%
    \node[scale=0.7] at (0.6em,0.4em) {O};%
    \node[scale=0.8] at (1.05em,0.25em) {B};%
    \node at (1.55em,0.3em) {L};%
    \node[scale=0.7] at (1.75em,0.45em) {E};%
    \node at (2.35em,0.3em) {M};%
  \end{tikzpicture}%
}
\renewcommand{\theproblem}{\Roman{problem}}
\newenvironment{problem}{\refstepcounter{problem}\noindent\color{blue}\Problem\theproblem}{}

\crefname{problem}{\protect\Problem}{Problem}
\newcommand\Solution{\begin{tikzpicture}[baseline]%
    \node at (-0.04em,0.3em) {$\mathbb{S}$};%
    \node[scale=0.7] at (0.35em,0.4em) {O};%
    \node at (0.7em,0.3em) {\textit{L}};%
    \node[scale=0.7] at (0.95em,0.4em) {U};%
    \node[scale=1.1] at (1.19em,0.32em){T};%
    \node[scale=0.85] at (1.4em,0.24em){I};%
    \node at (1.9em,0.32em){$\mathcal{O}$};%
    \node[scale=0.75] at (2.3em,0.21em){\texttt{N}};%
  \end{tikzpicture}}
\newenvironment{solution}{\begin{proof}[\Solution]}{\end{proof}}
\title{\input{../../.subject}\input{../.number}}
\makeatletter
\newcommand\email[1]{\def\@email{#1}\def\@refemail{mailto:#1}}
\newcommand\schoolid[1]{\def\@schoolid{#1}}
\ifpreface
  \def\@maketitle{
  \raggedright
  {\Huge \bfseries \sffamily \@title }\\[1cm]
  {\Huge  \bfseries \sffamily\heiti\@author}\\[1cm]
  {\Huge \@schoolid}\\[1cm]
  {\Huge\href\@refemail\@email}\\[0.5cm]
  \Huge\@date\\[1cm]}
\else
  \def\@maketitle{
    \raggedright
    \begin{center}
      {\Huge \bfseries \sffamily \@title }\\[4ex]
      {\Large  \@author}\\[4ex]
      {\large \@schoolid}\\[4ex]
      {\href\@refemail\@email}\\[4ex]
      \@date\\[8ex]
    \end{center}}
\fi
\makeatother
\ifpreface
  \usepackage[placement=bottom,scale=1,opacity=1]{background}
\fi

\author{白永乐}
\schoolid{25110180002}
\email{ylbai25@m.fudan.edu.cn}

\def\to{\rightarrow}
\newcommand{\xor}{\vee}
\newcommand{\AND}{\wedge}
\newcommand{\OR}{\vee}
\newcommand{\bor}{\bigvee}
\newcommand{\band}{\bigwedge}
\newcommand{\xand}{\wedge}
\newcommand{\minus}{\mathbin{\backslash}}
\newcommand{\mi}[1]{\mathscr{P}(#1)}
\newcommand{\card}{\mathrm{card}}
\newcommand{\oto}{\leftrightarrow}
\newcommand{\hin}{\hat{\in}}
\newcommand{\gl}{\mathrm{GL}}
\newcommand{\im}{\mathrm{Im}}
\newcommand{\re }{\mathrm{Re }}
\newcommand{\rank}{\mathrm{rank}}
\newcommand{\tra}{\mathop{\mathrm{tr}}}
\renewcommand{\char}{\mathop{\mathrm{char}}}
\DeclareMathOperator{\ot}{ordertype}
\DeclareMathOperator{\dom}{dom}
\DeclareMathOperator{\ran}{ran}

\title{逻辑学第一次作业}
\begin{document}
\large
\setlength{\baselineskip}{1.2em}
\ifpreface
  \input{../../../global/preface}
  \newgeometry{left=2cm,right=2cm,top=2cm,bottom=2cm}
\else
  \newgeometry{left=2cm,right=2cm,top=2cm,bottom=2cm}
  \maketitle
\fi
%from_here_to_type
\begin{problem}\label{pro:1}
  下列语句是否表达命题? 为什么?
  \begin{enumerate}
    \item 没有耕耘, 哪来收获?
    \item 飞翔吧, 祖国的雄鹰!
    \item 为胜利而干杯!
  \end{enumerate}
\end{problem}

\begin{solution}
  \begin{enumerate}
    \item 不是,命题是陈述句,而它不是陈述句。
    \item 不是,命题是陈述句,而它不是陈述句。
    \item 不是,命题是陈述句,而它不是陈述句。
  \end{enumerate}

\end{solution}
\begin{problem}\label{pro:2}
  请根据假言推理的有关知识, 回答下列问题:
  \begin{enumerate}
    \item “要是降落的球不受外力影响, 它就不会改变降落的方向, 既然球受到了外力的影响, 因此, 它改变了降落方向。”这个推理对不对? 为什么?
    \item “了解情况, 才能避免主观性; 此人主观, 可见, 他不了解情况。”这个推理对不对? 为什么?
    \item “只有甲队体力强, 技术高, 配合好, 才能战胜乙队, 甲队体力不强, 或技术不高, 或配合不好; 所以,甲队不能战胜乙队。”这个推理对不对? 为什么?
  \end{enumerate}
\end{problem}

\begin{solution}
  \begin{enumerate}
    \item 不对。设\(p:\text{降落的球不受外力影响}, q:\text{不会改变降落的方向}\),\(p \to q\)不能得出\(\neg p \to \neg q\).
    \item 不对。设\(p:\text{避免主观性}, q: \text{了解情况}\),\(p \leftarrow \neg q\) 不能得出\(q \to \neg p\)。
    \item 对。设\(p:\text{甲队体力强, 技术高, 配合好}, q: \text{能战胜乙队}\),\(p \leftarrow q\) 能得出\(\neg p \to \neg q\)。
  \end{enumerate}

\end{solution}

\begin{problem}\label{pro:3}
  下列推理各属何种形式的二难推理?
  \begin{enumerate}
    \item 如果这是一部好作品, 那么它思想性一定好; 如果这是一部好作品,那么它的艺术性一定高; 而这部作品或者思想性不好, 或者艺术性不高; 所以, 这不是一部好作品。
    \item 如果承认矛可以戳穿盾, 这说明盾没有他所夸的那么好; 如果承认矛戳不穿盾, 这就说明矛并没有象他所说的那么好; 或者矛可以戳穿盾, 或者矛戳不穿盾; 可见,或者他的盾不好,或者他的矛不好。
    \item “如果你考上大学, 那么, 你要利用在校时间努力学习; 如果你考不上大学, 你要在业余时间坚持自学; 你或者考上大学, 或者考不上大学; 总之, 你或者要利用在校时间努力学习,或者你要在业余时间坚持自学。”
  \end{enumerate}
\end{problem}
\begin{solution}
  \begin{enumerate}
    \item 设\(p:\text{作品好}\),\(q:\text{作品思想好}\),\(r:\text{作品艺术性高}\)。
      符号化后\((p \to q)\AND (p \to r) \vdash (\neg q \AND \neg r \to \neg p)\)。简单破坏式。
    \item 设\(p : \text{矛穿盾}\), \(q :\text{盾不好}\),\(r:\text{矛不好}\)。
      符号化后\((p \to q)\AND (\neg p \to r) \vdash ( p \OR \neg p \to q \OR r )\)。复杂构成式。
    \item 设\(p:\text{考上大学}\), \(q:\text{在校时间努力学习}\),\(r:\text{业余时间坚持自学}\)。
      符号化后\((p \to q)\AND (\neg p \to r) \vdash ( p \OR \neg p \to q \OR r )\)。复杂构成式。
  \end{enumerate}

\end{solution}

\begin{problem}\label{pro:4}
  请运用二难推理有关知识,回答下列问题:

  有个法院审理一起盗窃案件, 某村的 $A 、 B 、 C$ 三人作为嫌疑犯被押上法庭。这个法院的法官是这样想的: 初审的时候, 在没有威迫的情况下, 假如不是盗窃犯就不会说假话; 相反, 真正的盗窃犯是一定会为了掩盖罪行而编造借口的。因此, 法官得出这样的结论: 说真话的肯定不是盗窃犯, 说假话的肯定就是盗窃犯。事后, 事实也证明法官的这个想法是正确的。
  审问开始了。
  法官先问 $A$ :
  “你是怎样作案的? 快快从实招来!”
  $A$ 回答了法官的问题。但是, $A$ 讲的是某地的方言, 法官根本听不懂 $A$讲的话是什么意思。
  于是, 法官就问 $B$ 和 $C$ :
  “刚才 $A$ 是如何回答我的问题?”
  $B$ 说: “禀告法官老爷: $A$ 的意思是说, 他并不是盗窃犯。”
  $C$ 说: “禀告法官老爷: $A$ 刚才招供了,他承认自己就是盗窃犯。”
  $B$ 和 $C$ 说的话法官是能听懂的。听罢 $B$ 和 $C$ 的话之后, 这位法官马上作出判决: $B$ 无罪释放, $C$ 是盗窃犯应予逮捕人狱。
  请问: 这位聪明的法官为什么能根据 $B$ 和 $C$ 的回答, 作出这样的判决?
\end{problem}
\begin{solution}
  若\(A\) 是窃贼,则\(A\) 会撒谎说自己不是窃贼; 若\(A\) 不是窃贼,则\(A\) 会诚实地说自己不是窃贼。\(A\) 或者是窃贼,或者不是窃贼,故\(A\) 一定说自己不是窃贼。

  \(B\) 说\(A\) 说自己不是窃贼,因此\(B\) 没有说谎,所以\(B\) 不是窃贼。而\(C\) 说\(A\) 说自己是窃贼, 因此\(C\) 在说谎,故\(C\) 一定是窃贼。
\end{solution}
\begin{problem}\label{pro:5}

  下列推理属何种推理? 请列出它们的推理形式, 并说明是否有效?为什么?
  \begin{enumerate}
    \item 如果寒潮到来, 气温就要明显下降。所以, 如果气温没有明显下降, 就是寒潮没有到来。
    \item 只有充分发展商品生产, 才能把我国的经济搞活; 只有把我国的经济搞活, 才能加快四化建设的速度; 所以, 如果要加快四化建设的速度,就要充分发展商品生产。
    \item 如果要建设社会主义的物质文明, 那么就要大力发展社会生产; 如果要建设社会主义精神文明, 那么就要大力加强和改善思想政治工作; 我们既要建设社会主义的物质文明, 又要建设社会主义的精神文明; 所以, 我们又要大力发展社会生产, 又要加强和改善思想政治工作。
  \end{enumerate}
\end{problem}
\begin{solution}
  \begin{enumerate}
    \item 设\(p:\text{寒潮降临}\),\(q:\text{气温明显下降}\)。
      推理形式是\((p \to q ) \vdash (\neg q \to \neg p)\), 属于充分条件假言易位推理,属于有效推理,因为\((p \to q) \leftrightarrow (\neg q \to \neg p)\)。
    \item 设\(p:\text{充分发展商品生成}\), \(q:\text{把我国经济搞活}\), \(r:\text{加快四化建设速度}\)。
      推理形式是\((p \leftarrow q) \AND (q \leftarrow r) \vdash (p \leftarrow r)\), 属于假言联锁推理,属于有效推理,因为它是必要条件肯定式。
    \item 设\(p:\text{建设社会主义物质文明}\), \(q:\text{大力发展社会生产}\),
      \\\(r:\text{建设社会主义精神文明}\), \(t:\text{大力加强和改善思想政治工作}\)。
      推理形式是\((p \to q) \AND (r \to t) \AND(p \AND r) \vdash (q \AND t)\), 属于假言联言推理,属于有效推理,因为属于肯定式。
  \end{enumerate}

\end{solution}

\begin{problem}\label{pro:6}

  首先指出下列命题中所包含的简单命题, 然后使用符号表示它们, 并且指出其中的主联结词, 给出它们的真值表。
  \begin{enumerate}
    \item 高水平的舞蹈演员不怕吃苦, 而且还有常人不具备的毅力。
    \item 如果逻辑学使人更聪明, 那么人人都应认真学习逻辑学。
    \item 张三能推出答案当且仅当他具有推理能力。
  \end{enumerate}

\end{problem}
\begin{solution}
  \begin{enumerate}
    \item 设\(p:\text{舞蹈演员的水平高}\), \(q: \text{不怕吃苦}\),\(r : \text{具有常人不具备的毅力}\)。
      \(p \to (q \AND r)\), 主联结词是蕴含,真值表为:
      \[
        \begin{matrix}
          p & q & r & q \AND r & p \to (q \AND r) \\
          1 & 1 & 1 & 1        & 1                \\
          1 & 0 & 1 & 0        & 0                \\
          1 & 1 & 0 & 0        & 0                \\
          1 & 0 & 0 & 0        & 0                \\
          0 & 1 & 1 & 1        & 1                \\
          0 & 0 & 1 & 0        & 1                \\
          0 & 1 & 0 & 0        & 1                \\
          0 & 0 & 0 & 0        & 1                \\
        \end{matrix}
      \]
    \item 设\(p:\text{此人学逻辑学}\), \(q:\text{此人聪明}\),\(r: \text{人人应认真学习逻辑学}\)。
      \((p \to q)\to r\),主联结词是蕴含,真值表为:
      \[
        \begin{matrix}
          p & q & r & p \to q & (p \to q) \to r \\
          1 & 1 & 1 & 1       & 1               \\
          0 & 1 & 1 & 1       & 1               \\
          1 & 0 & 1 & 0       & 1               \\
          0 & 0 & 1 & 1       & 1               \\
          1 & 1 & 0 & 1       & 0               \\
          1 & 0 & 0 & 0       & 1               \\
          0 & 1 & 0 & 1       & 0               \\
          0 & 0 & 0 & 1       & 0               \\
        \end{matrix}
      \]
    \item 设\(p:\text{张三推出答案}\), \(q:\text{张三具有推理能力}\)。
      \(p \leftrightarrow q\),主联结词是等价,真值表为:
      \[
        \begin{matrix}
          p & q & p \leftrightarrow q \\
          1 & 1 & 1                   \\
          1 & 0 & 0                   \\
          0 & 1 & 0                   \\
          0 & 0 & 1                   \\
        \end{matrix}
      \]
  \end{enumerate}

\end{solution}

\begin{problem}\label{pro:7}

  判定下列每组命题中两个命题之间是否逻辑等值。
  \begin{enumerate}
    \item $\mathrm{B} \wedge \mathrm{A} \rightarrow \neg \mathrm{C}$ 和 $\neg \mathrm{A} \vee \neg \mathrm{B} \rightarrow \mathrm{C}$
    \item $\neg(\mathrm{A} \vee(\mathrm{C} \rightarrow \mathrm{B}))$ 和 $\neg \mathrm{A} \wedge \mathrm{C} \wedge \neg \mathrm{B}$
  \end{enumerate}

\end{problem}

\begin{solution}
  \begin{enumerate}
    \item 当\(A = 0, B=1,C=1\), 那么\(\neg A \OR \neg B = 1, \neg C=0\), 而\(B \AND A \to \neg C = 1\),\(\neg A \AND \neg B \to C = 0\)。真值表不一样故不等价。
    \item \(\neg(A \OR(C \to B)) \iff \neg A \AND \neg (\neg C \OR B) \iff \neg A \AND (\neg \neg C \AND \neg B )\iff \neg A \AND C \AND \neg B\)。故逻辑等值。
  \end{enumerate}

\end{solution}

\begin{problem}\label{pro:8}

  命题 “如果商品价廉并且物美, 那么商品畅销” 与下列哪些命题是逻辑等值的:
  \begin{enumerate}
    \item 如果商品不畅销, 那么价不廉并且物不美。
    \item \label{tt1} 如果商品不畅销, 那么价不廉或者物不美。
    \item \label{tt2}如果商品价廉,那么如果它物美就会畅销。
    \item \label{tt3}如果商品价廉,那么或者它物美或者它不会畅销。
  \end{enumerate}

\end{problem}

\begin{solution}
  设\(p:\text{商品畅销}\),\(q:\text{商品廉价}\),\(r:\text{商品物美}\)。
  那么“如果商品价廉并且物美, 那么商品畅销”,表示为\(q \AND r \to p\)。
  上述句子可以表示为:
  \begin{enumerate}
    \item \(\neg p \to \neg q \AND \neg r\)。
    \item \(\neg p \to \neg q \OR \neg r\)。
    \item \(q \to (r \to p)\)。
    \item \(q \to r \OR \neg p\)。
  \end{enumerate}
  故原语句和\ref{tt1},\ref{tt2},\ref{tt3}等价。
\end{solution}
\begin{problem}\label{pro:9}

  考虑如下推理: 某地发生一起凶杀案。经过分析, 凶手是两人合谋。初步确定 a、b、c、d、e 五人是嫌疑犯。警方掌握了以下情况:
  \begin{enumerate}
    \item a 和 d 两人中至少有一人是凶手。
    \item 如果 d 是凶手, 那么 e 也是凶手。
    \item 如果 b 是凶手,那么 c 是凶手。
    \item 如果 b 不是凶手,那么 a 也不可能是凶手。
    \item c 不是凶手。
  \end{enumerate}
  根据以上情况, 警方推断 d 和 e 是凶手。请写出警方推理的形式。
\end{problem}
\begin{solution}
  设\(p_1:\text{a 和 d 两人中至少有一人是凶手}\)\\
  \(p_2:\text{如果 d 是凶手,那么 e 也是凶手}\)\\
  \(p_3:\text{如果 b 是凶手,那么 c 是凶手}\)\\
  \(p_4:\text{如果 b 不是凶手,那么 a 也不可能是凶手}\)\\
  \(p_5:\text{c 不是凶手}\) 。
  由于\(p_5 \AND p_3  \to \text{b 不是凶手}\), \(\text{b 不是凶手} \AND p_4 \to \text{a 不是凶手}\),
  \(\text{a 不是凶手} \AND p_1 \to \text{d 是凶手}\), \(\text{d 是凶手} \AND p_2 \to \text{e 是凶手}\)。
\end{solution}

\end{document}
