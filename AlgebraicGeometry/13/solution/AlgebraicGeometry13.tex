%!Mode:: "TeX:UTF-8"
%!TEX encoding = UTF-8 Unicode
%!TEX TS-program = xelatex
\documentclass{ctexart}
\newif\ifpreface
\prefacetrue
\usepackage{fontspec}
\usepackage{bbm}
\usepackage{tikz}
\usepackage{amsmath,amssymb,amsthm,color,mathrsfs}
\usepackage{fixdif}
\usepackage{hyperref}
\usepackage{cleveref}
\usepackage{enumitem}%
\usepackage{expl3}
\usepackage{lipsum}
\usepackage[margin=0pt]{geometry}
\usepackage{listings}
\definecolor{mGreen}{rgb}{0,0.6,0}
\definecolor{mGray}{rgb}{0.5,0.5,0.5}
\definecolor{mPurple}{rgb}{0.58,0,0.82}
\definecolor{backgroundColour}{rgb}{0.95,0.95,0.92}

\lstdefinestyle{CStyle}{
  backgroundcolor=\color{backgroundColour},
  commentstyle=\color{mGreen},
  keywordstyle=\color{magenta},
  numberstyle=\tiny\color{mGray},
  stringstyle=\color{mPurple},
  basicstyle=\footnotesize,
  breakatwhitespace=false,
  breaklines=true,
  captionpos=b,
  keepspaces=true,
  numbers=left,
  numbersep=5pt,
  showspaces=false,
  showstringspaces=false,
  showtabs=false,
  tabsize=2,
  language=C
}
\usetikzlibrary{calc}
\theoremstyle{remark}
\newtheorem{lemma}{Lemma}
\usepackage{fontawesome5}
\usepackage{xcolor}
\newcounter{problem}
\newcommand{\Problem}{\begin{tikzpicture}[baseline]%
    \node at (-0.02em,0.3em) {$\mathbb{P}$};%
    \node[scale=0.7] at (0.2em,-0.0em) {R};%
    \node[scale=0.7] at (0.6em,0.4em) {O};%
    \node[scale=0.8] at (1.05em,0.25em) {B};%
    \node at (1.55em,0.3em) {L};%
    \node[scale=0.7] at (1.75em,0.45em) {E};%
    \node at (2.35em,0.3em) {M};%
  \end{tikzpicture}%
}
\renewcommand{\theproblem}{\Roman{problem}}
\newenvironment{problem}{\refstepcounter{problem}\noindent\color{blue}\Problem\theproblem}{}

\crefname{problem}{\protect\Problem}{Problem}
\newcommand\Solution{\begin{tikzpicture}[baseline]%
    \node at (-0.04em,0.3em) {$\mathbb{S}$};%
    \node[scale=0.7] at (0.35em,0.4em) {O};%
    \node at (0.7em,0.3em) {\textit{L}};%
    \node[scale=0.7] at (0.95em,0.4em) {U};%
    \node[scale=1.1] at (1.19em,0.32em){T};%
    \node[scale=0.85] at (1.4em,0.24em){I};%
    \node at (1.9em,0.32em){$\mathcal{O}$};%
    \node[scale=0.75] at (2.3em,0.21em){\texttt{N}};%
  \end{tikzpicture}}
\newenvironment{solution}{\begin{proof}[\Solution]}{\end{proof}}
\title{\input{../../.subject}\input{../.number}}
\makeatletter
\newcommand\email[1]{\def\@email{#1}\def\@refemail{mailto:#1}}
\newcommand\schoolid[1]{\def\@schoolid{#1}}
\ifpreface
  \def\@maketitle{
  \raggedright
  {\Huge \bfseries \sffamily \@title }\\[1cm]
  {\Huge  \bfseries \sffamily\heiti\@author}\\[1cm]
  {\Huge \@schoolid}\\[1cm]
  {\Huge\href\@refemail\@email}\\[0.5cm]
  \Huge\@date\\[1cm]}
\else
  \def\@maketitle{
    \raggedright
    \begin{center}
      {\Huge \bfseries \sffamily \@title }\\[4ex]
      {\Large  \@author}\\[4ex]
      {\large \@schoolid}\\[4ex]
      {\href\@refemail\@email}\\[4ex]
      \@date\\[8ex]
    \end{center}}
\fi
\makeatother
\ifpreface
  \usepackage[placement=bottom,scale=1,opacity=1]{background}
\fi

\author{白永乐}
\schoolid{202011150087}
\email{202011150087@mail.bnu.edu.cn}

\def\to{\rightarrow}
\newcommand{\xor}{\vee}
\newcommand{\bor}{\bigvee}
\newcommand{\band}{\bigwedge}
\newcommand{\xand}{\wedge}
\newcommand{\minus}{\mathbin{\backslash}}
\newcommand{\mi}[1]{\mathscr{P}(#1)}
\newcommand{\card}{\mathrm{card}}
\newcommand{\oto}{\leftrightarrow}
\newcommand{\hin}{\hat{\in}}
\newcommand{\gl}{\mathrm{GL}}
\newcommand{\im}{\mathrm{Im}}
\newcommand{\re }{\mathrm{Re }}
\newcommand{\rank}{\mathrm{rank}}
\newcommand{\tra}{\mathop{\mathrm{tr}}}
\renewcommand{\char}{\mathop{\mathrm{char}}}
\DeclareMathOperator{\ot}{ordertype}
\DeclareMathOperator{\dom}{dom}
\DeclareMathOperator{\ran}{ran}

\begin{document}
\large
\setlength{\baselineskip}{1.2em}
\ifpreface
    \backgroundsetup{contents={%
    \begin{tikzpicture}
      \fill [white] (current page.north west) rectangle ($(current page.north east)!.3!(current page.south east)$) coordinate (a);
      \fill [bgc] (current page.south west) rectangle (a);
\end{tikzpicture}}}
\definecolor{word}{rgb}{1,1,0}
\definecolor{bgc}{rgb}{1,0.95,0}
\setlength{\parindent}{0pt}
\thispagestyle{empty}
\begin{tikzpicture}%
  % \node[xscale=2,yscale=4] at (0cm,0cm) {\sffamily\bfseries \color{word} under};%
  \node[xscale=4.5,yscale=10] at (10cm,1cm) {\sffamily\bfseries \color{word} Graduate Homework};%
  \node[xscale=4.5,yscale=10] at (8cm,-2.5cm) {\sffamily\bfseries \color{word} In Mathematics};%
\end{tikzpicture}
\ \vspace{1cm}\\
\begin{minipage}{0.25\textwidth}
  \textcolor{bgc}{王胤雅是傻逼}
\end{minipage}
\begin{minipage}{0.75\textwidth}
  \maketitle
\end{minipage}
\vspace{4cm}\ \\
\begin{minipage}{0.2\textwidth}
  \
\end{minipage}
\begin{minipage}{0.8\textwidth}
  {\Huge
    \textinconsolatanf{}
  }General fire extinguisher
\end{minipage}
\newpage\backgroundsetup{contents={}}\setlength{\parindent}{2em}

\newgeometry{left=2cm,right=2cm,top=2cm,bottom=2cm}
\else
\newgeometry{left=2cm,right=2cm,top=2cm,bottom=2cm}
\maketitle
\fi
%from_here_to_type
\begin{problem}
  Assume \(\Omega \subset \mathbb{C}\) is a domain. Prove that \(f\) is meromorphic map over \(\Omega\) is equiv for \(\Omega\) as opensubset of \(\mathbb{C}\) and as Remian surface. 
\end{problem}

\begin{solution}
  First we assume \(f:\Omega \to \mathbb{C}_{\infty}\) is holomorphic. Let \(T:=\{x \in \Omega:f(x)=\infty\}\). 
  Now we prove \(f\) is meromorphic over \(\Omega\). 
  Since \(\forall x \in T,f(x)=\infty\) and \(f\) is continous, we get \(\forall x \in T,\lim_{y \to x}f(y)=\infty\). 
  Now we only need to prove every \(x \in T\) is isolated point. If not, we will prove \(f \equiv \infty\). 
  Let \(V:=\{x \in \Omega:f(x)=\infty \wedge \exists x_n \in \Omega,x_n \neq x,x_n \to x,f(x_n)=\infty\} \neq \varnothing\). 
  Easily \(V\) is closed in \(\Omega\), now we prove it's open, too. 
  Assume \(x \in V\), and \(x_n \in \Omega,x_n \neq x,x_n \to x,f(x_n)=\infty\). 
  Since \(f\) is holomorphic, we get \(\exists V:\infty \in V \subset \mathbb{C}_{\infty}\) is open, \(\exists U:x \in U \subset \Omega\) is open, 
  such that \(g:=\phi \circ \res{f}{U} \circ \text{id}:U \to \mathbb{C}\) is holomorphic, where \(\phi(x)=\frac{1}{x}\). 
  Then \(g(x)=g(x_n)=0\). So \(\res{g}{U}\equiv 0\). So we get \(\res{f}{U} \equiv 0\). So \(U \subset V\). 
  So \(V\) is open in \(\Omega\). Since \(\Omega \) is connected, we get \(V=\Omega\). 
  So \(f \equiv \infty\). 

  Second we assume \(f:W \to \mathbb{C}\) is holomorphic, where \(W \subset \Omega\) is open, 
  and \(\forall x \in T:=\Omega \setminus W,\lim_{y \to x}f(y)=\infty\), and \(x\) is isolated point. 
  Now let \(h:\Omega \to \mathbb{C}_{\infty},\res{h}{W}=f,\res{h}{T}\equiv \infty\). 
  We only need to prove \(h\) is holomorphic from \(\Omega\)(as Remian surface) to \(\mathbb{C}_{\infty}\). 
  Only need to prove \( h\) is holomorphic on \(x \in T\). Let \(\phi:\mathbb{C}\setminus \{0\}\to \mathbb{C},x \mapsto \frac{1}{x}\). 
  Let \(U \subset h^{-1}(\mathbb{C} \setminus \{0\}) \subset \Omega\) is a neibor of \(x\) and \(\forall y \in U \setminus \{x\},h(y)\neq \infty\). 
  Now we prove \(g:=\phi \circ h \circ \text{id}:U \to \mathbb{C}\) is holomorphic. 
  Easily \(h(z)= \frac{\psi(z)}{(z-x)^n}\) for some \(n \in \mathbb{N}^+\) and holomorphic map \(\psi\) where \(0 \notin \psi(U)\). 
  So we get \(g(z)=\frac{1}{h(z)}=\frac{(z-x)^n}{\psi(z)}\) is holomorphic over \(U\). 
\end{solution}

\begin{problem}
  Let \(M\) is a Remian surface, \(f:M \to \mathbb{C}_{\infty}\) is holomorphic. Assume \(p \in M\). 
  Prove that \(\ord_p(f)\) is well-defined. 
\end{problem}

\begin{solution}
  Assume \(U \subset M\) is openset and \(p \in U\), and \(\phi,\psi:U \to \mathbb{C}_{\infty}\) are holomorphic. 
  Consider \(f_\phi:=f \circ \phi^{-1}:\phi(U)\to \mathbb{C}\) and \(f_\psi:=f \circ \psi^{-1}:\psi(U) \to \mathbb{C}\) are meromorphic. 
  We only need prove \(f_\phi\) at \(\phi(p)\) and \(f_\psi\) at \(\psi(p)\) have the same ord. 
  Since \(M\) is Remian surface we know \(\phi \circ \psi^{-1}:\psi(U)\to \phi(U)\) is holomorphic. 
  Without loss of generality we can assume \(\phi(p)=\psi(p)=0\). 
  Assume \(f_\phi(z)=z^nh(z)\), where \(h(z)\) is holomorphic near \(0\) and \(h(0)\neq 0\). 
  Then \(f_\psi(z)=f_{\phi}\circ (\phi \circ \psi^{-1})(z)=\phi \circ \psi^{-1} (z)^n h(\phi \circ \psi^{-1}(z))\)
  Since \(\phi(p)=\psi(p)=0\) we get \(\phi \circ \psi^{-1}(0)=0\). So \(h(\phi \circ \psi^{-1}(0)) \neq 0\). 
  And \(\lim_{z \to 0}\frac{1}{z}\phi \circ \psi^{-1}(z)\) exists. 
  For the same reason we get \(\lim_{z \to 0}\frac{1}{z}\psi \circ \phi^{-1}(z)\) exists. 
  So \(\lim_{z \to 0}\phi \circ \psi^{-1}(z)\neq 0\). 
  So finally we get \(\ord_p(f)\) is well-defined. 
\end{solution}

\begin{problem}
  Assume \(p(z)\in \mathbb{C}[z]\) is a poly and not const. 
  Consider \(f:\mathbb{C}_{\infty}\to \mathbb{C}_{\infty},f(z)= \begin{cases}
  p(z) & z \in \mathbb{C}\\
  \infty & z=\infty
  \end{cases}\). 
  Prove that \(f\) is holomorphic. 
\end{problem}

\begin{solution}
  Obviously for \(z \in \mathbb{C}\) we have \(f\) is holomorphic near \(z\). So we only need to prove \(f\) is holomorphic near \(\infty\). 
  Let \(V \subset \mathbb{C}_{\infty}\) is a neibor of \(\infty\) and \(0 \notin V\). 
  Since \(\deg p>0\), we get \(\lim_{z \to \infty}p(z)=\infty\). 
  So \(\exists U \subset \mathbb{C}_{\infty}\) such that \(f(U)\subset V\). without loss of generality assume \(0 \notin U\). 
  Now we only need to prove \(\phi \circ f \circ \phi\) is holomorphic at \(0\), where \(\phi(z)=\frac{1}{z}\). 
  Assume \(f(z)=\sum_{i=0}^n a_i z^i\) and \(a_n \neq 0\), then \(\frac{1}{f(\frac{1}{z})}=\frac{z^n}{\sum_{i=0}^n a_i z^{n-i}}\). 
  Since \(a_n \neq 0\) we get \(\frac{1}{f(\frac{1}{z})}\) is holomorphic near \(0\). 
\end{solution}

\begin{problem}
  Let \(\omega_1,\omega_1 \in \mathbb{C}^*\) and are \(\mathbb{R}\)-linear independent. 
  Let \(\Lambda\) is the addtive group generated by \(\omega_1,\omega_2\) and \(k \in \mathbb{N}^+,k \geq 3 \). 
  Prove that \(g(z)=\sum_{\omega \in \Lambda} \frac{1}{(z-\omega)^k}\) locally uniformly converge in \(\mathbb{C} \setminus \Lambda\). 
  And \(g(z)=g(z+\omega),\forall \omega \in \Lambda\). 
\end{problem}

\begin{solution}
  Write \(\Lambda=\{z_n:n \in \mathbb{N}^+\}\), then we only need to prove \(\sum_{n=1}^{\infty} \frac{1}{(z-z_n)^k}\) locally uniformly converge in \(\mathbb{C} \setminus \Lambda\). 
  Without loss of generality we can assume \(|z_n| \leq |z_{n+1}|,\forall n \in \mathbb{N}^+\). 
  Easily we know \(\card(B(0,n)\cap \Lambda)=O(n^2),n \to \infty\). 
  So we get \(|z_n|=O(\sqrt{n}),n \to \infty\). 
  Assume \(M \subset \mathbb{C}\setminus L\) is cpt, now we need to prove the series converge uniformly in \(M\). 
  Let \(\lambda:=d(M,\Lambda)>0\), then we get \(|z-z_n|>\lambda\). 
  And \(|z_n-z| \geq |z_n|-|z| \geq |z_n|-\max\{|z|:z \in M\}=O(\sqrt{n}),n \to \infty\). 
  So \(\exists N \in \mathbb{N}^+,t \in \mathbb{R}^+,\forall n>N,|z_n-z| \geq t \sqrt{n}\). 
  For \(n \leq N\), we have \(|z_n-z| \geq \lambda\), without loss of generality assume \(t<\frac{\lambda}{\sqrt{N}}\), then \(|z_n-z| >t \sqrt{n}\), too. 
  So finally we get \(\left|\frac{1}{(z_n-z)^k}\right| \leq \frac{t}{n^{ \frac{k}{2}}}, \forall z \in M\). 
  Since \(\sum_{n \in \mathbb{N}^+} \frac{t}{n^{ \frac{k}{2}}}<\infty\), we get the series converge uniformly in \(M\). 

  Now we prove \(g(z)=g(z+\omega)\). If \(z \in \Lambda\), then easily \(g(z)=\infty=g(z+\omega)\). 
  Now we assume \(z \notin \Lambda\). Then we easily get \(|z-z_n|=O(\sqrt{n})\), so \(\sum_{n=1}^{\infty} \frac{1}{(z-z_n)^k}\) converge absolutely. 
  So we can change the order of sum. Since \(\Lambda \to \Lambda,x \mapsto x+\omega\) is bijection, we get \(\exists \sigma:\mathbb{N}^+ \to \mathbb{N}^+\) is bijection 
  and \(z_{\sigma(n)}=z_n-\omega\). 
  So we get \(g(z+\omega)=\sum_{n=1}^{\infty} \frac{1}{(z-z_n+\omega)^k}=\sum_{n=1}^{\infty} \frac{1}{(z-z_{\sigma(n)})^k}=\sum_{n=1}^{\infty} \frac{1}{(z-z_n)^k}=g(z)\). 
\end{solution}

\begin{problem}
  Let \(\Lambda\) be same as above, let \(\wp(z):=\sum_{\omega \in \Lambda \setminus \{0\}} \left(\frac{1}{(z-\omega)^2} -\frac{1}{\omega^2}\right)+\frac{1}{z^2}\). 
  Prove this series converge locally uniformly in \(\mathbb{C} \setminus \Lambda\). 
  And \(\wp(z)=\wp(z+\omega),\forall \omega \in \Lambda\). 
\end{problem}

\begin{solution}
  First we prove \(\wp(z)=\wp(z+\omega)\). If \(\omega=0\) or \(z \in \Lambda\) then it's obvious. Now we assume \(\omega \neq 0 \wedge z \notin \Lambda\). 
  We write \(\Lambda\setminus \{0\}=\{z_n:n \in \mathbb{N}^+\}\) as above, then \(|z_n|=O(\sqrt{n})\). 
  we first prove \(\sum_{n \in \mathbb{N}^+} \left(\frac{1}{(z-z_n)^2}-\frac{1}{z_n^2}\right)\) converge absolutely for \(z \notin \Lambda\). 
  Since \(\frac{1}{(z-z_n)^2}-\frac{1}{z_n^2}=\frac{2z z_n-z^2}{(z-z_n)^2z_n^2}\) and \(|z_n|=O(\sqrt{n})\), 
  we get \(\left|\frac{1}{(z-z_n)^2}-\frac{1}{z_n^2}\right|=O(\frac{1}{n^{ \frac{3}{2}} })\). 
  So we get the series is absolutely converge. 
  So we can change the order to sum. 
  Write \(f(z,t)=\frac{1}{(z-t)^2}\) and \(g(t)=\begin{cases}
  \frac{1}{t^2} & t \neq 0\\
  0 & t = 0 
  \end{cases}\) for \(z \in \mathbb{C}\) and \(t \in \Lambda\). Then \(\wp(z)=\sum_{t \in \Lambda}f(z,t)-g(t)\). 
  For \(t_1,t_2 \in \Lambda\), we define \(t_1 \sim t_2 \iff \frac{t_1-t_2}{\omega} \in \mathbb{Z}\). 
  Then \(\sim\) is a equivalence relation of \(\Lambda\). 
  We choose a class of reperentation element, write \(T\). 
  Then \(\wp(z)=\sum_{t \in T}\sum_{n \in \mathbb{Z}}f(z,t+n \omega)-g(t+n \omega)\). 
  And \(\wp(z+\omega)=\sum_{t \in T}\sum_{n \in \mathbb{Z}}f(z+\omega,t+n \omega)-g(t+n \omega)\). 
  So to prove \(\wp(z)=\wp(z+\omega)\), we only need to prove \(\forall t \in T,\sum_{n \in \mathbb{Z}}f(z,t+n \omega)-g(t+n \omega)=\sum_{n \in \mathbb{Z}}f(z+\omega,t+n \omega)-g(t+n \omega)\). 
  We can easily get \(|f(z,t+n \omega)|=O(frac{1}{n^2}),|g(t+n \omega)|=O(frac{1}{n^2})\), so we get \(\sum_{n in \mathbb{Z}}f(z,t+ n \omega)\) and \(\sum_{n \in \mathbb{Z}}g(t+n \omega)\) converge absolutely. 
  So we get \(\sum_{n \in \mathbb{Z}}f(z,t+n \omega)-g(t+n \omega)=\sum_{n \in \mathbb{Z}}f(z,t+n \omega)-\sum_{n \in \mathbb{Z}}g(t+n \omega)\). 
  For the same reason we get \(\sum_{n \in \mathbb{Z}} f(z+\omega,t+n \omega)-g(t+n \omega)=\sum_{n \in \mathbb{Z}}f(z+\omega,t+n \omega)-\sum_{n \in \mathbb{Z}}g(t+n \omega)\). 
  So we only need to prvoe \(\sum_{n \in \mathbb{Z}}f(z,t+n \omega)=\sum_{n \in \mathbb{Z}}f(z+\omega,t+n \omega)\). 
  Since \(f(z+\omega,t+n \omega)=f(z,t+(n-1)\omega)\) and \(n \mapsto n-1\) is bijection on \(\mathbb{Z}\), we get 
  \(\sum_{n \in \mathbb{Z}}f(z+\omega,t+n \omega)=\sum_{n \in \mathbb{Z}}f(z,t+(n-1)\omega)=\sum_{n \in \mathbb{Z}}f(z,t+n \omega)\).
  So finally we get \(\wp(z)=\wp(z+\omega),\forall \omega \in \Lambda\). 

  Now we prove the series converge locally uniformly in \(\mathbb{C}\setminus \Lambda\). 
  Let \(M \subset\mathbb{C}\setminus \Lambda\) is cpt. Now we prove the series converge uniformly in \(M\). 
  Only need to prove \(\sum_{n \in \mathbb{N}^+} \frac{2z z_n-z^2}{(z-z_n)^2 z_n^2}\) converge uniformly. 
  Since \(|z|\) is bounded and \(|z_n|=O(\sqrt{n})\), we get \(\exists s \in \mathbb{R}^+,| 2z z_n-z^2| \leq s \sqrt{n}\). 
  Since \(M\) is cpt we get \(d(M,\Lambda)>0\), so \(|z-z_n|\) has positive inf. 
  So as above we get \(\exists t \in \mathbb{R}^+,|z-z_n| \geq t \sqrt{n}\). 
  Without loss of generality we can assume \(t\) is little enough such that \(|z_n|\geq t \sqrt{n}\). 
  So finally we get \(\left|\frac{2z z_n - z^2}{(z-z_n)^2 z_n^2}\right| \leq \frac{s \sqrt{n}}{t^4 n^2}=\frac{s}{t^4} n^{-\frac{3}{2}}\). 
  Since \(\sum_{n \in \mathbb{N}^+} n^{-\frac{3}{2}}<\infty\), we get the series converge uniformly in \(M\). 
  
\end{solution}
\end{document}
