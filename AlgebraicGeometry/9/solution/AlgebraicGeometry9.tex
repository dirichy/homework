%!Mode:: "TeX:UTF-8"
%!TEX encoding = UTF-8 Unicode
%!TEX TS-program = xelatex
\documentclass{ctexart}
\newif\ifpreface
\prefacetrue
\usepackage{fontspec}
\usepackage{bbm}
\usepackage{tikz}
\usepackage{amsmath,amssymb,amsthm,color,mathrsfs}
\usepackage{fixdif}
\usepackage{hyperref}
\usepackage{cleveref}
\usepackage{enumitem}%
\usepackage{expl3}
\usepackage{lipsum}
\usepackage[margin=0pt]{geometry}
\usepackage{listings}
\definecolor{mGreen}{rgb}{0,0.6,0}
\definecolor{mGray}{rgb}{0.5,0.5,0.5}
\definecolor{mPurple}{rgb}{0.58,0,0.82}
\definecolor{backgroundColour}{rgb}{0.95,0.95,0.92}

\lstdefinestyle{CStyle}{
  backgroundcolor=\color{backgroundColour},
  commentstyle=\color{mGreen},
  keywordstyle=\color{magenta},
  numberstyle=\tiny\color{mGray},
  stringstyle=\color{mPurple},
  basicstyle=\footnotesize,
  breakatwhitespace=false,
  breaklines=true,
  captionpos=b,
  keepspaces=true,
  numbers=left,
  numbersep=5pt,
  showspaces=false,
  showstringspaces=false,
  showtabs=false,
  tabsize=2,
  language=C
}
\usetikzlibrary{calc}
\theoremstyle{remark}
\newtheorem{lemma}{Lemma}
\usepackage{fontawesome5}
\usepackage{xcolor}
\newcounter{problem}
\newcommand{\Problem}{\begin{tikzpicture}[baseline]%
    \node at (-0.02em,0.3em) {$\mathbb{P}$};%
    \node[scale=0.7] at (0.2em,-0.0em) {R};%
    \node[scale=0.7] at (0.6em,0.4em) {O};%
    \node[scale=0.8] at (1.05em,0.25em) {B};%
    \node at (1.55em,0.3em) {L};%
    \node[scale=0.7] at (1.75em,0.45em) {E};%
    \node at (2.35em,0.3em) {M};%
  \end{tikzpicture}%
}
\renewcommand{\theproblem}{\Roman{problem}}
\newenvironment{problem}{\refstepcounter{problem}\noindent\color{blue}\Problem\theproblem}{}

\crefname{problem}{\protect\Problem}{Problem}
\newcommand\Solution{\begin{tikzpicture}[baseline]%
    \node at (-0.04em,0.3em) {$\mathbb{S}$};%
    \node[scale=0.7] at (0.35em,0.4em) {O};%
    \node at (0.7em,0.3em) {\textit{L}};%
    \node[scale=0.7] at (0.95em,0.4em) {U};%
    \node[scale=1.1] at (1.19em,0.32em){T};%
    \node[scale=0.85] at (1.4em,0.24em){I};%
    \node at (1.9em,0.32em){$\mathcal{O}$};%
    \node[scale=0.75] at (2.3em,0.21em){\texttt{N}};%
  \end{tikzpicture}}
\newenvironment{solution}{\begin{proof}[\Solution]}{\end{proof}}
\title{\input{../../.subject}\input{../.number}}
\makeatletter
\newcommand\email[1]{\def\@email{#1}\def\@refemail{mailto:#1}}
\newcommand\schoolid[1]{\def\@schoolid{#1}}
\ifpreface
  \def\@maketitle{
  \raggedright
  {\Huge \bfseries \sffamily \@title }\\[1cm]
  {\Huge  \bfseries \sffamily\heiti\@author}\\[1cm]
  {\Huge \@schoolid}\\[1cm]
  {\Huge\href\@refemail\@email}\\[0.5cm]
  \Huge\@date\\[1cm]}
\else
  \def\@maketitle{
    \raggedright
    \begin{center}
      {\Huge \bfseries \sffamily \@title }\\[4ex]
      {\Large  \@author}\\[4ex]
      {\large \@schoolid}\\[4ex]
      {\href\@refemail\@email}\\[4ex]
      \@date\\[8ex]
    \end{center}}
\fi
\makeatother
\ifpreface
  \usepackage[placement=bottom,scale=1,opacity=1]{background}
\fi

\author{白永乐}
\schoolid{202011150087}
\email{202011150087@mail.bnu.edu.cn}

\def\to{\rightarrow}
\newcommand{\xor}{\vee}
\newcommand{\bor}{\bigvee}
\newcommand{\band}{\bigwedge}
\newcommand{\xand}{\wedge}
\newcommand{\minus}{\mathbin{\backslash}}
\newcommand{\mi}[1]{\mathscr{P}(#1)}
\newcommand{\card}{\mathrm{card}}
\newcommand{\oto}{\leftrightarrow}
\newcommand{\hin}{\hat{\in}}
\newcommand{\gl}{\mathrm{GL}}
\newcommand{\im}{\mathrm{Im}}
\newcommand{\re }{\mathrm{Re }}
\newcommand{\rank}{\mathrm{rank}}
\newcommand{\tra}{\mathop{\mathrm{tr}}}
\renewcommand{\char}{\mathop{\mathrm{char}}}
\DeclareMathOperator{\ot}{ordertype}
\DeclareMathOperator{\dom}{dom}
\DeclareMathOperator{\ran}{ran}

\begin{document}
\large
\setlength{\baselineskip}{1.2em}
\ifpreface
	\backgroundsetup{contents={%
    \begin{tikzpicture}
      \fill [white] (current page.north west) rectangle ($(current page.north east)!.3!(current page.south east)$) coordinate (a);
      \fill [bgc] (current page.south west) rectangle (a);
\end{tikzpicture}}}
\definecolor{word}{rgb}{1,1,0}
\definecolor{bgc}{rgb}{1,0.95,0}
\setlength{\parindent}{0pt}
\thispagestyle{empty}
\begin{tikzpicture}%
  % \node[xscale=2,yscale=4] at (0cm,0cm) {\sffamily\bfseries \color{word} under};%
  \node[xscale=4.5,yscale=10] at (10cm,1cm) {\sffamily\bfseries \color{word} Graduate Homework};%
  \node[xscale=4.5,yscale=10] at (8cm,-2.5cm) {\sffamily\bfseries \color{word} In Mathematics};%
\end{tikzpicture}
\ \vspace{1cm}\\
\begin{minipage}{0.25\textwidth}
  \textcolor{bgc}{王胤雅是傻逼}
\end{minipage}
\begin{minipage}{0.75\textwidth}
  \maketitle
\end{minipage}
\vspace{4cm}\ \\
\begin{minipage}{0.2\textwidth}
  \
\end{minipage}
\begin{minipage}{0.8\textwidth}
  {\Huge
    \textinconsolatanf{}
  }General fire extinguisher
\end{minipage}
\newpage\backgroundsetup{contents={}}\setlength{\parindent}{2em}

	\newgeometry{left=2cm,right=2cm,top=2cm,bottom=2cm}
\else
	\newgeometry{left=2cm,right=2cm,top=2cm,bottom=2cm}
	\maketitle
\fi
%from_here_to_type
\begin{problem}
Assume \(V\) is irreducible algebraic set in \(\mathbbm{A}_k^{n}\).
Assume \(\theta:\mathbbm{A}_k^{n} \to \mathbb{P}_k^{n}\) is the imbedding.
Prove that \(\overline{\theta(V)} \subset \mathbb{P}_k^{n}\) is irreducible.
\end{problem}

\begin{solution}
  If not, assume \(\overline{\theta(V)}= W_1 \cup W_2\), where \(W_1,W_2 \subsetneq \theta(V)\) are algebraic set. 
  First we prove \(\overline{\theta(V)} \cap U_0=\theta(V)\), where \(U_0=\{[x_0,\cdots,x_n]:x_0 \neq 0\}\). 
  Obviously \(\theta(V)\subset \overline{\theta(V)} \cap U_0\), so we only need to prove \(\overline{\theta(V)} \cap U_0 \subset \theta(V)\). 
  Consider \(\Phi:k[x_0,\cdots,x_n] \to k[x_1,\cdots,x_n],f(x_0,\cdots,x_n)\mapsto f(1,x_1,\cdots,x_n)\). 
  Then we have for any algebraic set \(W\) in \(\mathbb{P}_k^n\), \(\Phi(\ide(W))=\ide(\theta^{-1}(W\cap U_0))\). 
  Let \(W=\overline{\theta(V)}\), we get \(\ide(\theta^{-1}(W \cap U_0))=\Phi(\ide(W))=\ide(V)\). 
  So \(\theta^{-1}(W \cap U_0) \subset V\). 

  Now consider \(W_1 \cap U_0,W_2 \cap U_0\). Since \(\overline{W_1 \cap U_0}\subset W_1 \subsetneq V=\overline{\theta(V)}\), 
  we get \(W_1 \cap U_0,W_2 \cap U_0 \subsetneq \theta(V)\), so \(V=\theta^{-1}(W_1 \cap U_0) \cup \theta^{-1}(W_2 \cap U_0)\) is reducible, contradiction! 
  So we get \(\overline{(\theta(V))}\) is irreducible. 
\end{solution}
\begin{problem}
  Assume \(k\) is a algebraic closed field, \(f \in k[x_1,\cdots,x_n]\) is irreducible, prove that \(\van(f)\subset \mathbbm{A}_k^{n}\) is irreducible algebraic set. 
\end{problem}
\begin{solution}
  Only need to prove \((f)\) is prime ideal. i.e., \(f \mid gh \to f \mid g \vee f \mid h\). 
  Since \(k[x_1,\cdots,x_n]\) is Unique factorization domain, we get \(f\) is prime element. 
  So it's obvious. 
\end{solution}

\begin{problem}
  Assume \(V\) is irreducible algebraic set in \(\mathbbm{A}_k^{n}\), and \(U \subset V\) is nonempty open set in \(V\). 
  Assume \(f,g \in k(V)\) and \(\forall p \in U,f(p)=g(p)\). Prove that \(f=g\) in \(k(V)\). 
\end{problem}

\begin{solution}
  Assume \(f=\frac{f_1}{f_2},g=\frac{g_1}{g_2}\), where \(f_1,f_2,g_1,g_2 \in k[x_1,\cdots,x_n]\).
  And without loss of generality we assume \(U \subset \dom(f_2),\dom(g_2)\), or we use \(\dom(f_2) \cap \dom(g_2)\) replace \(U\). 
  To prove \(f=g\) in \(k(V)\), we only need to prove \(f_1g_2-f_2g_1=0\). 
  Consider \(h=f_1 g_2 - f_2 g_1 \in k[x_1,\cdots,x_n]\). We have \(U \subset \van(h)\). 
  So \((V\setminus U) \cup \van(h)=V\). Since \(V\) is irreducible, we get \(V\setminus U=V \vee \van(h)=V\). 
  Since \(U \neq \varnothing\), we get \(\van(h)=V\), so \(h=0\) in \(k(V)\).
\end{solution}
\end{document}
