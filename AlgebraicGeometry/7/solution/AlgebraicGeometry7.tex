%!Mode:: "TeX:UTF-8"
%!TEX encoding = UTF-8 Unicode
%!TEX TS-program = xelatex
\documentclass{ctexart}
\newif\ifpreface
\prefacetrue
\usepackage{fontspec}
\usepackage{bbm}
\usepackage{tikz}
\usepackage{amsmath,amssymb,amsthm,color,mathrsfs}
\usepackage{fixdif}
\usepackage{hyperref}
\usepackage{cleveref}
\usepackage{enumitem}%
\usepackage{expl3}
\usepackage{lipsum}
\usepackage[margin=0pt]{geometry}
\usepackage{listings}
\definecolor{mGreen}{rgb}{0,0.6,0}
\definecolor{mGray}{rgb}{0.5,0.5,0.5}
\definecolor{mPurple}{rgb}{0.58,0,0.82}
\definecolor{backgroundColour}{rgb}{0.95,0.95,0.92}

\lstdefinestyle{CStyle}{
  backgroundcolor=\color{backgroundColour},
  commentstyle=\color{mGreen},
  keywordstyle=\color{magenta},
  numberstyle=\tiny\color{mGray},
  stringstyle=\color{mPurple},
  basicstyle=\footnotesize,
  breakatwhitespace=false,
  breaklines=true,
  captionpos=b,
  keepspaces=true,
  numbers=left,
  numbersep=5pt,
  showspaces=false,
  showstringspaces=false,
  showtabs=false,
  tabsize=2,
  language=C
}
\usetikzlibrary{calc}
\theoremstyle{remark}
\newtheorem{lemma}{Lemma}
\usepackage{fontawesome5}
\usepackage{xcolor}
\newcounter{problem}
\newcommand{\Problem}{\begin{tikzpicture}[baseline]%
    \node at (-0.02em,0.3em) {$\mathbb{P}$};%
    \node[scale=0.7] at (0.2em,-0.0em) {R};%
    \node[scale=0.7] at (0.6em,0.4em) {O};%
    \node[scale=0.8] at (1.05em,0.25em) {B};%
    \node at (1.55em,0.3em) {L};%
    \node[scale=0.7] at (1.75em,0.45em) {E};%
    \node at (2.35em,0.3em) {M};%
  \end{tikzpicture}%
}
\renewcommand{\theproblem}{\Roman{problem}}
\newenvironment{problem}{\refstepcounter{problem}\noindent\color{blue}\Problem\theproblem}{}

\crefname{problem}{\protect\Problem}{Problem}
\newcommand\Solution{\begin{tikzpicture}[baseline]%
    \node at (-0.04em,0.3em) {$\mathbb{S}$};%
    \node[scale=0.7] at (0.35em,0.4em) {O};%
    \node at (0.7em,0.3em) {\textit{L}};%
    \node[scale=0.7] at (0.95em,0.4em) {U};%
    \node[scale=1.1] at (1.19em,0.32em){T};%
    \node[scale=0.85] at (1.4em,0.24em){I};%
    \node at (1.9em,0.32em){$\mathcal{O}$};%
    \node[scale=0.75] at (2.3em,0.21em){\texttt{N}};%
  \end{tikzpicture}}
\newenvironment{solution}{\begin{proof}[\Solution]}{\end{proof}}
\title{\input{../../.subject}\input{../.number}}
\makeatletter
\newcommand\email[1]{\def\@email{#1}\def\@refemail{mailto:#1}}
\newcommand\schoolid[1]{\def\@schoolid{#1}}
\ifpreface
  \def\@maketitle{
  \raggedright
  {\Huge \bfseries \sffamily \@title }\\[1cm]
  {\Huge  \bfseries \sffamily\heiti\@author}\\[1cm]
  {\Huge \@schoolid}\\[1cm]
  {\Huge\href\@refemail\@email}\\[0.5cm]
  \Huge\@date\\[1cm]}
\else
  \def\@maketitle{
    \raggedright
    \begin{center}
      {\Huge \bfseries \sffamily \@title }\\[4ex]
      {\Large  \@author}\\[4ex]
      {\large \@schoolid}\\[4ex]
      {\href\@refemail\@email}\\[4ex]
      \@date\\[8ex]
    \end{center}}
\fi
\makeatother
\ifpreface
  \usepackage[placement=bottom,scale=1,opacity=1]{background}
\fi

\author{白永乐}
\schoolid{202011150087}
\email{202011150087@mail.bnu.edu.cn}

\def\to{\rightarrow}
\newcommand{\xor}{\vee}
\newcommand{\bor}{\bigvee}
\newcommand{\band}{\bigwedge}
\newcommand{\xand}{\wedge}
\newcommand{\minus}{\mathbin{\backslash}}
\newcommand{\mi}[1]{\mathscr{P}(#1)}
\newcommand{\card}{\mathrm{card}}
\newcommand{\oto}{\leftrightarrow}
\newcommand{\hin}{\hat{\in}}
\newcommand{\gl}{\mathrm{GL}}
\newcommand{\im}{\mathrm{Im}}
\newcommand{\re }{\mathrm{Re }}
\newcommand{\rank}{\mathrm{rank}}
\newcommand{\tra}{\mathop{\mathrm{tr}}}
\renewcommand{\char}{\mathop{\mathrm{char}}}
\DeclareMathOperator{\ot}{ordertype}
\DeclareMathOperator{\dom}{dom}
\DeclareMathOperator{\ran}{ran}

\crefname{enumi}{}{}
\begin{document}
\large
\setlength{\baselineskip}{1.2em}
\ifpreface
    \backgroundsetup{contents={%
    \begin{tikzpicture}
      \fill [white] (current page.north west) rectangle ($(current page.north east)!.3!(current page.south east)$) coordinate (a);
      \fill [bgc] (current page.south west) rectangle (a);
\end{tikzpicture}}}
\definecolor{word}{rgb}{1,1,0}
\definecolor{bgc}{rgb}{1,0.95,0}
\setlength{\parindent}{0pt}
\thispagestyle{empty}
\begin{tikzpicture}%
  % \node[xscale=2,yscale=4] at (0cm,0cm) {\sffamily\bfseries \color{word} under};%
  \node[xscale=4.5,yscale=10] at (10cm,1cm) {\sffamily\bfseries \color{word} Graduate Homework};%
  \node[xscale=4.5,yscale=10] at (8cm,-2.5cm) {\sffamily\bfseries \color{word} In Mathematics};%
\end{tikzpicture}
\ \vspace{1cm}\\
\begin{minipage}{0.25\textwidth}
  \textcolor{bgc}{王胤雅是傻逼}
\end{minipage}
\begin{minipage}{0.75\textwidth}
  \maketitle
\end{minipage}
\vspace{4cm}\ \\
\begin{minipage}{0.2\textwidth}
  \
\end{minipage}
\begin{minipage}{0.8\textwidth}
  {\Huge
    \textinconsolatanf{}
  }General fire extinguisher
\end{minipage}
\newpage\backgroundsetup{contents={}}\setlength{\parindent}{2em}

\newgeometry{left=2cm,right=2cm,top=2cm,bottom=2cm}
\else
\newgeometry{left=2cm,right=2cm,top=2cm,bottom=2cm}
\maketitle
\fi
%from_here_to_type
\begin{problem}
  Assume \(k\) is an infinite field. \(f \in k[x_0,\cdots ,x_n]\). 
  knd \(\forall t \in t,f(t x_0,\cdots ,t x_n)=t^d f(x_0,\cdots ,x_n)\), where \(d \in \mathbb{N}\) is a constant. 
  Prove \(f\) is homogeneous. 
\end{problem}

\begin{solution}
  Consider \(g(t,x_0,\cdots ,x_n) = f(t x_0,\cdots t x_n)-t^d f(x_0,\cdots x_n) \in k[t,x_0,\cdots ,x_n]\). 
  We get for \(p \in \mathbbm{A}_k^n,g(p)=0\). Since \(k\) is infinite, we get \(g=0\). 
  Assume \(f(x_0,\cdots ,x_n)=\sum_{i}a_i x_1^{i_1} x_2^{i_2}\cdots x_n^{i_n} \). 
Then we get \(g(t,x_0,\cdots x_n)=\sum_{i}(t^d-t^{|i|})x_1^{i_1} x_2^{i_2}\cdots x_n^{i_n}=0\). 
Where \(|i|=\sum_{j=1}^{n} i_j\). So we get \(\sum_{j=1}^{n}i_j=d \). 
So \(f\) is homogeneous with degree \(d\). 
\end{solution}

\begin{problem}
  For an ideal \(I\) of \(k[x_0,\cdots ,x_n]\), prove following to definition of homogeneous ideal is equivalent. 
  \begin{enumerate}[ref=\theproblem.\arabic*]
    \item \label{it:21}\(\forall f \in I,f=\sum_{t}f_t \), where \(f_t\) are homogeneous poly with different degree. 
      then \(f_t \in I, \forall t\). 
    \item \label{it:22}\(\exists g_1,\cdots g_n\) are homogeneous such that \(I=(g_1,\cdots ,g_n)\). 
  \end{enumerate}
\end{problem}

\begin{solution}
  \begin{enumerate}[ref=\theproblem.\arabic*]
   \item \(\Cref{it:21} \implies \Cref{it:22}\): 
     Assume \(I=(f_1,f_2,\cdots ,f_n)\) and \(f_k=\sum_{t=1}^{a_k}g_{kt}  \), where \(g_{kt} \) are homogeneous. 
     Then \(I=(g_{kt} :k=1,2,\cdots ,n,t =1,2,\cdots ,a_k)\). 
   \item \(\Cref{it:22} \implies \Cref{it:21}\): 
     Assume \(I=(g_1,\cdots ,g_n)\), where \(g_k\) are homogeneous. Now consider \(f \in I\). 
     Assume \(f=\sum_{j=1}^{n}h_j g_j \). Assume \( h_j=\sum_{i=1}^{b_j}l_{ij}  \), where \(l_{ij} \) are homogeneous. 
     Then \(f=\sum_{j=1}^{n}\sum_{i=1}^{b_j}l_{ij} g_j  \). 
     Assume \(l_{ij} g_j\) has degree \(d_{ij} \), then we have \(f=\sum_{d} (\sum_{i,j:d_{ij} =d}l_{ij} g_j) \) is the homogeneous deconposition of \(f\). 
     Easily \(\sum_{i,j:d_{ij} =d}l_{ij} g_j \in I \) since \(g_j \in I\). 
 \end{enumerate} 
\end{solution}

\begin{problem}
 \begin{enumerate}[ref=\theproblem.\arabic*]
   \item Assume \(J_{\lambda}\) are homogeneous ideal. Prove that \(\van(\sum_{\lambda}J_{\lambda} )=\bigcap_{\lambda} \van (J_{\lambda})\). 
   \item Assume \(J_1,J_2\) are homogeneous ideal, then \(\van (J_1 \cap J_2)=\van (J_1) \cup \van (J_2)\). 
 \end{enumerate} 
\end{problem}

\begin{solution}
  \begin{enumerate}
    \item On one hand, assume \(p \in \van ( J_{\lambda}), \forall \lambda\), to prove \(p \in \van (\sum_{\lambda}J_{\lambda} )\). 
      Consider \( f \in \sum_{\lambda}J_{\lambda},f=\sum_{t=1}^{n} f_t \), where \(f_t \in J_{\lambda_{t}} \) for some \(\lambda_{t}\). 
      Then \(f(p)=\sum_{t=1}^{n}f_t(p)=0 \). So we get \( p \in \van ( \sum_{\lambda}J_{\lambda} )\). 

      On the other hand, assume \(p \in \van(\sum_{\lambda} J_{\lambda})\), to prove \(p \in \van(J_{\lambda}), \forall \lambda\). 
      Since \(J_{\lambda} \subset \sum_{\lambda}J_{\lambda} \), we get \(\van(\sum_{\lambda} J_{\lambda})\subset \van ( J_{\lambda})\). 
      So \(p \in \van (J_{\lambda})\). 
    \item On one hand, assume \(p \in \van(J_1 \cap J_2)\), to prove \(p \in \van(J_1) \vee p \in \van(J_2)\). 
      If not, assuem \(f_1 \in J_1,f_2 \in J_2,f_1(p),f_2(p) \neq 0\), then we get \(f_1 f_2)(p) \neq 0\). 
      But \(f_1 f_2 \in J_1 \cap J_2\), contradiction! 

      On the other hand, assume \(p \in \van(J_1) \cup \van(J_2)\), to prove \(p \in \van(J_1 \cap J_2)\). 
      Without loss of generality we assume \(p \in \van (J_1)\), then since \(J_1 \cap J_2 \subset J_1\) we get \(p \in \van(J_1 \cap J_2)\). 
  \end{enumerate}
\end{solution}
\begin{problem}
  Assume \(C=\{(x,y) \in \mathbb{C}^2:y^2=x(x-1)(x-2)\}\) and \(\hat{C}=\{[z,x,y]:y^2z=x(x-z)(x-2z)\} \in \mathbb{P}_{\mathbb{C}}^2 \). 
  Prove \(\hat{C}\) is one point compactification of \(C\). 
\end{problem}

\begin{solution}
  Write \(C=\{[z,x,y]:y^2z=x(x-z)(x-2z),z \neq 0\} \subset \mathbb{P}_{\mathbb{C}}^2\). 
  Then \(\hat{C}\setminus C=\{[z,x,y]:y^2z=x(x-z)(x-2z),z=0\}=\{[0,x,y]:x^3=0\}=\{[0,0,y]\}=\{[0,0,1]\}\). 
  So \(\hat{C}\) is one point compactification of \(C\). 
\end{solution}

\begin{problem}
  Assume \(X \subset \mathbb{P}_k^n\) is an algebraic set, then \(\van(\ide(X))=X\). 
\end{problem}

\begin{solution}
  Assume \(X=\van(J)\) for some homogeneous ideal \(J\). 
  Easily we get \(X \subset \van(\ide(X))\) and \(J \subset \ide(\van(J))\). 
  So we get \(\van(J) \supset \van(\ide(\van(J)))=\van(\ide(X))\). 
  So \(X=\van(\ide(X))\). 
\end{solution}

\begin{problem}
  Assume \(J\) is a homogeneous ideal, then \(\sqrt{J}\) is homogeneous ideal, too. 
\end{problem}

\begin{solution}
  Consider \(g \in \sqrt{J}\) and \(g=\sum_{t=1}^{n} g_t\) and \(g_t\) are homogeneous poly with different degree. 
  Without loss of generality assume \(d_1<d_2<\cdots<d_n\), where \(d_t\) is the degree of \(g_t\). 
  Now we need to prove \(g_t \in \sqrt{J}\). If not, assume \(j\) is the least such that \(g_j \notin \sqrt{J}\). 
  Let \(g'=g-\sum_{t=1}^{j-1} g_t\), then \(g' \in \sqrt{J}\) because \(g_j \in \sqrt{J}\) for \(t<j\). 
  Assume \(g'^{m}\in J\), then \((\sum_{t=j}^{n} g_t)^{m} \in J\). 
  We consider the homogeneous deconposition of \(g'^{m}\) with degree \(m d_j\).
  Since \(d_j<d_{j+1}<\cdots<d_n\), we get the part is \(g_j^{m}\). 
  Since \(J\) is homogeneous, we get \(g_j^{m} \in J\), i.e., \(g_j \in \sqrt{J}\), contradiction! 
  So finally we get \(\forall t, g_t \in \sqrt{J}\)
\end{solution}

\begin{problem}
  Assume \(J\) is a homogeneous ideal, then \(\sqrt{J}\subset \ide(\van(J))\). 
\end{problem}

\begin{solution}
  Assume \(f \in \sqrt{J},f^{n} \in J\). From \Cref{pro:1} we know \(\sqrt{J}\) is homogeneous, 
  so we only need to prove \(\forall p \in \van(J),f(p)=0\). 
  Since \(f^n \in J\) we get \(f^n(p) =0\). So \(f(p)=0\). 
  So we obtain \(\sqrt{J}\subset \ide(\van(J))\). 
\end{solution}
\end{document}
