% arara: xelatex
%!Mode:: "TeX:UTF-8"
%!TEX encoding = UTF-8 Unicode
%!TEX TS-program = xelatex
%!language = zh
\documentclass{ctexart}
\newif\ifpreface
\prefacetrue
\usepackage{fontspec}
\usepackage{bbm}
\usepackage{tikz}
\usepackage{amsmath,amssymb,amsthm,color,mathrsfs}
\usepackage{fixdif}
\usepackage{hyperref}
\usepackage{cleveref}
\usepackage{enumitem}%
\usepackage{expl3}
\usepackage{lipsum}
\usepackage[margin=0pt]{geometry}
\usepackage{listings}
\definecolor{mGreen}{rgb}{0,0.6,0}
\definecolor{mGray}{rgb}{0.5,0.5,0.5}
\definecolor{mPurple}{rgb}{0.58,0,0.82}
\definecolor{backgroundColour}{rgb}{0.95,0.95,0.92}

\lstdefinestyle{CStyle}{
  backgroundcolor=\color{backgroundColour},
  commentstyle=\color{mGreen},
  keywordstyle=\color{magenta},
  numberstyle=\tiny\color{mGray},
  stringstyle=\color{mPurple},
  basicstyle=\footnotesize,
  breakatwhitespace=false,
  breaklines=true,
  captionpos=b,
  keepspaces=true,
  numbers=left,
  numbersep=5pt,
  showspaces=false,
  showstringspaces=false,
  showtabs=false,
  tabsize=2,
  language=C
}
\usetikzlibrary{calc}
\theoremstyle{remark}
\newtheorem{lemma}{Lemma}
\usepackage{fontawesome5}
\usepackage{xcolor}
\newcounter{problem}
\newcommand{\Problem}{\begin{tikzpicture}[baseline]%
    \node at (-0.02em,0.3em) {$\mathbb{P}$};%
    \node[scale=0.7] at (0.2em,-0.0em) {R};%
    \node[scale=0.7] at (0.6em,0.4em) {O};%
    \node[scale=0.8] at (1.05em,0.25em) {B};%
    \node at (1.55em,0.3em) {L};%
    \node[scale=0.7] at (1.75em,0.45em) {E};%
    \node at (2.35em,0.3em) {M};%
  \end{tikzpicture}%
}
\renewcommand{\theproblem}{\Roman{problem}}
\newenvironment{problem}{\refstepcounter{problem}\noindent\color{blue}\Problem\theproblem}{}

\crefname{problem}{\protect\Problem}{Problem}
\newcommand\Solution{\begin{tikzpicture}[baseline]%
    \node at (-0.04em,0.3em) {$\mathbb{S}$};%
    \node[scale=0.7] at (0.35em,0.4em) {O};%
    \node at (0.7em,0.3em) {\textit{L}};%
    \node[scale=0.7] at (0.95em,0.4em) {U};%
    \node[scale=1.1] at (1.19em,0.32em){T};%
    \node[scale=0.85] at (1.4em,0.24em){I};%
    \node at (1.9em,0.32em){$\mathcal{O}$};%
    \node[scale=0.75] at (2.3em,0.21em){\texttt{N}};%
  \end{tikzpicture}}
\newenvironment{solution}{\begin{proof}[\Solution]}{\end{proof}}
\title{\input{../../.subject}\input{../.number}}
\makeatletter
\newcommand\email[1]{\def\@email{#1}\def\@refemail{mailto:#1}}
\newcommand\schoolid[1]{\def\@schoolid{#1}}
\ifpreface
  \def\@maketitle{
  \raggedright
  {\Huge \bfseries \sffamily \@title }\\[1cm]
  {\Huge  \bfseries \sffamily\heiti\@author}\\[1cm]
  {\Huge \@schoolid}\\[1cm]
  {\Huge\href\@refemail\@email}\\[0.5cm]
  \Huge\@date\\[1cm]}
\else
  \def\@maketitle{
    \raggedright
    \begin{center}
      {\Huge \bfseries \sffamily \@title }\\[4ex]
      {\Large  \@author}\\[4ex]
      {\large \@schoolid}\\[4ex]
      {\href\@refemail\@email}\\[4ex]
      \@date\\[8ex]
    \end{center}}
\fi
\makeatother
\ifpreface
  \usepackage[placement=bottom,scale=1,opacity=1]{background}
\fi

\author{白永乐}
\schoolid{25110180002}
\email{ylbai25@m.fudan.edu.cn}

\def\to{\rightarrow}
\newcommand{\xor}{\vee}
\newcommand{\AND}{\wedge}
\newcommand{\OR}{\vee}
\newcommand{\bor}{\bigvee}
\newcommand{\band}{\bigwedge}
\newcommand{\xand}{\wedge}
\newcommand{\minus}{\mathbin{\backslash}}
\newcommand{\mi}[1]{\mathscr{P}(#1)}
\newcommand{\card}{\mathrm{card}}
\newcommand{\oto}{\leftrightarrow}
\newcommand{\hin}{\hat{\in}}
\newcommand{\gl}{\mathrm{GL}}
\newcommand{\im}{\mathrm{Im}}
\newcommand{\re }{\mathrm{Re }}
\newcommand{\rank}{\mathrm{rank}}
\newcommand{\tra}{\mathop{\mathrm{tr}}}
\renewcommand{\char}{\mathop{\mathrm{char}}}
\DeclareMathOperator{\ot}{ordertype}
\DeclareMathOperator{\dom}{dom}
\DeclareMathOperator{\ran}{ran}

\begin{document}
\large
\setlength{\baselineskip}{1.2em}
\ifpreface
\input{../../../global/preface}
\newgeometry{left=2cm,right=2cm,top=2cm,bottom=2cm}
\else
\newgeometry{left=2cm,right=2cm,top=2cm,bottom=2cm}
\maketitle
\fi
%from_here_to_type
\begin{problem}
  Assume \(V \subset \mathbbm{A}_k^{n}\) is irreducible, and \(p \in V\).
  Let \(m_p := \{f \in k[V] : f(p) = 0\}, M_p := \{f \in k[x_1,\cdots,x_n]: f(p) = 0\}\).
  Prove that \(m_p \cong M_p/\ide(V)\).
\end{problem}

\begin{solution}
  Consider \( \theta : M_p \to m_p ,\theta(f) := f + \ide(V)\).
  Since \( p \in V\) we get \(\theta\) is well-defined.
  And easily we get \(\theta\) is homomorphism. Now consider \(\ker \theta\).
  Obviously \(\ide(V) \subset \ker \theta\), now we prove \(\ker \theta \subset \ide(V)\).
  Assume \( f \in \ker\theta\), to prove \(f \in \ide(V)\).
  Since \(\theta(f)=\ide(V)\), we get \(f+\ide(V)=\ide(V)\), so \(f \in \ide(V)\).
  So \(\ker \theta = \ide(V)\). And easily \(\theta\) is surjective, so we get
  \(m_p=M_p/\ker \theta=M_p / \ide(V)\).
\end{solution}

\begin{problem}
  Prove that \(M_p/M_p^2+\ide(V) \cong m_p/m_p^2\).
\end{problem}
\begin{solution}
  Consider \(\theta : M_p \to m_p/m_p^2, f \mapsto f|_V+m_p^2\). Easily \(\theta\) is homomorphism and surjective.
  Now we prove \(\ker \theta = M_p^2+\ide(V)\).

  On one hand, assume \(f \in \ker \theta\), i.e., \(f|_V \in m_p^2\).
  Then \(\exists g_1,\cdots,g_n,h_1,\cdots,h_n \in m_p\) such that \(f|_V=\sum_{k=1}^{n} g_k h_k\).
  Assume \(g_k=G_k|_V,h_k=H_k|_V\). Then \(G_k,H_k \in M_p\).
  Consider \(f-\sum_{k=1}^{n} G_k H_k=:h \in k[x_1,\cdots,x_n]\), easily to know \(h(x)=0,\forall x \in V\).
  So \(h \in \ide(V)\), thus \(f \in M_p^2+\ide(V)\).

  On the other hand, assume \(f \in M_p^2+\ide(V)\), to prove \(\theta f =0\).
  Assume \(f=\sum_{k=1}^{n} G_k H_k+h\), where \(G_k,H_k \in M_p\) and \(h \in \ide(V)\).
  Then \(\theta f = \sum_{k=1}^{n} g_k h_k +m_p^2\), where \(g_k=G_k|_V,h_k=H_k|_V\).
  So we get \(\theta f = m_p^2=0\).

  Finally we get \(m_p/m_p^{2} = M_p/ \ker \theta = M_p / M_p^2+\ide(V)\).
\end{solution}
\begin{problem}\label{pro:1}
  Assume \(V \subset \mathbbm{A}_k^n\) is irreducible, and \(p \in V,f \in k[V], f(p)\neq 0\).
  Consider \(V_f:= \{x \in V: f(x)\neq 0 \}\). Let \(\theta: V_f \to \mathbbm{A}_k^{n+1},x \mapsto (x,\frac{1}{f(x)})\).
  Let \(U = \theta(V_f)\), prove that \(T_p V \cong T_{\theta(p)} U\).
\end{problem}
\begin{solution}
  Write \(k[\mathbbm{A}_k^n]=k[x_1,\cdots , x_n ,y]\). Assume \(V = \van(I)= \van(f_1,\cdots, f_m)\), where \(f_i \in k[x_1,\cdots,x_n ]\).
  Then \(U = \van(f_1,\cdots,f_n, yf-1)\). Now consider \(\tau: T_p(V) \to \mathbbm{A}_k^{n+1}, \tau(x):=(x,\frac{1}{f(p)}-\frac{f_p^{(1)}(x)}{f^2(p)})\).
  Now we prove \(\tau(T_p V)=T_{\theta(p)} U\).
  Only need to prove \((yf-1)_{\theta(p)}^{(1)}(\tau(x))=0\).
  i.e., \(\frac{f_p^{(1)}(x)}{f(p)}+f(p)(y-\frac{1}{f(p)})=0\), where \(y = \frac{1}{f(p)}-\frac{f_p^{(1)}}{f^2(p)}\).
  Substitute \(y\) into the equation, we get is't obvious.

  Obviously \(\tau\) is injective, so it's isomorphic.
\end{solution}
\end{document}


