%!Mode:: "TeX:UTF-8"
%!TEX encoding = UTF-8 Unicode
%!TEX TS-program = xelatex
\documentclass{ctexart}
\newif\ifpreface
%\prefacetrue
\usepackage{fontspec}
\usepackage{bbm}
\usepackage{tikz}
\usepackage{amsmath,amssymb,amsthm,color,mathrsfs}
\usepackage{fixdif}
\usepackage{hyperref}
\usepackage{cleveref}
\usepackage{enumitem}%
\usepackage{expl3}
\usepackage{lipsum}
\usepackage[margin=0pt]{geometry}
\usepackage{listings}
\definecolor{mGreen}{rgb}{0,0.6,0}
\definecolor{mGray}{rgb}{0.5,0.5,0.5}
\definecolor{mPurple}{rgb}{0.58,0,0.82}
\definecolor{backgroundColour}{rgb}{0.95,0.95,0.92}

\lstdefinestyle{CStyle}{
  backgroundcolor=\color{backgroundColour},
  commentstyle=\color{mGreen},
  keywordstyle=\color{magenta},
  numberstyle=\tiny\color{mGray},
  stringstyle=\color{mPurple},
  basicstyle=\footnotesize,
  breakatwhitespace=false,
  breaklines=true,
  captionpos=b,
  keepspaces=true,
  numbers=left,
  numbersep=5pt,
  showspaces=false,
  showstringspaces=false,
  showtabs=false,
  tabsize=2,
  language=C
}
\usetikzlibrary{calc}
\theoremstyle{remark}
\newtheorem{lemma}{Lemma}
\usepackage{fontawesome5}
\usepackage{xcolor}
\newcounter{problem}
\newcommand{\Problem}{\begin{tikzpicture}[baseline]%
    \node at (-0.02em,0.3em) {$\mathbb{P}$};%
    \node[scale=0.7] at (0.2em,-0.0em) {R};%
    \node[scale=0.7] at (0.6em,0.4em) {O};%
    \node[scale=0.8] at (1.05em,0.25em) {B};%
    \node at (1.55em,0.3em) {L};%
    \node[scale=0.7] at (1.75em,0.45em) {E};%
    \node at (2.35em,0.3em) {M};%
  \end{tikzpicture}%
}
\renewcommand{\theproblem}{\Roman{problem}}
\newenvironment{problem}{\refstepcounter{problem}\noindent\color{blue}\Problem\theproblem}{}

\crefname{problem}{\protect\Problem}{Problem}
\newcommand\Solution{\begin{tikzpicture}[baseline]%
    \node at (-0.04em,0.3em) {$\mathbb{S}$};%
    \node[scale=0.7] at (0.35em,0.4em) {O};%
    \node at (0.7em,0.3em) {\textit{L}};%
    \node[scale=0.7] at (0.95em,0.4em) {U};%
    \node[scale=1.1] at (1.19em,0.32em){T};%
    \node[scale=0.85] at (1.4em,0.24em){I};%
    \node at (1.9em,0.32em){$\mathcal{O}$};%
    \node[scale=0.75] at (2.3em,0.21em){\texttt{N}};%
  \end{tikzpicture}}
\newenvironment{solution}{\begin{proof}[\Solution]}{\end{proof}}
\title{\input{../../.subject}\input{../.number}}
\makeatletter
\newcommand\email[1]{\def\@email{#1}\def\@refemail{mailto:#1}}
\newcommand\schoolid[1]{\def\@schoolid{#1}}
\ifpreface
  \def\@maketitle{
  \raggedright
  {\Huge \bfseries \sffamily \@title }\\[1cm]
  {\Huge  \bfseries \sffamily\heiti\@author}\\[1cm]
  {\Huge \@schoolid}\\[1cm]
  {\Huge\href\@refemail\@email}\\[0.5cm]
  \Huge\@date\\[1cm]}
\else
  \def\@maketitle{
    \raggedright
    \begin{center}
      {\Huge \bfseries \sffamily \@title }\\[4ex]
      {\Large  \@author}\\[4ex]
      {\large \@schoolid}\\[4ex]
      {\href\@refemail\@email}\\[4ex]
      \@date\\[8ex]
    \end{center}}
\fi
\makeatother
\ifpreface
  \usepackage[placement=bottom,scale=1,opacity=1]{background}
\fi

\author{白永乐}
\schoolid{202011150087}
\email{202011150087@mail.bnu.edu.cn}

\def\to{\rightarrow}
\newcommand{\xor}{\vee}
\newcommand{\bor}{\bigvee}
\newcommand{\band}{\bigwedge}
\newcommand{\xand}{\wedge}
\newcommand{\minus}{\mathbin{\backslash}}
\newcommand{\mi}[1]{\mathscr{P}(#1)}
\newcommand{\card}{\mathrm{card}}
\newcommand{\oto}{\leftrightarrow}
\newcommand{\hin}{\hat{\in}}
\newcommand{\gl}{\mathrm{GL}}
\newcommand{\im}{\mathrm{Im}}
\newcommand{\re }{\mathrm{Re }}
\newcommand{\rank}{\mathrm{rank}}
\newcommand{\tra}{\mathop{\mathrm{tr}}}
\renewcommand{\char}{\mathop{\mathrm{char}}}
\DeclareMathOperator{\ot}{ordertype}
\DeclareMathOperator{\dom}{dom}
\DeclareMathOperator{\ran}{ran}

\begin{document}
\large
\setlength{\baselineskip}{1.2em}
\ifpreface
    \backgroundsetup{contents={%
    \begin{tikzpicture}
      \fill [white] (current page.north west) rectangle ($(current page.north east)!.3!(current page.south east)$) coordinate (a);
      \fill [bgc] (current page.south west) rectangle (a);
\end{tikzpicture}}}
\definecolor{word}{rgb}{1,1,0}
\definecolor{bgc}{rgb}{1,0.95,0}
\setlength{\parindent}{0pt}
\thispagestyle{empty}
\begin{tikzpicture}%
  % \node[xscale=2,yscale=4] at (0cm,0cm) {\sffamily\bfseries \color{word} under};%
  \node[xscale=4.5,yscale=10] at (10cm,1cm) {\sffamily\bfseries \color{word} Graduate Homework};%
  \node[xscale=4.5,yscale=10] at (8cm,-2.5cm) {\sffamily\bfseries \color{word} In Mathematics};%
\end{tikzpicture}
\ \vspace{1cm}\\
\begin{minipage}{0.25\textwidth}
  \textcolor{bgc}{王胤雅是傻逼}
\end{minipage}
\begin{minipage}{0.75\textwidth}
  \maketitle
\end{minipage}
\vspace{4cm}\ \\
\begin{minipage}{0.2\textwidth}
  \
\end{minipage}
\begin{minipage}{0.8\textwidth}
  {\Huge
    \textinconsolatanf{}
  }General fire extinguisher
\end{minipage}
\newpage\backgroundsetup{contents={}}\setlength{\parindent}{2em}

\newgeometry{left=2cm,right=2cm,top=2cm,bottom=2cm}
\else
\newgeometry{left=2cm,right=2cm,top=2cm,bottom=2cm}
\maketitle
\fi
%from_here_to_type
\newcommand{\A}{\mathbbm{A}}
\begin{problem}
 $P$ is an ideal of a unitary commutative ring $A$, then $P$ is prime ideal of $A\iff A/P$ is integral domain.
\end{problem}
\begin{solution}
 $\Rightarrow $:

 Since $A$ is a unitary commutative ring, so $A/P$ is unitary commutative ring, too. So we only need to prove $[ab]=[0]\Rightarrow [a]=[0]\xor [b]=[0]$. 
 Obviously $[ab]=0\iff ab\in P\iff a\in P\xor b\in P\iff [a]=[0]\xor [b]=[0]$. 

 $\Leftarrow $:

 As the same, $ab\in P\iff [ab]=[0]\Rightarrow [a]=[0]\xor [b]=[0]\iff a\in P\xor b\in P$, so $P$ is prime ideal. 
\end{solution}

\begin{problem}
 $M$ is an ideal of a unitary commutative ring $A$, then $M$ is maximal ideal of $A \iff A/M$ is a field.   
\end{problem}
\begin{solution}
 $\Rightarrow $:

 Consider $[a]\in A/M\minus [0]$, we will prove it has a reverse.  Consider $N:=\{xm+ya:x,y\in A,m\in M\}$ is the minimum ideal of $A$ contains $M$ and $a$. 
 Since $[a]\neq [0]$ we know $a\notin M$, so $M\subsetneqq N$. 
 Noting $M$ is maximal, so $N=A$. That means $\exists x,y\in A,m\in M,xm+ya=1$. So $[xm+ya]=[1]$. Since $[xm]=[0]$ we get $[y][a]=1$, i.e., $[y]=[a]^{-1}$. 

 $\Leftarrow $:

 Consider $a\in A\minus M, N:=\{xp+ya:x,y\in A,p\in P\}$, we will prove $N=A$, which means $M$ is maximal. 
 Since $A/M$ is field, $\exists y\in A, [y]=[a]^{-1}$. That's means $ay-1\in M\subset N$. Noting $ay\in N$, so $1\in N$, thus $N=A$. 
\end{solution}

\begin{problem}
 A ring $A$ is noetherian, $I\subset A$ is an ideal of $A$, then $A/I$ is noetherian. 
\end{problem}
\begin{solution}
 Consider an ideal $J\subset A/I$, let $M:=\{x\in A:[x]\in J\}$. Then $\forall a\in A,x\in M,[ax]=[a][x]\in J$, so $ax\in M$. $\forall a,b\in M,[a-b]=[a]-[b]\in J$, so $a-b\in M$. So $M$ is an ideal of $A$. 
 Since $A$ is noetherian, we can assume $M=(f_i,i=1,2,\cdots n)$. Now we will prove $J=([f_i],i=1,2,\cdots n)$. 
 Consider $[f]\in J$, from definition of $M$ we know $f\in M$, so $f=\sum_{i=1}^na_if_i,a_i\in A$, thus $[f]=\left[\sum_{i=1}^na_if_i\right]=\sum_{i=1}^n[a_i][f_i]$. So $J=([f_i],i=1,2,\cdots n)$. 
\end{solution}

\begin{problem}
 \label{pro:4}
 $K$ is a field, $A=K[x_1,x_2,\cdots x_n],\A^n_K=K^n$. For ideal $I$ of $A$ let $V(I):=\{p\in\A_K^n:f(p)=0,\forall f\in I\}$. Then $V(I_1)\cup V(I_2)=V(I_1\cap I_2)=V(I_1I_2)$
\end{problem}
\begin{solution}
 \begin{lemma}
     \label{lem:1}
     $I\subset J\Rightarrow V(I)\supset V(J)$.
 \end{lemma}
 \begin{proof}
     Consider $p\in V(J)$, we get $\forall f\in J,f(p)=0$. Since $I\subset J$, so $\forall f\in I,f(p)=0$, i.e., $f\in V(I)$. 
 \end{proof}
 From Lemma \Cref{lem:1} we know $V(I_1\cap I_2)\subset V(I_1I_2)$, so we only need to prove $V(I_1)\cup V(I_2)\subset V(I_1\cap I_2),V(I_1I_2)\subset V(I_1)\cup V(I_2)$. 
 \begin{itemize}
     \item $V(I_1)\cup V(I_2)\subset V(I_1\cap I_2)$:
     From \Cref{lem:1} It's obvious. 
     \item $V(I_1I_2)\subset V(I_1)\cup V(I_2)$:
     Consider $p\in V(I_1I_2)$. 
     If $p\notin V(I_1)\cup V(I_2)$, then $\exists f_1\in I_1,f_2\in I_2, f_1(p)\neq 0,f_2(p)\neq 0$. Now consider $f=f_1f_2\in I_1I_2$, we get $f(p)=f_1(p)f_2(p)\neq 0$, so $p\notin V(I_1I_2)$, it's a contradiction. 
 \end{itemize}
\end{solution}

\begin{problem}
 $K$ is an infinite field, then $\A_k^n$ is not Hausdorff. 
\end{problem}
\begin{solution}
 \begin{lemma}
     \label{lem:2}
     $K$ is a infinite field, $f\in K[x_1,x_2,\cdots x_n]\setminus \{0\}$, then exists $p\in \A_K^n,f(p)\neq 0$. 
 \end{lemma}
 \begin{proof}
     Use MI. When $n=0,K[x_1,x_2,\cdots x_n]=K$, so it's obvious. Assume for $n=k$ it's right, when goes to $k+1$:

     Consider $h\in K(x_1,x_2,\cdots x_k)[x_{k+1}],h(x_{k+1}):=f(x_1,x_2,\cdots x_k,x_{k+1})$ is a non-zero polynomial so it has finite root. So exists $a\in K,h(a)\neq 0$. So $g:=f(x_1,x_2,\cdots x_k,a)\in K[x_1,x_2,\cdots x_k]\neq 0$. 
     By induction hypothesis we get $\exists b\in \A_K^k,g(b)\neq 0$. Let $p:=(b,a)\in \A_K^{k+1}$, then $f(p)\neq 0$. 
 \end{proof}
 In fact, it's not only not Hausdorff, it's kind of ''absolutly not Hausdorff'' because every pair of point can't be seperated. 
 Consider two point $p\neq q,p,q\in \A_K^n$. 
 Assume $p=(p_1,p_2,\cdots p_n),q=(q_1,q_2,\cdots q_n),p_1\neq q_1$. Assume two open set $V(I_1)^c,V(I_2)^c$ can seperate $p,q$, then $V(I_1)\cup V(I_2)=\A_K^n$. 
 From \Cref{pro:4} we know $V(I_1I_2)=\A_K^n$. So $\forall f\in V(I_1I_2),\forall p\in \A_K^n,f(p)=0$. 
 Then from Lemma\Cref{lem:2} we can get $f=0$. So $I_1I_2=\{0\}$. 
 Since $p\notin V(I_1),q\notin V(I_2)$, we know $I_1,I_2\neq \{0\}$. So $\exists f_1\in I_1\neq 0,\exists f_2\in I_2\neq 0$, and thus $f=f_1f_2\in I_1I_2\neq 0$, contradiction! 
 So these is not a pair of points can be seperated by two disjoint open set. 
\end{solution}
\end{document}
