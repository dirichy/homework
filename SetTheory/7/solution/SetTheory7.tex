%!Mode:: "TeX:UTF-8"
%!TEX encoding = UTF-8 Unicode
%!TEX TS-program = xelatex
\documentclass[a5paper]{ctexart}
\newif\ifpreface
\prefacetrue
\usepackage{fontspec}
\usepackage{bbm}
\usepackage{tikz}
\usepackage{amsmath,amssymb,amsthm,color,mathrsfs}
\usepackage{fixdif}
\usepackage{hyperref}
\usepackage{cleveref}
\usepackage{enumitem}%
\usepackage{expl3}
\usepackage{lipsum}
\usepackage[margin=0pt]{geometry}
\usepackage{listings}
\definecolor{mGreen}{rgb}{0,0.6,0}
\definecolor{mGray}{rgb}{0.5,0.5,0.5}
\definecolor{mPurple}{rgb}{0.58,0,0.82}
\definecolor{backgroundColour}{rgb}{0.95,0.95,0.92}

\lstdefinestyle{CStyle}{
  backgroundcolor=\color{backgroundColour},
  commentstyle=\color{mGreen},
  keywordstyle=\color{magenta},
  numberstyle=\tiny\color{mGray},
  stringstyle=\color{mPurple},
  basicstyle=\footnotesize,
  breakatwhitespace=false,
  breaklines=true,
  captionpos=b,
  keepspaces=true,
  numbers=left,
  numbersep=5pt,
  showspaces=false,
  showstringspaces=false,
  showtabs=false,
  tabsize=2,
  language=C
}
\usetikzlibrary{calc}
\theoremstyle{remark}
\newtheorem{lemma}{Lemma}
\usepackage{fontawesome5}
\usepackage{xcolor}
\newcounter{problem}
\newcommand{\Problem}{\begin{tikzpicture}[baseline]%
    \node at (-0.02em,0.3em) {$\mathbb{P}$};%
    \node[scale=0.7] at (0.2em,-0.0em) {R};%
    \node[scale=0.7] at (0.6em,0.4em) {O};%
    \node[scale=0.8] at (1.05em,0.25em) {B};%
    \node at (1.55em,0.3em) {L};%
    \node[scale=0.7] at (1.75em,0.45em) {E};%
    \node at (2.35em,0.3em) {M};%
  \end{tikzpicture}%
}
\renewcommand{\theproblem}{\Roman{problem}}
\newenvironment{problem}{\refstepcounter{problem}\noindent\color{blue}\Problem\theproblem}{}

\crefname{problem}{\protect\Problem}{Problem}
\newcommand\Solution{\begin{tikzpicture}[baseline]%
    \node at (-0.04em,0.3em) {$\mathbb{S}$};%
    \node[scale=0.7] at (0.35em,0.4em) {O};%
    \node at (0.7em,0.3em) {\textit{L}};%
    \node[scale=0.7] at (0.95em,0.4em) {U};%
    \node[scale=1.1] at (1.19em,0.32em){T};%
    \node[scale=0.85] at (1.4em,0.24em){I};%
    \node at (1.9em,0.32em){$\mathcal{O}$};%
    \node[scale=0.75] at (2.3em,0.21em){\texttt{N}};%
  \end{tikzpicture}}
\newenvironment{solution}{\begin{proof}[\Solution]}{\end{proof}}
\title{\input{../../.subject}\input{../.number}}
\makeatletter
\newcommand\email[1]{\def\@email{#1}\def\@refemail{mailto:#1}}
\newcommand\schoolid[1]{\def\@schoolid{#1}}
\ifpreface
  \def\@maketitle{
  \raggedright
  {\Huge \bfseries \sffamily \@title }\\[1cm]
  {\Huge  \bfseries \sffamily\heiti\@author}\\[1cm]
  {\Huge \@schoolid}\\[1cm]
  {\Huge\href\@refemail\@email}\\[0.5cm]
  \Huge\@date\\[1cm]}
\else
  \def\@maketitle{
    \raggedright
    \begin{center}
      {\Huge \bfseries \sffamily \@title }\\[4ex]
      {\Large  \@author}\\[4ex]
      {\large \@schoolid}\\[4ex]
      {\href\@refemail\@email}\\[4ex]
      \@date\\[8ex]
    \end{center}}
\fi
\makeatother
\ifpreface
  \usepackage[placement=bottom,scale=1,opacity=1]{background}
\fi

\author{白永乐}
\schoolid{202011150087}
\email{202011150087@mail.bnu.edu.cn}

\def\to{\rightarrow}
\newcommand{\xor}{\vee}
\newcommand{\bor}{\bigvee}
\newcommand{\band}{\bigwedge}
\newcommand{\xand}{\wedge}
\newcommand{\minus}{\mathbin{\backslash}}
\newcommand{\mi}[1]{\mathscr{P}(#1)}
\newcommand{\card}{\mathrm{card}}
\newcommand{\oto}{\leftrightarrow}
\newcommand{\hin}{\hat{\in}}
\newcommand{\gl}{\mathrm{GL}}
\newcommand{\im}{\mathrm{Im}}
\newcommand{\re }{\mathrm{Re }}
\newcommand{\rank}{\mathrm{rank}}
\newcommand{\tra}{\mathop{\mathrm{tr}}}
\renewcommand{\char}{\mathop{\mathrm{char}}}
\DeclareMathOperator{\ot}{ordertype}
\DeclareMathOperator{\dom}{dom}
\DeclareMathOperator{\ran}{ran}

\begin{document}
\large
\setlength{\baselineskip}{1.2em}
\ifpreface
    \backgroundsetup{contents={%
    \begin{tikzpicture}
      \fill [white] (current page.north west) rectangle ($(current page.north east)!.3!(current page.south east)$) coordinate (a);
      \fill [bgc] (current page.south west) rectangle (a);
\end{tikzpicture}}}
\definecolor{word}{rgb}{1,1,0}
\definecolor{bgc}{rgb}{1,0.95,0}
\setlength{\parindent}{0pt}
\thispagestyle{empty}
\begin{tikzpicture}%
  % \node[xscale=2,yscale=4] at (0cm,0cm) {\sffamily\bfseries \color{word} under};%
  \node[xscale=4.5,yscale=10] at (10cm,1cm) {\sffamily\bfseries \color{word} Graduate Homework};%
  \node[xscale=4.5,yscale=10] at (8cm,-2.5cm) {\sffamily\bfseries \color{word} In Mathematics};%
\end{tikzpicture}
\ \vspace{1cm}\\
\begin{minipage}{0.25\textwidth}
  \textcolor{bgc}{王胤雅是傻逼}
\end{minipage}
\begin{minipage}{0.75\textwidth}
  \maketitle
\end{minipage}
\vspace{4cm}\ \\
\begin{minipage}{0.2\textwidth}
  \
\end{minipage}
\begin{minipage}{0.8\textwidth}
  {\Huge
    \textinconsolatanf{}
  }General fire extinguisher
\end{minipage}
\newpage\backgroundsetup{contents={}}\setlength{\parindent}{2em}

\else
\maketitle
\fi
\newgeometry{a5paper,left=0cm,right=0cm,top=0cm,bottom=0cm}
\crefname{enumi}{}{}
%from_here_to_type
\begin{problem}\label{pro:1}
  Prove that there are arbitrarily large singular cardinals
\end{problem}
\begin{solution}
  For cardinal \(\lambda\), we cansider \(\aleph_{\lambda+\omega}\). 
  Easily \(\cof( \aleph_{\lambda+\omega})=\cof( \lambda+\omega)=\cof( \omega)=\aleph_0<\aleph_{\lambda+\omega}\), and 
  \(\aleph_{\lambda+\omega}\geq \lambda+\omega \geq \lambda\). 
\end{solution}
\begin{problem}\label{pro:2}
  There are arbitrarily large singular cardinals \(\aleph_{\alpha}\) such that \(\aleph_{\alpha}=\alpha\). 
\end{problem}
\begin{solution}
  For cardinal \(\lambda\), we let \(x_0=\lambda,x_{n+1}=\aleph_{x_{n}}\). Now consider \(\kappa=\sup_{n \in \omega}x_n\). 
  Easily \(\kappa\) is limit ordinal, so \(\aleph_{\kappa}=\sup_{\alpha<\kappa}\aleph_{\alpha}=\sup_{n \in \omega}\aleph_{x_n}=\sup_{n \in \omega}\aleph_{x_{n+1}}=\kappa\). 
  And since \(\kappa=\bigcup_{n \in \omega}x_n\), we get \(\cof( \kappa)\leq \omega\). Easily \(\kappa \geq x_2 =\aleph_{\aleph_{\lambda}} \geq \aleph_{\aleph_0}>\omega\). 
  So we get \(\kappa>\cof( \kappa)\). So \(\kappa\) is singular.
\end{solution}
\begin{problem}\label{pro:3}
  \begin{enumerate}
    \item \(\cof( \aleph+\beta)=\cof( \beta)\).
    \item \(\cof( \aleph_{\alpha})=\cof( \alpha)\) for limit ordinal \(\alpha\). 
    \item \(\cof( \aleph_{\alpha+1})=\aleph_{\alpha+1}\).
  \end{enumerate}
\end{problem}
\begin{solution}
  \begin{enumerate}
    \item First we prove \(\cof( \alpha+\beta) \leq \cof( \beta)\). Consider \(\theta:\cof( \beta) \to \beta\) is unbound. 
      Then we let \(\tau:\cof( \beta) \to \alpha+\beta,x \mapsto \alpha+\theta( x)\). Easily we get \(\tau\) is unbound. 
      So we get \(\cof( \alpha+\beta)\leq \cof( \beta)\). 

      Second we prove \(\cof( \alpha+\beta) \geq \cof( \beta)\). Consider \(\theta:\cof( \alpha+\beta) \to \alpha+\beta\) is unbound. 
      Now we consider \(B:=\{ x \in \alpha+\beta:x \geq \alpha\}\) and \(A=\theta_{-1}[ B]\). 
      Easily we get \(B \cong \beta\), and \(\text{ordertype}( A)\leq \cof( \alpha+\beta)\). 
      And \(\res{\theta}{A}:A \to B\) is unbounded, so easily we get \(\cof( \beta)\leq \cof( \alpha+\beta)\). 

      Finally we get \(\cof( \alpha+\beta)=\cof( \beta)\). 
    \item First we prove \(\cof( \aleph_{\alpha})\leq \cof( \alpha)\). Assume \(\theta:\cof( \alpha) \to \alpha\) is unbound. 
      Consider \(\tau:\cof( \alpha) \to \aleph_\alpha,x \mapsto \aleph_{\theta( x)}\). 
      Since \(\alpha\) is limit ordinal, we get \(\aleph_\alpha=\sup_{\beta<\alpha}\aleph_\beta\). 
      So we get \(\tau\) is unbounded. So we get \(\cof( \aleph_\alpha)\leq \alpha\). 

      Second we prove \(\cof( \alpha) \leq \cof( \aleph_\alpha)\). Assume \(\theta:\cof( \aleph_\alpha) \to \aleph_\alpha\) is unbounded. 
      Let \(f:\ord \to \ord,f( x):=\min\{ y \in \ord:\aleph_y \geq x\}\). 
      Let \(\tau:\cof( \aleph_\alpha)\to \alpha,x \mapsto f( \theta( x))\). 
      Since \(\theta( x)<\aleph_\alpha = \sup_{\beta<\alpha}\aleph_\beta\), we get \(\exists \beta < \alpha,\theta( x)<\aleph_\beta\). 
      So we get \(f( \theta( x)) \leq \beta < \alpha\). So \(\tau\) is well-defined. 
      Easily to get \(\tau\) is unbounded. So we get \(\cof( \aleph_\alpha)\leq \cof( \alpha)\). 

      Finally we get \(\cof( \aleph_\alpha)=\cof( \alpha)\). 
  \end{enumerate}
\end{solution}
\begin{problem}\label{pro:4}
  Assume GCH, prove that for cardinal \(\lambda,\kappa>\omega\), we have:
    \[\kappa^\lambda=\begin{cases}
    \kappa & \lambda<\cof( \kappa)\\
    \kappa^+ & \cof( \kappa)\leq \lambda \leq \kappa\\
    \lambda^+ & \kappa<\lambda
    \end{cases}\]
\end{problem}
\begin{solution}
  Use MI to \(\kappa\). For \(\kappa=\omega\), when \(\lambda = \omega\) we get \(\kappa^\lambda=2^\omega=\omega^+\). 
  When \(\lambda>\omega\), we get \(\kappa^\lambda \geq 2^\lambda=\lambda^+\). And \(\kappa^\lambda \leq (2^\lambda)^\lambda=2^{\lambda \times \lambda}=2^\lambda=\lambda^+\). 
  Now assume for \(\alpha:\omega \leq \alpha <\kappa\) it's right, consider \(\kappa\). 
  \begin{itemize}
    \item \(\lambda<\cof( \kappa)\). 

      Since \(\lambda < \cof( \kappa)\), we get every \(f:\lambda \to \kappa\) is bounded. 
      So we get \(\fun{\lambda}{\kappa}=\bigcup_{\alpha<\kappa}\fun{\lambda}{\alpha}\). 
      So we get \(\kappa^\lambda \leq \sum_{\alpha<\kappa}\alpha^\lambda=\kappa \sup_{\alpha<\kappa}\alpha^\lambda\).
      Since \(\alpha<\kappa\), we get \(\alpha^\lambda=\begin{cases}
      \alpha & \lambda < \cof( \alpha)\\
      \alpha^+ & \cof( \alpha)\leq \lambda \leq \alpha\\
      \lambda^+ & \alpha<\lambda
      \end{cases}\). Anyway, since \(\alpha,\lambda<\kappa\), we get \(\alpha^\lambda \leq \kappa\). 
      So we get \(\kappa^\lambda \leq \kappa \sup_{\alpha<\kappa}\alpha^\lambda \leq \kappa \kappa = \kappa\). 
    \item \(\cof( \kappa)\leq \lambda \leq \kappa\). 

      Easily \(\kappa^\lambda \leq \kappa^\kappa \leq 2^{\kappa \kappa}=2^\kappa = \kappa^+\). Now we only need \(\kappa^+ \leq \kappa^\lambda\). 
      Only need to prove \(\kappa^{\cof( \kappa)} >\kappa\). 
      If not, assume \(f:\kappa \to \fun{\cof( \kappa)}{\kappa}\) is bijection. 
      Assume \(\theta:\cof( \kappa) \to \kappa\) is unbounded. Without loss of generality assume \(\theta\) is injective.  Let \(\tau:\kappa \to \cof( \kappa),x \mapsto \min\{ y \in \cof( \kappa):\theta( y)\geq x\}\). 
      Now consider \(A_\alpha :=\tau_{-1}[ \alpha]\) for \(\alpha < \cof( \kappa)\). 
      Easily we get \(\forall y \in \tau_{-1}[ \alpha],\alpha > \theta( y)\). Since \(\theta\) is injective, we get \(\card( A_\alpha) \leq \card( \alpha) <\kappa\). 
      Let \(B_\alpha:=\{f(x)( \alpha):x \in A_\alpha\}\), then easily \(\card( B_\alpha) \leq \card( A_\alpha)<\kappa\). 
      Now consider \(g:\cof( \kappa)\to \kappa,g( \alpha):=\min( \kappa \setminus B_\alpha)\). 
      Since \(f\) is bijection, we get \(\exists x \in \kappa,g=f( x)\). 
      But \(f( x)( \tau( x)) \in B_{\tau( x)}\), and \(g( \tau( x))=\min( \kappa \setminus B_{\tau( x)}) \notin B_{\tau( x)}\), contradiction! 
      So we get \(\kappa^\lambda \geq \kappa^{\cof( \kappa)} > \kappa\), then \(\kappa^\lambda \geq \kappa^+\). 
    \item \(\kappa<\lambda\). 

      We get \(\lambda^+=2^\lambda \leq \kappa^\lambda \leq 2^{\lambda \lambda}=2^\lambda =\lambda^+\). 
      So \(\kappa^\lambda = \lambda^+\). 
  \end{itemize}
\end{solution}
\begin{problem}\label{pro:5}
  Assume a linearly ordered set \(P\) has a countable dense subset, then \(\card P \leq 2^{\aleph_0}\). 
\end{problem}
\begin{solution}
  Assume \(A \subset P\) is a countable dense subset. Now consider \(f:P \to \mathcal{P}( A),x \mapsto \{ y \in A:y < x\}\). 
  Easily \(\card \mathcal{P}( A)=2^{\aleph_0}\), so we only need to prove \(f\) is injection. 
  Assume \(f( x)=f( y)\). Without loss of generality assume \(x \leq y\). If \(x \neq y\), then we get \(x < y\). 
  Since \(A\) is dense, we get \(\exists z,w \in A\) such that \(x \leq z < w \leq y\).
  So we get \(z \in f( y)\) but \(z \notin f( x)\), contradiction! 
  So we get \(x = y\). So \(f\) is injective, so \(\card P \leq 2^{\aleph_0}\). 
\end{solution}
\begin{problem}\label{pro:6}
  Find the cardinal of all null sets of reals.
\end{problem}
\begin{solution}
  Let \(\mathcal{A} \subset \mathcal{P}( \mathbb{R})\) is the set of all of null sets. 
  Then we get \(\card \mathcal{A} \leq \card \mathcal{P}( \mathbb{R})=2^{\mathfrak{c}}\). 
  Now we prove \(\card \mathcal{A} \geq 2^\mathfrak{c}\). 
  Consider \(C \subset \mathbb{R}\) is the Canter set. We have \(C\) is null and \(\card C=\mathfrak{c}\). 
  So we get \(\mathcal{P}( C)\subset \mathcal{A}\), then \(\card \mathcal{A} \geq \card \mathcal{P}( C)=2^\mathfrak{c}\). 
\end{solution}
\begin{problem}\label{pro:7}
  Prove that \(\fun{\mathbb{N}}{\mathbb{N}}\) is uncountable. 
\end{problem}
\begin{solution}
  Easily we have \(\card \fun{\mathbb{N}}{\mathbb{N}}=\aleph_0^{\aleph_0} \geq 2^{\aleph_0}>\aleph_0\) is uncountable.
\end{solution}

\end{document}
