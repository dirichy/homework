%!Mode:: "TeX:UTF-8"
%!TEX encoding = UTF-8 Unicode
%!TEX TS-program = xelatex
\documentclass{ctexart}
\newif\ifpreface
\prefacetrue
\usepackage{fontspec}
\usepackage{bbm}
\usepackage{tikz}
\usepackage{amsmath,amssymb,amsthm,color,mathrsfs}
\usepackage{fixdif}
\usepackage{hyperref}
\usepackage{cleveref}
\usepackage{enumitem}%
\usepackage{expl3}
\usepackage{lipsum}
\usepackage[margin=0pt]{geometry}
\usepackage{listings}
\definecolor{mGreen}{rgb}{0,0.6,0}
\definecolor{mGray}{rgb}{0.5,0.5,0.5}
\definecolor{mPurple}{rgb}{0.58,0,0.82}
\definecolor{backgroundColour}{rgb}{0.95,0.95,0.92}

\lstdefinestyle{CStyle}{
  backgroundcolor=\color{backgroundColour},
  commentstyle=\color{mGreen},
  keywordstyle=\color{magenta},
  numberstyle=\tiny\color{mGray},
  stringstyle=\color{mPurple},
  basicstyle=\footnotesize,
  breakatwhitespace=false,
  breaklines=true,
  captionpos=b,
  keepspaces=true,
  numbers=left,
  numbersep=5pt,
  showspaces=false,
  showstringspaces=false,
  showtabs=false,
  tabsize=2,
  language=C
}
\usetikzlibrary{calc}
\theoremstyle{remark}
\newtheorem{lemma}{Lemma}
\usepackage{fontawesome5}
\usepackage{xcolor}
\newcounter{problem}
\newcommand{\Problem}{\begin{tikzpicture}[baseline]%
    \node at (-0.02em,0.3em) {$\mathbb{P}$};%
    \node[scale=0.7] at (0.2em,-0.0em) {R};%
    \node[scale=0.7] at (0.6em,0.4em) {O};%
    \node[scale=0.8] at (1.05em,0.25em) {B};%
    \node at (1.55em,0.3em) {L};%
    \node[scale=0.7] at (1.75em,0.45em) {E};%
    \node at (2.35em,0.3em) {M};%
  \end{tikzpicture}%
}
\renewcommand{\theproblem}{\Roman{problem}}
\newenvironment{problem}{\refstepcounter{problem}\noindent\color{blue}\Problem\theproblem}{}

\crefname{problem}{\protect\Problem}{Problem}
\newcommand\Solution{\begin{tikzpicture}[baseline]%
    \node at (-0.04em,0.3em) {$\mathbb{S}$};%
    \node[scale=0.7] at (0.35em,0.4em) {O};%
    \node at (0.7em,0.3em) {\textit{L}};%
    \node[scale=0.7] at (0.95em,0.4em) {U};%
    \node[scale=1.1] at (1.19em,0.32em){T};%
    \node[scale=0.85] at (1.4em,0.24em){I};%
    \node at (1.9em,0.32em){$\mathcal{O}$};%
    \node[scale=0.75] at (2.3em,0.21em){\texttt{N}};%
  \end{tikzpicture}}
\newenvironment{solution}{\begin{proof}[\Solution]}{\end{proof}}
\title{\input{../../.subject}\input{../.number}}
\makeatletter
\newcommand\email[1]{\def\@email{#1}\def\@refemail{mailto:#1}}
\newcommand\schoolid[1]{\def\@schoolid{#1}}
\ifpreface
  \def\@maketitle{
  \raggedright
  {\Huge \bfseries \sffamily \@title }\\[1cm]
  {\Huge  \bfseries \sffamily\heiti\@author}\\[1cm]
  {\Huge \@schoolid}\\[1cm]
  {\Huge\href\@refemail\@email}\\[0.5cm]
  \Huge\@date\\[1cm]}
\else
  \def\@maketitle{
    \raggedright
    \begin{center}
      {\Huge \bfseries \sffamily \@title }\\[4ex]
      {\Large  \@author}\\[4ex]
      {\large \@schoolid}\\[4ex]
      {\href\@refemail\@email}\\[4ex]
      \@date\\[8ex]
    \end{center}}
\fi
\makeatother
\ifpreface
  \usepackage[placement=bottom,scale=1,opacity=1]{background}
\fi

\author{白永乐}
\schoolid{202011150087}
\email{202011150087@mail.bnu.edu.cn}

\def\to{\rightarrow}
\newcommand{\xor}{\vee}
\newcommand{\bor}{\bigvee}
\newcommand{\band}{\bigwedge}
\newcommand{\xand}{\wedge}
\newcommand{\minus}{\mathbin{\backslash}}
\newcommand{\mi}[1]{\mathscr{P}(#1)}
\newcommand{\card}{\mathrm{card}}
\newcommand{\oto}{\leftrightarrow}
\newcommand{\hin}{\hat{\in}}
\newcommand{\gl}{\mathrm{GL}}
\newcommand{\im}{\mathrm{Im}}
\newcommand{\re }{\mathrm{Re }}
\newcommand{\rank}{\mathrm{rank}}
\newcommand{\tra}{\mathop{\mathrm{tr}}}
\renewcommand{\char}{\mathop{\mathrm{char}}}
\DeclareMathOperator{\ot}{ordertype}
\DeclareMathOperator{\dom}{dom}
\DeclareMathOperator{\ran}{ran}

\begin{document}
\large
\setlength{\baselineskip}{1.2em}
\ifpreface
    \backgroundsetup{contents={%
    \begin{tikzpicture}
      \fill [white] (current page.north west) rectangle ($(current page.north east)!.3!(current page.south east)$) coordinate (a);
      \fill [bgc] (current page.south west) rectangle (a);
\end{tikzpicture}}}
\definecolor{word}{rgb}{1,1,0}
\definecolor{bgc}{rgb}{1,0.95,0}
\setlength{\parindent}{0pt}
\thispagestyle{empty}
\begin{tikzpicture}%
  % \node[xscale=2,yscale=4] at (0cm,0cm) {\sffamily\bfseries \color{word} under};%
  \node[xscale=4.5,yscale=10] at (10cm,1cm) {\sffamily\bfseries \color{word} Graduate Homework};%
  \node[xscale=4.5,yscale=10] at (8cm,-2.5cm) {\sffamily\bfseries \color{word} In Mathematics};%
\end{tikzpicture}
\ \vspace{1cm}\\
\begin{minipage}{0.25\textwidth}
  \textcolor{bgc}{王胤雅是傻逼}
\end{minipage}
\begin{minipage}{0.75\textwidth}
  \maketitle
\end{minipage}
\vspace{4cm}\ \\
\begin{minipage}{0.2\textwidth}
  \
\end{minipage}
\begin{minipage}{0.8\textwidth}
  {\Huge
    \textinconsolatanf{}
  }General fire extinguisher
\end{minipage}
\newpage\backgroundsetup{contents={}}\setlength{\parindent}{2em}

\else
\maketitle
\fi
\newgeometry{left=2cm,right=2cm,top=2cm,bottom=2cm}
\crefname{enumi}{}{}
%from_here_to_type
\begin{problem}
  Assume \(A\) can be well-ordered, prove that \(\mathcal{P}(A)\) can be linear-orderd. 
\end{problem}
\begin{solution}
  Assume \((A,<)\) is a well-ordered set. 
  For \(X,Y \in \mathcal{P}(A),X \neq Y\), let \(X \prec Y \iff \min X \Delta Y \in X\). Now we prove \(\prec\) is linear-order. 
  
  First by defination we get \(X \not \prec X,\forall X \subset A\). 

  Second, for \(X,Y \in \mathcal{P},X \neq Y\), we have \(X \Delta Y \neq \varnothing\). 
  Since \(A\) is well-ordered, we get \(\min X \Delta Y\) exists. 
  And \(\min X \Delta Y \in X \Delta Y \in X \cup Y\). 
  So we get \(X \prec Y \vee Y \prec X\). 

  Finally, assume \(X \prec Y,Y \prec Z\), now we prove \(X \prec Z\). 
  Let \(x = \min X \Delta Y \in X,y=\min Y \Delta Z \in Y\). 
  Easily we get \(X \Delta Z = (X \Delta Y)\Delta (Y \Delta Z)\). 
  Assume \(t = \min X \Delta Z\). Only need to prove \(t \in X\). If not, we get \(t \in Z\). 
  If \(t \in X \Delta Y\), then \(t \geq x\). Since \(t \notin X \wedge x \in X\), we get \(t>x\). 
  So \(x \notin X \Delta Z\), so \(x \in Z\). Noting \(x \notin Y\), we get \(x \in Y \Delta Z\), so \(x >y\). 
  So \(y \notin X \Delta Y\), so \(y \in X\). Since \(y \notin Z\), we get \(y \in X \Delta Z\). 
  So \(t<y\). So \(t<y<x<t\), contradiction! 
  Else we get \(t \in Y \Delta Z\). Since \(t \in Z\) we get \(t \notin Y\). Then \(t>y\). 
  So \(y \notin X \Delta Z\). Since \(y \notin Z\), we get \(y \notin X\). 
  So \(y \in X \Delta Y\), thus \(y>x\). So \(t>x\), thus \(x \notin X \Delta Z\). 
  So \(x \in Z\), thus \(x \in Y \Delta Z\). Then \(x > y\), contradiction! 

  So we get \(\prec\) is a linear-order on \(\mathcal{P}(A)\). 
\end{solution}

\begin{problem}
  Assume \(\{X_i:i \in I\}\) and \(\{Y_i:i \in I\}\) are two disjoint families such that \(X_i \approx Y_i\). 
  Prove that \(\bigcup_{i \in I}X_i \approx \bigcup_{ i \in I} Y_i\)
\end{problem}

\begin{solution}
  Since \(X_i \approx Y_i\), we get \(\text{bij}(X_i,Y_i)\neq \varnothing\). 
  Let \(\theta:I \to \bigcup_{i \in I} \text{bij}(X_i,Y_i)\) is a choice function. i.e., \(\theta(i)\in \text{bij}(X_i,Y_i)\). 
  Now consider \(\tau=\bigcup \ran(\theta) \). We will prove \(\tau\) is bijection from \(X:=\bigcup_{i \in I} X_i\) to \(\bigcup_{i \in I} Y_i\). 

  First we prove \(\tau\) is a map. i.e., \(\forall x \in X,\exists ! y \in Y,(x,y)\in \tau\). 
  Since \(X_i \cap X_j = \varnothing , \forall i \neq j\), we get \(x \in X \to \exists ! i \in I,x \in X_i\). 
  So \((x,\theta(i)(x)) \in \tau\). If \((x,z)\in \tau\), we get \(\exists j \in I,(x,z)\in \theta(j)\). 
  Since \(x \in\dom(\theta(j))=X_j\), we get \(j=i\). Since \(\theta(i)\) is a map, we get \(z = \theta(i)(x)\). 

  Second we prove \(\tau\) is injection. Assume \(x,t \in X,\tau(x)=\tau(t)\). Now we prove \(x=t\). 
  Since \(Y_i \cap Y_j=\varnothing,\forall i \neq j\), we get \(\exists ! i \in I,\tau(x) \in Y_i\). 
  Since \(\ran(\theta(j))=Y_j,\forall j \in I\), we get \((x,\tau(x)) \in \theta(i)\). So \(\theta(i)(x)=\tau(x)\). 
  For the same reason we get \(\theta(i)(t)=\tau(t)\). So \(\theta(i)(x)=\theta(i)(t)\). 
  Since \(\theta(i)\) is bijection, we get \(x=t\). 

  Finally we prove \(\tau\) is surjection. Assume \(y \in Y\), then \(\exists i \in I,y \in Y_i\). 
  So \(\exists x \in X_i,\theta(i)(x)=y\). So \(\tau(x)=y\). So \(\tau\) is surjective. 
\end{solution}

\begin{problem}
  Prove that \(\prod_{0 < n < \omega} n=2^{\aleph_0}\). 
\end{problem}

\begin{solution}
  Obviously \(\prod_{0 < n < \omega} n=\prod_{n < \omega} (n+1)=\left(\sup_{n<\omega}(n+1) \right)^{|\omega|}=\aleph_0^{\aleph_0}=2^{\aleph_0} \). 
\end{solution}

\begin{problem}
  Prove that \(\prod_{n<\omega} \aleph_n=\aleph_\omega^{\aleph_0}\). 
\end{problem}

\begin{solution}
  Obviously \(\aleph_n>0\), so we get \(\prod_{n<\omega} \aleph_n = \left(\sup_{n<\omega}\aleph_n\right)^{\omega}=\aleph_\omega^{\aleph_0}\). 
\end{solution}

\begin{problem}
  Prove that \(\prod_{n<\omega+\omega} \aleph_n=\aleph_{\omega+\omega}^{\aleph_0}\).
\end{problem}

\begin{solution}
  Let \(f:\omega \to \omega+\omega\) be a bijection. Then \(\prod_{n<\omega + \omega} \aleph_n=\prod_{n<\omega} \aleph_{f(n)}\). 
  So we get \(\prod_{n<\omega+\omega} = \left(\sup_{n<\omega}\aleph_{f(n)}\right)^{\aleph_0}=\aleph_{\omega+\omega}^{\aleph_0}\). 
\end{solution}

\begin{problem}
  For every ordinal \(\alpha\) less than \(\omega_1\), prove that \(\exists X:\omega \to \mathcal{P}(\alpha)\) such that \(\text{ordertype}(X(n))\leq \alpha^n\) and \(\alpha=\bigcup \ran X \). 
\end{problem}

\begin{solution}
  If not, assume \(\beta\) is the least ordinal less than \(\omega_1\) don't meet the requirement. 
  If \(\beta=\alpha+1\), Since \(\alpha<\beta\), we get \(\exists X \in \fun{\omega}{\mathcal{P}(\alpha)}\) meet the requirement. 
  Now we let \(Y:\omega \to \mathcal{P}(\beta)\) and \(Y(0)=\{\alpha\},Y(n+1)=X(n)\). 
  Then easily \(Y\) meet the requirement, contradiction! 
  Else, \(\beta\) is limit ordinal. Since \(\beta<\omega_1\) we get \(\text{cf}(\beta)\leq \omega\). 
  Since \(\beta\) is limit ordinal we get \(\text{cf}(\beta)=\omega\). 
  Consider \(\theta \text{cf}(\beta) \to \beta\) is unbounded, then \(\beta=\bigcup \ran \theta \). 
  For \(n \in \text{cf}(\beta)\), we have \(\theta(n)<\beta\), so by AC, \(\exists X:\text{cf}(\beta) \times \omega \to \mathcal{P}(\beta)\) such that
  \(\text{ordertype}(X(n,m)) \leq \theta(n)^m\) and \(\theta(n) = \bigcup_{m \in \omega} X(n,m) \). 
  Now let \(Y:\omega \to \mathcal{P}(\beta)\) and \(Y(2^n(2m+1)-1)=X(n,m)\). 
  Then easily \(\text{ordertype}(Y(k)) \leq \beta^k\) and \(\beta=\bigcup \ran \theta=\bigcup_{n,m \in \omega} X(n,m)=\bigcup_{k \in \omega} Y(k) \). 
  contradiction! 
  So such \(\beta\) doesn't exist. 
\end{solution}
\begin{problem}
  If \(\kappa\) is a cardinal and \(\lambda < \cof(\kappa)\), then \(\kappa^\lambda=\sum_{\alpha<\kappa} |\alpha|^\lambda\). 
\end{problem}

\begin{solution}
  When \(\lambda=0\) it's obvious, now we assume \(\lambda>0\). 
  Easily \(\kappa \geq \omega\) is a cardinal, so we get \(\sum_{\alpha<\kappa} |\alpha|^\lambda=\kappa \sup_{\alpha<\kappa} |\alpha|^\lambda \leq \kappa \cdot \kappa^\lambda=\kappa^\lambda\). 
  Now consider \(f \in \fun{\lambda}{\kappa}\), we get \(f\) is bounded. So \(\fun{\lambda}{\kappa}=\bigcup_{\alpha<\kappa} \fun{\lambda}{\alpha}\). 
  So we get \(\kappa^{\lambda}\leq \sum_{\alpha<\kappa} |\alpha|^\lambda\). 
  Finally we get \(\kappa^\lambda=\sum_{\alpha<\kappa} |\alpha|^\lambda\).  
\end{solution}

\begin{problem}
  Prove that \(\aleph_\omega^{\aleph_1}=2^{\aleph_1}\cdot \aleph_\omega^{\aleph_0}\). 
\end{problem}

\begin{solution}
  Since \(2^{\aleph_1},\aleph_\omega^{\aleph_0} \leq \aleph_\omega^{\aleph_1}\), we get \(\aleph_\omega^{\aleph_1}\geq 2^{\aleph_1}\cdot \aleph_\omega^{\aleph_0}\). 
  Since \(\aleph_\omega=\sup_{n<\omega}\aleph_n\), we get \(\aleph_\omega=\prod_{n<\omega} \aleph_n^{\aleph_1}\). 
  By Hausdoff formula we get \(\aleph_{n+1}^{\aleph_1}=\aleph_n^{\aleph_1} \cdot \aleph_{n+1}\). 
  By MI we can easily get \(\aleph_n^{\aleph_1}\leq \aleph_{\omega}\cdot \aleph_0^{\aleph_1}\). 
  So finally we get \(\aleph_\omega^{\aleph_1}\leq \prod_{n<\omega} \aleph_{\omega}\cdot \aleph_0^{\aleph_1}=\aleph_\omega^{\aleph_0}\cdot \aleph_0^{\aleph_1 \cdot \aleph_0}\). 
  Easily we get \(\aleph_0^{\aleph_1 \cdot \aleph_0}=\aleph_0^{\aleph_1}=2^{\aleph_1}\), 
  so finally we get \(\aleph_\omega^{\aleph_1}=2^{\aleph_1} \cdot \aleph_\omega^{\aleph_0}\). 
\end{solution}
\end{document}
