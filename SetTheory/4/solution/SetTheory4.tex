%!Mode:: "TeX:UTF-8"
%!TEX encoding = UTF-8 Unicode
%!TEX TS-program = xelatex
\documentclass{ctexart}
\newif\ifpreface
\prefacetrue
\usepackage{fontspec}
\usepackage{bbm}
\usepackage{tikz}
\usepackage{amsmath,amssymb,amsthm,color,mathrsfs}
\usepackage{fixdif}
\usepackage{hyperref}
\usepackage{cleveref}
\usepackage{enumitem}%
\usepackage{expl3}
\usepackage{lipsum}
\usepackage[margin=0pt]{geometry}
\usepackage{listings}
\definecolor{mGreen}{rgb}{0,0.6,0}
\definecolor{mGray}{rgb}{0.5,0.5,0.5}
\definecolor{mPurple}{rgb}{0.58,0,0.82}
\definecolor{backgroundColour}{rgb}{0.95,0.95,0.92}

\lstdefinestyle{CStyle}{
  backgroundcolor=\color{backgroundColour},
  commentstyle=\color{mGreen},
  keywordstyle=\color{magenta},
  numberstyle=\tiny\color{mGray},
  stringstyle=\color{mPurple},
  basicstyle=\footnotesize,
  breakatwhitespace=false,
  breaklines=true,
  captionpos=b,
  keepspaces=true,
  numbers=left,
  numbersep=5pt,
  showspaces=false,
  showstringspaces=false,
  showtabs=false,
  tabsize=2,
  language=C
}
\usetikzlibrary{calc}
\theoremstyle{remark}
\newtheorem{lemma}{Lemma}
\usepackage{fontawesome5}
\usepackage{xcolor}
\newcounter{problem}
\newcommand{\Problem}{\begin{tikzpicture}[baseline]%
    \node at (-0.02em,0.3em) {$\mathbb{P}$};%
    \node[scale=0.7] at (0.2em,-0.0em) {R};%
    \node[scale=0.7] at (0.6em,0.4em) {O};%
    \node[scale=0.8] at (1.05em,0.25em) {B};%
    \node at (1.55em,0.3em) {L};%
    \node[scale=0.7] at (1.75em,0.45em) {E};%
    \node at (2.35em,0.3em) {M};%
  \end{tikzpicture}%
}
\renewcommand{\theproblem}{\Roman{problem}}
\newenvironment{problem}{\refstepcounter{problem}\noindent\color{blue}\Problem\theproblem}{}

\crefname{problem}{\protect\Problem}{Problem}
\newcommand\Solution{\begin{tikzpicture}[baseline]%
    \node at (-0.04em,0.3em) {$\mathbb{S}$};%
    \node[scale=0.7] at (0.35em,0.4em) {O};%
    \node at (0.7em,0.3em) {\textit{L}};%
    \node[scale=0.7] at (0.95em,0.4em) {U};%
    \node[scale=1.1] at (1.19em,0.32em){T};%
    \node[scale=0.85] at (1.4em,0.24em){I};%
    \node at (1.9em,0.32em){$\mathcal{O}$};%
    \node[scale=0.75] at (2.3em,0.21em){\texttt{N}};%
  \end{tikzpicture}}
\newenvironment{solution}{\begin{proof}[\Solution]}{\end{proof}}
\title{\input{../../.subject}\input{../.number}}
\makeatletter
\newcommand\email[1]{\def\@email{#1}\def\@refemail{mailto:#1}}
\newcommand\schoolid[1]{\def\@schoolid{#1}}
\ifpreface
  \def\@maketitle{
  \raggedright
  {\Huge \bfseries \sffamily \@title }\\[1cm]
  {\Huge  \bfseries \sffamily\heiti\@author}\\[1cm]
  {\Huge \@schoolid}\\[1cm]
  {\Huge\href\@refemail\@email}\\[0.5cm]
  \Huge\@date\\[1cm]}
\else
  \def\@maketitle{
    \raggedright
    \begin{center}
      {\Huge \bfseries \sffamily \@title }\\[4ex]
      {\Large  \@author}\\[4ex]
      {\large \@schoolid}\\[4ex]
      {\href\@refemail\@email}\\[4ex]
      \@date\\[8ex]
    \end{center}}
\fi
\makeatother
\ifpreface
  \usepackage[placement=bottom,scale=1,opacity=1]{background}
\fi

\author{白永乐}
\schoolid{25110180002}
\email{ylbai25@m.fudan.edu.cn}

\def\to{\rightarrow}
\newcommand{\xor}{\vee}
\newcommand{\AND}{\wedge}
\newcommand{\OR}{\vee}
\newcommand{\bor}{\bigvee}
\newcommand{\band}{\bigwedge}
\newcommand{\xand}{\wedge}
\newcommand{\minus}{\mathbin{\backslash}}
\newcommand{\mi}[1]{\mathscr{P}(#1)}
\newcommand{\card}{\mathrm{card}}
\newcommand{\oto}{\leftrightarrow}
\newcommand{\hin}{\hat{\in}}
\newcommand{\gl}{\mathrm{GL}}
\newcommand{\im}{\mathrm{Im}}
\newcommand{\re }{\mathrm{Re }}
\newcommand{\rank}{\mathrm{rank}}
\newcommand{\tra}{\mathop{\mathrm{tr}}}
\renewcommand{\char}{\mathop{\mathrm{char}}}
\DeclareMathOperator{\ot}{ordertype}
\DeclareMathOperator{\dom}{dom}
\DeclareMathOperator{\ran}{ran}

\crefname{enumi}{}{}
\begin{document}
\large
\setlength{\baselineskip}{1.2em}
\ifpreface
	\input{../../../global/preface}
\else
	\maketitle
\fi
\newgeometry{left=2cm,right=2cm,top=2cm,bottom=2cm}
%from_here_to_type
\begin{problem}
Consider \(\mathbb{Q}=\mathbb{Z} \times (\mathbb{Z} \setminus \{0\})/\sim\), where \((a,b)\sim(c,d) \iff ad=bc\).j
Define \(+_\mathbb{Q},\cdot_\mathbb{Q}\) and \(<_\mathbb{Q}\) and verify that your definitions
doesn't depend on the choice of representatives.
\end{problem}
\begin{solution}
	Let \([(a,b)]+_\mathbb{Q}[(c,d)]=[(ad+bc,bd)],[(a,b)]\cdot_\mathbb{Q}[(c,d)]=[(ac,bd)]\),
	and \([(a,b)]<_\mathbb{Q} [(c,d)]\iff a b d^2<c d b^2\).
	Now we prove they are well-defined, i.e., doesn't depend on the choice of representatives.

	For \(+_\mathbb{Q}\), assume \((a,b)\sim (e,f)\), we need to prove
	\((ad+bc,bd)\sim (ed+fc,df)\).
	Since \(af=be\), we have \((ad+bc)bf=ad^2f+bdcf=bed^2+bdcf=(ed+fc)bd\).
	So \(+_\mathbb{Q}\) is well defined.

	For \(\cdot_\mathbb{Q}\), assume \((a,b)\sim (e,f)\), we need to prove
	\((ac,bd)\sim (ec,fd)\).
	Since \(af=be\), we have \(acfd=bced=bdec\).
	So \(\cdot_\mathbb{Q}\) is well defined.

	For \(<_\mathbb{Q}\), assume \((a_1,b_1)\sim (a_2,b_2),(c_1,d_1)\sim(c_2,d_2)\) and \((a_1,b_1)<(c_1,d_1)\).
	Now we need to prove \((a_2,b_2)<(c_2,d_2)\).
	Since \(a_1b_2=a_2b_1,c_1d_2=c_2d_1\) we get
	\(a_1 b_1 d_2^2<c_2 d_2 b_1^2\)
\end{solution}

\begin{problem}
The set of all continuous functions $f: \mathbb{R} \rightarrow \mathbb{R}$ has cardinality $\mathfrak{c}$ (while the set of all functions has cardinality $2^{\mathfrak{c}}$ ).
[A continuous function on $\mathbb{R}$ is determined by its values at rational points.]
\end{problem}

\begin{solution}
  Consider \(\theta:\fun{\mathbb{R}}{\mathbb{R}} \to 2^{\mathbb{Q}}, f \mapsto \{(a,b) \in \mathbb{Q}: f(a) < b\}\). 
  Now we prove \(f\) is a injection. 
  Assume \(\theta(f)=\theta(g)\), to prove \(f=g\). 
  First we prove for \(x\in \mathbb{Q}\) we have \(f(x)=g(x)\). 
  We have \(f(x)=\sup \{y \in \mathbb{Q}:y<f(x)\}=\sup\{y \in \mathbb{Q}:(x,y) \in \theta(f)\}=\sup\{y \in \mathbb{Q} : (x,y) \in \theta (g)\}=g(x)\).
  For \(x \in \mathbb{R}\), choose a sequence \(x_n \in \mathbb{Q}\) such that \(x_n \to x\), 
  then \(f(x)= \lim_{n \to \infty }f(x_n)=\lim_{n \to \infty }g(x_n)=g(x)\). 
  So we get \(f=g\). So \(\card \fun{\mathbb{R}}{\mathbb{R}} \leq \card 2^{\mathbb{Q}}=2^{\aleph_0}\).
  Obviously \(\card\fun{\mathbb{R}}{\mathbb{R}} \geq 2^{\aleph_0}\), so we get they are equal. 
\end{solution}
\begin{problem}
There are at least $\mathfrak{c}$ countable order-types of linearly ordered sets.
\end{problem}

\begin{solution}
For every sequence $a=\left\langle a_n: n \in \mathbb{N}\right\rangle$ of natural numbers consider the ordertype
$$
\tau_a=\{(x,y) \in \mathbb{Z}\times \mathbb{N}:2 \nmid y \wedge 0<x<a_{\frac{y}{2}}\}
$$
And for \((x,y),(z,w) \in \tau_a\) we define \((x,y)<(z,w) \iff y<w \wedge y=w,x<z\).
Now we will show that if $a \neq b$, then $\tau_a \neq \tau_b$. 
Assume \(\tau_a \cong \tau_b\), we need to prove \(a=b\). assume \(\theta:\tau_a \to \tau_b\) is the isomorfism. 

We know \((x,0)\) can be defined as \(\phi(p)=\exists_{k=1}^{x-1} t_k,\wedge_{1 \leq i < j \leq x-1}t_i \neq t_j, \forall k=1,\cdots x-1, t_k<p\). 
And \(\theta\) is isomorphism. So \(\theta(x,0)=(x,0)\). 
For \((x,1)\), we let \(b_0\) satisfy \(\theta(0,1)=(b_0,m)\). 
Since the set \(\{(x,y):y=1\}\) can be defined by \(\psi(p)=\forall r,s(r,s<p \wedge \tau(r)\wedge \tau(s) \to \card[r,s] < \infty )\), 
where \(\tau(r):= \{s:s<r\}\) and \([r,s]=\{y:r<y<s\}\). 
we get \(\theta[\{(x,y):y=1\}]=\{(x,y):y=1\}\). 
So we can delete the element whose second coordinary is \(0,1\), and \(\theta\) is isomorphism, too. 
Do this repeatedly, we get \(\theta(x,2n+1)=(x,2n+1)\). 
So \(a_n=\card\{(x,2n+1)\in \tau_a\}=\card \{(x,2n+1) \in \tau_b\}=b_n\) and thus \(a=b\).
\end{solution}
\begin{problem}\label{pro:2}
The set of all algebraic reals is countable.
\end{problem}

\begin{solution}
  Assume \(\{f_n:n \in \mathbb{N}\}\) is the set of all integral coefficient polynomial.
  Consider \(A_n:=\{x \in \mathbb{C}:f(x)=0\}\) is finite set. Then we get \(\cup_{n \in \mathbb{N}}A_n\) is at most countable. 
  Obviouly \(\bigcup_{n \in \mathbb{N} } A_n\) is infinite, so it's countable. 
\end{solution}
\begin{problem}\label{pro:1}
If $S$ is a countable set of reals, then $|\mathbb{R}-S|=\mathfrak{c}$.
[Use $\mathbb{R} \times \mathbb{R}$ rather than $\mathbb{R}$ (because $|\mathbb{R} \times \mathbb{R}|=2^{\aleph_0}$ ).]
\end{problem}

\begin{solution}
  Assume \(\theta:\mathbb{R} \to \mathbb{R} \times \mathbb{R}\) is bijection, and \(T=\theta(S)\). 
  Then \(T\) is countable. And \(\card (\mathbb{R}\setminus S)= \card (\mathbb{R} \times \mathbb{R} \setminus T)\). 
  So we only need to prove \(\mathbb{R}\times \mathbb{R} \approx \mathbb{R} \times \mathbb{R} \setminus T\). 
  Obviously \(\card \mathbb{R}\times \mathbb{R} \setminus T \leq \card \mathbb{R} \times \mathbb{R}\), so we only need
  \(\mathbb{R}\times \mathbb{R} \setminus T \geq \mathbb{R}\). 
  Since \(T\) is countable, we get \(\{x:\exists y,(x,y) \in T\}\) is countable. 
  Choose \(t \notin \{x:\exists y,(x,y) \in T\}\). 
  Let \(f:\mathbb{R} \to \mathbb{R} \times \mathbb{R} \setminus T,x \mapsto (t,x)\). 
  Easily we get \(f\) is injection. So \(\card \mathbb{R} \times \mathbb{R} \setminus T=\mathfrak{c}\). 
\end{solution}

\begin{problem}
  \begin{enumerate}
    \item Prove that the set of all irrational number in \(\mathbb{R}\) has cardinality \(\mathfrak{c}\).  
    \item Prove that the set of all transcendental numbers  in \(\mathbb{R}\) has cardinality \(\mathfrak{c}\). 
  \end{enumerate}
\end{problem}

\begin{solution}
  \begin{enumerate}
    \item Since \(\card \mathbb{Q}\) is countable, from \Cref{pro:1} we get \(\card(\mathbb{R} \setminus \mathbb{Q})=\mathfrak{c}\). 
    \item Since the set of all algebraic nbumber is countable(\Cref{pro:2}), for the same reason we get the set of transcendental number has cardinality \(\mathfrak{c}\). 
  \end{enumerate}
\end{solution}
\begin{problem}
  Assume \(T\) is a tree. 
  \begin{enumerate}[ref=\theproblem.\arabic*]
    \item \label{it:1}If \(s,t,u \in T\), then \(R_{stu} :=\{\delta_{st} ,\delta_{tu} ,\delta_{us} \}\) has at most \(2\) elements. 
      And if \(p,q \in R_{stu}\), then \(p \subset q \vee q \subset p\). 
    \item \label{it:2}\(\prec \) is a linear ordering of \(T\) which extends \(\sqsubset\). 
    \item \label{it:3}For every \(t \in T\), Prove \(T^t:=\{s \in T:t \sqsubset s\}\) is an interval in \((T,\prec)\). 
  \end{enumerate}
\end{problem}

\begin{solution}
  \begin{enumerate}
    \item First we prove for \(p,q \in R_{stu} \) we have \(p \subset q \vee q \subset p\). 
      Without loss of generality assume \(p = \delta_{st} ,q=\delta_{tu} \). We have 
      \(p,q \subset (\cdot, t)\). Since \((\cdot,t)\) is well ordered, and easily \(p,q\) are initial segment, 
      so \(p \subset q \vee q \subset p\). 

      Now we prove there are at most two elements. 
      From above we know \((R_{stu},\subset)\) is linear order set, and it's finite. 
      Without loss of generality we assume \(\delta_{st} \subset \delta_{tu} \subset \delta_{us} \). 
      Then we get \(\delta_{tu}=\delta_{tu}\cap \delta_{us}=(\cdot,t)\cap(\cdot,u)\cap(\cdot,s) \subset \delta_{st}\).
      That means \(\delta_{st}=\delta_{tu}\), so there is at most two elements. 
    \item Easily to prove \(\sqsubset \subset \prec \). Now we prove \(\prec\) is linear ordered. 
      Consider a bigger linear ordered set \(Y\) is obtained by adding a minimum, \(-\infty\), in \(X\). 
      Consider the tree \(U:=\fun{<\alpha}{Y}\). We try to make a map from \(T\) to \(B_U\). 
      Let \(\theta:T \to B_U,\theta(f)(\beta):=\begin{cases}
        f(\beta), \beta \in \dom f\\
        -\infty,\beta \notin \dom f
      \end{cases}\,\,\, \forall \beta \in \alpha,f \in T\). 
      Then we it's easily to prove \(\theta\) is injective and \(f \prec g \iff \theta(f)(\beta)< \theta(g)(\beta)\), where \(\beta = \min\{t \in \alpha:\theta(f)(t) \neq \theta(g)(t)\}\). 
      We define \(f,g \in B_U,f <g \iff f(\beta)<g(\beta)\), where \(\beta=\min\{t \in \alpha:f(t)\neq g(t)\}\). 
      Now we only need to prove \((B_U,<)\) is linear ordered. 
      Easily \(f \not< f , \forall f \in B_U\). And for \(f \neq g,f<g \vee g<f\). 
      Assume \(f<g<h\), to prove \(f<h\).

      If \(n_{fg}< n_{gh} \) then we get \(f(n_{fg})<g(n_{fg})=h(n_{fg})\). 
      So \(n_{fh}\leq n_{fg}\). From \Cref{it:1} we get \(n_{fh}=n_{fg} \vee n_{fh}=n_{gh}\). 
      So \(n_{fh}=n_{fg}\), and thus \(f<h\). 

      If \(n_{fg}>n_{gh}\), then we get \(h(n_{gh})>g(n_{gh})=f(n_{gh})\). Same as above we get \(n_{fh}=n_{gh}\), so \(f<h\). 

      If \(n_{fg}=n_{gh}\), it's obvious \(f<h\). 
      
      So we have proved \(B_U\) is linear ordered, and thus \((T,\prec)\) is linear orderd. 
    \item Only need to prove if \(t \sqsubset u,t \sqsubset v,u \prec v\), then \(\forall s:u \prec s \prec v,t \sqsubset s\). 
      If \(u \sqsubseteq s\) then \(t \sqsubset u \sqsubseteq s\). 
      Else we get \(u \not \sqsubseteq s\). So we get \(s \not \sqsubseteq u \wedge s(n_{su})>u(n_{su})\). 
      From \Cref{it:2} we get \(t \prec s\). So if \(t \not \sqsubset s \) then \(s \not \sqsubseteq t \wedge s(n_{st})>t(n_{st})\). 
      Since \(t \sqsubseteq v\) we get \(s(n_{st})>t(n_{st})=v(n_{st})\). 
      Since \(t \sqsubset v\) we get \(n_{st}=n_{sv}\), so \(v \prec s\), contradiction! 
      So \(t \sqsubset s\). 
  \end{enumerate}
\end{solution}

\begin{problem}
  \begin{enumerate}
    \item Prove that \(\prec\) is linear ordered on \(T \cup B_T\). 
    \item For every \(t \in T\), prove that \(B_t = \{f \in B_T:t \in f\}\cup\{f \in T:t \sqsubset f\}\) is interval in \((T \cup B_T,\prec)\). 
  \end{enumerate}
\end{problem}

\begin{solution}
  \begin{enumerate}
    \item consider a bigger tree \(U\). For \(f \in T\) let \(\theta(f)= f\), for \(f \in B_T\) we let \(\theta(f)\) is a map 
      from \(ordertype(\dom(f))\) to \(X\), and \(\theta(f)(\beta):=g(\beta)\), where \(g \in f \) and \(\beta \in \dom(g)\). 
      Let \(U=\theta(T \cup B_T)\). Then easily \(T \subset U\). Now we prove \(\theta\) is isomorphic from \((T \cup B_T,\prec)\) to \((U,\prec)\). 
      Easily for \(f,g \in T\) we have \(f \sqsubset g \iff \theta(f) \sqsubset \theta(g)\). 
 
      And for \(f \in T,g \in B_T\) we have \(f \in g \iff \theta(f)\sqsubset \theta(g)\). 
      So from the defination of \(\prec\) we get \(\theta\) is isomorphic. 
      Since we have proved \((U,\prec)\) is linear order(\Cref{it:2}), we get \((T \cup B_T,\prec)\) is linear order, too. 
    \item Since \(\theta(B_t)=U^{\theta(t)}\) is an interval(\Cref{it:3}), we get \(B_t\) is interval, too.
  \end{enumerate}
  
\end{solution}

\begin{problem}
   Prove that if \(X\) is Suslinnnn line, then \(X^2\) is not c.c.c.
\end{problem}

\begin{solution}
  Let \(\mathcal{A}:=\{Y \subset X:\card (X\setminus Y) \leq\omega\}\). 
  For every \(Y \in \mathcal{A}\), let \(\theta(Y):=\{(a,c) \subset X:(a,c) \cap Y =\varnothing \wedge a<c\}\). 
  Since \(X\) is not separable, we know \(Y\) is not dense in \(X\), so \(\theta(Y) \neq \varnothing\).
  So we can use AC to get a map \(\tau:\mathcal{A} \to X \times X \times X,Y \mapsto (a,b,c)\), where \((a,c) \cap Y = \varnothing\) and \(a<b<c\). 
  Now we give a map from \(\omega_1\) to \(X \times X \times X\) recursively. 
  Assume for some cardinary \(\alpha\) we have defined \(f(\beta)\) for every \(\beta<\alpha\), now we give \(f(\alpha)\). 
  Let \(f(\alpha):=\tau(\{b:\exists a,c \in X,(a,b,c) \in f[\alpha]\})\). 
  Since \(\alpha<\omega_1\) is countable, we get the set \(\{b:\exists a,c \in X,(a,b,c) \in f[\alpha]\}\) is countalbe, so it's in \(\dom(\tau)\). 
  So the defination of \(f\) is well. 

  Now consider \(\{(a,b)\times(b,c):(a,b,c) \in \ran(f)\}\). From the defination of \(f\) we easily get they are disjoint open interval. 
  And easily we get \(f\) is injective, so \(\ran(f)\) is uncountable. So we get a class of uncountable disjoint open interval in \(X \times X\). 
  So we obtain \(X \times X \) is not c.c.c.
\end{solution}

\begin{problem}\label{pro:3}
  Assume \(P\) is a perfect set and \(P \cap (a,b) \neq \varnothing\). Prove that \(\card(P \cap (a,b))=\mathfrak{c}\). 
\end{problem}
\begin{lemma}\label{lem:1}
  If \(P\) is a perfect set and \(P \cap (a,b) \neq \varnothing\), then \(\exists c \in (a,b) \cap P\) such that \(P \cap (a,c),P \cap(c,b)\neq \varnothing\). 
\end{lemma}
\begin{proof}
  Sicne \(P \cap(a,b) \neq \varnothing\), take \(x \in P \cap (a,b)\). 
  Since \(P\) is perfect set, and \((a,b)\) is open, we get \(\exists y \neq x,y \in P \cap (a,b)\). 
  without loss of generality we assume \(x<y\). Then \(y \in (x,c) \cap P\), so \(\exists z \neq y,z \in (x,c) \cap P\) for the same reason. 
  Without loss of generality we assume \(y<z\). Now let \(c=y\), we get \(c \in P \cap (a,b)\) and \(x \in (a,c) \cap P,z \in (c,b) \cap P\). 
  So the lemma have been proved. 
\end{proof}
\begin{solution}
  To prove \(\card(P \cap(a,b))=\mathfrak{c}\), we only need to find a injective from \((0,1)\) to \((P \cap (a,b)) \). 
  First we consider the tree \(T:=\fun{<\omega}{2}\). We define \(\tau:T \to \{(m,n) \in X \times X:m<n,\exists x \in P,m<x<n\}\) recursively. 
  Let \(\tau(\varnothing)=(a,b)\). Assume for \(x \in T\) we have defined \(\tau\) on \((\cdot,x)_T\). Now we need to give \(\tau(x)\). 
  Assume \(ordertype(x)=n+1\) for some \(n \in \mathbb{N}\). Then \(\dom(x)=n+1\). 
  Consider \(y=x|_{n} <_T x\), assume \(\tau(y)=(m,n)\). 
  From \Cref{lem:1} we know \(\exists l \in P \cap (m,n)\) such that \(P \cap (m,l),P \cap (l,n) \neq \varnothing\). 
  If \(x(n)=0\) we let \(\tau(x)=(m,l)\); else we get \(x(n)=1\), we let \(\tau(x)=(l,n)\). 

  From the defination of \(\tau\) we can easily get \(f \sqsubset g \to \tau(f)_1 \leq \tau(g)_1 \leq \tau(g)_2 \leq \tau(f)_2\). 
  Now we construct \(\theta:(0,1) \to P \cap (a,b)\). For \(x=\sum_{k=1}^{\infty} \frac{a_k}{2^k} \in (0,1)\), let \(f:\omega \to 2,f(k)=a_k\). 
  Consider \(f|_{n} \in T\), assume \(\theta(f|_{n})=(m_k,n_k)\). 
  Then we get \(\bigcap_{k \in \mathbb{N}} [m_k,n_k]\cap P \neq \varnothing\) since they are descending compect set. 
  Let \(\theta(x):= \min \bigcap_{k \in \mathbb{N}} [m_k,n_k]\cap P\). 

  First we prove \(\theta(x) \in (a,b) \cap P\). Obviously \(\theta(x) \in P \cap [a,b]\), since \([m_k,n_k] \subset [a,b]\). 
  Since \(x \in (0,1),\exists k,a_k =0\). Then \([m_k,n_k]= [m_{k-1},l_{k-1}]\), where \(l_{k-1}<m_{k-1} \leq b\). 
  So we get \(\theta(x)<b\). For the same reason we get \(\theta(x)>a\). 

  Now we only need to prove \(\theta\) is injective. Assume \(x \neq y\). 
  Assume \(x=\sum_{k=1}^{\infty} \frac{a_k}{2^k},y=\sum_{k=1}^{\infty} \frac{b_k}{2^k}\). Without loss of generality we assume \(x<y\). 
  And we can assume \(\lim a_k,\lim b_k \neq 1\). Assume \(t=\min\{k:a_k<b_k\}\). 
  Let \(f(k)=a_k,g(k)=b_k\), and \(\tau(f|_{t+1})=(m,n),\tau(g|_{t+1})=(n,l)\). 
  Then \(\theta(x) \leq n \leq \theta(y)\). Now we only need to prove \(\theta(x)<n\). 
  Since \(\lim a_k neq 1\), we obtain \(\exists i>t,a_i=0\). Then \(\tau(f|_{k+1})_2<n\), and thus \(\theta(x)<n \leq \theta(y)\). 

  So we have found a injective from \((0,1)\) to \(P cap (a,b)\). And obviously \((a,b) \cap P \subset \mathbb{R}\), we get \(\card((a,b) \cap P)=\mathfrak{c}\). 
\end{solution}

\begin{problem}\label{pro:6}
  If \(P_1,P_2\) are perfect set and \(P_1 \subsetneq P_2\), then \(\card(P_2\setminus P_1)=\mathfrak{c}\). 
\end{problem}

\begin{solution}
Take \(x \in P_2 \setminus P_1\). Since \(P_1\) is closed and \(x \notin P_1\), we know \(\exists a,b \in \mathbb{R},a<x<b,(a,b) \cap P_1 = \varnothing\). 
Now consider \((a,b) \cap P_{2}\), since \(x \in (a,b) \cap P_2 \neq \varnothing\), from \Cref{pro:3} we get \(\card((a,b) \cap P_2)=\mathfrak{c}\). 
Noting \((a,b) \cap P_2 \subset P_1^{c} \cap P_2 = P_2 \setminus P_1 \subset \mathbb{R}\), we finally get \(\card(P_2 \setminus P_1)=\mathfrak{c}\). 
\end{solution}

\begin{problem}
  If \(P\) is perfect set, then \(P^{*}=P\). 
\end{problem}

\begin{solution}
  First we prove \(P \subset P^{*}\). Take \(a \in P\), consider a neibor of \(a\), named \(I\). 
  Since \(a \in I \cap P \neq \varnothing\), from \Cref{pro:3} we get \(\card(P \cap I)=\mathfrak{c}\). 
  So \(a \in P^{*}\). 

  Second we prove \(P^{*} \subset P\). From the defination of \(P^{*}\) we know \(P^{*}\subset \overline{P} = P\). 
  So we get \(P = P^{*}\). 
\end{solution}

\begin{problem}\label{pro:4}
  If \(F\) is colsed and \(P \subset F\) is a perfect set, then \(P \subset F^{*}\). 
\end{problem}

\begin{solution}
  From the defination of \({}^{*}\) we easily get \(A \subset B \to A^{*} \subset B^{*}\). 
  So we get \(P = P^{*} \subset F^{*}\). 
\end{solution}

\begin{problem}\label{pro:5}
  Assume \(F=P \cup S\) is closed set, where \(P\) is perfect and \(S\) is at most countable. 
  Prove that \(F^{*}=P\). 
\end{problem}

\begin{solution}
  From \Cref{pro:4} we get \(F^{*} \supset P\), now we only need to prove \(F^{*}\subset P\). 
  Assume \(a \in F^{*}\), then for any neibor of \(a\), named \(I\), we get \(I \cap F\) is uncountable. 
  Since \(F=P \cup S\) and \(S\) is countable, we easily get \(I \cap P \neq \varnothing\). 
  So we get \(a \in \overline{P}=P\). So \(F^{*} \subset P\). 
\end{solution}

\begin{problem}
  Assume \(F\) is uncountable closed set, prove that \(F=F^{*} \cup (F\setminus F^{*})\) is the unique way to part \(F\) into a perfect set and an at most countable set. 
\end{problem}

\begin{solution}
  First we prove \(F^{*} \neq \varnothing\). If \(F^{*}=\varnothing\), then \(\forall a \in F^{*},\exists I_a\) is neibor of \(a\) such that 
  \(I_a \cap F\) is at most countable. 
  Since \(F= \bigcup_{n \in \mathbb{N}} F \cap [-n,n]\) is uncountable, we know at least one of \(F \cap [-n,n]\) is uncountable, name it \(E\). 
  Then \(E\) is compect. Notin \(\{I_a:a \in E\}\) is a covery of \(E\), so there exists finite subcovery. 
  Then we get \(E=\bigcap_{k=1}^{n} I_{a_k}\) is countable, contradiction! 
  So \(F^{*} \neq \varnothing\).

  Next we prove \(F^{*}\) is colsed. From the defination of \({}^{*}\) it's obvious.

  Then we prove \(F \setminus F^{*}\) is at most countable. 
  Consider \(F_n:=\{x \in F\setminus F^{*}:d(x,F^{*})\geq\frac{1}{n}\}\). We get \(F \setminus F^{*}=\bigcup_{n \in \mathbb{N}^{+}} F_n \). 
  So we only need to prove \(F_n\) is countable. If not, since \(F_n\) is uncountable colsed set, from above we easily know \(F_n^{*} \neq \varnothing\). 
  But \(F_n^{*} \subset F_n \subset F\setminus F^{*}\) and since \(F_n \subset F\) we know \(F_n^{*}\subset F^{*}\), contradiction! 
  So we get \(F\setminus F^{*}\) is at most countable. 

  Now we prove \(F^{*}\) is perfect set. Only need to prove every \(a \in F^{*}\) is limit point. 
  For any neibor \(I\) of \(a\), we easily know \(I \cap F\) is uncountable. 
  Sicne \(F \setminus F^{*}\) is countable, we get \(I \cap F^{*}\neq \varnothing\). 
  So \(a\) is limit point. 

  Finally we prove it's unique. Assume \(F=P_1 \cup S_1=P_2 \cup S_2\), where \(P_1,P_2\) are perfect sets and \(S_1,S_2\) are atmost countable. 
  Consider \(S:=S_1 \cup S_2,P:=P_1 \cap P_2 = F\setminus S\). Since \(S\) is atmost countable, we get \(P \neq \varnothing\). 
  So \(P\) is perfect, too. 
  If \(P \subsetneq P_1\), then from \Cref{pro:6} we get \(\card(P_1\setminus P)=\mathfrak{c}\). 
  But \(P_1\setminus P \subset S_2\) is at most countable, contradiction! 
  So \(P=P_1\). For the same reason we get \(P=P_2\). So \(P_1=P_2\) and thus \(S_1=S_2\). 
\end{solution}

\begin{problem}
  If \(B\) is Borel and \(f\) is continuous, then \(f_{-1}(B)\) is Borel. 
\end{problem}

\begin{solution}
  Let \(\mathcal{A}:=\{A \in \mathcal{B}:f_{-1}(A) \in \mathcal{B}\}\). 
  Since \(f\) is continuous we get for open set \(I \subset \mathbb{R}, f_{-1}(I)\) is open, thus Borel. 
  So \(I \in \mathcal{A}\). Now we need to prove \(\mathcal{A}=\mathcal{B}\), only need to prove \(\mathcal{A}\) is \(\sigma\)-field. 
  
  First, for \(A \in \mathcal{A}\), we have \(f_{-1}(A^{c})=(f_{-1}(A))^{c} \in \mathcal{B}\). 

  Second, for \(A_n \in \mathcal{A}, \forall n \in \mathbb{N}\), we have \(f_{-1}(\bigcup_{n \in \mathbb{N}} A_n)=\bigcup_{n \in \mathbb{N}} f_{-1}(A_n) \in \mathcal{B}\). 

  Finally, it's obviously that \(\varnothing,\mathbb{R} \in \mathcal{A}\). 

  So we get \(\mathcal{A}\) is \(\sigma\)-field, thus \(\mathcal{B}=\mathcal{A}\). 
  So \(B \in \mathcal{A}\), thus \(f_{-1}(B) \in \mathcal{B}\). 
\end{solution}

\begin{problem}
  For closed set \(F \subset \mathcal{N}\), let \(T_F:=\{s \in \mathrm{Seq}:\exists f \in F,s \subset f\}\). 
  Prove that \(T_F\) has no maximal node. And prove \(F \mapsto T_F\) is bijection from closed set in \(\mathcal{N}\) and sequnetial trees with no maximal node. 
\end{problem}

\begin{solution}
  For \(s \in T_F\), we assume \(s \subset f \in F\). 
  Since \(s\) is finite, we can assume \(s = f|_{n}\). Then \(s \subsetneq f|_{n+1} \in T_F\), 
  so \(s\) is not maximal node. So \(T_F\) has no maximal node. 

  Now we prove \(F \mapsto T_F\) is injective. 
  Assume \(T_F=T_G\), where \(F,G\) are closed subset of \(\mathcal{N}\). 
  Only need to prove \(F=G\). By symetry we only need to prove \(F \subset G\). 
  For \(f \in F\), from the defination of \(T_F\) we know \(\forall n \in \mathbb{N},f_n:=f|_{n} \in T_F=T_G\). 
  From defination of \(T_G\) we get \(\exists g_n \in G,g_n|_{n} = f|_{n}\). Then \(d(g_n,f) \leq \frac{1}{2^{n}}\). 
  So \(g_n \to f\). Since \(G\) is closed, we get \(f \in G\). 

  Finally we prove \(F \to T_F\) is surjective. 
  Assume \(T \subset \mathrm{Seq}\) has no maximal node. Let \(F:=\{f \in \mathcal{N}:\forall n,f|_{n} \in T\}\). 
  Now we prove \(T_F = T\). Easily we get \(T_F \subset T\), only need \(T \subset T_F\). 
  For \(s \in T\), we need to prove \(s \in T_F\), i.e., \(\exists f \in F,s \subset f\). 
  Now we construct \(f\) recursively. For \(n \in \dom(s)\) we let \(f(n)=s(n)\). 
  Now assume \(n \notin \dom(s)\) and for \(m<n\) we have defined \(f(m)\) and \(f|_{n} \in T\). 
  Since \(T\) has no maximal, we get \(f|_{n}\) is not maximal. So \(\{t \in \mathbb{N}:f|_{n}\cup \{(n,t)\} \in T\} \neq \varnothing\). 
  Let \(f(n):=\min \{t \in \mathbb{N}:f|_{n} \cup \{(n,t)\} \in T\}\) is OK. 
  Easily we get \(f|_{n} \in T,\forall n \in \mathbb{N}\). So \(f \in F\), and \(s \subset f\). 
\end{solution}
\end{document}
