%!Mode:: "TeX:UTF-8"
%!TEX encoding = UTF-8 Unicode
%!TEX TS-program = xelatex × 2
\documentclass{ctexart}
\newif\ifpreface
\prefacetrue
\usepackage{fontspec}
\usepackage{bbm}
\usepackage{tikz}
\usepackage{amsmath,amssymb,amsthm,color,mathrsfs}
\usepackage{fixdif}
\usepackage{hyperref}
\usepackage{cleveref}
\usepackage{enumitem}%
\usepackage{expl3}
\usepackage{lipsum}
\usepackage[margin=0pt]{geometry}
\usepackage{listings}
\definecolor{mGreen}{rgb}{0,0.6,0}
\definecolor{mGray}{rgb}{0.5,0.5,0.5}
\definecolor{mPurple}{rgb}{0.58,0,0.82}
\definecolor{backgroundColour}{rgb}{0.95,0.95,0.92}

\lstdefinestyle{CStyle}{
  backgroundcolor=\color{backgroundColour},
  commentstyle=\color{mGreen},
  keywordstyle=\color{magenta},
  numberstyle=\tiny\color{mGray},
  stringstyle=\color{mPurple},
  basicstyle=\footnotesize,
  breakatwhitespace=false,
  breaklines=true,
  captionpos=b,
  keepspaces=true,
  numbers=left,
  numbersep=5pt,
  showspaces=false,
  showstringspaces=false,
  showtabs=false,
  tabsize=2,
  language=C
}
\usetikzlibrary{calc}
\theoremstyle{remark}
\newtheorem{lemma}{Lemma}
\usepackage{fontawesome5}
\usepackage{xcolor}
\newcounter{problem}
\newcommand{\Problem}{\begin{tikzpicture}[baseline]%
    \node at (-0.02em,0.3em) {$\mathbb{P}$};%
    \node[scale=0.7] at (0.2em,-0.0em) {R};%
    \node[scale=0.7] at (0.6em,0.4em) {O};%
    \node[scale=0.8] at (1.05em,0.25em) {B};%
    \node at (1.55em,0.3em) {L};%
    \node[scale=0.7] at (1.75em,0.45em) {E};%
    \node at (2.35em,0.3em) {M};%
  \end{tikzpicture}%
}
\renewcommand{\theproblem}{\Roman{problem}}
\newenvironment{problem}{\refstepcounter{problem}\noindent\color{blue}\Problem\theproblem}{}

\crefname{problem}{\protect\Problem}{Problem}
\newcommand\Solution{\begin{tikzpicture}[baseline]%
    \node at (-0.04em,0.3em) {$\mathbb{S}$};%
    \node[scale=0.7] at (0.35em,0.4em) {O};%
    \node at (0.7em,0.3em) {\textit{L}};%
    \node[scale=0.7] at (0.95em,0.4em) {U};%
    \node[scale=1.1] at (1.19em,0.32em){T};%
    \node[scale=0.85] at (1.4em,0.24em){I};%
    \node at (1.9em,0.32em){$\mathcal{O}$};%
    \node[scale=0.75] at (2.3em,0.21em){\texttt{N}};%
  \end{tikzpicture}}
\newenvironment{solution}{\begin{proof}[\Solution]}{\end{proof}}
\title{\input{../../.subject}\input{../.number}}
\makeatletter
\newcommand\email[1]{\def\@email{#1}\def\@refemail{mailto:#1}}
\newcommand\schoolid[1]{\def\@schoolid{#1}}
\ifpreface
  \def\@maketitle{
  \raggedright
  {\Huge \bfseries \sffamily \@title }\\[1cm]
  {\Huge  \bfseries \sffamily\heiti\@author}\\[1cm]
  {\Huge \@schoolid}\\[1cm]
  {\Huge\href\@refemail\@email}\\[0.5cm]
  \Huge\@date\\[1cm]}
\else
  \def\@maketitle{
    \raggedright
    \begin{center}
      {\Huge \bfseries \sffamily \@title }\\[4ex]
      {\Large  \@author}\\[4ex]
      {\large \@schoolid}\\[4ex]
      {\href\@refemail\@email}\\[4ex]
      \@date\\[8ex]
    \end{center}}
\fi
\makeatother
\ifpreface
  \usepackage[placement=bottom,scale=1,opacity=1]{background}
\fi

\author{白永乐}
\schoolid{202011150087}
\email{202011150087@mail.bnu.edu.cn}

\def\to{\rightarrow}
\newcommand{\xor}{\vee}
\newcommand{\bor}{\bigvee}
\newcommand{\band}{\bigwedge}
\newcommand{\xand}{\wedge}
\newcommand{\minus}{\mathbin{\backslash}}
\newcommand{\mi}[1]{\mathscr{P}(#1)}
\newcommand{\card}{\mathrm{card}}
\newcommand{\oto}{\leftrightarrow}
\newcommand{\hin}{\hat{\in}}
\newcommand{\gl}{\mathrm{GL}}
\newcommand{\im}{\mathrm{Im}}
\newcommand{\re }{\mathrm{Re }}
\newcommand{\rank}{\mathrm{rank}}
\newcommand{\tra}{\mathop{\mathrm{tr}}}
\renewcommand{\char}{\mathop{\mathrm{char}}}
\DeclareMathOperator{\ot}{ordertype}
\DeclareMathOperator{\dom}{dom}
\DeclareMathOperator{\ran}{ran}

\DeclareMathOperator{\ord}{Ord}
\DeclareMathOperator{\otype}{OrderType}
\newcommand{\ini}{\mathrm{\mathop{ini}}}
\newtheorem{example}{Example}
\newcommand{\N}{\mathbb{N}}
\newcommand{\calA}{\mathcal{A}}
\newcommand{\len}{\mathop{\mathrm{len}}}
\newcommand{\Q}{\mathbb{Q}}
\crefname{enumi}{}{}
\begin{document}
\large
\setlength{\baselineskip}{1.2em}
\ifpreface
\backgroundsetup{contents={%
    \begin{tikzpicture}
      \fill [white] (current page.north west) rectangle ($(current page.north east)!.3!(current page.south east)$) coordinate (a);
      \fill [bgc] (current page.south west) rectangle (a);
\end{tikzpicture}}}
\definecolor{word}{rgb}{1,1,0}
\definecolor{bgc}{rgb}{1,0.95,0}
\setlength{\parindent}{0pt}
\thispagestyle{empty}
\begin{tikzpicture}%
  % \node[xscale=2,yscale=4] at (0cm,0cm) {\sffamily\bfseries \color{word} under};%
  \node[xscale=4.5,yscale=10] at (10cm,1cm) {\sffamily\bfseries \color{word} Graduate Homework};%
  \node[xscale=4.5,yscale=10] at (8cm,-2.5cm) {\sffamily\bfseries \color{word} In Mathematics};%
\end{tikzpicture}
\ \vspace{1cm}\\
\begin{minipage}{0.25\textwidth}
  \textcolor{bgc}{王胤雅是傻逼}
\end{minipage}
\begin{minipage}{0.75\textwidth}
  \maketitle
\end{minipage}
\vspace{4cm}\ \\
\begin{minipage}{0.2\textwidth}
  \
\end{minipage}
\begin{minipage}{0.8\textwidth}
  {\Huge
    \textinconsolatanf{}
  }General fire extinguisher
\end{minipage}
\newpage\backgroundsetup{contents={}}\setlength{\parindent}{2em}

\else
\maketitle
\fi
\newgeometry{left=2cm,right=2cm,top=2cm,bottom=2cm}
%from_here_to_type
\section{Question}
\begin{problem}
Let $(U,\le),(V,\prec)$ be two well-orderings. Consider $f:=\{(x,y):x\in U\xand y\in V\xand (U_x,\le)\cong (V_y,\prec)\}$, prove $f$ is isomorphism from some initial segment of $U$ to some initial segment of $V$.
\end{problem}

\begin{problem}\label{pro:2}
The relation ``$(P,\le)\cong(Q,\le)$'' is an equivalence relation (on the class of all partially ordered sets).
\end{problem}


\begin{problem}
Let $\mathcal{A}$ denote the class of all well orderings. For any $a,b\in\mathcal{A}$, define $a\prec b\iff a$ is isomorphic to an initial segment of $b$. Show that $\prec$ is a well ordering on $\mathcal{A}/\cong$, where $\cong$ is the equivalence relation given in \Cref{pro:2}.
\end{problem}


\begin{problem}
\begin{enumerate}
\item If $(W,<)$ is a well ordering and $U \subset W$, then $(U,<\cap(U \times U))$ is a well ordering.
\item
If $\left(W_1,<_1\right)$ and $\left(W_2,<_2\right)$ are two well orderings and $W_1 \cap W_2=\varnothing$, then $W_1 \oplus W_2=\left(W_1 \cup W_2, \prec\right)$ is a well ordering, where
$$
\prec=<_1 \cup<_2 \cup\left\{(a, b) \mid a \in W_1 \wedge b \in W_2\right\}
$$
\item
If $\left(W_1,<_1\right)$ and $\left(W_2,<_2\right)$ are two well orderings, then $W_1 \otimes W_2=\left(W_1 \times W_2, \prec\right)$ is a well ordering, where
$$
\left(a_1, b_1\right) \prec\left(a_2, b_2\right) \leftrightarrow b_1<_2 b_2 \vee\left(b_1=b_2 \wedge a_1<_1 a_2\right)
$$
\end{enumerate}
\end{problem}


\begin{problem}
Show that the following are equivalent:
\begin{enumerate}[ref=\theproblem.\arabic*]
\item\label{it:1} $T$ is transitive;
\item\label{it:2}$\bigcup T \subseteq T$;
\item\label{it:3} $T \subseteq \mathscr{P}(T)$.
\end{enumerate}
\end{problem}


\begin{problem}
Let $\alpha, \beta, \gamma \in$ Ord and let $\alpha<\beta$. Then
\begin{enumerate}[label=\alph*,ref=\theproblem.\alph*]
\item\label{it:11} $\alpha+\gamma \leq \beta+\gamma$.
\item\label{it:12} $\alpha \cdot \gamma \leq \beta \cdot \gamma$.
\item\label{it:13} $\alpha^\gamma \leq \beta^\gamma$.
\end{enumerate}
Given examples to show that $\leq$ cannot be replaced by $<$ in either inequality.
\end{problem}

\begin{example}
\begin{enumerate}[label=\alph*,ref=Example\theexample.\alph*]
\item\label{it:71} Let $\alpha=0,\beta=1,\gamma=\omega$, then $\alpha<\beta$ but $\alpha+\gamma=\omega=1+\omega=\beta+\gamma$.
\item\label{it:72} Let $\alpha=1,\beta=2,\gamma=\omega$, then $\alpha\cdot \gamma=\omega=2\cdot\omega =\omega$.
\item\label{it:73}  Let $\alpha=2,\beta=3,\gamma=\omega$, then $\alpha^\gamma=\beta^\gamma$.
\end{enumerate}
\end{example}

\begin{problem}
Show that the following rules do not hold for all $\alpha, \beta, \gamma \in\ord$:
\begin{enumerate}[label=\alph*,ref=\theproblem.\alph*]
\item If $\alpha+\gamma=\beta+\gamma$ then $\alpha=\beta$.
\item If $\gamma>0$ and $\alpha \cdot \gamma=\beta \cdot \gamma$ then $\alpha=\beta$.
\item $(\beta+\gamma) \cdot \alpha=\beta \cdot \alpha+\gamma \cdot \alpha$.
\end{enumerate}
\end{problem}


\begin{problem}
Find a set $A \subset \mathbb{Q}$ such that $\left(A,<_{\mathbb{Q}}\right) \cong(\alpha, \in)$, where
\begin{enumerate}[label=\alph*,ref=\theproblem.\alph*]
\item  $\alpha=\omega+1$,
\item  $\alpha=\omega \cdot 2$,
\item  $\alpha=\omega \cdot \omega$,
\item  $\alpha=\omega^\omega$,
\item  $\alpha=\varepsilon_0$.
\item  $\alpha$ is any ordinal $<\omega_1$.
\end{enumerate}
\end{problem}



\begin{problem}
An ordinal $\gamma$ is a limit ordinal iff $\gamma=\omega \cdot \beta$ for some $\beta \in\ord$.
\end{problem}



\begin{problem}
Find the first three $\alpha>0$ s.t. $\xi+\alpha=\alpha$ for all $\xi<\alpha$.
\end{problem}



\begin{problem}
Find the least $\xi$ such that
\begin{enumerate}[label=\alph*,ref=\theproblem.\alph*]
\item  $\omega+\xi=\xi$.
\item  $\omega \cdot \xi=\xi, \xi \neq 0$.
\item  $\omega^{\xi}=\xi$.
\end{enumerate}
(Hint for (1): Consider a sequence $\left\langle\xi_n\right\rangle$ s.t. $\xi_{n+1}=\omega+\xi_n$.)
\end{problem}
\begin{lemma}\label{lem:f}
If $f:\ord\to\ord$ and $a\leq b\to f(a)\leq f(b)$ and $f(\sup B)=\sup f(B)$ for any $B$ is subset of $\ord$, let $a_0=0,a_{n+1}=f(a_n)$, then $\xi=\sup\{a_n:n\in\N\}$ is the least $\xi$ such that $f(\xi)=\xi$.
\end{lemma}
\begin{proof}
First we prove $a_{n+1}\geq a_n$. Use MI it's obvious.

Second we prove $f(\xi)=\xi$. Obviously $f(\xi)=f(\sup\{a_n\})=\sup\{f(a_n)\}=\sup\{a_{n+1}\}=\lim a_{n+1}=\lim a_n=\xi$.

Finally we prove $\xi$ is the least. Assume $f(\alpha)=\alpha$, then use MI we can easily prove $\alpha\geq a_n\forall n<\omega$. So $\alpha\geq \sup\{a_n\}=\xi$.
\end{proof}


