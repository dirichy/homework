%!Mode:: "TeX:UTF-8"
%!TEX encoding = UTF-8 Unicode
%!TEX TS-program = xelatex
\documentclass{ctexart}
\newif\ifpreface
\prefacetrue
\usepackage{fontspec}
\usepackage{bbm}
\usepackage{tikz}
\usepackage{amsmath,amssymb,amsthm,color,mathrsfs}
\usepackage{fixdif}
\usepackage{hyperref}
\usepackage{cleveref}
\usepackage{enumitem}%
\usepackage{expl3}
\usepackage{lipsum}
\usepackage[margin=0pt]{geometry}
\usepackage{listings}
\definecolor{mGreen}{rgb}{0,0.6,0}
\definecolor{mGray}{rgb}{0.5,0.5,0.5}
\definecolor{mPurple}{rgb}{0.58,0,0.82}
\definecolor{backgroundColour}{rgb}{0.95,0.95,0.92}

\lstdefinestyle{CStyle}{
  backgroundcolor=\color{backgroundColour},
  commentstyle=\color{mGreen},
  keywordstyle=\color{magenta},
  numberstyle=\tiny\color{mGray},
  stringstyle=\color{mPurple},
  basicstyle=\footnotesize,
  breakatwhitespace=false,
  breaklines=true,
  captionpos=b,
  keepspaces=true,
  numbers=left,
  numbersep=5pt,
  showspaces=false,
  showstringspaces=false,
  showtabs=false,
  tabsize=2,
  language=C
}
\usetikzlibrary{calc}
\theoremstyle{remark}
\newtheorem{lemma}{Lemma}
\usepackage{fontawesome5}
\usepackage{xcolor}
\newcounter{problem}
\newcommand{\Problem}{\begin{tikzpicture}[baseline]%
    \node at (-0.02em,0.3em) {$\mathbb{P}$};%
    \node[scale=0.7] at (0.2em,-0.0em) {R};%
    \node[scale=0.7] at (0.6em,0.4em) {O};%
    \node[scale=0.8] at (1.05em,0.25em) {B};%
    \node at (1.55em,0.3em) {L};%
    \node[scale=0.7] at (1.75em,0.45em) {E};%
    \node at (2.35em,0.3em) {M};%
  \end{tikzpicture}%
}
\renewcommand{\theproblem}{\Roman{problem}}
\newenvironment{problem}{\refstepcounter{problem}\noindent\color{blue}\Problem\theproblem}{}

\crefname{problem}{\protect\Problem}{Problem}
\newcommand\Solution{\begin{tikzpicture}[baseline]%
    \node at (-0.04em,0.3em) {$\mathbb{S}$};%
    \node[scale=0.7] at (0.35em,0.4em) {O};%
    \node at (0.7em,0.3em) {\textit{L}};%
    \node[scale=0.7] at (0.95em,0.4em) {U};%
    \node[scale=1.1] at (1.19em,0.32em){T};%
    \node[scale=0.85] at (1.4em,0.24em){I};%
    \node at (1.9em,0.32em){$\mathcal{O}$};%
    \node[scale=0.75] at (2.3em,0.21em){\texttt{N}};%
  \end{tikzpicture}}
\newenvironment{solution}{\begin{proof}[\Solution]}{\end{proof}}
\title{\input{../../.subject}\input{../.number}}
\makeatletter
\newcommand\email[1]{\def\@email{#1}\def\@refemail{mailto:#1}}
\newcommand\schoolid[1]{\def\@schoolid{#1}}
\ifpreface
  \def\@maketitle{
  \raggedright
  {\Huge \bfseries \sffamily \@title }\\[1cm]
  {\Huge  \bfseries \sffamily\heiti\@author}\\[1cm]
  {\Huge \@schoolid}\\[1cm]
  {\Huge\href\@refemail\@email}\\[0.5cm]
  \Huge\@date\\[1cm]}
\else
  \def\@maketitle{
    \raggedright
    \begin{center}
      {\Huge \bfseries \sffamily \@title }\\[4ex]
      {\Large  \@author}\\[4ex]
      {\large \@schoolid}\\[4ex]
      {\href\@refemail\@email}\\[4ex]
      \@date\\[8ex]
    \end{center}}
\fi
\makeatother
\ifpreface
  \usepackage[placement=bottom,scale=1,opacity=1]{background}
\fi

\author{白永乐}
\schoolid{202011150087}
\email{202011150087@mail.bnu.edu.cn}

\def\to{\rightarrow}
\newcommand{\xor}{\vee}
\newcommand{\bor}{\bigvee}
\newcommand{\band}{\bigwedge}
\newcommand{\xand}{\wedge}
\newcommand{\minus}{\mathbin{\backslash}}
\newcommand{\mi}[1]{\mathscr{P}(#1)}
\newcommand{\card}{\mathrm{card}}
\newcommand{\oto}{\leftrightarrow}
\newcommand{\hin}{\hat{\in}}
\newcommand{\gl}{\mathrm{GL}}
\newcommand{\im}{\mathrm{Im}}
\newcommand{\re }{\mathrm{Re }}
\newcommand{\rank}{\mathrm{rank}}
\newcommand{\tra}{\mathop{\mathrm{tr}}}
\renewcommand{\char}{\mathop{\mathrm{char}}}
\DeclareMathOperator{\ot}{ordertype}
\DeclareMathOperator{\dom}{dom}
\DeclareMathOperator{\ran}{ran}

\begin{document}
\large
\setlength{\baselineskip}{1.2em}
\ifpreface
    \backgroundsetup{contents={%
    \begin{tikzpicture}
      \fill [white] (current page.north west) rectangle ($(current page.north east)!.3!(current page.south east)$) coordinate (a);
      \fill [bgc] (current page.south west) rectangle (a);
\end{tikzpicture}}}
\definecolor{word}{rgb}{1,1,0}
\definecolor{bgc}{rgb}{1,0.95,0}
\setlength{\parindent}{0pt}
\thispagestyle{empty}
\begin{tikzpicture}%
  % \node[xscale=2,yscale=4] at (0cm,0cm) {\sffamily\bfseries \color{word} under};%
  \node[xscale=4.5,yscale=10] at (10cm,1cm) {\sffamily\bfseries \color{word} Graduate Homework};%
  \node[xscale=4.5,yscale=10] at (8cm,-2.5cm) {\sffamily\bfseries \color{word} In Mathematics};%
\end{tikzpicture}
\ \vspace{1cm}\\
\begin{minipage}{0.25\textwidth}
  \textcolor{bgc}{王胤雅是傻逼}
\end{minipage}
\begin{minipage}{0.75\textwidth}
  \maketitle
\end{minipage}
\vspace{4cm}\ \\
\begin{minipage}{0.2\textwidth}
  \
\end{minipage}
\begin{minipage}{0.8\textwidth}
  {\Huge
    \textinconsolatanf{}
  }General fire extinguisher
\end{minipage}
\newpage\backgroundsetup{contents={}}\setlength{\parindent}{2em}

\else
\maketitle
\fi
\newgeometry{left=2cm,right=2cm,top=2cm,bottom=2cm}
\crefname{enumi}{}{}
%from_here_to_type
\begin{problem}
  Prove:\(F \subset \mathcal{N}\) is closed set \(\iff F = [T]\) for some \(T \subset \fun{<\omega}{\omega}\). 
\end{problem}
\begin{solution}
  \begin{itemize}
    \item \(\implies\): Let \(T:=T_F\), now we need to prove \(F=[T]\). 
      Form the defination of \(T_F\) and \([T]\) easily we get \(F \subset [T]\). 
      Now we prove \([T] \subset F\). For \(f \in [T]\), we get \(f\upharpoonright n \in T\). 
      i.e., \(\forall n \in \mathbb{N}, \res{f}{n}=\res{g}{n}\) for some \(g \in F\). 
      So \(d(f,F) \leq d(f,g)=\frac{1}{2^n}\). Since \(F\) is closed, we get \(f \in F\). 

    \item \(\impliedby\): For any \([T] \in \fun{<\omega}{\omega}\), we need to prove \([T]\) is closed. 
      Assume \(f \in \overline{[T]}\), then \(\forall n \in \mathbb{N},\exists g \in [T],\res{f}{n}=\res{g}{n}\). 
      Since \(g \in [T]\) we get \(\res{f}{n}=\res{g}{n} \in T\). So \(f \in [ T]\). 
      So \([T]\) is closed. 
  \end{itemize}
\end{solution}

\begin{problem}
  Assume \(f\) is isolated point in closed set \(F \subset \mathcal{N}\), then 
  \(\exists n \in \mathbb{N},\forall g \in F,g \neq f \to \res{g}{n} \neq \res{f}{n}\). 
\end{problem}

\begin{solution}
  Since \(f\) is isolated, we get \(\exists n \in \mathbb{N},\forall g \in F \setminus \{f\},d(f,g)>\frac{1}{2^n}\). 
  Then \(\res{f}{n} \neq \res{g}{n}\). 
\end{solution}

\begin{problem}
  A closed set \(F \subset \mathcal{N}\) is perfect \(\iff T_F\) is perfecr tree. 
\end{problem}

\begin{solution}
  \begin{itemize}
    \item \(\implies\): For \(t \in T_F\), by defination we know \(\exists f \in F,n \in \mathbb{N},t =\res{f}{n}\).
      Since \(f\) is perfect we know \(\exists g \in F \wedge g \neq f,d(f,g)<\frac{1}{2^{n+1}}\). 
      Then \(t =\res{f}{n} \sqsubset g\). Since \(f \neq g\), we get \(\exists m \in \mathbb{N} \wedge m >n,\res{f}{m}\neq \res{g}{m}\). 
      So \(t \sqsubset \res{f}{m},t \sqsubset \res{g}{m}\), and \(\res{f}{m},\res{g}{m}\) are incomparable. 
    \item \(\impliedby\): For \(f \in F\), we need to prove \(f\) is limit point. 
      \(\forall n \in \mathbb{N},t:=\res{f}{n} \in T_F\). 
      So \(\exists s_1,s_2 \in T_F\) such that \(t \sqsubset s_1,s_2\) and \(s_1,s_2\) are incomparable. 
      Then \(s_1,s_2 \sqsubset f\) is impossible. Without loss of generality assume \(s_1 \not \sqsubset f\). 
      Then \(s_1=\res{g}{m}\) for some \(g \in F,m \in \mathbb{N}\). 
      So \(d(f,g) \leq \frac{1}{2^n}\). So \(f\) is not isolated. 
  \end{itemize}
\end{solution}

\begin{problem}
  For \(\alpha <\omega_1\), we let \(\Sigma_0=\) the set of all open set in \(\mathbb{R}\), 
  and \(\Pi_0=\) the set of all closed set in \(\mathbb{R}\). 
  And \(\Sigma_{\alpha+1}=\{\bigcup_{n \in \mathbb{N}} A(n):A \in \fun{\mathbb{N}}{\Pi_{\alpha}}\). 
  \(\Pi_{\alpha+1}=\{\mathbb{R}\setminus A:A \in \Sigma_{\alpha}\}\). 
  \(\Sigma_{\alpha}=\bigcup_{\beta<\alpha} \Sigma_{\beta},\Pi_{\alpha}=\bigcup_{\beta<\alpha} \Pi _{\beta}\) for limit ordinal \(\alpha\). 
  Prove that \(\mathcal{B}(\mathbb{R})=\bigcup_{\alpha<\omega_1} \Sigma_{\alpha}\). 
\end{problem}

\begin{solution}
  Use MI easily we get \(\bigcup_{\alpha < \omega_1} \Sigma_{\alpha} \subset \mathcal{B}(\mathbb{R})\). 
  Now we prove \(\mathcal{B}(\mathbb{R})\subset \bigcup_{\alpha<\omega_1} \Sigma_{\alpha}\). 
  Since open sets is subset of \(\bigcup_{\alpha<\omega_1} \Sigma_{\alpha}\), we only need to prove \(\bigcup_{\alpha<\omega_1} \Sigma_{\alpha}=:\mathcal{A}\) is \(\sigma\)-field. 
  Easily we get \(\Sigma_{\alpha} \subset \Sigma_{\alpha+2}\). 
  Obviously \(\mathbb{R} \in \mathcal{A}\). For \(A \in \mathcal{A}\), assume \(A \in \Sigma_{\alpha}\). 
  Then \(\mathbb{R}\setminus A \in \Pi_{\alpha+1} \subset \Sigma_{\alpha+1}\subset \mathcal{A}\). 
  Assume \(A \in \fun{\mathbb{N}}{\mathcal{A}}\), let \(f \in \fun{\mathbb{N}}{\omega_1},f(n)=\min\{\alpha \in \omega_1:A(n) \in \Sigma_{\alpha}\}\). 
  Consider \(\sup \ran f =:\gamma\). Since \(\forall \alpha \in \ran f,\alpha\) is countable. And \(\ran f\) is countable. 
  So \(\sup \ran f\) is countable, thus \(\sup \ran f <\omega_1\). 
  Then \(\ran A \subset \Pi_{\gamma+1} \). So we get \(\bigcup_{n \in \mathbb{N}} A(n) \subset \Sigma_{\gamma+2} \subset \mathcal{A}\). 
  So we get \(\mathcal{A} \) is \(\sigma\)-field. So \(\mathcal{B}(\mathbb{R}) \subset \mathcal{A}\), thus \(\mathcal{A}=\mathcal{B}(\mathbb{R})\). 
\end{solution}

\begin{problem}
  Show that \(\mathcal{M}:=\{A \subset \mathbb{R}:A \text{ is measurable}\}\) is a \(\sigma\)-field. 
\end{problem}

\begin{lemma}\label{lem:0}
  For \(A \in \fun{\mathbb{N}}{\mathcal{P}(\mathbb{R})}\), we have \(\mu^{*}(\bigcup_{n \in \mathbb{N}} A(n)) \leq \sum_{n \in \mathbb{N}} \mu^{*}(A(n))\). 
\end{lemma}

\begin{proof}
  For any \(\varepsilon>0,n \in \mathbb{N}, \exists O(n) \in \mathcal{O},A(n)\subset O(n) \wedge \mu^{*} (A(n)) \leq |O(n)|+\frac{\varepsilon}{2^{n+1}}\). 
  Let \(U:= \bigcup_{ n \in \mathbb{N}} O(n)\), then \(\bigcup_{ n \in \mathbb{N}} A(n)\subset U\). 
  So \(\mu^{*}(\bigcup_{n \in \mathbb{N}} A(n)) \leq |U| \leq \sum_{n \in \mathbb{N}} |O(n)| \leq \sum_{ n \in \mathbb{N}} \mu ^{*}(A(n))+\varepsilon\). 
  Since \(\varepsilon\) is arbitry, we get the lemma. 
\end{proof}
\begin{lemma}\label{lem:1}
  If \(G \in G_{\delta}\), then \(\forall \varepsilon >0,\exists O \in \mathcal{O},G \subset O \wedge \mu^{*}(O\setminus G)\leq \varepsilon\). 
\end{lemma}

\begin{proof}
  We first consider \(G\) is bonded. Assume \(G \subset [-M,M],M >0\). Assume \(G =\bigcap_{n \in \mathbb{N}} O_n\), where \(O_n \in \mathcal{O}\). 
  Then \(G=\bigcap_{n \in \mathbb{N}} (O_n \cap (-M-1,M+1))\). By convinence we assume \(O_n \subset (-M-1,M+1)\). 
  And \(G=\bigcap_{n \in \mathbb{N}} \bigcap_{m=1}^n O_m \), by convinence we assume \(O_n \supset O_{n+1} \). 
  
\end{proof}
\begin{solution}
  First, for \(A=\mathbb{R}\), easily we can let \(F=G=\mathbb{R}\). Then \(F\) is \(F_{\sigma}\) and \(G\) is \(G_{\delta}\). 
  Second, assume \(A \in \mathcal{M}\), consider \(B=\mathbb{R}\setminus A\). 
  Assume \(F \subset A \subset G\) and \(\mu^{*}(G\setminus F)=0\). Then \(G^c \subset B \subset F^c\). 
  And \(G^c\) is \(F_{\sigma}\), \(F^c\) is \(G_{\delta}\). And \(\mu^{*}(F^c\setminus G^c)=\mu^{*}(G\setminus F)=0\). 
  So \(B \in \mathcal{M}\). 
  Finally, assume \(A \in \fun{\mathbb{N}}{\mathcal{M}}\), we need to prove \(\bigcup_{n \in \mathbb{N}} A(n) =:A\in \mathcal{M}\). 
  Use AC we can find \(F \in \fun{ \mathbb{N}}{F_{\sigma}},G \in \fun{ \mathbb{N}}{G_{\delta}}\) such that \(F(n)\subset A(n) \subset G(n),\mu^{*}(G(n)-F(n))=0\). 
  Let \(T=\bigcup_{n \in \mathbb{N}} F(n)\). Since \(F(n)\) is \(F_{\sigma}\), we get \(T \in F_{\sigma}\). 
  And easily \(T=\bigcup_{n \in \mathbb{N}} F(n) \subset \bigcup_{n \in \mathbb{N}} A(n)=A\). 
\end{solution}

\begin{problem}
  Show that \(\mathcal{A}:=\{A \subset \mathbb{R}:A \text{ has property of Baire}\}\) is \(\sigma\)-field. 
\end{problem}

\begin{solution}
  Easily \(\mathbb{R} \Delta \mathbb{R}\) is meager, so \(\mathbb{R} \in \mathcal{A}\). 

  If \(A \in \mathcal{A}\), we need to prove \(\mathbb{R} \setminus A \in \mathcal{A}\). 
  Assume \(G \in \mathcal{O}\) and \( A \Delta G\) is meager, write \(B=\mathbb{R}\setminus A\), 
  only need to prove \(\exists U \in \mathcal{O}\), such that \(B\setminus U,U\setminus B\) are meager. 
  Let \(U=\mathbb{R} \setminus \overline{G}\). Then \(B \setminus U = A \setminus \overline{G}\) is meager. 
  Now only need to prove \(U \setminus B = \overline{ G}\setminus A\) is meager. 
  Since \(G \setminus A\) is meager, we only need to prove \(\overline{G}\setminus G\) is meager. 
  In fact, we can prove \(\overline{G}\setminus G\) is nowhere dense. 
  Consider \(I \in \mathcal{O}\), we need to prove \(\exists J \subset I,J \in \mathcal{O}, J \cap \partial G = \varnothing\). 
  If \(I \cap \partial G = \varnothing\), we can let \(J = I\). 
  Else, assume \( a \in I \cap \partial G\). Form the defination of \(\partial G\), 
  we get \(\exists b \in I \cap G\). Let \(J = I \cap G \neq \varnothing\) is OK. 
  So \(B \Delta U\) is meager. 

  Assume \(A \in \fun{\mathbb{N}}{\mathcal{P}(\mathcal{A})}\), we need to prove \(\bigcup_{n \in \mathbb{N}} A(n)=:A \in \mathcal{A}\). 
  Assume \(G(n) \in \mathcal{O}\) and \(A(n) \Delta G(n)\) is meager. Consider \(G:= \bigcup_{ n \in \mathbb{N}} G(n)\). 
  We only need to prove \(G \Delta A\) is meager. Only need \(G \setminus A,A \setminus G\) is meager. 
  Since \(G \setminus A \subset \bigcup_{n \in \mathbb{N}} G(n) \setminus A(n)\) and \(G(n)\setminus A(n)\) is meager, we get \(G \setminus A\) is meager. 
  For the same reason, we get \(A \setminus G \subset \bigcup_{n \in \mathbb{N}} A(n)\setminus G(n)\) is meager. 

  So finally we get \(\mathcal{A}\) is \(\sigma\)-field. 
\end{solution}
\end{document}
