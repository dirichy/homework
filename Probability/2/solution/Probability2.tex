%!Mode:: "TeX:UTF-8"
%!TEX TS-program = xelatex
%!TEX language = zh
%arara: xelatex
\documentclass{ctexart}
\newif\ifpreface
\prefacetrue
\usepackage{fontspec}
\usepackage{bbm}
\usepackage{tikz}
\usepackage{amsmath,amssymb,amsthm,color,mathrsfs}
\usepackage{fixdif}
\usepackage{hyperref}
\usepackage{cleveref}
\usepackage{enumitem}%
\usepackage{expl3}
\usepackage{lipsum}
\usepackage[margin=0pt]{geometry}
\usepackage{listings}
\definecolor{mGreen}{rgb}{0,0.6,0}
\definecolor{mGray}{rgb}{0.5,0.5,0.5}
\definecolor{mPurple}{rgb}{0.58,0,0.82}
\definecolor{backgroundColour}{rgb}{0.95,0.95,0.92}

\lstdefinestyle{CStyle}{
  backgroundcolor=\color{backgroundColour},
  commentstyle=\color{mGreen},
  keywordstyle=\color{magenta},
  numberstyle=\tiny\color{mGray},
  stringstyle=\color{mPurple},
  basicstyle=\footnotesize,
  breakatwhitespace=false,
  breaklines=true,
  captionpos=b,
  keepspaces=true,
  numbers=left,
  numbersep=5pt,
  showspaces=false,
  showstringspaces=false,
  showtabs=false,
  tabsize=2,
  language=C
}
\usetikzlibrary{calc}
\theoremstyle{remark}
\newtheorem{lemma}{Lemma}
\usepackage{fontawesome5}
\usepackage{xcolor}
\newcounter{problem}
\newcommand{\Problem}{\begin{tikzpicture}[baseline]%
    \node at (-0.02em,0.3em) {$\mathbb{P}$};%
    \node[scale=0.7] at (0.2em,-0.0em) {R};%
    \node[scale=0.7] at (0.6em,0.4em) {O};%
    \node[scale=0.8] at (1.05em,0.25em) {B};%
    \node at (1.55em,0.3em) {L};%
    \node[scale=0.7] at (1.75em,0.45em) {E};%
    \node at (2.35em,0.3em) {M};%
  \end{tikzpicture}%
}
\renewcommand{\theproblem}{\Roman{problem}}
\newenvironment{problem}{\refstepcounter{problem}\noindent\color{blue}\Problem\theproblem}{}

\crefname{problem}{\protect\Problem}{Problem}
\newcommand\Solution{\begin{tikzpicture}[baseline]%
    \node at (-0.04em,0.3em) {$\mathbb{S}$};%
    \node[scale=0.7] at (0.35em,0.4em) {O};%
    \node at (0.7em,0.3em) {\textit{L}};%
    \node[scale=0.7] at (0.95em,0.4em) {U};%
    \node[scale=1.1] at (1.19em,0.32em){T};%
    \node[scale=0.85] at (1.4em,0.24em){I};%
    \node at (1.9em,0.32em){$\mathcal{O}$};%
    \node[scale=0.75] at (2.3em,0.21em){\texttt{N}};%
  \end{tikzpicture}}
\newenvironment{solution}{\begin{proof}[\Solution]}{\end{proof}}
\title{\input{../../.subject}\input{../.number}}
\makeatletter
\newcommand\email[1]{\def\@email{#1}\def\@refemail{mailto:#1}}
\newcommand\schoolid[1]{\def\@schoolid{#1}}
\ifpreface
  \def\@maketitle{
  \raggedright
  {\Huge \bfseries \sffamily \@title }\\[1cm]
  {\Huge  \bfseries \sffamily\heiti\@author}\\[1cm]
  {\Huge \@schoolid}\\[1cm]
  {\Huge\href\@refemail\@email}\\[0.5cm]
  \Huge\@date\\[1cm]}
\else
  \def\@maketitle{
    \raggedright
    \begin{center}
      {\Huge \bfseries \sffamily \@title }\\[4ex]
      {\Large  \@author}\\[4ex]
      {\large \@schoolid}\\[4ex]
      {\href\@refemail\@email}\\[4ex]
      \@date\\[8ex]
    \end{center}}
\fi
\makeatother
\ifpreface
  \usepackage[placement=bottom,scale=1,opacity=1]{background}
\fi

\author{白永乐}
\schoolid{25110180002}
\email{ylbai25@m.fudan.edu.cn}

\def\to{\rightarrow}
\newcommand{\xor}{\vee}
\newcommand{\AND}{\wedge}
\newcommand{\OR}{\vee}
\newcommand{\bor}{\bigvee}
\newcommand{\band}{\bigwedge}
\newcommand{\xand}{\wedge}
\newcommand{\minus}{\mathbin{\backslash}}
\newcommand{\mi}[1]{\mathscr{P}(#1)}
\newcommand{\card}{\mathrm{card}}
\newcommand{\oto}{\leftrightarrow}
\newcommand{\hin}{\hat{\in}}
\newcommand{\gl}{\mathrm{GL}}
\newcommand{\im}{\mathrm{Im}}
\newcommand{\re }{\mathrm{Re }}
\newcommand{\rank}{\mathrm{rank}}
\newcommand{\tra}{\mathop{\mathrm{tr}}}
\renewcommand{\char}{\mathop{\mathrm{char}}}
\DeclareMathOperator{\ot}{ordertype}
\DeclareMathOperator{\dom}{dom}
\DeclareMathOperator{\ran}{ran}

\begin{document}
\large
\setlength{\baselineskip}{1.2em}
\ifpreface
\input{../../../global/preface}
\else
\maketitle
\fi
\newtheorem*{remark}{remark}
\newgeometry{left=2cm,right=2cm,top=2cm,bottom=2cm}
%from_here_to_type

\begin{problem}\label{pro:1.4.21}
  设\(\mu^* \)是\(\mu \)生成的外测度,证明测度空间\((\Omega,\mathcal{A},\mu)\)是完全的当且仅当\(\mathcal{A} \supset \{A \in \Omega:\mu^*(A)=0\} \)。
\end{problem}
\begin{solution}
  先证充分性。设\(\mathcal{A} \supset \{A \in \Omega:\mu^*(A)=0\}\)。
  考察\(\mu \)-零测集\(A \),由定义知\(\exists B \in \mathcal{A},A \subset B,\mu(B)=0 \)。
  故\(\{B\} \)是\(A \)的一个覆盖,从而\(\mu^*(A) \leq \mu(B)=0 \),故\(\mu^*(A)=0 \),从而\(A \in \mathcal{A} \)。

  再证必要性。设\(A \)满足\(\mu^*(A)=0 \)。则由定义知\(\exists A_1,A_2,\cdots \in \mathcal{A},B \subset \bigcup_{n}A_n,\mu(A_n)=0 \)。
  由\(\mathcal{A} \)为\(\sigma \)-代数知\(\bigcup_{n}A_n \in \mathcal{A} \),故\(A \)也是\(\mu \)-零测集。
  由完全性的定义知\(A \in \mathcal{A} \)。
\end{solution}

\begin{problem}\label{pro:1.4.22}
  $\mathscr{S}$ 是半集代数,$\mu$ 是 $\mathscr{S}$ 上有限测度。记 $\left(\Omega, \mathscr{A}^*, \mu^*\right)$ 是 $\mu$ 扩张至 $\sigma(\mathscr{S})$的完全化,令
  $$
  \begin{aligned}
    \mu_*(A) & =\sup \left\{\sum_n \mu\left(A_n\right): A_n \in \mathscr{S} \text { 两两不交 },\sum_n A_n \subset A\right\}, \\
    \mathscr{A}_* & =\left\{A \subset \Omega: \mu^*(A)=\mu_*(A)\right\} .
  \end{aligned}
  $$
  试证: \(\mathscr{A}^* \supset \mathscr{A}_*\)
\end{problem}
\begin{solution}
  设\(A \in \mathcal{A}_{*} \),即\(\mu^{*}(A)=\mu_{*}(A) \),下证\(A \in \mathcal{A}^* \)。
  由外测度的定义,结合\(\mu \)有限,可得 \(\forall \varepsilon =\frac{1}{n} >0,\exists \mathcal{R} \subset \mathscr{S} \)  为\(A \)的可数覆盖,且\(\sum_{X \in \mathcal{R}}\mu(X) < \mu^*(A) + \varepsilon \)。
  令\(A_n := \bigcup \mathcal{R} \),则\(A_n \supset A \)且\(\mu(A_n)\searrow \mu^*(A) \)。
  同样地,可以找到\(B_n \in \sigma(\mathscr{S}) \),满足\(B_n \subset A \)且\(\mu(B_n) \nearrow \mu_*(A) \)。
  令\(O=\bigcap_{n}A_n \setminus \bigcup_{n} B_n \),则\(\forall n, \mu(O) \leq \mu(A_n \setminus B_n)=\mu(A_n)-\mu(B_n) \to 0 \),故\(\mu(O)=0 \)。
  又\(A \setminus \bigcup_{n}B_n \subset O\),从而\(A=\bigcup_{n}B_n \cup (A \setminus \bigcup_{n}B_n) \in \mathcal{A}^* \)。
\end{solution}
\begin{remark}
  反向的包含一般不成立。
  令\(\Omega=\mathbb{N},\mathscr{S}:=\{\{n\}:n \in \mathbb{N}_{+}\}\cup \{\{0\}\cup \{n,n+1,\cdots\}:n \in \mathbb{N}_{+}\} \),
  则易于验证\(\mathscr{S} \)是半集代数。令\(\mu:\mathscr{S} \to \mathbb{R},\mu(\{n\})=0,\mu(\{0\}\cup\{n,n+1,\cdots\})=1 ,\forall n \in \mathbb{N}_{+}\),
  则易于验证\(\mu \)是测度。考查\(\{0\} \),由\(\{0\} \in \sigma(\mathscr{S}) \)可知\(\{0\} \in \mathcal{A}^* \),但易知\(\mu^*(\{0\})=1,\mu_{*}(\{0\})=0 \)。
\end{remark}

\begin{problem}\label{pro:1.4.23}
  设 $(\Omega, \mathscr{A}, \mu)$ 为测度空间,$\mu^*$ 为由 $\mu$ 生成的外测度。证明 $N \subset \Omega$ 为 $\mu$零测集
  当且仅当 $\mu^*(N)=0$ .
\end{problem}

\begin{solution}
  一方面,设\(\mu^*(N)=0 \),则由\ref{pro:1.4.22}中证明可知\(\exists A_n \in \mathcal{A} ,N \subset A_n\)使\(\mu(A_n) \to 0 \)。
  故\(\mu(\bigcap_{n}A_n) \leq \mu(A_n) \to 0 \),从而\(N \subset \bigcap_{n}A_n,\mu(\bigcap_{n}A_n)=0 \)。
  故\(N \)是\(\mu \)零测集。

  另一方面,设\(N\)是\(\mu \)零测集,则\(\exists M \in \mathcal{A},N \subset M,\mu(M)=0 \)。
  从而\(\{M\} \)为\(N \)的可数覆盖,\(\mu^*(N) \leq \mu(M)=0 \)。
\end{solution}

\begin{problem}\label{pro:1.4.12}
  举例说明即使\(\Omega\)可数,半集代数\(\mathcal{T}\)生成的\(\sigma\)-代数不能表示为:
  \[
    \sigma(\mathcal{T})=\{\sum_{n=1}^{\infty} A_n:\forall n \geq 1, A_n \in \mathcal{T}\}
  \]
\end{problem}
\begin{solution}
  令\(\Omega = \mathbb{N} \),令\(\mathcal{T}:=\{A \subset \mathbb{N}:0 \in A \text{且\( A^c \) 有限或}0 \notin A \text{且\( A\) 有限} \} \)。
  先证\(\mathcal{T} \)是半集代数。只需证\(\mathcal{T} \)是集代数。
  \begin{enumerate}
    \item \( \Omega=\mathbb{N} \)满足\(0 \in \Omega,\Omega^c=\emptyset \) 有限,故\(\Omega \in \mathcal{T} \)。
    \item 设\(A,B \in \mathcal{T} \)。须证\(A \setminus B \in \mathcal{T} \)。若\(A \)有限且\(0 \notin A \),则有\(A \setminus B \)也有限且\(0 \notin A \setminus B \),故\(A \setminus B \in \mathcal{T} \)。
      若\(A^c \)有限且\(0 \in A \),\(B \)有限且\(0 \notin B \),则\(( A \setminus B  )^c = A^c \cup B\)也有限且\(0 \in A \setminus B \),故\(A \setminus B \in \mathcal{T} \)。
      若\(A^c,B^c \)有限且\(0 \in A,B \),则\(A \setminus B=A \cap B^c \)有限,且\(0 \notin A \setminus B \),故\(A \setminus B \in \mathcal{T} \)。
      综上,\(\forall A,B \in \mathcal{T},A \setminus B \in \mathcal{T} \)。
  \end{enumerate}
  令\(\mathcal{S}:= \{\sum_{n=1}^{\infty} A_n:\forall n \geq 1, A_n \in \mathcal{T}\} \),下证\(\mathcal{S} \neq \sigma(\mathcal{T}) \)。
  只需证\(\{0\} \in \sigma(\mathcal{T}),\{0\} \notin \mathcal{S} \)。
  由\(\{n\} \in \mathcal{T},\forall n \geq 1 \)可得\(\mathbb{N}_+ \in \sigma(\mathcal{T}) \),故\(\{0\}=\mathbb{N}_+^c \in \sigma(\mathcal{T}) \)。
  反设 \( \{0\} \in \mathcal{S} \),则\(\{0\}=\sum_{n}A_n,A_n \in \mathcal{T} \)。由\(\{0\} \)有限知\(A_n \)有限,由\(\mathcal{T} \)的定义知\(0 \notin A_n \),故\(0 \notin \sum_{n}A_n=\{0\} \),矛盾!
  故\(\{0\}\notin \mathcal{S} \)。从而 \( \mathcal{S} \neq \sigma(\mathcal{T}) \)。
\end{solution}

\begin{problem}\label{pro:2.5.2}
  \begin{enumerate}
    \item   设\(g \)是\((\overline{\mathbb{R}}^n,\overline{\mathcal{B}}^n) \)上的实(复)可测函数,\(f_1,\cdots,f_n \)是\((\Omega,\mathcal{A}) \)上的实可测函数。
      则\(g(f_1,\cdots,f_n) \)是\((\Omega,\mathcal{A}) \)上的实(复)可测函数。
    \item   设\(g \)是\((\overline{\mathbb{C}}^n,\overline{\mathcal{B}}^n_c) \)上的实(复)可测函数,\(f_1,\cdots,f_n \)是\((\Omega,\mathcal{A}) \)上的复可测函数。
      则\(g(f_1,\cdots,f_n) \)是\((\Omega,\mathcal{A}) \)上的实(复)可测函数。
  \end{enumerate}
\end{problem}
\begin{solution}
  \begin{enumerate}
    \item
      由定理2.6(2)可知\(F:=(f_1,\cdots,f_n) \)是{ \((\Omega,\mathcal{A}) \) }上的\(n \) 维{实(复)可测函数}。
      故由定理2.7可得\(g \circ F=g(f_1,\cdots,f_n) \)是\((\Omega,\mathcal{A}) \)上的{实(复)可测函数}。
    \item
      由定理2.6(2)可知\(F:=(f_1,\cdots,f_n) \)是{ \((\Omega,\mathcal{A}) \) }上的\(n \) 维{实(复)可测函数}。
      故由定理2.7可得\(g \circ F=g(f_1,\cdots,f_n) \)是\((\Omega,\mathcal{A}) \)上的{实(复)可测函数}。
  \end{enumerate}

\end{solution}

\begin{problem}\label{pro:2.5.3}
  \begin{enumerate}
    \item   设\(g \)是\((\overline{\mathbb{R}}^n,\overline{\mathcal{B}}^n) \)上的实(复)可测函数,\(f_1,\cdots,f_n \)是\((\Omega,\mathcal{A},\mathbb{P}) \)上的随机变量。
      且\(\mathbb{P}(|g(f_1,\cdots,f_n)|=\infty)=0 \)。 则\(g(f_1,\cdots,f_n) \)是\((\Omega,\mathcal{A},\mathbb{P} ) \)上的实(复)随机变量。
    \item   设\(g \)是\((\overline{\mathbb{C}}^n,\overline{\mathcal{B}}^n_c) \)上的实(复)可测函数,\(f_1,\cdots,f_n \)是\((\Omega,\mathcal{A},\mathbb{P}) \)上的复随机变量。
      且\(\mathbb{P}(|g(f_1,\cdots,f_n)|=\infty)=0 \)。 则\(g(f_1,\cdots,f_n) \)是\((\Omega,\mathcal{A},\mathbb{P} ) \)上的实(复)随机变量。
  \end{enumerate}
\end{problem}
\begin{solution}
  由\ref{pro:2.5.2}可知\(g(f_1,f_2,\cdots,f_n) \)是实(复)可测函数。
  又由\(\mathbb{P}(|g(f_1,\cdots,f_n)|=\infty)=0 \)知其几乎处处有限,故有\(g(f_1,\cdots,f_n) \overset{\text{a.s.}}{=} \mathbbm{1}_{|g(f_1,\cdots,f_n)|<\infty}g(f_1,\cdots,f_n) \)。
  而后者值域为\(\mathbb{R} \)(\(\mathbb{C} \)),故在几乎处处相等的意义下\(g(f_1,\cdots,f_n) \)是实(复)随机变量。
\end{solution}

\end{document}
