%!Mode:: "TeX:UTF-8"
%!TEX TS-program = xelatex
%arara: xelatex
\documentclass{ctexart}
\newif\ifpreface
\prefacetrue
\usepackage{fontspec}
\usepackage{bbm}
\usepackage{tikz}
\usepackage{amsmath,amssymb,amsthm,color,mathrsfs}
\usepackage{fixdif}
\usepackage{hyperref}
\usepackage{cleveref}
\usepackage{enumitem}%
\usepackage{expl3}
\usepackage{lipsum}
\usepackage[margin=0pt]{geometry}
\usepackage{listings}
\definecolor{mGreen}{rgb}{0,0.6,0}
\definecolor{mGray}{rgb}{0.5,0.5,0.5}
\definecolor{mPurple}{rgb}{0.58,0,0.82}
\definecolor{backgroundColour}{rgb}{0.95,0.95,0.92}

\lstdefinestyle{CStyle}{
  backgroundcolor=\color{backgroundColour},
  commentstyle=\color{mGreen},
  keywordstyle=\color{magenta},
  numberstyle=\tiny\color{mGray},
  stringstyle=\color{mPurple},
  basicstyle=\footnotesize,
  breakatwhitespace=false,
  breaklines=true,
  captionpos=b,
  keepspaces=true,
  numbers=left,
  numbersep=5pt,
  showspaces=false,
  showstringspaces=false,
  showtabs=false,
  tabsize=2,
  language=C
}
\usetikzlibrary{calc}
\theoremstyle{remark}
\newtheorem{lemma}{Lemma}
\usepackage{fontawesome5}
\usepackage{xcolor}
\newcounter{problem}
\newcommand{\Problem}{\begin{tikzpicture}[baseline]%
    \node at (-0.02em,0.3em) {$\mathbb{P}$};%
    \node[scale=0.7] at (0.2em,-0.0em) {R};%
    \node[scale=0.7] at (0.6em,0.4em) {O};%
    \node[scale=0.8] at (1.05em,0.25em) {B};%
    \node at (1.55em,0.3em) {L};%
    \node[scale=0.7] at (1.75em,0.45em) {E};%
    \node at (2.35em,0.3em) {M};%
  \end{tikzpicture}%
}
\renewcommand{\theproblem}{\Roman{problem}}
\newenvironment{problem}{\refstepcounter{problem}\noindent\color{blue}\Problem\theproblem}{}

\crefname{problem}{\protect\Problem}{Problem}
\newcommand\Solution{\begin{tikzpicture}[baseline]%
    \node at (-0.04em,0.3em) {$\mathbb{S}$};%
    \node[scale=0.7] at (0.35em,0.4em) {O};%
    \node at (0.7em,0.3em) {\textit{L}};%
    \node[scale=0.7] at (0.95em,0.4em) {U};%
    \node[scale=1.1] at (1.19em,0.32em){T};%
    \node[scale=0.85] at (1.4em,0.24em){I};%
    \node at (1.9em,0.32em){$\mathcal{O}$};%
    \node[scale=0.75] at (2.3em,0.21em){\texttt{N}};%
  \end{tikzpicture}}
\newenvironment{solution}{\begin{proof}[\Solution]}{\end{proof}}
\title{\input{../../.subject}\input{../.number}}
\makeatletter
\newcommand\email[1]{\def\@email{#1}\def\@refemail{mailto:#1}}
\newcommand\schoolid[1]{\def\@schoolid{#1}}
\ifpreface
  \def\@maketitle{
  \raggedright
  {\Huge \bfseries \sffamily \@title }\\[1cm]
  {\Huge  \bfseries \sffamily\heiti\@author}\\[1cm]
  {\Huge \@schoolid}\\[1cm]
  {\Huge\href\@refemail\@email}\\[0.5cm]
  \Huge\@date\\[1cm]}
\else
  \def\@maketitle{
    \raggedright
    \begin{center}
      {\Huge \bfseries \sffamily \@title }\\[4ex]
      {\Large  \@author}\\[4ex]
      {\large \@schoolid}\\[4ex]
      {\href\@refemail\@email}\\[4ex]
      \@date\\[8ex]
    \end{center}}
\fi
\makeatother
\ifpreface
  \usepackage[placement=bottom,scale=1,opacity=1]{background}
\fi

\author{白永乐}
\schoolid{25110180002}
\email{ylbai25@m.fudan.edu.cn}

\def\to{\rightarrow}
\newcommand{\xor}{\vee}
\newcommand{\AND}{\wedge}
\newcommand{\OR}{\vee}
\newcommand{\bor}{\bigvee}
\newcommand{\band}{\bigwedge}
\newcommand{\xand}{\wedge}
\newcommand{\minus}{\mathbin{\backslash}}
\newcommand{\mi}[1]{\mathscr{P}(#1)}
\newcommand{\card}{\mathrm{card}}
\newcommand{\oto}{\leftrightarrow}
\newcommand{\hin}{\hat{\in}}
\newcommand{\gl}{\mathrm{GL}}
\newcommand{\im}{\mathrm{Im}}
\newcommand{\re }{\mathrm{Re }}
\newcommand{\rank}{\mathrm{rank}}
\newcommand{\tra}{\mathop{\mathrm{tr}}}
\renewcommand{\char}{\mathop{\mathrm{char}}}
\DeclareMathOperator{\ot}{ordertype}
\DeclareMathOperator{\dom}{dom}
\DeclareMathOperator{\ran}{ran}

\begin{document}
\large
\setlength{\baselineskip}{1.2em}
\ifpreface
\input{../../../global/preface}
\else
\maketitle
\fi
\newgeometry{left=2cm,right=2cm,top=2cm,bottom=2cm}
%from_here_to_type
% p46 5, 7, 9, 11, 12
% p47 15,18,19,20,25
\begin{problem}\label{pro:5}
  分布函数是否是不降的?举出反例或者给出证明。
\end{problem}
\begin{solution}
  \(1 \)
  维情形由定义显然是不降的,高维情形未必,下面给出一个例子。
  令\(F(x,y)=\mathrm{e}^{x+y}-x-y \),则有\(\Delta_{a,b} F = F(b_1,b_2)-F(a_1,b_2)-F(b_1,a_2)+F(a_1,a_2)
    = \mathrm{e}^{b_1+b_2}-\mathrm{e}^{a_1+b_2}-\mathrm{e}^{b_1+a_2}+\mathrm{e}^{a_1+a_2}-b_1-b_2+a_1+b_2+b_1+a_2-a_1-a_2
    = (\mathrm{e}^{b_1}-\mathrm{e}^{a_1})(\mathrm{e}^{b_2}-\mathrm{e}^{a_2}) \geq 0
  \),但\(F(-1,-1)\geq 2 > 0 = F(0,0) \)。
\end{solution}

\begin{problem}\label{pro:7}
  证明若\(F(x)=\mathbb{P}(\xi < x) \)是连续的,则\(\eta = F(\xi) \)具有\((0,1) \)上的均匀分布。
\end{problem}

\begin{solution}
  只需证明\(\mathbb{P}(F(\xi)< x)=x,x \in (0,1) \)。
  令\(G(x):=\inf\{y \in \mathbb{R}:F(y)\geq x\},x \in [0,1] \)。
  考查事件\(\xi < G(x) \),由\(G(x) \)的定义知其等价于\(F(\xi)<x \),故\(\mathbb{P}(F(\xi)<x)=\mathbb{P}(\xi \leq G(x)) \)。
  又由\(F \)是连续的,得\(\mathbb{P}(\xi \leq G(x))=\mathbb{P}(\xi < G(x))=F(G(x)) \)。若\(G(x)=\pm \infty \),易于验证\(F(G(x))=x \)。
  下设\(G(x)\in \mathbb{R} \)。

  由\(G(x) \)的定义知\(\exists y_n \searrow G(x), F(y_n) \geq x \),结合\(F \)的连续性可知\(F(G(x)) \geq x\)。
  同样由\(G(x) \)的定义可知\(\forall y < G(x),F(y)<x \),令\(y \nearrow G(x) \),由\(F \)的连续性可知\(F(G(x)) \leq x \)。
  故\(F(G(x))=x \),从而\(\mathbb{P}(\xi \leq G(x))=x \),即\(F(\xi) \)是\((0,1) \)上的均匀分布。
\end{solution}

\begin{problem}\label{pro:9}
  设\(\xi_n,n \in \mathbb{N}_{+} \)为\(\mathrm{i.i.d.}  \)随机变量,分布为\(\mu \)。
  给定\(A \in \mathcal{B} \),\(\mu(A) >0 \),定义\(\tau = \inf\{k:\xi_k \in A\} \)。
  证明\(\xi_{\tau} \)的分布为\(\frac{\mu(\cdot \cap A)}{\mu(A)} \)。
\end{problem}

\begin{solution}
  \(\mathbb{P}(\xi_{\tau} \in B)=\sum_{n =1}^{\infty} \mathbb{P}(\tau=n,\xi_{\tau} \in B )= \sum_{n =1}^{\infty} \mathbb{P}(\tau=n,\xi_{n} \in B )\)。
  注意到\(\{\tau=n\}=\{\forall k <n,\xi_k \notin A\} \cap \{\xi_n \in A\} \),且\(\xi_n \)相互独立,故有
  \(\mathbb{P}(\tau=n,\xi_n \in B)=\mathbb{P}(\tau=n)\mathbb{P}(\xi_n \in B \mid \xi_n \in A)=\mathbb{P}(\tau=n) \frac{\mu(B \cap A)}{\mu(A)} \)。
  从而\(\mathbb{P}(\xi_{\tau} \in B)=\sum_{n = 1}^{\infty} \mathbb{P}(\tau=n) \frac{\mu(B \cap A)}{\mu(A)}=\frac{\mu(B \cap A)}{\mu(A)} \)。
\end{solution}

\begin{problem}\label{pro:11}
  若\(\mathcal{C}_1,\cdots,\mathcal{C}_n \)为独立的包含\(\Omega \) 的\(\pi \)-系,那么\(\sigma(\mathcal{C}_1),\cdots,\sigma(\mathcal{C}_n)\)独立。
\end{problem}
\begin{solution}
  先证\(\sigma(\mathcal{C}_1),\mathcal{C}_2,\cdots,\mathcal{C}_n \)独立。
  令\(\mathcal{A}:=\{A \in \sigma( \mathcal{C}_1 ): \forall C_k \in \mathcal{C}_k,k=2,\cdots,n,\mathbb{P}(A \cap \bigcap_{k=2}^{n}C_k)=\mathbb{P}(A)\prod_{k=2}^{n}\mathbb{P}(C_k)\} \),只需证\(\mathcal{A} = \sigma(\mathcal{C}_1) \)。
  显然\(\mathcal{C}_1 \subset \mathcal{A} \),只需证\(\mathcal{A} \)是\(\sigma \)-代数。由\(\mathcal{C}_1 \)是\(\pi \)-系,只需证\(\mathcal{A} \)是\(\lambda \)-系。
  \begin{itemize}
    \item 由定义知\(\Omega \in \mathcal{C}_1 \subset \mathcal{A} \)。
    \item 设\(A,B \in \mathcal{A} , B \subset A\),则有\(\mathbb{P}((A\setminus B) \cap \bigcap_{k=2}^{n}C_k) = \mathbb{P}(A \cap \bigcap_{k=2}^{n}C_k)-\mathbb{P}(B \cap \bigcap_{k=2}^{n}C_k)= ( \mathbb{P}(A) -\mathbb{P}(B))\prod_{k=2}^{n}\mathbb{P}(C_k)=( \mathbb{P}(A\setminus B))\prod_{k=2}^{n}\mathbb{P}(C_k)\)。
    \item 设\(A_k \in \mathcal{A},A_k \nearrow A \),则有\(\mathbb{P}(A \cap \bigcap_{k=2}^{n}C_k) = \lim_{i}\mathbb{P}(A_i \cap \bigcap_{k=2}^{n}C_k)= \lim_{i}\mathbb{P}(A_i) \prod_{k=2}^{n}\mathbb{P}(C_k)=\mathbb{P}(A)\prod_{k=2}^{n}\mathbb{P}(C_k)\)。
  \end{itemize}
  故\(\mathcal{A} \)是\(\lambda \)-系,从而\(\sigma(\mathcal{C}_1),\cdots,\mathcal{C}_n \)独立。

  显然\(\sigma(\mathcal{C}_k) \)也是包含\(\Omega \)的\(\pi \)-系,故可以重复上述过程,最后得到\(\sigma(\mathcal{C}_1),\cdots,\sigma(\mathcal{C}_n) \)独立。
\end{solution}

\begin{problem}\label{pro:12}
  \begin{enumerate}
    \item 设\(\{A_n\}_{n \geq 1} \) 为独立事件序列,令\(\mathcal{J} = \bigcap_{n=1}^{\infty} \sigma\{A_n,A_{n + 1},\cdots\}. \)
      证明\(\forall A \in \mathcal{J} \),有\(\mathbb{P} (A) =0 \)或\(1 \).
    \item 设 \(\{\xi_n\}_{n \geq 1} \)为独立随机变量,令\(\mathcal{J} = \bigcap_{n=1}^{\infty} \sigma\{\xi_n,\xi_{n + 1},\cdots\} \)。
      证明\(\forall A \in \mathcal{J} \),有\(\mathbb{P}(A)=0 \)或\(1 \).
  \end{enumerate}
\end{problem}
\begin{solution}
  \begin{enumerate}
    \item 可取\(\xi_n=\mathbbm{1}_{A_n} \),故只证\ref{it:1}。
    \item \label{it:1} 由\(A \in \mathcal{J} \)知\(A \in \sigma(\xi_n,\cdots) \),故\(A \)与\(\xi_1,\cdots,\xi_{n-1}  \)独立。
      故\(A \)与包含\(\Omega \)的 \(\pi \)-系\(\mathcal{C}_n:=\{ \bigcap_{k=1}^{n-1} \{\xi_k\in B_k\} :B_k \in \mathcal{B} \} \)是独立的。
      从而\(A \)与\(\bigcup_{n \geq 1}\mathcal{C}_n \)独立。故由\ref{pro:11} 知
      \(A \)与\(\sigma(\xi_1,\cdots)=\sigma(\bigcup_{n \geq 1}\mathcal{C}_n) \)独立。但\(A \in \sigma(\xi_1,\cdots) \),故\(A \)与\(A \)独立,从而\(\mathbb{P}(A)=\mathbb{P}(A \cap A)=\mathbb{P}(A)\mathbb{P}(A) \),
      故\(\mathbb{P}(A) \in \{0,1\} \)。
  \end{enumerate}
\end{solution}
% p47 15,18,19,20,25
\begin{problem}\label{pro:15}
  设\(\{\xi_1,\cdots\} \)是\(\mathrm{i.i.d.} \)的取值于\(\{1,\cdots,r\} \)的随机变量,且\(\mathbb{P}(\xi_i=k)=p(k)>0,\forall 1 \leq k \leq r \)。
  令\(\pi_n(\omega)=\prod_{k=1}^{n}p(\xi_k(\omega)) \),证明
  \(-n^{-1} \log \pi_n \overset{\mathbb{P}}{\to} H \overset{\Delta}{=}-\sum_{k=1}^{r}p(k)\log p(k) \)。
  这里\(H \)称为Shannon信息熵。
\end{problem}

\begin{problem}\label{pro:18}
  设\(\xi_n \)关于\(n \)单调上升,且\(\xi_n \to \xi \),求证\(\xi_n \overset{\mathrm{a.e.}}{\to} \xi \)。
\end{problem}

\begin{problem}\label{pro:19}
  \begin{enumerate}
    \item 设\(\xi_n \overset{\mathrm{a.e.}}{\to} \xi\),则
      \(S_n \overset{\Delta}{=} \frac{1}{n}\sum_{k = 1}^{n}\xi_k \overset{\mathrm{a.e.}}{\to} \xi \)。
    \item 若\(\xi_n \overset{\mathbb{P}}{\to} \xi \),则\(S_n \overset{\mathbb{P}}{\to} \xi \)是否成立?
  \end{enumerate}
\end{problem}

\begin{problem}\label{pro:20}
  若\(\Omega \)存在划分\(\{A_n\}_{n \geq 1} \)使\(\mathcal{A} = \sigma(\{A_n : 1 \leq n <\infty\}) \),
  则称\((\Omega,\mathcal{A},\mathbb{P}) \)为纯原子概率空间,每个非空的\(A_n \)称为一个原子。
  证明在纯原子概率空间上,随机变量序列依概率收敛等价于几乎处处收敛。
\end{problem}

\begin{problem}\label{pro:25}
  设随机变量\(\xi_n,\xi \)的分布函数分别为\(F_n,F \)。
  若\(\xi_n \overset{d}{\to} \xi \),则对\(F \)的任意连续点\(x \)有\(\mathbb{P}(\xi_n \leq x)\to \mathbb{P}(\xi \leq x),\mathbb{P}(\xi_n >x) \to \mathbb{P}(\xi >x) \)。
\end{problem}

\end{document}
