%arara: xelatex
%!Mode:: "TeX:UTF-8"
\documentclass{ctexart}
\newif\ifpreface
\prefacetrue
\usepackage{fontspec}
\usepackage{bbm}
\usepackage{tikz}
\usepackage{amsmath,amssymb,amsthm,color,mathrsfs}
\usepackage{fixdif}
\usepackage{hyperref}
\usepackage{cleveref}
\usepackage{enumitem}%
\usepackage{expl3}
\usepackage{lipsum}
\usepackage[margin=0pt]{geometry}
\usepackage{listings}
\definecolor{mGreen}{rgb}{0,0.6,0}
\definecolor{mGray}{rgb}{0.5,0.5,0.5}
\definecolor{mPurple}{rgb}{0.58,0,0.82}
\definecolor{backgroundColour}{rgb}{0.95,0.95,0.92}

\lstdefinestyle{CStyle}{
  backgroundcolor=\color{backgroundColour},
  commentstyle=\color{mGreen},
  keywordstyle=\color{magenta},
  numberstyle=\tiny\color{mGray},
  stringstyle=\color{mPurple},
  basicstyle=\footnotesize,
  breakatwhitespace=false,
  breaklines=true,
  captionpos=b,
  keepspaces=true,
  numbers=left,
  numbersep=5pt,
  showspaces=false,
  showstringspaces=false,
  showtabs=false,
  tabsize=2,
  language=C
}
\usetikzlibrary{calc}
\theoremstyle{remark}
\newtheorem{lemma}{Lemma}
\usepackage{fontawesome5}
\usepackage{xcolor}
\newcounter{problem}
\newcommand{\Problem}{\begin{tikzpicture}[baseline]%
    \node at (-0.02em,0.3em) {$\mathbb{P}$};%
    \node[scale=0.7] at (0.2em,-0.0em) {R};%
    \node[scale=0.7] at (0.6em,0.4em) {O};%
    \node[scale=0.8] at (1.05em,0.25em) {B};%
    \node at (1.55em,0.3em) {L};%
    \node[scale=0.7] at (1.75em,0.45em) {E};%
    \node at (2.35em,0.3em) {M};%
  \end{tikzpicture}%
}
\renewcommand{\theproblem}{\Roman{problem}}
\newenvironment{problem}{\refstepcounter{problem}\noindent\color{blue}\Problem\theproblem}{}

\crefname{problem}{\protect\Problem}{Problem}
\newcommand\Solution{\begin{tikzpicture}[baseline]%
    \node at (-0.04em,0.3em) {$\mathbb{S}$};%
    \node[scale=0.7] at (0.35em,0.4em) {O};%
    \node at (0.7em,0.3em) {\textit{L}};%
    \node[scale=0.7] at (0.95em,0.4em) {U};%
    \node[scale=1.1] at (1.19em,0.32em){T};%
    \node[scale=0.85] at (1.4em,0.24em){I};%
    \node at (1.9em,0.32em){$\mathcal{O}$};%
    \node[scale=0.75] at (2.3em,0.21em){\texttt{N}};%
  \end{tikzpicture}}
\newenvironment{solution}{\begin{proof}[\Solution]}{\end{proof}}
\title{\input{../../.subject}\input{../.number}}
\makeatletter
\newcommand\email[1]{\def\@email{#1}\def\@refemail{mailto:#1}}
\newcommand\schoolid[1]{\def\@schoolid{#1}}
\ifpreface
  \def\@maketitle{
  \raggedright
  {\Huge \bfseries \sffamily \@title }\\[1cm]
  {\Huge  \bfseries \sffamily\heiti\@author}\\[1cm]
  {\Huge \@schoolid}\\[1cm]
  {\Huge\href\@refemail\@email}\\[0.5cm]
  \Huge\@date\\[1cm]}
\else
  \def\@maketitle{
    \raggedright
    \begin{center}
      {\Huge \bfseries \sffamily \@title }\\[4ex]
      {\Large  \@author}\\[4ex]
      {\large \@schoolid}\\[4ex]
      {\href\@refemail\@email}\\[4ex]
      \@date\\[8ex]
    \end{center}}
\fi
\makeatother
\ifpreface
  \usepackage[placement=bottom,scale=1,opacity=1]{background}
\fi

\author{白永乐}
\schoolid{202011150087}
\email{202011150087@mail.bnu.edu.cn}

\def\to{\rightarrow}
\newcommand{\xor}{\vee}
\newcommand{\bor}{\bigvee}
\newcommand{\band}{\bigwedge}
\newcommand{\xand}{\wedge}
\newcommand{\minus}{\mathbin{\backslash}}
\newcommand{\mi}[1]{\mathscr{P}(#1)}
\newcommand{\card}{\mathrm{card}}
\newcommand{\oto}{\leftrightarrow}
\newcommand{\hin}{\hat{\in}}
\newcommand{\gl}{\mathrm{GL}}
\newcommand{\im}{\mathrm{Im}}
\newcommand{\re }{\mathrm{Re }}
\newcommand{\rank}{\mathrm{rank}}
\newcommand{\tra}{\mathop{\mathrm{tr}}}
\renewcommand{\char}{\mathop{\mathrm{char}}}
\DeclareMathOperator{\ot}{ordertype}
\DeclareMathOperator{\dom}{dom}
\DeclareMathOperator{\ran}{ran}

\begin{document}
\large
\setlength{\baselineskip}{1.2em}
\ifpreface
\backgroundsetup{contents={%
    \begin{tikzpicture}
      \fill [white] (current page.north west) rectangle ($(current page.north east)!.3!(current page.south east)$) coordinate (a);
      \fill [bgc] (current page.south west) rectangle (a);
\end{tikzpicture}}}
\definecolor{word}{rgb}{1,1,0}
\definecolor{bgc}{rgb}{1,0.95,0}
\setlength{\parindent}{0pt}
\thispagestyle{empty}
\begin{tikzpicture}%
  % \node[xscale=2,yscale=4] at (0cm,0cm) {\sffamily\bfseries \color{word} under};%
  \node[xscale=4.5,yscale=10] at (10cm,1cm) {\sffamily\bfseries \color{word} Graduate Homework};%
  \node[xscale=4.5,yscale=10] at (8cm,-2.5cm) {\sffamily\bfseries \color{word} In Mathematics};%
\end{tikzpicture}
\ \vspace{1cm}\\
\begin{minipage}{0.25\textwidth}
  \textcolor{bgc}{王胤雅是傻逼}
\end{minipage}
\begin{minipage}{0.75\textwidth}
  \maketitle
\end{minipage}
\vspace{4cm}\ \\
\begin{minipage}{0.2\textwidth}
  \
\end{minipage}
\begin{minipage}{0.8\textwidth}
  {\Huge
    \textinconsolatanf{}
  }General fire extinguisher
\end{minipage}
\newpage\backgroundsetup{contents={}}\setlength{\parindent}{2em}

\else
\maketitle
\fi
\newgeometry{left=2cm,right=2cm,top=2cm,bottom=2cm}
%from_here_to_type
%p22 2,3,4,6,9,11,12
\begin{problem}\label{pro:1.4.1}
  证明\(\sigma \)-代数是集代数。
\end{problem}
\begin{solution}
  设\(\mathcal{A} \)是\(\sigma \)-代数,则由定义知\(\Omega \in \mathcal{A} \)。
  下面证\( \forall A,B \in \mathcal{A},A \setminus B \in \mathcal{A} \)。
  易知\(A \setminus B = A \cap B^c = (A^c \cup B)^c \)。
  由\(A \in \mathcal{A} \)及\(\mathcal{A} \)是\(\sigma \)-代数知\(A^c \in \mathcal{A} \)。
  又\(A^c,B \in \mathcal{A} \),可取\(A_1=A^c,A_2=B,A_n=\emptyset,\forall n \geq 3 \),可知\(A^c \cup B = \bigcup_{n \geq 1}A_n \in \mathcal{A} \)。
  最后由\(A^c \cup B \in \mathcal{A} \)知\(A \setminus B = (A^c \cup B)^c  \in \mathcal{A}\)。

  综上知\(\mathcal{A} \)是集代数。
\end{solution}

\begin{problem}\label{pro:1.4.2}
  设\(\mathcal{C} \)是集类,
  则\(\forall A \in \sigma(\mathcal{C}),\exists \mathcal{C}_1 \subset \mathcal{C},|\mathcal{C}_1| \leq \aleph_0,A \in \sigma(\mathcal{C}_1) \)。
\end{problem}
\begin{solution}
  定义\(\mathcal{D}:=\{ A \in \sigma(\mathcal{C}):\exists \mathcal{C}_1 \subset \mathcal{C},|\mathcal{C}_1| \leq \aleph_0,A \in \sigma(\mathcal{C}_1) \} \)。
  对于\(A \in \mathcal{C} \),易知\(A \in \sigma\{A\} \),故\(A \in \mathcal{D} \)。
  从而\(\mathcal{C} \subset \mathcal{D} \)。又显然有\(\mathcal{D} \subset \sigma(\mathcal{C}) \),故要证\(\mathcal{D} =\sigma(\mathcal{C})\),只需证\(\mathcal{D} \)是\(\sigma \)-代数。

  \begin{enumerate}
    \item 由\(\Omega \in \{\emptyset,\Omega\} = \sigma(\emptyset) \)知\(\Omega \in \mathcal{D} \)。
    \item 对\(A \in \mathcal{D} \),取满足条件的\(\mathcal{C}_1 \),则由 \(A \in \sigma(\mathcal{C}_1) \) 知\(A^c \in \sigma(\mathcal{C}_1) \),故\(A^c \in \mathcal{D} \)。
    \item 对\(A_n \in \mathcal{D},n=1,2,\cdots \),分别取满足条件的\(\mathcal{C}_n \)。考查\(\mathcal{C}_0=\bigcup_{n=1}^{\infty} \mathcal{C}_n \)。
      由\(A_n \in \sigma(\mathcal{C}_n) \subset \sigma(\mathcal{C}_0) \)及\(\sigma(\mathcal{C}_0) \)为\(\sigma \)-代数可知\(\bigcup_{n=1}^{\infty} A_n \in \sigma(\mathcal{C}_0) \)。
      又由\(\mathcal{C}_n \)是可数的,得\(\mathcal{C}_0 \)是可数的。
      故\(\bigcup_{n=1}^{\infty} A_n \in \mathcal{D} \)。
  \end{enumerate}
  综上知\(\mathcal{D} \)为\(\sigma \)-代数,故\(\mathcal{D}=\sigma(\mathcal{C}) \),也即\(\forall A \in \sigma(\mathcal{C}),\exists \mathcal{C}_1 \subset \mathcal{C},|\mathcal{C}_1| \leq \aleph_0,A \in \sigma(\mathcal{C}_1) \)。
\end{solution}

\begin{problem}\label{pro:1.4.3}
  \(\sigma \)-代数\(\mathcal{A} \)称为可数生成的,如果存在可数的子集类\(\mathcal{C} \subset \mathcal{A} \)使\(\sigma(\mathcal{C})=\mathcal{A} \)。
  证明\(\mathcal{B}^d \)是可数生成的。
\end{problem}
\begin{solution}
  令\(\mathcal{C}:=\{B(x,r):x=(x_1,\cdots,x_d),x_i \in \mathbb{Q},i=1,\cdots,d, r \in \mathbb{Q}\} \)。易知\(\mathcal{C} \)可数。
  令\(\mathcal{O} \)为全体开集。易知\(\mathcal{C} \subset \mathcal{O} \),故\(\sigma(\mathcal{C}) \subset \mathcal{B}^d \)。
  故要证\(\sigma(\mathcal{C})=\mathcal{B}^d \),只需\(\sigma(\mathcal{C}) \supset \mathcal{B}^k \)。
  又由\(\sigma(\mathcal{C}) \)为\(\sigma \)-代数,且\(\mathcal{B}^k=\sigma(\mathcal{O}) \),故只需证\(\mathcal{O} \subset \sigma(\mathcal{C}) \)。
  对于\(O \in \mathcal{O}\),我们只需证\(O = \bigcup_{C \in \mathcal{C}:C \subset O} C \in \sigma(\mathcal{C}) \) (由\(\mathcal{C} \)可数知其为可数并)。
  显然有\( O \supset \bigcup_{C \in \mathcal{C}:C \subset O} C \),故只需证
  \( O \subset \bigcup_{C \in \mathcal{C}:C \subset O} C  \)。
  取\(x \in O \),由\(O \)为开集知\(\exists r >0,B(x,r) \subset O \)。
  设\(x=(x_1,\cdots,x_d) \),由有理数的稠密性知\(\exists y_i \in \mathbb{Q},|x_i-y_i| < \frac{r}{100d} \)。
  再取\(s \in \mathbb{Q} \)使\(\frac{r}{3} < s < \frac{r}{2} \)。
  则易知\(B(y,s) \subset B(x,r) \subset O \),且\(x \in B(y,s) \in \mathcal{C} \)。
  故\(x \in \bigcup_{C \in \mathcal{C}:C \subset O} C \)。

  综上 \(\mathcal{B}^k=\sigma(\mathcal{C}) \) 是可数生成的。
\end{solution}
\begin{problem}\label{pro:1.4.4}
  设\(\mathcal{C}_n \)是单调上升的集类。
  \begin{enumerate}
    \item
      设\(\mathcal{C}_n \)是集代数,则\(\bigcup_{n=1}^{\infty} \mathcal{C}_n\)是集代数
    \item 设\(\mathcal{C}_n \)是\(\sigma \)-代数,则\(\bigcup_{n=1}^{\infty}\mathcal{C}_n \)未必是\(\sigma \)-代数。
  \end{enumerate}
\end{problem}
\begin{solution}
  \begin{enumerate}
    \item
      \begin{enumerate}
        \item 显然\(\Omega \in \mathcal{C}_n \subset \mathcal{C} \)。
        \item 设\(A,B \in \mathcal{C} \),则\(\exists n,m \)使\(A \in \mathcal{C}_n,B \in \mathcal{C}_m \)。
          取\(k >n,m \),则\(A,B \in \mathcal{C}_k \)。故\(A \setminus B \in \mathcal{C}_k \subset \mathcal{C} \)。
      \end{enumerate}
      故\(\mathcal{C} \)是集代数。
    \item 令\(\Omega=\mathbb{N} \),令\(\mathcal{C}_n \)是由\(\{\{1\},\{2\},\cdots,\{n\}\} \)生成的\(\sigma \)代数。
      考察\(\bigcup_{n=1}^{\infty}\mathcal{C}_n =\{A:A \text{有限} \vee A^c \text{有限}\}\),易知\(\forall n,\{2n\} \in \bigcup_{n=1}^{\infty}\mathcal{C}_n \),
      但\(2 \mathbb{N} \notin \bigcup_{n=1}^{\infty}\mathcal{C}_n \)。故其不是\(\sigma \)-代数。
  \end{enumerate}
\end{solution}

\begin{problem}\label{pro:1.4.6}
  证明\(\sigma \)代数不能与\(\mathbb{N} \)等势。
\end{problem}
\begin{solution}
  设\(\mathcal{F} \)是一个\(\sigma \)-代数。考察其原子集族\(\mathcal{A} \)。
  若\(\mathcal{A} \)是\(\Omega \)的覆盖,则考察\(|\mathcal{A}| \)。若\(\mathcal{A} \)有限,则易知\(|\mathcal{F}|=2^{|\mathcal{A}|} \)也有限。
  若\(\mathcal{A} \)无限,则取一个\(\mathcal{A} \)的可数子集\(\mathcal{A}_1 \),加上\(\Omega \setminus \bigcup \mathcal{A}_1\) 构成一个\(\Omega \)的可数分割,记为\(\mathcal{B} \)。
  故\(|\mathcal{F}|\geq |\sigma(\mathcal{B})|=2^{|\mathcal{B}|}=2^{\aleph_0}>\aleph_0 \)。

  若\(\mathcal{A} \)不是\(\Omega \)的覆盖,则\(\mathcal{F} \)中一定有无穷递降列\(F_1 \supset F_2 \supset \cdots \)。
  令\(A_n :=F_n \setminus F_{n+1} \),则\(A_n \in \mathcal{F} \)两两不交。取\(\mathcal{B}:=\{A_n:n \in \mathbb{N}_{+}\} \cup \{\Omega \setminus \bigcup_{n=1}^{\infty}A_n\} \),
  则由前面的讨论知\(\mathcal{F} \)不可数。
\end{solution}

\begin{problem}\label{pro:1.4.a}
  设\(\mathcal{C} \)是\(\Omega \)中任一集代数,则存在\(\Omega \)中的单调类\(\mathcal{M}_0 \)满足:
  \begin{enumerate}
    \item \label{temp:1} \(\mathcal{C} \subset \mathcal{M}_0 \),
    \item \label{temp:2}对于包含\(\mathcal{C} \)的单调类\(\mathcal{M} \),有\(\mathcal{M}_0 \subset \mathcal{M} \)。
  \end{enumerate}
  称这样的单调类为\(\mathcal{C} \)生成的单调类,记作\(\mathcal{M}(\mathcal{A}) \)。
\end{problem}
\begin{solution}
  令
  \begin{equation}\label{equ:1}
    \mathcal{M}_0:=\bigcap_{\mathcal{M}:\mathcal{C} \subset \mathcal{M}, \mathcal{M} \text{为\(\Omega \)上的单调类}}\mathcal{M}
  \end{equation}
  。
  显然\(2^{\Omega} \)是包含\(\mathcal{C} \)的单调类,故\(\mathcal{M}_0 \)良定义。
  由\(\mathcal{M}_0 \)的定义易知\ref{temp:1},\ref{temp:2} 是成立的。故只需证\(\mathcal{M}_0 \)是单调类。

  对\(A_1,\cdots \in \mathcal{M}_0, A_1 \subset A_2 \subset \cdots\),我们要证\(A:=\bigcup_{n=1}^{\infty} A_n \in \mathcal{M}_0 \)。
  对于\eqref{equ:1}中的\(\mathcal{M} \),由\(\mathcal{M}_0 \)的定义知\(\mathcal{M}_0 \subset \mathcal{M} \),故\(A_n \in \mathcal{M} \)。
  又由\(\mathcal{M} \)为单调类,故\(A \in \mathcal{M} \)。
  由\(\mathcal{M} \)的任意性可知\(A \in \mathcal{M}_0 \)。
  对单调下降集列同理可得。故\(\mathcal{M}_0 \)是单调类,从而\(\mathcal{M}(\mathcal{A}) \)是良定义的。
\end{solution}

\begin{problem}\label{pro:1.4.b}
  设\(\Omega_i ,i=1,2,\cdots,n\)是\(n \)个集合,\(\mathcal{A}_i \)是\(\Omega_i \)上的\(\sigma \)-代数。
  证明\(\mathcal{C}=\{A_1 \times \cdots \times A_n:A_i \in \mathcal{A}_i\} \)为半集代数。
\end{problem}
\begin{solution}
  事实上只需\(\mathcal{A}_i \)是半集代数就够了。
  \begin{enumerate}
    \item 显然\(\Omega=\prod_{i=1}^{n} \Omega_i \in \mathcal{C} \),且\(\emptyset=\prod_{i=1}^{n} \emptyset  \in \mathcal{C}\)。
    \item 设\(A=\prod_{i}A_i,B=\prod_{i}B_i \in \mathcal{C} \),则\(A \cap B=\prod_{i}A_i \cap B_i \in \mathcal{C} \)。
    \item 设\(A=\prod_{i}A_i,B_0=\prod_{i}B_{0i} \in \mathcal{C} \),且\(B \subset A \)。
      则有\(\forall i,B_{0i} \subset A_i \)。由\(\mathcal{A}_i \)为半集代数可知\(\exists B_{1i},\cdots,B_{n_ii} \in \mathcal{A}_i,A_i=\sum_{j=1}^{n_i} B_{ji} \)。
      于是\(A=\sum_{i_1,i_2,\cdots,i_n:0 \leq i_n \leq n_i} \prod_{j=1}^{n} B_{i_j} \),\(i_1=i_2=\cdots=i_n=0 \)时取到\(B \)。
  \end{enumerate}
  故\(\mathcal{C} \)为\(\prod_{i}\Omega_i \)上的半集代数。
\end{solution}

\begin{problem}\label{pro:1.4.9}
  举例说明可加测度未必有限可加。
\end{problem}
\begin{solution}
  令\(\Omega=\{1,2,3\},\mathcal{C}=\{\{1\},\{2\},\{3\},\{1,2,3\}\} \)。
  令\(\mu:\mathcal{C} \to \mathbb{R},\mu(\{1\})=\mu(\{2\})=\mu(\{3\})=\mu(\Omega)=1 \)。
  由于\(\mathcal{C} \)中任两个集都相交,故可加性显然满足。但\(\mu(\Omega)=1 \neq 3 = \mu(\{1\})+\mu(\{2\})+\mu(\{3\}) \)。
\end{solution}

\begin{problem}
  设\((\Omega_n,\mathcal{A}_n,\mu_{n}),n \geq 1 \)为一列测度空间,\(\Omega_n \)两两不交。令 \[
    \Omega=\sum_{n=1}^{\infty} \Omega_n, \mathcal{A}=\{A \subset \Omega : \forall n \geq 1,A \cap \Omega_n \in \mathcal{A}_n\}, \mu (A)=\sum_{n=1}^{\infty} \mu_{n} (A \cap \Omega_n), A \in \mathcal{A}
  \]
  证明\((\Omega , \mathcal{A} , \mu ) \)为测度空间。
\end{problem}
\begin{solution}
  先证\(\mathcal{A} \)为\(\sigma \)-代数。
  \begin{enumerate}
    \item \(\Omega = \bigcup_{n=1}^{\infty}\Omega_n \in \mathcal{A} \)。
    \item 设\(A_n \in \mathcal{A} \),则\(( \bigcup_{n}A_n ) \cap \Omega_m = \bigcup_{n}(A_n \cap \Omega_m) \in \mathcal{A}_m\),故\(\bigcup_{n}A_n \in \mathcal{A} \)。
    \item 设\(A \in \mathcal{A} \),则\(A^c \cap \Omega_m = \Omega_m \setminus (A \cap \Omega_m) \in \mathcal{A}_m \),故\(A^c \in \mathcal{A} \)。
  \end{enumerate}
  再证\(\mu \)为测度。显然\(\mu(A) \geq 0 \)。
  设\(A_n \in \mathcal{A} \)两两不交,则
  \(\mu(\bigcup_{n}A_n) = \sum_{k=1}^{\infty}\mu_k(\bigcup_{n}A_n \cap \Omega_k) = \sum_{k=1}^{\infty}\sum_{m=1}^{\infty}\mu_k(A_m \cap \Omega_k) = \sum_{m=1}^{\infty}\mu(A_m) \)。
  故\(\mu \)是测度。
\end{solution}

\begin{problem}
  设\(\Omega \)为一无穷集,令\(\mathcal{F} \)为\(\Omega \)中的有限集或者余有限集构成的集合,\(\mathbb{P} \)在次两类集合上取值
  分别为\(0 \)或\(1 \)。
  \begin{itemize}
    \item 证明\(\mathcal{F} \)为集代数,\(\mathbb{P} \)为有限可加。
    \item 若\(\Omega \)为可数集,则\(\mathbb{P} \)不可能为\(\sigma \)可加。
    \item 若\(\Omega \)为不可数集,则\(\mathbb{P} \)为可数可加。
  \end{itemize}

\end{problem}
\begin{solution}

\end{solution}

\begin{problem}\label{pro:1.4.12}
  举例说明半集代数\(\mathcal{T}\)生成的\(\sigma\)-代数不能一般性地表述为
  \[
    \sigma(\mathcal{T})=\{\sum_{n=1}^{\infty} A_n:\forall n \geq 1, A_n \in \mathcal{T}\}
  \]
  但如果\(\Omega\)至多可数时,如上表述是正确的。
\end{problem}
\begin{solution}
  % 令\(\Omega=\mathbb{R} \),令\(\mathcal{T}:=\{A \subset \mathbb{R}:|A| < \aleph_0 \text{或} |A^c| < \aleph_0 \} \)。
  % 易于验证\(\mathcal{T} \)是半集代数(事实上它是集代数)。令\(\mathcal{S}:= \{\sum_{n=1}^{\infty} A_n:\forall n \geq 1, A_n \in \mathcal{T}\} \),
  % 下证\(\mathcal{S} \neq \sigma(\mathcal{T}) \)。事实上可以证明\(\mathbb{N}^c \notin \mathcal{S},\mathbb{N}^c \in \sigma( \mathcal{T} ) \)。
  % 首先由\(\{n\} \in \mathcal{T} \)知\(\mathbb{N}=\bigcup_{n \in \mathbb{N}}\{n\} \in \sigma(\mathcal{T}) \),故\(\mathbb{N}^c \in \sigma ( \mathcal{T} ) \)。
  % 若\(\mathbb{N}^c = \sum_{n=1}^{\infty} A_n,A_n \in \mathcal{T} \),则由\(\mathbb{N}^c \)不可数知\(\exists n, A_n \)不可数。
  % 由\(\mathcal{T} \)的定义知\(A_n^c \)是有限的。但\(\mathbb{N}=(\mathbb{N}^c)^c \)是无限集,矛盾!
  % 故\(\mathbb{N}^c \notin \mathcal{S} \)。
  %
  令\(\Omega = \mathbb{N} \),令\(\mathcal{T}:=\{A \subset \mathbb{N}:0 \in A \text{且\( A^c \) 有限或}0 \notin A \text{且\( A\) 有限} \} \)。
  先证\(\mathcal{T} \)是半集代数。只需证\(\mathcal{T} \)是集代数。
  \begin{enumerate}
    \item \( \Omega=\mathbb{N} \)满足\(0 \in \Omega,\Omega^c=\emptyset \) 有限,故\(\Omega \in \mathcal{T} \)。
      % 同理\(\emptyset \in \mathcal{T} \)。
      % \item 设\(A,B \in \mathcal{T} \),需证\(A \cap B \in \mathcal{T} \)。若\(A,B \)中有一个有限且不含\(0 \),则\(A \cap B \)也有限且不含\(0 \)。
      %   若\(A^c,B^c \)均有限且\(A,B \)均含\(0 \),则\((A \cap B)^c = A^c \cup B^c \)也有限,\(A \cap B \)也含\(0 \)。
      %   故无论如何都有\(A \cap B \in \mathcal{T} \)。
    \item 设\(A,B \in \mathcal{T} \)。须证\(A \setminus B \in \mathcal{T} \)。若\(A \)有限且\(0 \notin A \),则有\(A \setminus B \)也有限且\(0 \notin A \setminus B \),故\(A \setminus B \in \mathcal{T} \)。
      若\(A^c \)有限且\(0 \in A \),\(B \)有限且\(0 \notin B \),则\(( A \setminus B  )^c = A^c \cup B\)也有限且\(0 \in A \setminus B \),故\(A \setminus B \in \mathcal{T} \)。
      若\(A^c,B^c \)有限且\(0 \in A,B \),则\(A \setminus B=A \cap B^c \)有限,且\(0 \notin A \setminus B \),故\(A \setminus B \in \mathcal{T} \)。
      综上,\(\forall A,B \in \mathcal{T},A \setminus B \in \mathcal{T} \)。
  \end{enumerate}
  令\(\mathcal{S}:= \{\sum_{n=1}^{\infty} A_n:\forall n \geq 1, A_n \in \mathcal{T}\} \),下证\(\mathcal{S} \neq \sigma(\mathcal{T}) \)。
  只需证\(\{0\} \in \sigma(\mathcal{T}),\{0\} \notin \mathcal{S} \)。
  由\(\{n\} \in \mathcal{T},\forall n \geq 1 \)可得\(\mathbb{N}_+ \in \sigma(\mathcal{T}) \),故\(\{0\}=\mathbb{N}_+^c \in \sigma(\mathcal{T}) \)。
  反设 \( \{0\} \in \mathcal{S} \),则\(\{0\}=\sum_{n}A_n,A_n \in \mathcal{T} \)。由\(\{0\} \)有限知\(A_n \)有限,由\(\mathcal{T} \)的定义知\(0 \notin A_n \),故\(0 \notin \sum_{n}A_n=\{0\} \),矛盾!
  故\(\{0\}\notin \mathcal{S} \)。从而 \( \mathcal{S} \neq \sigma(\mathcal{T}) \)。

  % 下面证明对可数的\(\Omega \)上面的表述是正确的。
  % 易知\(\mathrm{L.H.S} \supset \mathrm{R.H.S} \supset \mathcal{T} \),故只需证\(\mathrm{R.H.S} \)是\(\sigma \)-代数。
  % 方便起见同样将其记为\(\mathcal{S} \)。
  % \begin{enumerate}
  %   \item 显然\(\Omega \in \mathcal{T} \subset \mathcal{S} \)。
  %   \item 设\(A_n \in \mathcal{S} ,n=1,2,\cdots \),则由\(\mathcal{S} \)的定义可知\(A_n= \sum_{i} A_{i n},A_{i n} \in \mathcal{T}\)。
  %     故\(\bigcup_{n}A_n = \bigcup_{i,n}A_{i n} \)为可数并,方便起见将其记作\(\bigcup_{n}A_n=\bigcup_{n}B_n,B_n \in \mathcal{T} \)。
  %     令\(C_1=B_1,C_{n+1}=B_{n+1} \setminus B_n \)。则\(\bigcup_{n}A_n=\sum_{n}C_n \)。
  %     令\(D_{1,1}=C_1,k_1=1 \)。对\(n+1>1 \),由\(\mathcal{T} \)为半集代数知\(B_{n+1} \cap B_n \in \mathcal{T} \)。
  %     又\(B_{n + 1}\cap B_n \subset B_{n + 1} \),由半集代数的性质可知\(\exists D_{n + 1,1},D_{n + 1,2},\cdots,D_{n + 1,k_{n + 1}} \in \mathcal{T},B_{n + 1}=\sum_{i=0}^{k_{n + 1}} D_{n+1,i} \),其中\(D_{n+1,0}=B_{n+1} \cap B_{n} \)。
  %     故\(C_{n+1}=\sum_{i=1}^{k_{n + 1}} D_{n+1,i} \)。
  %     故\(\bigcup_{n}A_n = \sum_{n=1}^{\infty} \sum_{i=1}^{k_{n}} D_{n+1,i} \in \mathcal{S} \)。
  % \end{enumerate}

\end{solution}

\end{document}
