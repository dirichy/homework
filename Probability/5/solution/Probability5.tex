%!Mode:: "TeX:UTF-8"
%!TEX TS-program = xelatex
%arara: xelatex
\documentclass{ctexart}
\newif\ifpreface
\prefacefalse
\usepackage{fontspec}
\usepackage{bbm}
\usepackage{tikz}
\usepackage{amsmath,amssymb,amsthm,color,mathrsfs}
\usepackage{fixdif}
\usepackage{hyperref}
\usepackage{cleveref}
\usepackage{enumitem}%
\usepackage{expl3}
\usepackage{lipsum}
\usepackage[margin=0pt]{geometry}
\usepackage{listings}
\definecolor{mGreen}{rgb}{0,0.6,0}
\definecolor{mGray}{rgb}{0.5,0.5,0.5}
\definecolor{mPurple}{rgb}{0.58,0,0.82}
\definecolor{backgroundColour}{rgb}{0.95,0.95,0.92}

\lstdefinestyle{CStyle}{
  backgroundcolor=\color{backgroundColour},
  commentstyle=\color{mGreen},
  keywordstyle=\color{magenta},
  numberstyle=\tiny\color{mGray},
  stringstyle=\color{mPurple},
  basicstyle=\footnotesize,
  breakatwhitespace=false,
  breaklines=true,
  captionpos=b,
  keepspaces=true,
  numbers=left,
  numbersep=5pt,
  showspaces=false,
  showstringspaces=false,
  showtabs=false,
  tabsize=2,
  language=C
}
\usetikzlibrary{calc}
\theoremstyle{remark}
\newtheorem{lemma}{Lemma}
\usepackage{fontawesome5}
\usepackage{xcolor}
\newcounter{problem}
\newcommand{\Problem}{\begin{tikzpicture}[baseline]%
    \node at (-0.02em,0.3em) {$\mathbb{P}$};%
    \node[scale=0.7] at (0.2em,-0.0em) {R};%
    \node[scale=0.7] at (0.6em,0.4em) {O};%
    \node[scale=0.8] at (1.05em,0.25em) {B};%
    \node at (1.55em,0.3em) {L};%
    \node[scale=0.7] at (1.75em,0.45em) {E};%
    \node at (2.35em,0.3em) {M};%
  \end{tikzpicture}%
}
\renewcommand{\theproblem}{\Roman{problem}}
\newenvironment{problem}{\refstepcounter{problem}\noindent\color{blue}\Problem\theproblem}{}

\crefname{problem}{\protect\Problem}{Problem}
\newcommand\Solution{\begin{tikzpicture}[baseline]%
    \node at (-0.04em,0.3em) {$\mathbb{S}$};%
    \node[scale=0.7] at (0.35em,0.4em) {O};%
    \node at (0.7em,0.3em) {\textit{L}};%
    \node[scale=0.7] at (0.95em,0.4em) {U};%
    \node[scale=1.1] at (1.19em,0.32em){T};%
    \node[scale=0.85] at (1.4em,0.24em){I};%
    \node at (1.9em,0.32em){$\mathcal{O}$};%
    \node[scale=0.75] at (2.3em,0.21em){\texttt{N}};%
  \end{tikzpicture}}
\newenvironment{solution}{\begin{proof}[\Solution]}{\end{proof}}
\title{\input{../../.subject}\input{../.number}}
\makeatletter
\newcommand\email[1]{\def\@email{#1}\def\@refemail{mailto:#1}}
\newcommand\schoolid[1]{\def\@schoolid{#1}}
\ifpreface
  \def\@maketitle{
  \raggedright
  {\Huge \bfseries \sffamily \@title }\\[1cm]
  {\Huge  \bfseries \sffamily\heiti\@author}\\[1cm]
  {\Huge \@schoolid}\\[1cm]
  {\Huge\href\@refemail\@email}\\[0.5cm]
  \Huge\@date\\[1cm]}
\else
  \def\@maketitle{
    \raggedright
    \begin{center}
      {\Huge \bfseries \sffamily \@title }\\[4ex]
      {\Large  \@author}\\[4ex]
      {\large \@schoolid}\\[4ex]
      {\href\@refemail\@email}\\[4ex]
      \@date\\[8ex]
    \end{center}}
\fi
\makeatother
\ifpreface
  \usepackage[placement=bottom,scale=1,opacity=1]{background}
\fi

\author{白永乐}
\schoolid{25110180002}
\email{ylbai25@m.fudan.edu.cn}

\def\to{\rightarrow}
\newcommand{\xor}{\vee}
\newcommand{\AND}{\wedge}
\newcommand{\OR}{\vee}
\newcommand{\bor}{\bigvee}
\newcommand{\band}{\bigwedge}
\newcommand{\xand}{\wedge}
\newcommand{\minus}{\mathbin{\backslash}}
\newcommand{\mi}[1]{\mathscr{P}(#1)}
\newcommand{\card}{\mathrm{card}}
\newcommand{\oto}{\leftrightarrow}
\newcommand{\hin}{\hat{\in}}
\newcommand{\gl}{\mathrm{GL}}
\newcommand{\im}{\mathrm{Im}}
\newcommand{\re }{\mathrm{Re }}
\newcommand{\rank}{\mathrm{rank}}
\newcommand{\tra}{\mathop{\mathrm{tr}}}
\renewcommand{\char}{\mathop{\mathrm{char}}}
\DeclareMathOperator{\ot}{ordertype}
\DeclareMathOperator{\dom}{dom}
\DeclareMathOperator{\ran}{ran}

\begin{document}
\large
\setlength{\baselineskip}{1.2em}
\ifpreface
\input{../../../global/preface}
\newgeometry{left=2cm,right=2cm,top=2cm,bottom=2cm}
\else
\newgeometry{left=2cm,right=2cm,top=2cm,bottom=2cm}
\maketitle
\fi
%from_here_to_type
%p75: 38,39,40
\begin{problem}\label{pro:3.38}
  (原题目定义不清,此处换用更清楚的定义)
  设\(C \)为 Cantor 集,对每个\(x \in (0,1) \),定义\(x_n \)为\(x \)的三进制表示,即\(x = \sum_{n=0}^{\infty}\frac{x_n}{3^n} \)。
  定义\(f(x):= \inf_{x_n = 1}n \)。
  % 对于有多种三进制表示的\(x \),我们总取使\(f(x) \)最小的一种表示。
  定义:
  \begin{equation}\label{equ:1}
    F(x) =
    \begin{cases}
      0 & x \leq 0\\
      1 & x \geq 1\\
      \sum_{n=0}^{f(x)}\frac{\mathbbm{1}_{x_n>0}}{2^{n}} & x \in (0,1)
    \end{cases}
  \end{equation}
  此\(F \)称为 Cantor 集上的均匀分布。证明:
  \begin{enumerate}
    \item \(F \)是连续的;
    \item \(F \)是 Lebesgue 奇异的。
  \end{enumerate}
\end{problem}
\begin{solution}
  先证\(F \)是良定义的,即对不同的三进制表示取值相同。
  设\(x = \sum_{n=0}^{\infty}\frac{x_n}{3^n} = \sum_{n=0}^{\infty}\frac{y_n}{3^n} \)是\(x \)的两种三进制表示,
  设\(\exists N,\forall n >N,x_n=0,y_n=2 \)。令\(m = \inf\{N:\forall n > N,y_n=2\} \),则\(x_m=y_m+1,\forall n>m, x_n=0,y_n=2; \forall n<m,x_n=y_n  \)。
  若\(\exists n<m,x_n = 1 \),则用\(x_n \)和\(y_n \)得到的\(F(x) \)显然由定义是相等的。现在假设\(\forall n<m,x_n \neq 1 \)。

  若\(y_m=0,x_m=1 \),则两种表示得到的值分别为\(\sum_{n=0}^{\infty}\frac{\mathbbm{1}_{y_n>0}}{2^n} \)和\(\sum_{n=0}^{m}\frac{\mathbbm{1}_{x_n>0}}{2^n} \),
  差为\(\sum_{n=m+1}^{\infty} \frac{1}{2^n}-\frac{1}{2^m}=0 \)。

  若\(y_m=1,x_m=2 \),则两种表示得到的值分别为\(\sum_{n=0}^{m}\frac{\mathbbm{1}_{y_n>0}}{2^n} \)和\(\sum_{n=0}^{\infty}\frac{\mathbbm{1}_{x_n>0}}{2^n} \),
  由\(\forall n>m,x_n=0 \)立得它们相等。

  综上,\(F \)是良定义的。
  \begin{enumerate}
    \item 先证\(F \)单调不减。设\(0<x<y<1 \),取\(x,y \)的三进制表示\(x_n,y_n \),则\(\exists m,\forall n<m,x_n=y_n, x_m<y_m \),于是\(y_m>0 \)。
      若\(\exists n<m,x_n=y_n=1 \),则\[F(x) = F(y) \]。现设\(\forall n<m,x_n \neq 1 \)。
      若\(x_m=0 \),则\(F(y)-F(x) \geq \frac{1}{2^m}-\sum_{n=m+1}^{\infty} \frac{1}{2^n}=0 \)。
      若\(x_m >0 \),则\(x_m=1,y_m=2 \),于是\(f(x)=m \),故\(F(y) \geq F(x) \)明显成立。
      于是\(F \)在\((0,1) \)上单调不减。又由\(F \)的定义明显看出\(F \)在\(\mathbb{R} \)上是单调的。

      再证\(F \)的值域是\([0,1] \)。显然\(F(\mathbb{R}) \subset [0,1] \),故只需\(\forall y \in [0,1],\exists x \in \mathbb{R},F(x)=y \)。
      对\(y=0 \)和\(y=1 \)显然成立,下设\(y \in (0,1) \)。取\(y \)的二进制表示\(y = \sum_{n=0}^{\infty}\frac{y_n}{2^n} \)。
      令\(x:=\sum_{n=0}^{\infty}\frac{2y_n}{3^n} \),则显然有\(F(x)=y \)。

      故\(F \)是单调的值域连通的函数,从而连续。
    \item 由Lebesgue分解定理知\(F=F_c+F_d+F_s \)。由\(F \)连续知\(F_d \)差分为\(0 \),不妨设\(F_d=0 \)。
      在 Cantor 关于\([0,1] \)的补集中的每个开区间\((a,b) \)上,一定有\(a=\sum_{k=0}^{m} \frac{a_k}{3^k},b=\sum_{k = 0}^{m}\frac{b_k}{3^k},\forall n<m,a_n = b_n, a_m=1,b_m=2 \)。
      于是\(F(a)=F(b) \),也就是\(\Delta F(a,b)=0 \)。于是\(F \)所对应的测度只集中在 Cantor 集上。又由 Cantor 集是零测的,得\(F_c=0 \)。
      于是\(F=F_d \),故\(F \)是奇异的。
  \end{enumerate}
\end{solution}

\begin{problem}\label{pro:3.39}
  设\(\mu_1,\mu_2 \)是有限符号测度,令\(\mu_1 \vee \mu_2 = \mu_1 + (\mu_2 - \mu_1)^{+},\mu_1 \wedge \mu_2 = \mu_1 - (\mu_1 - \mu_2)^{+} \),
  则\(\mu_1 \vee \mu_2 \)是满足\(\nu \geq \mu_i,i=1,2 \)的最小符号测度;
  \(\mu_1 \wedge \mu_2 \)是满足\(\nu \leq \mu_i,i=1,2 \)的最大符号测度。
\end{problem}
\begin{solution}
  对\(\mu_1-\mu_{2} \)进行 Hahn 分解得到\(D,D^c \),使\((\mu_1-\mu_{2})^+(A)=\mu_1(A \cap D)-\mu_2(A \cap D) \)。
  则自然有\((\mu_2-\mu_{1})^+(A)=(\mu_1-\mu_{2})^-(A)=\mu_2(A \setminus D)-\mu_1(A \setminus D) \)。

  且显然成立\((\mu_1-\mu_{2})^+ \geq \mu_1 - \mu_2 ,(\mu_2-\mu_{1})^+ \geq \mu_2 - \mu_1 \)。

  一方面,对于任何的\(A \),有\(\mu_1 \vee \mu_2(A)= \mu_1(A) + (\mu_2-\mu_1)^+(A) \geq \mu_1(A) + \max\{0,\mu_2(A)-\mu_1(A)\}=\max\{\mu_1(A),\mu_2(A)\}\);
  另一方面,设\(\nu\)满足\(\nu \geq \mu_i \),须证\(\nu \geq \mu_1 \vee \mu_2 \)。反设不成立,则存在\(A \)使\(\nu(A)<\mu_1 \vee \mu_2(A) \),
  则\(\nu(A \cap D)<\mu_1 \vee \mu_2(A \cap D)=\mu_1(A \cap D) + 0 \)或\(\nu(A \setminus D) < \mu_1 \vee \mu_2(A \setminus D) =\mu_1(A \setminus D) + \mu_2(A \setminus D)-\mu_1(A \setminus D)=\mu_2(A \setminus D) \)。
  均与\(\nu \geq \mu_1,\mu_{2} \)矛盾!故\(\nu \geq \mu_1 \vee \mu_{2} \)。

  同理,一方面有\(\mu_1\wedge \mu_2(A)=\mu_1(A)-(\mu_1-\mu_{2})^-(A) \leq \mu_1(A)-\max\{0,\mu_1(A)-\mu_2(A)\}=\min\{\mu_1(A),\mu_2(A)\} \)。
  另一方面,设\(\nu\)满足\(\nu \leq \mu_i \),须证\(\nu \leq \mu_1 \vee \mu_2 \)。反设不成立,则存在\(A \)使\(\nu(A)>\mu_1 \wedge \mu_2(A) \),
  则\(\nu(A \cap D)<\mu_1 \wedge \mu_2(A \cap D)=\mu_2(A \cap D) \) 或\(\nu(A \setminus D) < \mu_1 \wedge \mu_2(A \setminus D) =\mu_1(A \setminus D) + 0=\mu_1(A \setminus D) \)。
  均与\(\nu \leq \mu_1,\mu_{2} \)矛盾!故\(\nu \leq \mu_1 \wedge \mu_{2} \)。
\end{solution}

\begin{problem}\label{pro:3.40}
  设\(\mu \)为\((\Omega,\mathcal{A}) \)上的\(\sigma \)有限测度,\(\mathcal{A} \)包含单点集。则集合\(\{x \in \Omega: \mu(\{x\}) > 0\} \)至多可数。
\end{problem}
\begin{solution}
  令\(A_n := \{x \in \Omega:\mu(\{x\}) \geq \frac{1}{n}\} \)。若\(\exists n,A_n \)为无限集,则可取其可数子集\(B_n \),
  有\(\mu(B_n) = \sum_{x \in B_n}\mu(\{x\}) \geq \sum_{x \in B_n}\frac{1}{n} = \frac{q}{n} \cdot \infty = \infty\)。
  与\(\mu(B_n) \leq \mu(\Omega) < \infty \)矛盾。故\(\forall n,A_n \)为有限集。

  注意到\(\{x \in \Omega:\mu(\{x\}) >0\} = \bigcup_{n = 1}^{\infty}A_n \),而每个\(A_n \)有限,故\(\{x \in \Omega:\mu(\{x\})>0\} \)至多可数。
\end{solution}
\begin{problem}\label{pro:4.4}
  设\((\Omega_i,\mathcal{A}_i),i = 1,2,3 \)为三个可测空间,\(\lambda \)为\(\Omega_2 \times \mathcal{A}_{3} \)上的\(\sigma \)有限测转移测度,\(f \)是\(\mathcal{A}_1 \times \mathcal{A}_{3} \)
  可测函数。若\(\forall (\omega_1,\omega_{2})\in \Omega_1 \times \Omega_{2} \),积分\(g(\omega_1,\omega_{2}):=\int_{\Omega_{3}} f(\omega_1,\omega_{3}) \lambda(\omega_2,\d \omega_{3}) \)存在。
  证明\(g \)是\(\mathcal{A}_1 \times \mathcal{A}_{2} \)可测的。
\end{problem}
\begin{solution}
  令\(L:=\{f \in \mathcal{A}_1 \times \mathcal{A}_3: \int_{\Omega_{3}} f(\omega_1,\omega_{3}) \lambda(\omega_2,\d \omega_{3})\text{可测}\} \)。
  先证\(L \)是\(\mathcal{L} \)-系。
  \begin{enumerate}
    \item \(\int_{\Omega_{3}} 1 \lambda(\omega_2,\d \omega_{3})= \lambda(\omega_2,\Omega_{3}) \),由转移测度的定义知其可测。故\(1 \in L \)。
    \item 设\(f_{1},f_{2} \in L,a,b \in \mathbb{R},a f_{1} + b f_{2} \)有意义。则\(\int_{\Omega_{3}} (a f_{1} + b f_{2})(\omega_1,\omega_{3}) \lambda(\omega_2,\d \omega_{3}) =a\int_{\Omega_{3}} f_{1}(\omega_1,\omega_{3}) \lambda(\omega_2,\d \omega_{3}) +b\int_{\Omega_{3}} f_{2}(\omega_1,\omega_{3}) \lambda(\omega_2,\d \omega_{3}) \)。
      故\(a f_{1} + b f_{2} \in L \)。
    \item 设\(0 \leq f_n \nearrow f,f_n \in L \)。则\(\int_{\Omega_{3}} f(\omega_1,\omega_{3}) \lambda(\omega_2,\d \omega_{3}) = \lim_{n}\int_{\Omega_{3}} f_n(\omega_1,\omega_{3}) \lambda(\omega_2 \d \omega_{3}) \)可测,于是\(f \in L \)。
  \end{enumerate}
  于是\(L \)是\(\mathcal{L} \)-系。对于\(A_{1} \times A_{3} \in \mathcal{A}_1 \times \mathcal{A}_{3} \),有\(\int_{\Omega_{3}} \mathbbm{1}_{A_{1} \times A_{3}}(\omega_1,\omega_{3}) \lambda(\omega_2,\d \omega_{3}) = \mathbbm{1}_{A_{1}}(\omega_{1}) \int_{A_{3}} 1 \lambda(\omega_2,\d \omega_{3}) = \mathbbm{1}_{A_{1}}(\omega_{1}) \lambda(\omega_2,A_{3})\)。
  由转移测度的定义知\(\int_{A_{3}} 1 \lambda(\omega_2,\d \omega_{3}) \)可测,又显然\(\mathbbm{1}_{A_{1}}(\omega_{1}) \)可测,故其可测。于是\(\mathbbm{1}_{A_{1} \times A_{3}} \in L \)。
  故所有\(\mathcal{A}_1 \times \mathcal{A}_{3} \)可测函数均在\(L \)中,证毕。
\end{solution}

\begin{problem}\label{pro:4.6}
  若矩阵\(P = (p_{ij})_{i,j = 1}^\infty \)满足\(p_{ij} \geq 0,\sum_{j = 1}^{\infty} p_{ij} = 1, \forall i \geq 1 \),则称\(P \)为转移概率矩阵。
  令\(\lambda(i,A) = \sum_{j \in A}p_{ij} \),证明\(\lambda \)是转移概率。
\end{problem}
\begin{solution}
  对固定的\(i \)而言,须证\(\lambda(i,\cdot) \)是概率。注意到\(\mu_{ij}(A) = \mathbbm{1}_{j \in A} p_{ij} \)是测度,而\(\lambda(i,A) = \sum_{j}\mu_{ij}(A) \),于是为测度。
  又\(\lambda(i,\Omega) = \sum_{j}p_{ij} = 1 \),故为概率。

  对固定的\(A \)而言,须证\(\lambda(\cdot,A) \)可测。在离散空间上显然所有函数都可测。

  综上,\(\lambda \)是转移概率。
\end{solution}

\begin{problem}\label{pro:4.9}
  设\(\mu_k,\nu_k \)分别为\((\Omega_k,\mathcal{A}_k) \)上\(\sigma \)有限测度,\(\nu_k \ll \mu_k,k = 1,2 \)。
  证明\(\nu_1 \times \nu_2 \ll \mu_1 \times \mu_{2} \)且
  \(\frac{\d(\nu_1 \times \nu_{2})}{\d (\mu_1 \times \mu_{2})}(\omega_1,\omega_{2}) = \frac{\d \nu_{1}}{\d \mu_{1}}(\omega_{1}) \frac{\d \nu_{2}}{\d \mu_{2}}(\omega_{2}), \mu_1 \times \mu_2 \mathrm{-a.e.} \)
\end{problem}

\begin{solution}
  由测度扩张的唯一性,只需验证对于\(A_{1} \times A_{2} \in \mathcal{A}_1 \times \mathcal{A}_{2} \),有\(\nu_1 \times \nu_{2}(A_{1} \times A_{2}) = \int_{\Omega_1 \times \Omega_{2}} \frac{\d \nu_{1}}{\d \mu_{1}}(\omega_{1}) \frac{\d \nu_{2}}{\d \mu_{2}}(\omega_{2}) \d \mu_1 \times \mu_{2} \)即可。
  显然有
  \(\nu_1 \times \nu_2(A_{1} \times A_{2}) = \nu_1 (A_{1}) \nu_2(A_{2}) = \int_{\Omega_{1}} \frac{\d \nu_{1}}{\d \mu_{1}}(\omega_{1}) \d \mu_{1} + \int_{\Omega_{2}} \frac{\d \nu_{2}}{\d \mu_{2}}(\omega_{2}) \d \mu_{2} = \int_{\Omega_1 \times \Omega_{2}} \frac{\d \nu_{1}}{\d \mu_{1}}(\omega_{1}) \frac{\d \nu_{2}}{\d \mu_{2}}(\omega_{2}) \d \mu_1 \times \mu_{2} \)。
\end{solution}

\begin{problem}\label{pro:4.10}
  设\((\Omega_t,\mathcal{A}_t)_{t \in T} \)为一族可测空间,\(\mathcal{C}_t \subset \mathcal{A}_t,\sigma(\mathcal{C}_t) = \mathcal{A}_t,t \in T \)。
  证明\(\prod_{t \in T}\mathcal{A}_t = \sigma\left(\bigcup_{t \in T} J_t^{-1}(\mathcal{C}_t)\right) \),其中\(J_t:\prod_{t \in T} \ni \omega \mapsto \omega_t \in \Omega_t \)。
\end{problem}
\begin{solution}
  显然有\(\prod_{t \in T}\mathcal{A}_t \supset \sigma\left(\bigcup_{t \in T} J_t^{-1}(\mathcal{C}_t)\right) \)。只需证\( \sigma\left(\bigcup_{t \in T} J_t^{-1}(\mathcal{C}_t)\right) \)包含所有有限维柱集即可。
  对于有限维柱集有\(\prod_{t \in T_{0}}A_t \times \prod_{t \in T \setminus T_{0}} \Omega_t= \bigcap_{t \in T_{0}} J_t^{-1}(A_t)\),故只需证明\(J_t^{-1}(A_t) \in \sigma(J_t^{-1}(\mathcal{C}_t))\)。
  由\(\sigma(\mathcal{C}_t) = \mathcal{A}_t \)及\(A_t \in \mathcal{A}_t \)知其成立。故\( \sigma\left(\bigcup_{t \in T} J_t^{-1}(\mathcal{C}_t)\right) \)包含所有有限维柱集,
  从而\(\prod_{t \in T}\mathcal{A}_t = \sigma\left(\bigcup_{t \in T} J_t^{-1}(\mathcal{C}_t)\right) \)。
\end{solution}
\end{document}
