%!Mode:: "TeX:UTF-8"
%!TEX TS-program = xelatex
%arara: xelatex
\documentclass{ctexart}
\usepackage{catppuccinpalette}
\newcommand{\res}[2]{#1_{#2}}
\newif\ifpreface
\prefacetrue
\usepackage{fontspec}
\usepackage{bbm}
\usepackage{tikz}
\usepackage{amsmath,amssymb,amsthm,color,mathrsfs}
\usepackage{fixdif}
\usepackage{hyperref}
\usepackage{cleveref}
\usepackage{enumitem}%
\usepackage{expl3}
\usepackage{lipsum}
\usepackage[margin=0pt]{geometry}
\usepackage{listings}
\definecolor{mGreen}{rgb}{0,0.6,0}
\definecolor{mGray}{rgb}{0.5,0.5,0.5}
\definecolor{mPurple}{rgb}{0.58,0,0.82}
\definecolor{backgroundColour}{rgb}{0.95,0.95,0.92}

\lstdefinestyle{CStyle}{
  backgroundcolor=\color{backgroundColour},
  commentstyle=\color{mGreen},
  keywordstyle=\color{magenta},
  numberstyle=\tiny\color{mGray},
  stringstyle=\color{mPurple},
  basicstyle=\footnotesize,
  breakatwhitespace=false,
  breaklines=true,
  captionpos=b,
  keepspaces=true,
  numbers=left,
  numbersep=5pt,
  showspaces=false,
  showstringspaces=false,
  showtabs=false,
  tabsize=2,
  language=C
}
\usetikzlibrary{calc}
\theoremstyle{remark}
\newtheorem{lemma}{Lemma}
\usepackage{fontawesome5}
\usepackage{xcolor}
\newcounter{problem}
\newcommand{\Problem}{\begin{tikzpicture}[baseline]%
    \node at (-0.02em,0.3em) {$\mathbb{P}$};%
    \node[scale=0.7] at (0.2em,-0.0em) {R};%
    \node[scale=0.7] at (0.6em,0.4em) {O};%
    \node[scale=0.8] at (1.05em,0.25em) {B};%
    \node at (1.55em,0.3em) {L};%
    \node[scale=0.7] at (1.75em,0.45em) {E};%
    \node at (2.35em,0.3em) {M};%
  \end{tikzpicture}%
}
\renewcommand{\theproblem}{\Roman{problem}}
\newenvironment{problem}{\refstepcounter{problem}\noindent\color{blue}\Problem\theproblem}{}

\crefname{problem}{\protect\Problem}{Problem}
\newcommand\Solution{\begin{tikzpicture}[baseline]%
    \node at (-0.04em,0.3em) {$\mathbb{S}$};%
    \node[scale=0.7] at (0.35em,0.4em) {O};%
    \node at (0.7em,0.3em) {\textit{L}};%
    \node[scale=0.7] at (0.95em,0.4em) {U};%
    \node[scale=1.1] at (1.19em,0.32em){T};%
    \node[scale=0.85] at (1.4em,0.24em){I};%
    \node at (1.9em,0.32em){$\mathcal{O}$};%
    \node[scale=0.75] at (2.3em,0.21em){\texttt{N}};%
  \end{tikzpicture}}
\newenvironment{solution}{\begin{proof}[\Solution]}{\end{proof}}
\title{\input{../../.subject}\input{../.number}}
\makeatletter
\newcommand\email[1]{\def\@email{#1}\def\@refemail{mailto:#1}}
\newcommand\schoolid[1]{\def\@schoolid{#1}}
\ifpreface
  \def\@maketitle{
  \raggedright
  {\Huge \bfseries \sffamily \@title }\\[1cm]
  {\Huge  \bfseries \sffamily\heiti\@author}\\[1cm]
  {\Huge \@schoolid}\\[1cm]
  {\Huge\href\@refemail\@email}\\[0.5cm]
  \Huge\@date\\[1cm]}
\else
  \def\@maketitle{
    \raggedright
    \begin{center}
      {\Huge \bfseries \sffamily \@title }\\[4ex]
      {\Large  \@author}\\[4ex]
      {\large \@schoolid}\\[4ex]
      {\href\@refemail\@email}\\[4ex]
      \@date\\[8ex]
    \end{center}}
\fi
\makeatother
\ifpreface
  \usepackage[placement=bottom,scale=1,opacity=1]{background}
\fi

\author{白永乐}
\schoolid{25110180002}
\email{ylbai25@m.fudan.edu.cn}

\def\to{\rightarrow}
\newcommand{\xor}{\vee}
\newcommand{\AND}{\wedge}
\newcommand{\OR}{\vee}
\newcommand{\bor}{\bigvee}
\newcommand{\band}{\bigwedge}
\newcommand{\xand}{\wedge}
\newcommand{\minus}{\mathbin{\backslash}}
\newcommand{\mi}[1]{\mathscr{P}(#1)}
\newcommand{\card}{\mathrm{card}}
\newcommand{\oto}{\leftrightarrow}
\newcommand{\hin}{\hat{\in}}
\newcommand{\gl}{\mathrm{GL}}
\newcommand{\im}{\mathrm{Im}}
\newcommand{\re }{\mathrm{Re }}
\newcommand{\rank}{\mathrm{rank}}
\newcommand{\tra}{\mathop{\mathrm{tr}}}
\renewcommand{\char}{\mathop{\mathrm{char}}}
\DeclareMathOperator{\ot}{ordertype}
\DeclareMathOperator{\dom}{dom}
\DeclareMathOperator{\ran}{ran}

\begin{document}
\large
\setlength{\baselineskip}{1.2em}
\ifpreface
\input{../../../global/preface}
\newgeometry{left=2cm,right=2cm,top=2cm,bottom=2cm}
\else
\newgeometry{left=2cm,right=2cm,top=2cm,bottom=2cm}
\maketitle
\fi
%from_here_to_type

\begin{problem}\label{pro:1}
  举例说明\(\mathcal{C} \not \subset \mathcal{C}' \)时,未必有\(\mathbb{E}(\mathbb{E}(\xi \mid \mathcal{C}')\mid \mathcal{C}) = \mathbb{E}(\xi \mid \mathcal{C}) \)。
\end{problem}
\begin{solution}
  令\(\xi \sim B(1,\frac{1}{2}) \),令\(\mathcal{C}'=\{\emptyset,\Omega\},\mathcal{C}=\sigma(\xi) \)。
  则\(\mathbb{E}(\xi \mid \mathcal{C}) = \mathbb{E}(\xi \mid \sigma(\xi)) = \xi \),
  但\(\mathbb{E}(\mathbb{E}(\xi \mid \mathcal{C}') \mid \mathcal{C}) = \mathbb{E}(\frac{1}{2} \mid \mathcal{C}) = \frac{1}{2} \)。
  显然\(\xi \neq \frac{1}{2} \)。
\end{solution}

\begin{problem}\label{pro:2}
  设\(\xi \)可积。设\(\mathcal{C} \)为\(\mathcal{A} \)的子\(\sigma \)代数,\(\phi(B)=\int_{B} \xi \d \mathbb{P} , B \in \mathcal{C}  \), 则\(\phi  \)是\( \mathcal{C} \) 上\(\sigma \)可加集函数,
  \(\phi \ll \res{\mathbb{P}}{\mathcal{C}}\)。从而存在\(\mathbb{E}(\xi \mid \mathcal{C}) = \frac{\d \phi}{\d \mathbb{P}_{\mathcal{C}}}, \mathbb{P}_{\mathcal{C}}\)-a.e.
\end{problem}
\begin{solution}
  显然\(\phi:\mathcal{C} \to \mathbb{R} \)。只需证\(\phi \)为\(\mathcal{C} \)上的\(\sigma \)可加集函数。
  取\(B_n \in \mathcal{C},n \in \mathbb{N}_{+} \)两两不交。考查\(\xi_n:=\xi \mathbbm{1}_{\bigcup_{k=1}^{n}B_k} = \sum_{k=1}^{n} \xi \mathbbm{1}_{B_k} \),有\(|\xi_n|<|\xi| \),且\(\xi \)可积。
  故由控制收敛定理可得\(\lim_{n \to \infty} \int \xi_n \d \mathbb{P} = \int \lim_{n} \xi_n \d \mathbb{P} \),
  即\(\sum_{k=1}^{\infty} \phi(B_n)=\lim_{n}\sum_{k=1}^{n} \phi(B_n) = \lim_{n} \sum_{k=1}^{n} \int_{B_k} \xi \d \mathbb{P} = \int \mathbbm{1}_{\bigcup_{k=1}^{\infty} B_k} \xi \d \mathbb{P} = \phi(\bigcup_{k=1}^{\infty} B_k) \)。

  显然\(\mathbb{P}(B)=0 \implies \phi(B)=\int_{B} \xi \d \mathbb{P} =0\)。故\(\phi \ll \res{\mathbb{P}}{\mathcal{C}} \)。从而存在\(\mathbb{E}(\xi \mid \mathcal{C}) = \frac{\d \phi}{ \d \mathbb{P}_{\mathcal{C}}} \)。
\end{solution}
\begin{problem}\label{pro:5}
  证明赫尔德不等式\(\mathbb{E}(\xi \eta \mid \mathcal{C}) \leq \mathbb{E}(|\xi|^p \mid \mathcal{C})^{\frac{1}{p}} \mathbb{E}(|\eta|^q \mid \mathcal{C})^{\frac{1}{q}}, p >1, \frac{1}{p} + \frac{1}{q} =1 \)。
\end{problem}
\begin{solution}
  不妨设\(\xi,\eta \geq 0 \),否则用\(|\xi|,|\eta| \)代替。
  记\(\xi_n = \frac{\xi}{ (\mathbb{E}(\xi^p \mid \mathcal{C})+ \frac{1}{n})^{\frac{1}{p}}}, \eta_n = \frac{\eta}{ (\mathbb{E}(\eta^q \mid \mathcal{C})+ \frac{1}{n})^{\frac{1}{q}}} \)。
  由 Young 不等式知\(\xi_n \eta_n \leq \frac{1}{p} \xi_n^p + \frac{1}{q} \eta_n^q \)。
  注意到\( (\mathbb{E}(\xi^p \mid \mathcal{C})+ \frac{1}{n})^{\frac{1}{p}}, (\mathbb{E}(\eta^q \mid \mathcal{C})+ \frac{1}{n})^{\frac{1}{q}} \),关于\(\mathcal{C} \)可测,结合条件期望的单调性知
  \begin{equation}
    \begin{aligned}
      &\frac{\mathbb{E}(\xi \eta \mid \mathcal{C})}{ (\mathbb{E}(\xi^p \mid \mathcal{C})+ \frac{1}{n})^{\frac{1}{p}}(\mathbb{E}(\eta^q \mid \mathcal{C})+ \frac{1}{n})^{\frac{1}{q}} }\\
      =&\mathbb{E}(\xi_n \eta_n \mid \mathcal{C}) \\
      \leq & \mathbb{E} (\frac{1}{p} \xi_n^p + \frac{1}{q} \eta_n^q \mid \mathcal{C}) \\
      =& \frac{1}{p} \frac{\mathbb{E}(\xi^p \mid \mathcal{C})}{ \mathbb{E}(\xi^p \mid \mathcal{C}) + \frac{1}{n}} + \frac{1}{q} \frac{\mathbb{E}(\eta^q \mid \mathcal{C})}{ \mathbb{E}(\eta^q \mid \mathcal{C}) + \frac{1}{n}} \\
      \leq & \frac{1}{p} + \frac{1}{q} \\
      =& 1
    \end{aligned}
  \end{equation}
  ,整理得\(\mathbb{E}(\xi \eta \mid \mathcal{C}) \leq  (\mathbb{E}(\xi^p \mid \mathcal{C})+ \frac{1}{n})^{\frac{1}{p}} (\mathbb{E}(\eta^q \mid \mathcal{C})+ \frac{1}{n})^{\frac{1}{q}}  \)。
  令\(n \to \infty \),得到\(\mathbb{E}(\xi \eta \mid \mathcal{C}) \leq \mathbb{E}(|\xi|^p \mid \mathcal{C})^{\frac{1}{p}} \mathbb{E}(|\eta|^q \mid \mathcal{C})^{\frac{1}{q}}\)。
\end{solution}

\begin{problem}\label{pro:7}
  叙述并证明关于条件期望 Jensen 不等式和 Minikowski 不等式。
  \begin{enumerate}
    \item 设\(\xi \)可积,\(\phi \)是区间上的下凸函数,且\(\xi(\omega) \in \dom \phi \)几乎必然成立。则\(\mathbb{E}(\phi(\xi) \mid \mathcal{C} ) \geq \phi(\mathbb{E}(\xi \mid \mathcal{C})) \)。
  \end{enumerate}
\end{problem}
\begin{lemma}\label{lem:1}
  \(\phi \) 为区间上的下凸函数,且在端点处(如果有定义的话)连续。那么存在可数的\(A \subset \mathbb{R}^2 \),使\(\phi(x)=\sup_{(a,b) \in A} ax+b \)。
\end{lemma}
\begin{proof}
  若区间为单点集,则显然成立。下设区间不是单点集。

  由于\(\phi \)在端点处连续,故只需考虑定义域内部的点\(x \)。故可不妨设该区间为开区间。

  熟知开区间上的下凸函数必然连续,且\(\phi_+',\phi_-' \)都存在,且\(\forall x<y \),有\(\phi_-'(x) \leq \phi_+'(x) \leq \phi_-'(y) \leq \phi_+'(y) \)。
  对于任意的\(a \),考查直线\(y-\phi(a)=\phi_-'(a)(x-a) \)。
  由凸函数的性质,有\(\phi(x) \geq \phi(a) + \phi_-'(a)(x-a) = \phi_-'(a)x + \phi(a) - a \phi_-'(a) \)。
  让\(a \)取遍\(\mathbb{Q} \cap \dom \phi \),即令\(A:=\{(\phi_-'(a),\phi(a)-a \phi_-'(a)):a \in \mathbb{Q} \cap \dom \phi\} \),显然\(A \)可数,下证\(A \)满足条件。

  按照\(A \)的定义显然有\(\phi(x) \geq \sup_{(a,b) \in A}ax+b \),下面证明\( \phi(x) \leq \sup_{(a,b) \in A}ax+b \)。
  取一列\(\mathbb{Q} \cap \dom \phi \ni a_n \nearrow x \),有\((\phi_-'(a_n),\phi(a_n) - a_n \phi_-'(a_n)) \in A \)。
  考查\(y_n = \phi_-'(a_n)x + \phi(a_n) - a_n \phi_-'(a_n) \),有\(\phi(x)-y_n = \phi(x) - \phi(a_n) + \phi_-'(a_n)(x-a_n) \to 0 \)。
  于是\(y_n \to \phi(x) \),故\(\phi(x) = \sup_{(a,b) \in A} ax+b \)。
\end{proof}

\begin{solution}
  \begin{enumerate}
    \item 若\(\dom \phi\)为单点集,即\(\xi \)为常数,则显然成立。下设\(\dom \phi\) 不是单点集。

      记\(\dom \phi\)的内部为\((a,b) \),则\(\phi \)在\((a,b) \)上连续且单侧可导。
      由\(\xi \leq b \)几乎必然成立,知\(\mathbb{E}(\xi \mathbbm{1}_{C}) \leq b \mathbb{P}(C) \)对所有\(C \in \mathcal{C} \)成立,故\(\mathbb{E}(\xi \mid \mathcal{C}) \leq b \)几乎必然成立。
      同理\(\mathbb{E}(\xi \mid \mathcal{C}) \geq a \)几乎必然成立。
      对\(x \in [a,b] \)记 \(\psi(x)= \lim_{(a,b)\ni y \to x}\phi(y) \),则\(\psi(x) \)在\([a,b] \)上连续且下凸。
      且对\(x \in (a,b) \)有\(\phi(x) = \psi(x) \),且\(\phi(a) \geq \psi(a),\phi(b) \geq \psi(b) \)。

      由引理 \ref{lem:1}知存在可数的\(A \subset \mathbb{R}^2 \)使\(\psi(x) = \sup_{(a,b) \in A}ax + b \)。
      对任何\((a,b) \in A \),有\(\mathbb{E}(\psi(\xi) \mid \mathcal{C}) \geq \mathbb{E}(a \xi + b \mid \mathcal{C}) = a \mathbb{E}(\xi \mid \mathcal{C}) + b \)几乎处处成立。
      由\(A \)是可数集,可对\((a,b)\in A \)取上确界,得\(\mathbb{E}(\psi(\xi) \mid \mathcal{C}) \geq \psi(\mathbb{E}(\xi \mid \mathcal{C})) \)。

      对于\(\omega \)使\(\mathbb{E}(\xi \mid \mathcal{C})(\omega) \in (a,b) \),有\(\mathbb{E}(\phi(\xi) \mid \mathcal{C})(\omega) \geq \mathbb{E}(\psi(\xi) \mid \mathcal{C})(\omega) \geq \psi(\mathbb{E}(\xi \mid \mathcal{C})(\omega)) = \phi(\mathbb{E}(\xi \mid \mathcal{C})(\omega)) \)。
      否则有\(\mathbb{E}(\xi \mid \mathcal{C})(\omega)= a \)或\(b \),不妨设\(\mathbb{E}(\xi \mid \mathcal{C})(\omega) = b \)。
      考查\(B:=\{\omega \in \Omega:\mathbb{E}(\xi \mid \mathcal{C}) = b\} \in \mathcal{C} \),有\(\mathbb{E}(\xi \mathbbm{1}_{B}) = b \mathbb{P}(B) \)。
      结合\(\xi \leq b \)知\(\forall \omega \in B,\xi(\omega)= b \)。于是对于任何\(B \)的\(\mathcal{C} \)可测子集\(C \),成立
      \begin{align}
        \int_{C} \mathbb{E}(\phi(\xi) \mid \mathcal{C}) \d \mathbb{P} &= \int_{\Omega} \mathbb{E}(\phi(\xi) \mathbbm{1}_{C} \mid \mathcal{C}) \d \mathbb{P} = \int_{\Omega} \mathbb{E}(\phi(b) \mathbbm{1}_{C} \mid \mathcal{C}) \d \mathbb{P} = \mathbb{P}(C) \phi(b);\\
        \int_{C} \phi(\mathbb{E}(\xi \mid \mathcal{C})) \d \mathbb{P} &= \int_{C} \phi(b) \d \mathbb{P} = \mathbb{P}(C) \phi(b).
      \end{align}
      故\(\mathbb{E}(\phi(\xi) \mid \mathcal{C})(\omega) \geq \phi(\mathbb{E}(\xi \mid \mathcal{C})(\omega)) \)对于\(\omega \in B \)几乎处处成立(事实上是等于)。
      对于\(\mathbb{E}(\xi \mid \mathcal{C})(\omega) = a \)也同理。

      综上所述,\(\mathbb{E}(\phi(\xi) \mid \mathcal{C}) \geq \phi(\mathbb{E}(\xi \mid \mathcal{C})) \)几乎处处成立。
  \end{enumerate}
\end{solution}

\begin{problem}\label{pro:9}
  设\(\mathcal{C}_1,\mathcal{C}_2 \)为\(\mathcal{A} \)的子\(\sigma \)代数。举例说明\(\mathbb{E}(\xi \mid \mathcal{C}_1 \cap \mathcal{C}_2) \neq \mathbb{E}(\mathbb{E}(\xi \mid \mathcal{C}_1) \mid \mathcal{C}_2) \)。
\end{problem}

\(\int f g \leq (\int f^r)^{\frac{1}{r}} (\int g^s)^{\frac{1}{s}} \),where \(\frac{1}{r} + \frac{1}{s} = 1 \).
wlog assume \(\int f^r = \int g^s = 1 \).
\(\int f g \leq \int \frac{1}{s} g^s + \frac{1}{r} f^r = 1 \).

\(\ln \int f g \leq \frac{1}{r} \ln\int f^r + \frac{1}{s} \ln\int g^s \).
\(\frac{\ln a}{s} + \frac{\ln b}{r} \leq \ln(\frac{a}{s} +\frac{b}{r}) \).
\(a^{\frac{1}{s}} b^{\frac{1}{r} } \leq \frac{a}{s} + \frac{b}{r} \).
\(ab \leq \frac{1}{s} a^s + \frac{1}{r} b^r \).
\end{document}
