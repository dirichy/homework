%!Mode:: "TeX:UTF-8"
%!TEX encoding = UTF-8 Unicode
%!TEX TS-program = xelatex
\documentclass{ctexart}
\newif\ifpreface
%\prefacetrue
\usepackage{fontspec}
\usepackage{bbm}
\usepackage{tikz}
\usepackage{amsmath,amssymb,amsthm,color,mathrsfs}
\usepackage{fixdif}
\usepackage{hyperref}
\usepackage{cleveref}
\usepackage{enumitem}%
\usepackage{expl3}
\usepackage{lipsum}
\usepackage[margin=0pt]{geometry}
\usepackage{listings}
\definecolor{mGreen}{rgb}{0,0.6,0}
\definecolor{mGray}{rgb}{0.5,0.5,0.5}
\definecolor{mPurple}{rgb}{0.58,0,0.82}
\definecolor{backgroundColour}{rgb}{0.95,0.95,0.92}

\lstdefinestyle{CStyle}{
  backgroundcolor=\color{backgroundColour},
  commentstyle=\color{mGreen},
  keywordstyle=\color{magenta},
  numberstyle=\tiny\color{mGray},
  stringstyle=\color{mPurple},
  basicstyle=\footnotesize,
  breakatwhitespace=false,
  breaklines=true,
  captionpos=b,
  keepspaces=true,
  numbers=left,
  numbersep=5pt,
  showspaces=false,
  showstringspaces=false,
  showtabs=false,
  tabsize=2,
  language=C
}
\usetikzlibrary{calc}
\theoremstyle{remark}
\newtheorem{lemma}{Lemma}
\usepackage{fontawesome5}
\usepackage{xcolor}
\newcounter{problem}
\newcommand{\Problem}{\begin{tikzpicture}[baseline]%
    \node at (-0.02em,0.3em) {$\mathbb{P}$};%
    \node[scale=0.7] at (0.2em,-0.0em) {R};%
    \node[scale=0.7] at (0.6em,0.4em) {O};%
    \node[scale=0.8] at (1.05em,0.25em) {B};%
    \node at (1.55em,0.3em) {L};%
    \node[scale=0.7] at (1.75em,0.45em) {E};%
    \node at (2.35em,0.3em) {M};%
  \end{tikzpicture}%
}
\renewcommand{\theproblem}{\Roman{problem}}
\newenvironment{problem}{\refstepcounter{problem}\noindent\color{blue}\Problem\theproblem}{}

\crefname{problem}{\protect\Problem}{Problem}
\newcommand\Solution{\begin{tikzpicture}[baseline]%
    \node at (-0.04em,0.3em) {$\mathbb{S}$};%
    \node[scale=0.7] at (0.35em,0.4em) {O};%
    \node at (0.7em,0.3em) {\textit{L}};%
    \node[scale=0.7] at (0.95em,0.4em) {U};%
    \node[scale=1.1] at (1.19em,0.32em){T};%
    \node[scale=0.85] at (1.4em,0.24em){I};%
    \node at (1.9em,0.32em){$\mathcal{O}$};%
    \node[scale=0.75] at (2.3em,0.21em){\texttt{N}};%
  \end{tikzpicture}}
\newenvironment{solution}{\begin{proof}[\Solution]}{\end{proof}}
\title{\input{../../.subject}\input{../.number}}
\makeatletter
\newcommand\email[1]{\def\@email{#1}\def\@refemail{mailto:#1}}
\newcommand\schoolid[1]{\def\@schoolid{#1}}
\ifpreface
  \def\@maketitle{
  \raggedright
  {\Huge \bfseries \sffamily \@title }\\[1cm]
  {\Huge  \bfseries \sffamily\heiti\@author}\\[1cm]
  {\Huge \@schoolid}\\[1cm]
  {\Huge\href\@refemail\@email}\\[0.5cm]
  \Huge\@date\\[1cm]}
\else
  \def\@maketitle{
    \raggedright
    \begin{center}
      {\Huge \bfseries \sffamily \@title }\\[4ex]
      {\Large  \@author}\\[4ex]
      {\large \@schoolid}\\[4ex]
      {\href\@refemail\@email}\\[4ex]
      \@date\\[8ex]
    \end{center}}
\fi
\makeatother
\ifpreface
  \usepackage[placement=bottom,scale=1,opacity=1]{background}
\fi

\author{白永乐}
\schoolid{25110180002}
\email{ylbai25@m.fudan.edu.cn}

\def\to{\rightarrow}
\newcommand{\xor}{\vee}
\newcommand{\AND}{\wedge}
\newcommand{\OR}{\vee}
\newcommand{\bor}{\bigvee}
\newcommand{\band}{\bigwedge}
\newcommand{\xand}{\wedge}
\newcommand{\minus}{\mathbin{\backslash}}
\newcommand{\mi}[1]{\mathscr{P}(#1)}
\newcommand{\card}{\mathrm{card}}
\newcommand{\oto}{\leftrightarrow}
\newcommand{\hin}{\hat{\in}}
\newcommand{\gl}{\mathrm{GL}}
\newcommand{\im}{\mathrm{Im}}
\newcommand{\re }{\mathrm{Re }}
\newcommand{\rank}{\mathrm{rank}}
\newcommand{\tra}{\mathop{\mathrm{tr}}}
\renewcommand{\char}{\mathop{\mathrm{char}}}
\DeclareMathOperator{\ot}{ordertype}
\DeclareMathOperator{\dom}{dom}
\DeclareMathOperator{\ran}{ran}

\begin{document}
\large
\setlength{\baselineskip}{1.2em}
\ifpreface
    \input{../../../global/preface}
\newgeometry{left=2cm,right=2cm,top=2cm,bottom=2cm}
\else
\newgeometry{left=2cm,right=2cm,top=2cm,bottom=2cm}
\maketitle
\fi
%from_here_to_type
\begin{problem}
  Assume \(G=\langle  a\rangle \) is the \(n\)-ranked cyclic group. Prove that 
  \[
    f(a^r)=\sum_{j=0}^{n-1}k_j \xi^{rj}
  \]
  where \(f:G \to \mathbb{C}\) is a function and \(\xi=\mathrm{e}^{\frac{2\pi \mathrm{i}}{n}}\) and 
  \[
    k_j=\frac{1}{n} \sum_{r=0}^{n-1}f( a^r)\xi^{-rj}
  \]
\end{problem}

\begin{solution}
  Easily we have \(\hat{G} = \{ \phi^j:j=0,\cdots,n-1\}\), where \(\phi( a^r)=\xi^r\). 
  So we have \(f( a^r)=\frac{1}{\sqrt{n}}\sum_{j=0}^{n-1}\hat{f}( \phi^j) \phi^j( a^r)=\frac{1}{\sqrt{n}}\sum_{j=0}^{n-1}\hat{f}( \phi^j) \xi^{rj}\). 
  And \(\hat{f}( \phi^j)=\frac{1}{\sqrt{n}}\sum_{r=0}^{n-1}f( a^r)\overline{\phi^j( a^r)}=\frac{1}{\sqrt{n}}\sum_{r=0}^{n-1}f( a^r)\xi^{-rj}\). 
  So we finally get the given equation. 
\end{solution}

\begin{problem}
  Assume \(f:G \to \mathbb{C}\) and \(G\) is Abel group. Prove that \(f\) is const \(\iff \hat{f}( \phi)=0,\forall \phi \in \hat{G} \setminus \{ 1\}\). 
\end{problem}

\begin{solution}
  ``\(\implies\)'': Easily we have \(\hat{f}( \phi)=\frac{1}{\sqrt{n}}\sum_{g \in G}f( g)\phi( g)\). 
  Assume \(\phi \neq 1\), since \(f\) is const, we only need to prove \(\sum_{g \in G}\phi( g)=0\). 
  Since \(\phi \perp 1\), we get \(\sum_{g \in G}\phi( g) =0\). 
  ``\(\impliedby\)'': Easily we have \(f( g)=\sum_{\phi \in \hat{G}}\frac{1}{\sqrt{n}}\sum_{\phi \in \hat{G}} \hat{f}( \phi)\phi( g)\). 
  Since \(\phi \neq 1 \to \hat{f}( \phi)=0\), we get \(f( g)=\frac{1}{\sqrt{n}} \hat{f}( 1)\) is a const. 
\end{solution}

\begin{problem}
  Find a \(3\)-dim irriducible complex character of \(S_4\).
\end{problem}

\begin{solution}
  Consider \(S_4 \times \Omega:=\{ 1,2,3,4\} \to \Omega,\sigma x=\sigma( x)\). Easily this group action is double transitive. 
  Let \(\phi\) is the reperesentation obtained by this group action, then \(\phi=\phi_0 \oplus \phi_1\), where \(\phi_0\) is main reperesentation and \(\dim \phi_1=3\) and \(\phi_1\) is irriducible. 
  Easily \(S_4\) has \(5\) conjugate classes and \(\{ ( 1),( 12),( 123),( 12)( 34),( 1234)\}\) is reperesentation element. 
  Let \(\chi,\chi_0,\chi_1\) are character of \(\phi,\phi_0,\phi_1\), then \(\chi=\chi_0+\chi_1=1+\chi_1\). 
  So \(\chi_1( ( 1))=\chi( ( 1))-1=3,\chi_1( ( 12))=\chi( ( 12))-1=1,\chi_1( ( 123))=\chi(( 123))-1=0,\chi_1( ( 12)( 34))=\chi( ( 12)( 34))-1=-1,\chi_1( ( 1234))=\chi( ( 1234))-1=-1\). 
\end{solution}

\begin{problem}
  Find a \(4\)-dim irriducible complex character of \(A_5\).
\end{problem}

\begin{solution}
  Consider \(A_5 \times \Omega:=\{ 1,2,3,4,5\} \to \Omega,\sigma x:=\sigma( x)\), easily it's double transitive. 
  Let \(\phi\) is the reperesentation obtained by this group action, then \(\phi=\phi_0 \oplus \phi_1\), where \(\phi_0\) is main reperesentation and \(\phi_1\) is \(4\)-dim irriducible. 
  Let \(\chi,\chi_0,\chi_1\) are character of \(\phi,\phi_0,\phi_1\), then \(\chi=\chi_0+\chi_1=1+\chi_1\). 
  Easily \(A_5\) has \(5\) conjugate classes and \(\{ ( 1),( 123),( 12)( 34),( 12345),( 12543)\}\) is a group of reperesentation elements. 
  Finally we get \(\chi_1( ( 1))=4,\chi_1( ( 123))=1,\chi_1( ( 12)( 34))=0,\chi_1( ( 12345))=\chi_1( ( 12543))=-1\).
\end{solution}
\end{document}
