%!Mode:: "TeX:UTF-8"
%!TEX encoding = UTF-8 Unicode
%!TEX TS-program = xelatex
\documentclass{ctexart}
\newif\ifpreface
%\prefacetrue
\usepackage{fontspec}
\usepackage{bbm}
\usepackage{tikz}
\usepackage{amsmath,amssymb,amsthm,color,mathrsfs}
\usepackage{fixdif}
\usepackage{hyperref}
\usepackage{cleveref}
\usepackage{enumitem}%
\usepackage{expl3}
\usepackage{lipsum}
\usepackage[margin=0pt]{geometry}
\usepackage{listings}
\definecolor{mGreen}{rgb}{0,0.6,0}
\definecolor{mGray}{rgb}{0.5,0.5,0.5}
\definecolor{mPurple}{rgb}{0.58,0,0.82}
\definecolor{backgroundColour}{rgb}{0.95,0.95,0.92}

\lstdefinestyle{CStyle}{
  backgroundcolor=\color{backgroundColour},
  commentstyle=\color{mGreen},
  keywordstyle=\color{magenta},
  numberstyle=\tiny\color{mGray},
  stringstyle=\color{mPurple},
  basicstyle=\footnotesize,
  breakatwhitespace=false,
  breaklines=true,
  captionpos=b,
  keepspaces=true,
  numbers=left,
  numbersep=5pt,
  showspaces=false,
  showstringspaces=false,
  showtabs=false,
  tabsize=2,
  language=C
}
\usetikzlibrary{calc}
\theoremstyle{remark}
\newtheorem{lemma}{Lemma}
\usepackage{fontawesome5}
\usepackage{xcolor}
\newcounter{problem}
\newcommand{\Problem}{\begin{tikzpicture}[baseline]%
    \node at (-0.02em,0.3em) {$\mathbb{P}$};%
    \node[scale=0.7] at (0.2em,-0.0em) {R};%
    \node[scale=0.7] at (0.6em,0.4em) {O};%
    \node[scale=0.8] at (1.05em,0.25em) {B};%
    \node at (1.55em,0.3em) {L};%
    \node[scale=0.7] at (1.75em,0.45em) {E};%
    \node at (2.35em,0.3em) {M};%
  \end{tikzpicture}%
}
\renewcommand{\theproblem}{\Roman{problem}}
\newenvironment{problem}{\refstepcounter{problem}\noindent\color{blue}\Problem\theproblem}{}

\crefname{problem}{\protect\Problem}{Problem}
\newcommand\Solution{\begin{tikzpicture}[baseline]%
    \node at (-0.04em,0.3em) {$\mathbb{S}$};%
    \node[scale=0.7] at (0.35em,0.4em) {O};%
    \node at (0.7em,0.3em) {\textit{L}};%
    \node[scale=0.7] at (0.95em,0.4em) {U};%
    \node[scale=1.1] at (1.19em,0.32em){T};%
    \node[scale=0.85] at (1.4em,0.24em){I};%
    \node at (1.9em,0.32em){$\mathcal{O}$};%
    \node[scale=0.75] at (2.3em,0.21em){\texttt{N}};%
  \end{tikzpicture}}
\newenvironment{solution}{\begin{proof}[\Solution]}{\end{proof}}
\title{\input{../../.subject}\input{../.number}}
\makeatletter
\newcommand\email[1]{\def\@email{#1}\def\@refemail{mailto:#1}}
\newcommand\schoolid[1]{\def\@schoolid{#1}}
\ifpreface
  \def\@maketitle{
  \raggedright
  {\Huge \bfseries \sffamily \@title }\\[1cm]
  {\Huge  \bfseries \sffamily\heiti\@author}\\[1cm]
  {\Huge \@schoolid}\\[1cm]
  {\Huge\href\@refemail\@email}\\[0.5cm]
  \Huge\@date\\[1cm]}
\else
  \def\@maketitle{
    \raggedright
    \begin{center}
      {\Huge \bfseries \sffamily \@title }\\[4ex]
      {\Large  \@author}\\[4ex]
      {\large \@schoolid}\\[4ex]
      {\href\@refemail\@email}\\[4ex]
      \@date\\[8ex]
    \end{center}}
\fi
\makeatother
\ifpreface
  \usepackage[placement=bottom,scale=1,opacity=1]{background}
\fi

\author{白永乐}
\schoolid{25110180002}
\email{ylbai25@m.fudan.edu.cn}

\def\to{\rightarrow}
\newcommand{\xor}{\vee}
\newcommand{\AND}{\wedge}
\newcommand{\OR}{\vee}
\newcommand{\bor}{\bigvee}
\newcommand{\band}{\bigwedge}
\newcommand{\xand}{\wedge}
\newcommand{\minus}{\mathbin{\backslash}}
\newcommand{\mi}[1]{\mathscr{P}(#1)}
\newcommand{\card}{\mathrm{card}}
\newcommand{\oto}{\leftrightarrow}
\newcommand{\hin}{\hat{\in}}
\newcommand{\gl}{\mathrm{GL}}
\newcommand{\im}{\mathrm{Im}}
\newcommand{\re }{\mathrm{Re }}
\newcommand{\rank}{\mathrm{rank}}
\newcommand{\tra}{\mathop{\mathrm{tr}}}
\renewcommand{\char}{\mathop{\mathrm{char}}}
\DeclareMathOperator{\ot}{ordertype}
\DeclareMathOperator{\dom}{dom}
\DeclareMathOperator{\ran}{ran}

\begin{document}
\large
\setlength{\baselineskip}{1.2em}
\ifpreface
    \input{../../../global/preface}
\newgeometry{left=2cm,right=2cm,top=2cm,bottom=2cm}
\else
\newgeometry{left=2cm,right=2cm,top=2cm,bottom=2cm}
\maketitle
\fi
%from_here_to_type
\begin{problem}
  \(R\) is a ring with identity element.
  If every non zero element in \(R\) is inversible, we call \(R\) is division ring.
  Prove: if \( D\) is a division ring, then \(M_n(D)\) is a monocycle.
\end{problem}
\begin{solution}
  Assume \(I\) is a non-zero two-sided ideal of \(M_n(D)\), now we only need to prove \(I=M_n(D)\). 
  Only need to prove \(E_{ij} \in I,\forall i,j\). 
  Since \(I \neq \{0\}\), assume \(A \in I\) and \(A_{st} \neq 0\). 
  Then \(\forall i,j,E_{is}AE_{tj} \in I\). i.e., \(a_{st} E_{ij} \in I\). 
  So \(a_{st}^{-1}I_n a_{st}E_{ij} \in I\), i.e., \(E_{ij} \in I\). 
  So \(I = M_n(D)\). 
\end{solution}

\begin{problem}
  \(V\) is right module of division ring \(D\), let \(\hom_D(V,V)\) is the set of all module isomorphic of \(V\). 
  Given \(\dim_DV=n\), prove that \(\hom_D(V,V) \cong M_n(D)\). 
\end{problem}

\begin{solution}
  Assume \(\{a_1,\cdots,a_n\} \) is a maximal linearly indenpendent set of \(V\). 
  First we prove \(\forall x \in V,\exists ! d_1,\cdots,d_n \in D\) such that \(x = \sum_{k=1}^{n} d_ka_k\). 
  
  Existence: Since \(A\) is maximal linearly indenpendent set, we have \(A \cup \{ x \}\) is not linearly indenpendent. 
  So \(\exists t_1,\cdots,t_n,t \in D\) such that \(\sum_{k=1}^{n} t_ka_k +tx =0\) and \(t_1,\cdots,t_n,t\) are not all \(0\). 
  If \(t=0\),  then we get \(a_1,\cdots,a_n\) are not linearly indenpendent, contradiction! 
  So \(t \neq 0\). Since \(D\) is division ring, we get \(\exists w \in D\) such that \(tw=wt =1\). 
  Let \(d_k=-wt_k,k=1,\cdots,n\), then \(\sum_{k=1}^{n} d_ka_k=\sum_{k=1}^{n}-wd_ka_k=wtx=x \).

  Uniqueness: If \(x=\sum_{k=1}^{n} d_ka_k=\sum_{k=1}^{n} t_k a_k\), then \(\sum_{k=1}^{n} (d_k-t_k)a_k=0\). 
  Since \(a_k,k=1,\cdots,n\) is linearly indenpendent, we get \(d_k-t_k=0,k=1,\cdots,n\). 
  So the form is unique. 

  Now let \(f_j:V \to D,\sum_{k=1}^{n} d_ka_k \mapsto d_j\), for \(j=1,\cdots,n\). Then \(f_k\) is well-defined. 
  Let \(F:\hom_D(V,V) \to M_n(D), F(\phi)_{ij}:=f_i(\phi(a_j))\). Now we prove \(F\) is isomorphic. 

  Since  \(\phi(\sum_{k=1}^{n} d_ka_k)=\sum_{k=1}^{n} d_k \phi(a_k)\), so 
  \(\phi(\sum_{k=1}^{n} d_ka_k)_i=\sum_{k=1}^{n} d_k F(\phi)_{ik}\). 
  So \(\phi\) can be reperesented by \(F(\phi)\), so \(F\) is injection. 

  For \(A \in M_n (D)\), assume \(A=(a_{ij})\). Let \(\phi(\sum_{k=1}^{n} d_ka_k)_i:=\sum_{k=1}^{n} d_ka_{ik}\). 
  Easily we get \(\phi \in \hom_D(V,V)\). And \(F(\phi)=A\). So \(F\) is surjective. 

  Easily \(F(\phi \psi)_{ij}=f_i(\phi \psi a_j)=f_i(\phi \sum_{k=1}^{n} f_k(\psi a_j)a_k)=\sum_{k=1}^{n} F(\phi)_{ik}F(\psi)_{kj}\). 
  So \(F(\phi \psi)=F(\phi) F(\psi)\). 
  And obviously \(F(a \phi+ b \psi)=aF(\phi)+b F(\psi)\). 

  So all in all we get \(F\) is a isomorphic. 
\end{solution}
\end{document}
