%!Mode:: "TeX:UTF-8"
%!TEX encoding = UTF-8 Unicode
%!TEX TS-program = xelatex
\documentclass{ctexart}
\newif\ifpreface
%\prefacetrue
\usepackage{fontspec}
\usepackage{bbm}
\usepackage{tikz}
\usepackage{amsmath,amssymb,amsthm,color,mathrsfs}
\usepackage{fixdif}
\usepackage{hyperref}
\usepackage{cleveref}
\usepackage{enumitem}%
\usepackage{expl3}
\usepackage{lipsum}
\usepackage[margin=0pt]{geometry}
\usepackage{listings}
\definecolor{mGreen}{rgb}{0,0.6,0}
\definecolor{mGray}{rgb}{0.5,0.5,0.5}
\definecolor{mPurple}{rgb}{0.58,0,0.82}
\definecolor{backgroundColour}{rgb}{0.95,0.95,0.92}

\lstdefinestyle{CStyle}{
  backgroundcolor=\color{backgroundColour},
  commentstyle=\color{mGreen},
  keywordstyle=\color{magenta},
  numberstyle=\tiny\color{mGray},
  stringstyle=\color{mPurple},
  basicstyle=\footnotesize,
  breakatwhitespace=false,
  breaklines=true,
  captionpos=b,
  keepspaces=true,
  numbers=left,
  numbersep=5pt,
  showspaces=false,
  showstringspaces=false,
  showtabs=false,
  tabsize=2,
  language=C
}
\usetikzlibrary{calc}
\theoremstyle{remark}
\newtheorem{lemma}{Lemma}
\usepackage{fontawesome5}
\usepackage{xcolor}
\newcounter{problem}
\newcommand{\Problem}{\begin{tikzpicture}[baseline]%
    \node at (-0.02em,0.3em) {$\mathbb{P}$};%
    \node[scale=0.7] at (0.2em,-0.0em) {R};%
    \node[scale=0.7] at (0.6em,0.4em) {O};%
    \node[scale=0.8] at (1.05em,0.25em) {B};%
    \node at (1.55em,0.3em) {L};%
    \node[scale=0.7] at (1.75em,0.45em) {E};%
    \node at (2.35em,0.3em) {M};%
  \end{tikzpicture}%
}
\renewcommand{\theproblem}{\Roman{problem}}
\newenvironment{problem}{\refstepcounter{problem}\noindent\color{blue}\Problem\theproblem}{}

\crefname{problem}{\protect\Problem}{Problem}
\newcommand\Solution{\begin{tikzpicture}[baseline]%
    \node at (-0.04em,0.3em) {$\mathbb{S}$};%
    \node[scale=0.7] at (0.35em,0.4em) {O};%
    \node at (0.7em,0.3em) {\textit{L}};%
    \node[scale=0.7] at (0.95em,0.4em) {U};%
    \node[scale=1.1] at (1.19em,0.32em){T};%
    \node[scale=0.85] at (1.4em,0.24em){I};%
    \node at (1.9em,0.32em){$\mathcal{O}$};%
    \node[scale=0.75] at (2.3em,0.21em){\texttt{N}};%
  \end{tikzpicture}}
\newenvironment{solution}{\begin{proof}[\Solution]}{\end{proof}}
\title{\input{../../.subject}\input{../.number}}
\makeatletter
\newcommand\email[1]{\def\@email{#1}\def\@refemail{mailto:#1}}
\newcommand\schoolid[1]{\def\@schoolid{#1}}
\ifpreface
  \def\@maketitle{
  \raggedright
  {\Huge \bfseries \sffamily \@title }\\[1cm]
  {\Huge  \bfseries \sffamily\heiti\@author}\\[1cm]
  {\Huge \@schoolid}\\[1cm]
  {\Huge\href\@refemail\@email}\\[0.5cm]
  \Huge\@date\\[1cm]}
\else
  \def\@maketitle{
    \raggedright
    \begin{center}
      {\Huge \bfseries \sffamily \@title }\\[4ex]
      {\Large  \@author}\\[4ex]
      {\large \@schoolid}\\[4ex]
      {\href\@refemail\@email}\\[4ex]
      \@date\\[8ex]
    \end{center}}
\fi
\makeatother
\ifpreface
  \usepackage[placement=bottom,scale=1,opacity=1]{background}
\fi

\author{白永乐}
\schoolid{25110180002}
\email{ylbai25@m.fudan.edu.cn}

\def\to{\rightarrow}
\newcommand{\xor}{\vee}
\newcommand{\AND}{\wedge}
\newcommand{\OR}{\vee}
\newcommand{\bor}{\bigvee}
\newcommand{\band}{\bigwedge}
\newcommand{\xand}{\wedge}
\newcommand{\minus}{\mathbin{\backslash}}
\newcommand{\mi}[1]{\mathscr{P}(#1)}
\newcommand{\card}{\mathrm{card}}
\newcommand{\oto}{\leftrightarrow}
\newcommand{\hin}{\hat{\in}}
\newcommand{\gl}{\mathrm{GL}}
\newcommand{\im}{\mathrm{Im}}
\newcommand{\re }{\mathrm{Re }}
\newcommand{\rank}{\mathrm{rank}}
\newcommand{\tra}{\mathop{\mathrm{tr}}}
\renewcommand{\char}{\mathop{\mathrm{char}}}
\DeclareMathOperator{\ot}{ordertype}
\DeclareMathOperator{\dom}{dom}
\DeclareMathOperator{\ran}{ran}

\begin{document}
\large
\setlength{\baselineskip}{1.2em}
\ifpreface
	\input{../../../global/preface}
	\newgeometry{left=2cm,right=2cm,top=2cm,bottom=2cm}
\else
	\newgeometry{left=2cm,right=2cm,top=2cm,bottom=2cm}
	\maketitle
\fi
%from_here_to_type
\begin{problem}
Assume \((\phi,V),(\psi,W)\) are two finite-dim reperentation of group \(G\), find the matrix of \(\phi \otimes \psi\).
\end{problem}

安塞赛赛按所打算\(dsadsadljksjalk\)懂爱萨东岸\(\)\(\)\(\)\(\)
\begin{solution}
	Assume \(\{ v_i:i=1,\cdots,n\},\{ w_i:i=1,\cdots ,m\}\) are basis of \(V,W\). Then we get \(\{ v_i \otimes w_j:1 \leq i \leq n,1 \leq j \leq m\}\) is a basis of \(V \otimes W\).
	Assume \(\Phi,\Psi,\Gamma\) is the matrix of \(\phi,\psi,\phi \otimes \psi\).
  We use \(\{ 1,\cdots,n\}^4\) as the dom of \(\Gamma( g)\). 
	Then we get \(( \phi \otimes \psi)( g)( v_i \otimes w_j)=\phi( g)( v_i)\otimes \psi( g)( w_j)\).
	So \(\Gamma( g)( e_{ij})=( \sum_{k=1}^{n}\sum_{t=1}^{m} \Phi( g)_{ki} \Psi( g)_{tj} e_{kt})\).
	So finally we get \(\Gamma( g)_{k t,i j}=\Phi( g)_{ki}\Psi( g)_{tj}\).
\end{solution}

\begin{problem}
Assume \(\sym^2 V:=\Span\{ v \otimes w + w \otimes v:v,w \in V\}\) and \(\bigwedge^2 V:=\Span\{ v \otimes w - w \otimes v:v,w \in V\}\).
Prove that \(V \otimes V = \sym^2 V \oplus \bigwedge^2 V\).
\end{problem}

\begin{solution}
	First since \(x \otimes y = \frac{x}{2} \otimes y + y \otimes \frac{x}{2} + \frac{x}{2}\otimes y - y \otimes \frac{x}{2}\) we get 
  \(x \otimes y \in \sym^2 V+ \bigwedge^2 V\). Since \(\Span\{ x \otimes y:x,y \in V\}=V \otimes V\), we get
  \(V \otimes V = \sym^2 V + \bigwedge^2 V\).
	Now assume \(\dim V=n\), we only need to prove \(\dim\sym^2 V + \dim \bigwedge^2 V \leq n^2\).
	Assume \(\{ v_i:1 \leq i \leq n\}\) is a basis of \(V\), then \(\{v_i \otimes v_j:1 \leq i,j \leq n\}\) is basis of \(V \otimes V\).
	Then easily \(\Span\{ v_i \otimes v_j + v_j \otimes v_i:1 \leq i,j \leq n\}=\sym^2 V,\Span\{ v_i \otimes v_j - v_j \otimes v_i : 1 \leq i,j \leq n\}=\bigwedge^2 V\).
	Since for \(i \neq j\) we get \(v_i \otimes v_j+v_j \otimes v_i=v_j \otimes v_i + v_i \otimes v_j\) we get \(\dim \sym^2 V \leq n+\frac{n^2-n}{2}\).
	Since for \(i \neq j\) we have \(v_i \otimes v_j - v_j \otimes v_i = -( v_j \otimes v_i - v_i \otimes v_j)\) and for \(i = j \) we have \(v_i \otimes v_j - v_j \otimes v_i=0\) we get \(\dim \bigwedge^2 V \leq \frac{n^2-n}{2}\).
	So finally we get \(\dim \text{ Sym}^2 V +\dim \bigwedge^2 V \leq n+\frac{n^2-n}{2}+\frac{n^2-n}{2}=n^2\).
	So \(V \otimes V=\sym^2 V \oplus \bigwedge^2 V\).
\end{solution}

\begin{problem}
Find the complex character table of the group \(D_5\).
\end{problem}

\begin{solution}
	First we should find all of irreducible complex reperentation of \(D_5\).
	Easily all of conjugate of \(D_5\) are \(\{ e\},\{ \sigma,\sigma^4\}, \{ \sigma^2,\sigma^3\},\{ \tau,\sigma \tau,\sigma^2 \tau,\sigma^3 \tau,\sigma^4 \tau\}\).
	So there are four different irreducible complex reperentation of \(D_5\).
	Now we try to find the one-dim irreducible complex reperentation. Easily we get \(D_5' = \left\langle  \sigma\right\rangle \).
	So \(D_5 / D_5' \cong \mathbb{Z}_2\). So \(D_5\) has two different irreducible complex reperentation, \(\phi_0,\phi_1\).
	Where \(\phi_0\) is the main reperentation, and \(\phi_1( \sigma^i )=1,\phi_1( \sigma^i \tau)=-1\).
	Now we try to find other reperentation of \(D_5\). Since \(| D_5|=10=1^2+1^2+2^2+2^2\), we get \(D_5\) has two different two-dim irreducible reperentation.
	Consider \(\phi_\theta( \sigma)=\begin{pmatrix}
		\cos \theta & -\sin \theta \\
		\sin \theta & \cos \theta  \\
	\end{pmatrix}\) and \(\phi_\theta( \tau)=\begin{pmatrix}
		-1 & 0 \\
		0  & 1 \\
	\end{pmatrix}\), let \(\phi_2=\phi_{\frac{2 \pi}{5}},\phi_3=\phi_{\frac{4 \pi}{5}}\). Easily \(\phi_2,\phi_3\) are irreducible and different.
	So all of different irreducible complex reperentation of \(D_5\) are \(\phi_0,\phi_1,\phi_2,\phi_3\).
	Now we let \(g_1=e,g_2=\sigma,g_3=\sigma^2,g_4=\tau\) and \(W_{ij}=\chi_{i-1}( g_j)\), we have
	\[
		W=\begin{pmatrix}
		  1 & 1                      & 1                      & 1  \\
			1 & 1                      & 1                      & -1 \\
			2 & 2\cos \frac{2 \pi}{5}  & 2 \cos \frac{4 \pi}{5} & 0  \\
			2 & 2 \cos \frac{4 \pi}{5} & 2\cos \frac{2 \pi}{5}  & 0  \\
		\end{pmatrix}
	\]
\end{solution}
\end{document}
