%!Mode:: "TeX:UTF-8"
%!TEX encoding = UTF-8 Unicode
%!TEX TS-program = xelatex
\documentclass{ctexart}
\newif\ifpreface
\prefacetrue
\usepackage{fontspec}
\usepackage{bbm}
\usepackage{tikz}
\usepackage{amsmath,amssymb,amsthm,color,mathrsfs}
\usepackage{fixdif}
\usepackage{hyperref}
\usepackage{cleveref}
\usepackage{enumitem}%
\usepackage{expl3}
\usepackage{lipsum}
\usepackage[margin=0pt]{geometry}
\usepackage{listings}
\definecolor{mGreen}{rgb}{0,0.6,0}
\definecolor{mGray}{rgb}{0.5,0.5,0.5}
\definecolor{mPurple}{rgb}{0.58,0,0.82}
\definecolor{backgroundColour}{rgb}{0.95,0.95,0.92}

\lstdefinestyle{CStyle}{
  backgroundcolor=\color{backgroundColour},
  commentstyle=\color{mGreen},
  keywordstyle=\color{magenta},
  numberstyle=\tiny\color{mGray},
  stringstyle=\color{mPurple},
  basicstyle=\footnotesize,
  breakatwhitespace=false,
  breaklines=true,
  captionpos=b,
  keepspaces=true,
  numbers=left,
  numbersep=5pt,
  showspaces=false,
  showstringspaces=false,
  showtabs=false,
  tabsize=2,
  language=C
}
\usetikzlibrary{calc}
\theoremstyle{remark}
\newtheorem{lemma}{Lemma}
\usepackage{fontawesome5}
\usepackage{xcolor}
\newcounter{problem}
\newcommand{\Problem}{\begin{tikzpicture}[baseline]%
    \node at (-0.02em,0.3em) {$\mathbb{P}$};%
    \node[scale=0.7] at (0.2em,-0.0em) {R};%
    \node[scale=0.7] at (0.6em,0.4em) {O};%
    \node[scale=0.8] at (1.05em,0.25em) {B};%
    \node at (1.55em,0.3em) {L};%
    \node[scale=0.7] at (1.75em,0.45em) {E};%
    \node at (2.35em,0.3em) {M};%
  \end{tikzpicture}%
}
\renewcommand{\theproblem}{\Roman{problem}}
\newenvironment{problem}{\refstepcounter{problem}\noindent\color{blue}\Problem\theproblem}{}

\crefname{problem}{\protect\Problem}{Problem}
\newcommand\Solution{\begin{tikzpicture}[baseline]%
    \node at (-0.04em,0.3em) {$\mathbb{S}$};%
    \node[scale=0.7] at (0.35em,0.4em) {O};%
    \node at (0.7em,0.3em) {\textit{L}};%
    \node[scale=0.7] at (0.95em,0.4em) {U};%
    \node[scale=1.1] at (1.19em,0.32em){T};%
    \node[scale=0.85] at (1.4em,0.24em){I};%
    \node at (1.9em,0.32em){$\mathcal{O}$};%
    \node[scale=0.75] at (2.3em,0.21em){\texttt{N}};%
  \end{tikzpicture}}
\newenvironment{solution}{\begin{proof}[\Solution]}{\end{proof}}
\title{\input{../../.subject}\input{../.number}}
\makeatletter
\newcommand\email[1]{\def\@email{#1}\def\@refemail{mailto:#1}}
\newcommand\schoolid[1]{\def\@schoolid{#1}}
\ifpreface
  \def\@maketitle{
  \raggedright
  {\Huge \bfseries \sffamily \@title }\\[1cm]
  {\Huge  \bfseries \sffamily\heiti\@author}\\[1cm]
  {\Huge \@schoolid}\\[1cm]
  {\Huge\href\@refemail\@email}\\[0.5cm]
  \Huge\@date\\[1cm]}
\else
  \def\@maketitle{
    \raggedright
    \begin{center}
      {\Huge \bfseries \sffamily \@title }\\[4ex]
      {\Large  \@author}\\[4ex]
      {\large \@schoolid}\\[4ex]
      {\href\@refemail\@email}\\[4ex]
      \@date\\[8ex]
    \end{center}}
\fi
\makeatother
\ifpreface
  \usepackage[placement=bottom,scale=1,opacity=1]{background}
\fi

\author{白永乐}
\schoolid{202011150087}
\email{202011150087@mail.bnu.edu.cn}

\def\to{\rightarrow}
\newcommand{\xor}{\vee}
\newcommand{\bor}{\bigvee}
\newcommand{\band}{\bigwedge}
\newcommand{\xand}{\wedge}
\newcommand{\minus}{\mathbin{\backslash}}
\newcommand{\mi}[1]{\mathscr{P}(#1)}
\newcommand{\card}{\mathrm{card}}
\newcommand{\oto}{\leftrightarrow}
\newcommand{\hin}{\hat{\in}}
\newcommand{\gl}{\mathrm{GL}}
\newcommand{\im}{\mathrm{Im}}
\newcommand{\re }{\mathrm{Re }}
\newcommand{\rank}{\mathrm{rank}}
\newcommand{\tra}{\mathop{\mathrm{tr}}}
\renewcommand{\char}{\mathop{\mathrm{char}}}
\DeclareMathOperator{\ot}{ordertype}
\DeclareMathOperator{\dom}{dom}
\DeclareMathOperator{\ran}{ran}

\begin{document}
\large
\setlength{\baselineskip}{1.2em}
\ifpreface
    \backgroundsetup{contents={%
    \begin{tikzpicture}
      \fill [white] (current page.north west) rectangle ($(current page.north east)!.3!(current page.south east)$) coordinate (a);
      \fill [bgc] (current page.south west) rectangle (a);
\end{tikzpicture}}}
\definecolor{word}{rgb}{1,1,0}
\definecolor{bgc}{rgb}{1,0.95,0}
\setlength{\parindent}{0pt}
\thispagestyle{empty}
\begin{tikzpicture}%
  % \node[xscale=2,yscale=4] at (0cm,0cm) {\sffamily\bfseries \color{word} under};%
  \node[xscale=4.5,yscale=10] at (10cm,1cm) {\sffamily\bfseries \color{word} Graduate Homework};%
  \node[xscale=4.5,yscale=10] at (8cm,-2.5cm) {\sffamily\bfseries \color{word} In Mathematics};%
\end{tikzpicture}
\ \vspace{1cm}\\
\begin{minipage}{0.25\textwidth}
  \textcolor{bgc}{王胤雅是傻逼}
\end{minipage}
\begin{minipage}{0.75\textwidth}
  \maketitle
\end{minipage}
\vspace{4cm}\ \\
\begin{minipage}{0.2\textwidth}
  \
\end{minipage}
\begin{minipage}{0.8\textwidth}
  {\Huge
    \textinconsolatanf{}
  }General fire extinguisher
\end{minipage}
\newpage\backgroundsetup{contents={}}\setlength{\parindent}{2em}

\else
\maketitle
\fi
\newgeometry{left=2cm,right=2cm,top=2cm,bottom=2cm}
%from_here_to_type
\begin{problem}
 Find all of $1-$dimentional complex representation of the alternating group $A_4$.
\end{problem}

\begin{solution}
 Consider the conjugacy classes of $A_4$. They are: $T_1=\{(1)\},T_2=\{(1\ 2)(3\ 4),(1\ 3)(2\ 4),(1\ 4)(2\ 3)\},T_3=\{(1\ 2\ 3),(1\ 4\ 2),(1\ 3\ 4),(2\ 4\ 3)\},T_4=\{(1\ 3\ 2),(1\ 2\ 4),(1\ 4\ 3),(2\ 3\ 4)\}$. Assume $\phi$ is the representation, then for $a\sim b$ we obtain $\phi(a)=\phi(g^{-1}bg)=\phi(g)^{-1}\phi(b)\phi(g)=\phi(b)$. 
 So $\tau:G/\sim\to \mathbb{C},[a]\mapsto \phi(a)$ is well-defined. 
 Since $T_2\subset A_4'$ we get $\tau(T_2)=1$. And easily $\tau(T_3)\tau(T_4)=1,\tau(T_3)^3=1$. So $\tau(T_3)=1,\omega,\bar{\omega}$, where $\omega=-\frac{1}{2}+\frac{\sqrt{3}\mathrm{i}}{2}$. 
 \begin{enumerate}
  \item $\tau(T_3)=1$, we get $\phi(a)=1,\forall a\in A_4$. 
  \item $\tau(T_3)=\omega$, we get $\phi(a)=\begin{cases}
   \omega & a\in T_3\\\bar{\omega} & a\in T_4 \\ 1 & otherwise 
  \end{cases}$. 
  \item $\tau(T_3)=\bar{\omega}$, we get $\phi(a)=\begin{cases}
   \bar{\omega} & a\in T_3\\\omega & a\in T_4 \\ 1 & otherwise 
  \end{cases}$. 
 \end{enumerate} 
\end{solution}

\begin{problem}\label{pro:2}
 Consider $N\trianglelefteq S_4$ and $N=\{(1),(12)(34),(13)(24),(14)(23)\}$. 
 \begin{enumerate}
  \item Prove: $S_4/N\cong S_3$. 
  \item Find a $2-$dimentional irreducible complex matrix representation of $S_4$. 
 \end{enumerate}
\end{problem}

\begin{solution}
 \begin{enumerate}
  \item Since $|S_4/N|=6$ and obviously $S_4/N\not\cong C_6$, because $[(1\ 2)][(1\ 3)]\neq [(1\ 3)][(1\ 2)]$, we get $S_4/N\cong S_3$. 
  \item Consider $\phi: S_3\to \gl_2(\mathbb{C}),(2\ 3)\mapsto \bar{\cdot},(1\ 2\ 3)\mapsto A$, where $A$ is the rotation of $\frac{2 \mathrm{\pi}}{3}$. Then easily $\phi$ is a group representation. 
  Obviously $\phi$ is irreducible, so $\bar{\phi}$ is irreducible. So $\bar{\phi}$ satisfy the requirement. 
 \end{enumerate}
\end{solution}

\begin{problem}
 Assume $K$ is a field and $m\in \mathbb{N}^*$. Let $\phi_m(t):=t^m,\forall t\in K^*$, then $\phi_m$ is a $1-$dimentional $K-$representation of $(K^*,\cdot)$. Use $\phi_m$ to find a $1-$ dimentional $K-$ representation of $\gl_n(K)$. 
\end{problem}

\begin{solution}
 Consider $f:\gl_n(K)\to K,f(A)=|A|$. Since $\phi_m$ is group representation, $\phi_m \circ f$ is group representation of $\gl_n(K)$. So $\bar{\phi_m}:\gl_n(K)\to K^*,A\mapsto |A|^m$ satisfy the requirement. 
\end{solution}

\begin{problem}
 Prove that if $\phi$ is $1-$dimentional complex representation of finite group $G$, then $G/\ker\phi$ is a cyclic group. 
\end{problem}

\begin{solution}
 Let $\phi(G)=:T\subset \mathbb{C}$. Since $G$ is finite we get $\forall x\in T,|x|=1$. Let $a\in T$ and $\arg a\in [0,2 \mathrm{\pi})$ is minimum. For $b\in T$, if $\arg a\nmid \arg b$, then assume $\arg b=\arg a \cdot n+\theta$, where $\theta\in (0,\arg a)$. Then we get $\mathrm{e}^{\theta}=ba^{-n}\in T$ since $T$ is subgroup. Contridiction to $\arg a$ is minimum. So $\forall b\in T,\arg a \mid \arg b$. That means $\exists n\in \mathbb{N},b=a^n$. So $T$ is cyclic group. Noting $G/\ker f\cong \ran(f)=T$, we get $G/\ker f$ is cyclic group. 
\end{solution}

\begin{problem}
 Prove: If $G$ is a non-cyclic finite group, then there is no faithful $1-$dimentional complex representation of $G$. 
\end{problem}

\begin{solution}
 Assume there is a faithful $\phi$. 
 
 If $G$ is not Abel, then exists $a,b\in G$ such that $aba^{-1}b^{-1}\neq e$. \\
 But $\phi(aba^{-1}b^{-1})=\phi(a)\phi(b)\phi(a)^{-1}\phi(b)^{-1}=1=\phi(e)$, contridiction! 

 If $G$ is Abel, then assume $G=\oplus_{k=1}^n G_k$, where $G_k$ is cyclic, and $|G_k|=p_k^{\alpha_k}$. If $\forall i\neq j, p_i\neq p_j$, then $G$ is cyclic, contridiction! So exists $i\neq j$ such that $p_i=p_j$. Assume $p_1= p_2$. Let $f_i:G\to G_i$ is projection, then $\phi_i:=\phi\circ f_i$ is group representation of $G_i$. Assume $G_1=\langle x\rangle,G_2=\langle y\rangle$, then $o(\phi_1(x^{\alpha_1-1}))=p_1=p_2$. So $\exists z\in G_2$ such that $\phi_2(z)=\phi_1(\phi_1(x^{\alpha_1-1}))$. Contridiction to $\phi$ is faithful! 
\end{solution}

\begin{problem}
 Assume $(\phi,V)$ and $(\psi,W)$ are two $K-$representation of group $G$. 
 Prove: $(\phi \dot{+}\psi)^*\approx \phi^*\dot{+}\psi^*$. 
\end{problem}

\begin{solution}
 First we prove $V^*\oplus W^*\cong(V\oplus W)^*$. Consider $\theta:V^*\oplus W^*\to(V\oplus W)^*,\theta(f,g)(u,v):=(f(u),g(v))$. Then obviously $\theta$ is a bijection. And $\theta(a(f,g)+b(h,l))(u,v)=\theta(af+bh,ag+bl)(u,v)=((af+bh)(u),(ag+bl)(v))=(af(u)+bh(u),ag(u)+bl(u))=a \theta(f,g)(u,v)+b \theta(h,l)(u,v)$, so $\theta$ is isomorphism. 

 Now we only need to prove $(\phi\dot{+}\psi)^*(a)\theta=\theta(\phi^*\dot{+}\psi^*)(a),\forall a\in G$. Forall $f\in V^*,g\in W^*$, we have $ (\phi\dot{+}\psi)^*(a)\theta(f,g)=\theta(f,g)(\phi\dot{+}\psi)(a)$. And $\theta(\phi^*\dot{+}\psi^*)(a)(f,g)=\theta(\phi^*(a)(f),\psi^*(a)(g))=\theta(f\phi(a),g\psi(a))$. Easily $\theta(f\phi(a),g\psi(a))=\theta(f,g)(\phi\dot{+}\psi)(a)$, so $\theta$ is isomorphism of $(\phi \dot{+}\psi)^*$ and $\phi^*\dot{+}\psi^*$. 
\end{solution}

\end{document}
