%!Mode:: "TeX:UTF-8"
%!TEX encoding = UTF-8 Unicode
%!TEX TS-program = xelatex
\documentclass{ctexart}
\newif\ifpreface
%\prefacetrue
\usepackage{fontspec}
\usepackage{bbm}
\usepackage{tikz}
\usepackage{amsmath,amssymb,amsthm,color,mathrsfs}
\usepackage{fixdif}
\usepackage{hyperref}
\usepackage{cleveref}
\usepackage{enumitem}%
\usepackage{expl3}
\usepackage{lipsum}
\usepackage[margin=0pt]{geometry}
\usepackage{listings}
\definecolor{mGreen}{rgb}{0,0.6,0}
\definecolor{mGray}{rgb}{0.5,0.5,0.5}
\definecolor{mPurple}{rgb}{0.58,0,0.82}
\definecolor{backgroundColour}{rgb}{0.95,0.95,0.92}

\lstdefinestyle{CStyle}{
  backgroundcolor=\color{backgroundColour},
  commentstyle=\color{mGreen},
  keywordstyle=\color{magenta},
  numberstyle=\tiny\color{mGray},
  stringstyle=\color{mPurple},
  basicstyle=\footnotesize,
  breakatwhitespace=false,
  breaklines=true,
  captionpos=b,
  keepspaces=true,
  numbers=left,
  numbersep=5pt,
  showspaces=false,
  showstringspaces=false,
  showtabs=false,
  tabsize=2,
  language=C
}
\usetikzlibrary{calc}
\theoremstyle{remark}
\newtheorem{lemma}{Lemma}
\usepackage{fontawesome5}
\usepackage{xcolor}
\newcounter{problem}
\newcommand{\Problem}{\begin{tikzpicture}[baseline]%
    \node at (-0.02em,0.3em) {$\mathbb{P}$};%
    \node[scale=0.7] at (0.2em,-0.0em) {R};%
    \node[scale=0.7] at (0.6em,0.4em) {O};%
    \node[scale=0.8] at (1.05em,0.25em) {B};%
    \node at (1.55em,0.3em) {L};%
    \node[scale=0.7] at (1.75em,0.45em) {E};%
    \node at (2.35em,0.3em) {M};%
  \end{tikzpicture}%
}
\renewcommand{\theproblem}{\Roman{problem}}
\newenvironment{problem}{\refstepcounter{problem}\noindent\color{blue}\Problem\theproblem}{}

\crefname{problem}{\protect\Problem}{Problem}
\newcommand\Solution{\begin{tikzpicture}[baseline]%
    \node at (-0.04em,0.3em) {$\mathbb{S}$};%
    \node[scale=0.7] at (0.35em,0.4em) {O};%
    \node at (0.7em,0.3em) {\textit{L}};%
    \node[scale=0.7] at (0.95em,0.4em) {U};%
    \node[scale=1.1] at (1.19em,0.32em){T};%
    \node[scale=0.85] at (1.4em,0.24em){I};%
    \node at (1.9em,0.32em){$\mathcal{O}$};%
    \node[scale=0.75] at (2.3em,0.21em){\texttt{N}};%
  \end{tikzpicture}}
\newenvironment{solution}{\begin{proof}[\Solution]}{\end{proof}}
\title{\input{../../.subject}\input{../.number}}
\makeatletter
\newcommand\email[1]{\def\@email{#1}\def\@refemail{mailto:#1}}
\newcommand\schoolid[1]{\def\@schoolid{#1}}
\ifpreface
  \def\@maketitle{
  \raggedright
  {\Huge \bfseries \sffamily \@title }\\[1cm]
  {\Huge  \bfseries \sffamily\heiti\@author}\\[1cm]
  {\Huge \@schoolid}\\[1cm]
  {\Huge\href\@refemail\@email}\\[0.5cm]
  \Huge\@date\\[1cm]}
\else
  \def\@maketitle{
    \raggedright
    \begin{center}
      {\Huge \bfseries \sffamily \@title }\\[4ex]
      {\Large  \@author}\\[4ex]
      {\large \@schoolid}\\[4ex]
      {\href\@refemail\@email}\\[4ex]
      \@date\\[8ex]
    \end{center}}
\fi
\makeatother
\ifpreface
  \usepackage[placement=bottom,scale=1,opacity=1]{background}
\fi

\author{白永乐}
\schoolid{202011150087}
\email{202011150087@mail.bnu.edu.cn}

\def\to{\rightarrow}
\newcommand{\xor}{\vee}
\newcommand{\bor}{\bigvee}
\newcommand{\band}{\bigwedge}
\newcommand{\xand}{\wedge}
\newcommand{\minus}{\mathbin{\backslash}}
\newcommand{\mi}[1]{\mathscr{P}(#1)}
\newcommand{\card}{\mathrm{card}}
\newcommand{\oto}{\leftrightarrow}
\newcommand{\hin}{\hat{\in}}
\newcommand{\gl}{\mathrm{GL}}
\newcommand{\im}{\mathrm{Im}}
\newcommand{\re }{\mathrm{Re }}
\newcommand{\rank}{\mathrm{rank}}
\newcommand{\tra}{\mathop{\mathrm{tr}}}
\renewcommand{\char}{\mathop{\mathrm{char}}}
\DeclareMathOperator{\ot}{ordertype}
\DeclareMathOperator{\dom}{dom}
\DeclareMathOperator{\ran}{ran}

\begin{document}
\large
\setlength{\baselineskip}{1.2em}
\ifpreface
    \backgroundsetup{contents={%
    \begin{tikzpicture}
      \fill [white] (current page.north west) rectangle ($(current page.north east)!.3!(current page.south east)$) coordinate (a);
      \fill [bgc] (current page.south west) rectangle (a);
\end{tikzpicture}}}
\definecolor{word}{rgb}{1,1,0}
\definecolor{bgc}{rgb}{1,0.95,0}
\setlength{\parindent}{0pt}
\thispagestyle{empty}
\begin{tikzpicture}%
  % \node[xscale=2,yscale=4] at (0cm,0cm) {\sffamily\bfseries \color{word} under};%
  \node[xscale=4.5,yscale=10] at (10cm,1cm) {\sffamily\bfseries \color{word} Graduate Homework};%
  \node[xscale=4.5,yscale=10] at (8cm,-2.5cm) {\sffamily\bfseries \color{word} In Mathematics};%
\end{tikzpicture}
\ \vspace{1cm}\\
\begin{minipage}{0.25\textwidth}
  \textcolor{bgc}{王胤雅是傻逼}
\end{minipage}
\begin{minipage}{0.75\textwidth}
  \maketitle
\end{minipage}
\vspace{4cm}\ \\
\begin{minipage}{0.2\textwidth}
  \
\end{minipage}
\begin{minipage}{0.8\textwidth}
  {\Huge
    \textinconsolatanf{}
  }General fire extinguisher
\end{minipage}
\newpage\backgroundsetup{contents={}}\setlength{\parindent}{2em}

\newgeometry{left=2cm,right=2cm,top=2cm,bottom=2cm}
\else
\newgeometry{left=2cm,right=2cm,top=2cm,bottom=2cm}
\maketitle
\fi
%from_here_to_type
\begin{problem}
  Find all of irreducible complex reperesentation of \(D_4\). 
\end{problem}
\begin{solution}
  Write \(D_4=\left\langle \sigma,\tau:\sigma^4=\tau^2=1,\tau \sigma \tau = \sigma^{-1}\right\rangle \). 
  First we find all of conjugate class of \(D_4\), they are \(C_1=\{1\},C_2=\{\sigma,\sigma^3\},C_3=\{\sigma^2\},C_4=\{\tau,\sigma^2 \tau\},C_5=\{\sigma \tau,\sigma^3 \tau\}\). 
  Second we find all of \(1\)-dimetional reperesentation of \(D_4\). Only need to find all \(1\)-dimetional reperesentation of \(D_4/D_4'\). 
  Easily we get \(D_4'=\{\sigma^2,1\}\), so \(D_4/D_4'=\{D_4',D_4'\sigma,D_4'\tau,D_4'\sigma \tau\}\). 
  So \(D_4/D_4'\) has \(4\) different reperesentation, write \(\overline{\phi_0},\overline{\phi_1},\overline{\phi_2},\overline{\phi_3}\), where \(\overline{\phi_0}\) is main reperesentation. 
  And let \(\overline{\phi_1}(D_4'\sigma)=-1,\overline{\phi_1}(D_4'\tau)=1,\overline{\phi_2}(D_4'\sigma)=1,\overline{\phi_2}(D_4'\tau)=-1,\overline{\phi_3}(D_4'\sigma)=-1,\overline{\phi_3}(D_4'\tau)=-1\). 
  Then improve them to \(D_4\), we get \(\phi_0,\phi_1,\phi_2,\phi_3\), where \(\phi_0\) is main reperesentation, and 
  \(\phi_i(x)=\overline{\phi_i}(D_4'x)\). They are all of \(1\)-dimetional reperesentation of \(D_4\). 
  Now we find other irreducible reperesentation of \(D_4\). 
  Since \(|D_4|=8=1^2+1^2+1^2+1^2+2^2\) we get \(D_4\) has a \(2\)-dimetional irreducible reperesentation. 
  Consider \(\phi_4:D_4 \to M_2(\mathbb{C}),\sigma \mapsto \begin{pmatrix}
    0 & -1 \\
    1 & 0
  \end{pmatrix},\tau \mapsto \begin{pmatrix}
    -1 & 0 \\
    0 & 1 
  \end{pmatrix}\). 
  Obviously it's irreducible reperesentation of \(D_4\). 
  So all of irreducible reperesentation are \(\phi_0,\phi_1,\phi_2,\phi_3,\phi_4\). 
\end{solution}

\begin{problem}
  Find all irreducible reperesentation of \(Q\), the quaternions group. 
\end{problem}

\begin{solution}
  Write \(Q=\{\pm 1,\pm i,\pm j,\pm k\}\). Easily we get \(C_0=\{1\},C_1=\{- 1\},C_2=\{\pm i\},C_3=\{\pm j\},C_4=\{\pm k\}\) are conjugate class of \(Q\). 
  So \(Q\) has \(5\) different irreducible reperesentation. 
  Easily we know \(Q'=\{\pm 1\}\) and \(Q / Q'=\{Q',Q'i,Q'j,Q'k=Q'ij\}\). 
  Easily \(Q / Q'\) has \(4\) different \(1\)-dimetional reperesentation, write \(\overline{\phi_0},\overline{\phi_1},\overline{\phi_2},\overline{\phi_3}\), where \(\overline{\phi_0}\) is main reperesentation.
  And \(\overline{\phi_1}(Q'i)=-1,\overline{\phi_1}(Q'j)=1;\overline{\phi_2}(Q'i)=1,\overline{\phi_2}(Q'j)=-1;\overline{\phi_3}(Q'i)=\overline{\phi_3}(Q'j)=-1\). 
  Improve them we get \(\phi_0,\phi_1,\phi_2,\phi_3\), and \(\phi_0\) is main reperesentation, and 
  \(\phi_t(x)=\overline{\phi_t}(Q'x)\). 
  Since \(|Q|=8=1^2+1^2+1^2+1^2+2^2\), we get the last reperesentation is \(2\)-dimetional. 
  Consider \(\phi_4:Q \to M_2(\mathbb{C}),i \mapsto \begin{pmatrix}
    \mathrm{i} & 0 \\ 
    0 & -\mathrm{i}
  \end{pmatrix}, j \mapsto \begin{pmatrix}
    0 & \mathrm{i} \\
    \mathrm{i} & 0
  \end{pmatrix}\). 
  Easily we get \(\phi_4\) is irreducible, so \(\phi_t,t=0,\cdots,4\) are all irreducible reperesentation of \(Q\). 
\end{solution}

\begin{problem}
  Find all of irreducible reperesentation of \(A_4\). 
\end{problem}

\begin{solution}
  Obviously \(A_4'=K_4=\{(1 2)(3 4),(1 3)(2 4),(1 4)(2 3),(1)\}\). 
  And \(C_1=\{(1)\}, C_2=K_4,C_3=\{(1 2 3),(243),(134),(142)\}, C_4=\{(132),(124),(143),(234)\}\) are all of conjugate class of \(A_4\). 
  Easily \(A_4/K_4=\{(123)K_4,(132)K_4,K_4\}\). So it has \(3\) different irreducible \(1\)-dimetional reperesentation. 
  Write \(\overline{\phi_0},\overline{\phi_1},\overline{\phi_2}\), where \(\overline{\phi_0}\) is main reperesentation. 
  And \(\overline{\phi_1}((123)K_4)=\omega,\overline{\phi_2}((123)K_4)=\omega^2\). 
  Now improve then to \(A_4\), we get \(\phi_0,\phi_1,\phi_2\), where \(\phi_0\) is main reperesentation, and 
  \(\phi_t(x)=\overline{\phi_t}(xK4)\). 
  Sicne \(|A_4|=1^2+1^2+1^2+1^2+3^2\), we know the last irreducible reperesentation is \(3\)-dimetional. 
  Consider \(\phi_3:A_4 \to M_3(\mathbb{C})\), and 
  \[
    \phi_3((123))=\begin{pmatrix}
      -1 & -1 & -1 \\
      1 & 0 & 0 \\
      0 & 0 & 1
    \end{pmatrix}, \phi_3((124))=\begin{pmatrix}
      -1 & -1 & -1 \\
      0 & 1 & 0 \\
      1 & 0 & 0
    \end{pmatrix}
  \]
  Easily we get \(\phi_3\) is irreducible. 
  So all irreducible reperesentation of \(A_4\) are \(\phi_0,\phi_1,\phi_2,\phi_3\). 
\end{solution}
\end{document}
