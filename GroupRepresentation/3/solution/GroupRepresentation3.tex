%!Mode:: "TeX:UTF-8"
%!TEX encoding = UTF-8 Unicode
%!TEX TS-program = xelatex × 2
\documentclass{ctexart}
\newif\ifpreface
%\prefacetrue
\usepackage{fontspec}
\usepackage{bbm}
\usepackage{tikz}
\usepackage{amsmath,amssymb,amsthm,color,mathrsfs}
\usepackage{fixdif}
\usepackage{hyperref}
\usepackage{cleveref}
\usepackage{enumitem}%
\usepackage{expl3}
\usepackage{lipsum}
\usepackage[margin=0pt]{geometry}
\usepackage{listings}
\definecolor{mGreen}{rgb}{0,0.6,0}
\definecolor{mGray}{rgb}{0.5,0.5,0.5}
\definecolor{mPurple}{rgb}{0.58,0,0.82}
\definecolor{backgroundColour}{rgb}{0.95,0.95,0.92}

\lstdefinestyle{CStyle}{
  backgroundcolor=\color{backgroundColour},
  commentstyle=\color{mGreen},
  keywordstyle=\color{magenta},
  numberstyle=\tiny\color{mGray},
  stringstyle=\color{mPurple},
  basicstyle=\footnotesize,
  breakatwhitespace=false,
  breaklines=true,
  captionpos=b,
  keepspaces=true,
  numbers=left,
  numbersep=5pt,
  showspaces=false,
  showstringspaces=false,
  showtabs=false,
  tabsize=2,
  language=C
}
\usetikzlibrary{calc}
\theoremstyle{remark}
\newtheorem{lemma}{Lemma}
\usepackage{fontawesome5}
\usepackage{xcolor}
\newcounter{problem}
\newcommand{\Problem}{\begin{tikzpicture}[baseline]%
    \node at (-0.02em,0.3em) {$\mathbb{P}$};%
    \node[scale=0.7] at (0.2em,-0.0em) {R};%
    \node[scale=0.7] at (0.6em,0.4em) {O};%
    \node[scale=0.8] at (1.05em,0.25em) {B};%
    \node at (1.55em,0.3em) {L};%
    \node[scale=0.7] at (1.75em,0.45em) {E};%
    \node at (2.35em,0.3em) {M};%
  \end{tikzpicture}%
}
\renewcommand{\theproblem}{\Roman{problem}}
\newenvironment{problem}{\refstepcounter{problem}\noindent\color{blue}\Problem\theproblem}{}

\crefname{problem}{\protect\Problem}{Problem}
\newcommand\Solution{\begin{tikzpicture}[baseline]%
    \node at (-0.04em,0.3em) {$\mathbb{S}$};%
    \node[scale=0.7] at (0.35em,0.4em) {O};%
    \node at (0.7em,0.3em) {\textit{L}};%
    \node[scale=0.7] at (0.95em,0.4em) {U};%
    \node[scale=1.1] at (1.19em,0.32em){T};%
    \node[scale=0.85] at (1.4em,0.24em){I};%
    \node at (1.9em,0.32em){$\mathcal{O}$};%
    \node[scale=0.75] at (2.3em,0.21em){\texttt{N}};%
  \end{tikzpicture}}
\newenvironment{solution}{\begin{proof}[\Solution]}{\end{proof}}
\title{\input{../../.subject}\input{../.number}}
\makeatletter
\newcommand\email[1]{\def\@email{#1}\def\@refemail{mailto:#1}}
\newcommand\schoolid[1]{\def\@schoolid{#1}}
\ifpreface
  \def\@maketitle{
  \raggedright
  {\Huge \bfseries \sffamily \@title }\\[1cm]
  {\Huge  \bfseries \sffamily\heiti\@author}\\[1cm]
  {\Huge \@schoolid}\\[1cm]
  {\Huge\href\@refemail\@email}\\[0.5cm]
  \Huge\@date\\[1cm]}
\else
  \def\@maketitle{
    \raggedright
    \begin{center}
      {\Huge \bfseries \sffamily \@title }\\[4ex]
      {\Large  \@author}\\[4ex]
      {\large \@schoolid}\\[4ex]
      {\href\@refemail\@email}\\[4ex]
      \@date\\[8ex]
    \end{center}}
\fi
\makeatother
\ifpreface
  \usepackage[placement=bottom,scale=1,opacity=1]{background}
\fi

\author{白永乐}
\schoolid{202011150087}
\email{202011150087@mail.bnu.edu.cn}

\def\to{\rightarrow}
\newcommand{\xor}{\vee}
\newcommand{\bor}{\bigvee}
\newcommand{\band}{\bigwedge}
\newcommand{\xand}{\wedge}
\newcommand{\minus}{\mathbin{\backslash}}
\newcommand{\mi}[1]{\mathscr{P}(#1)}
\newcommand{\card}{\mathrm{card}}
\newcommand{\oto}{\leftrightarrow}
\newcommand{\hin}{\hat{\in}}
\newcommand{\gl}{\mathrm{GL}}
\newcommand{\im}{\mathrm{Im}}
\newcommand{\re }{\mathrm{Re }}
\newcommand{\rank}{\mathrm{rank}}
\newcommand{\tra}{\mathop{\mathrm{tr}}}
\renewcommand{\char}{\mathop{\mathrm{char}}}
\DeclareMathOperator{\ot}{ordertype}
\DeclareMathOperator{\dom}{dom}
\DeclareMathOperator{\ran}{ran}

\crefname{enumi}{}{}
\renewcommand{\phi}{\varphi}

\begin{document}
\large
\setlength{\baselineskip}{1.2em}
\ifpreface
  \backgroundsetup{contents={%
    \begin{tikzpicture}
      \fill [white] (current page.north west) rectangle ($(current page.north east)!.3!(current page.south east)$) coordinate (a);
      \fill [bgc] (current page.south west) rectangle (a);
\end{tikzpicture}}}
\definecolor{word}{rgb}{1,1,0}
\definecolor{bgc}{rgb}{1,0.95,0}
\setlength{\parindent}{0pt}
\thispagestyle{empty}
\begin{tikzpicture}%
  % \node[xscale=2,yscale=4] at (0cm,0cm) {\sffamily\bfseries \color{word} under};%
  \node[xscale=4.5,yscale=10] at (10cm,1cm) {\sffamily\bfseries \color{word} Graduate Homework};%
  \node[xscale=4.5,yscale=10] at (8cm,-2.5cm) {\sffamily\bfseries \color{word} In Mathematics};%
\end{tikzpicture}
\ \vspace{1cm}\\
\begin{minipage}{0.25\textwidth}
  \textcolor{bgc}{王胤雅是傻逼}
\end{minipage}
\begin{minipage}{0.75\textwidth}
  \maketitle
\end{minipage}
\vspace{4cm}\ \\
\begin{minipage}{0.2\textwidth}
  \
\end{minipage}
\begin{minipage}{0.8\textwidth}
  {\Huge
    \textinconsolatanf{}
  }General fire extinguisher
\end{minipage}
\newpage\backgroundsetup{contents={}}\setlength{\parindent}{2em}

  \newgeometry{left=2cm,right=2cm,top=2cm,bottom=2cm}
\else
\newgeometry{left=2cm,right=2cm,top=2cm,bottom=2cm}
\maketitle
\fi
%from_here_to_type
\section{problem}
\begin{problem}\label{pro:1}
 Let $\phi$ is representation of $\gl_n(K)$ over $K^n$. And $\phi(A)\alpha:=A \alpha$. Prove:$\phi$ is faithful and irreducible and $n-$dimentional. 
\end{problem}
\begin{solution}
 Obviously it's $n$-dimentional. If $A\neq B$, then exists $\alpha\in K^n$ s.t. $(A-B)\alpha\neq 0$. So $\phi(A)\alpha\neq \phi(B)\alpha$. So $\phi(A)\neq \phi(B)$, so $\phi$ is faithful. To prove $\phi$ is irreducible, we only need to prove there is no invariant subspace of $K^n$. Obviously for $\alpha,\beta\in K^n\minus\{0\}$, obviously there exists $A\in \gl_n(K)$ such that $A \alpha=\beta$. So there is no nontrival invariant subspace of $K^n$. So it's irreducible. 
\end{solution}

\begin{problem}
 For $A\in\gl_n(K)$, let $\psi(A)X=AX,\forall X\in M_n(K)$. Then:
 \begin{enumerate}
  \item $\psi$ is $n^2-$dimentional representation of $\gl_n(K)$ over $K$. 
  \item For $j:1\leq j\leq n$, let $M_n^{(j)}(K):=\{(a_{ik})_{n\times n}:a_{ik}\neq 0\to k=j\}$. Prove $M_n^{(j)}$ is invariant subspace of $\gl_n(K)$. Let $\psi_j$ is subrepresentation of $\psi$ in $M_n^{(j)}$, prove $\psi_j$ is irreducible and $\psi=\bigoplus_{j=1}^n \psi_j$. 
  \item\label{it:1.3} Prove $\psi_j\cong \phi$, where $\phi=(\Cref{pro:1}).\phi$
 \end{enumerate}
\end{problem}

\begin{solution}
 \begin{enumerate}
  \item Obviously $M_n(K)$ is $n^2$-dimentional, so $\psi$ is $n^2$-dimentional. Easily we have $\psi(AB)X=ABX=\psi(A)BX=\psi(A)\psi(B)X$, so $\psi(AB)=\psi(A)\psi(B)$. So $\psi$ is representation. 
  \item For $X\in M_n^{(j)},A\in \gl_n(K)$, we have $(AX)_{ik}=\sum_{t}a_{it}x_{tk}$. So for $k\neq j$ we have $(AX)_{ik}=\sum_{t}a_{it}\cdot 0=0$. So $AX\in M_n^{(j)}$. So $M_n^{(j)}$ is invariant subspace. Easily $M_n(K)=\bigoplus_{j=1}^n M_n^{(j)}$, so $\psi=\bigoplus_{j=1}^n \psi_j$. 
  From \Cref{it:1.3} we know $\psi_j\cong \phi$, and from \Cref{pro:1} we get $phi$ is irreducible, so $\psi_j$ is irreducible. 
  \item Consider $\tau:M_n^{(j)}\to K^n,(\tau(X))_k:=x_{jk}$, then $\tau$ is isomophism. Easily get $\tau$ is isomophism form $\psi_j$ to $(\Cref{pro:1}).\phi$. So $\psi_j\cong\phi$. 
 \end{enumerate}
\end{solution}

\begin{problem}
 Let $K=\mathbb{C}$ and $n=2$ in (Group representation second homework).(\Cref{pro:old3}), prove the subrepresentation of $\phi$ over $M_2^0(\mathbb{C})$ is irreducible.
\end{problem}

\begin{solution}
 Since every matrix in $M_2(\mathbb{C})$ can be diagonalized, so $\forall X\in M_2^0(\mathbb{C}),\exists A\in \gl_2(\mathbb{C}),\phi(A)X=\left(\begin{array}{cc}\lambda&0\\0&-\lambda\end{array}\right)=\lambda\left(\begin{array}{cc}1&0\\0&-1\end{array}\right)$. So for a invariant subspace $V$, we have $X\in V\to \left(\begin{array}{cc}1&0\\0&-1\end{array}\right)\in V\to \forall Y\in M_2^0(\mathbb{C}),Y\in V$. So $\phi$ is irreducible. 
\end{solution}

\begin{problem}
 Assume $n\geq 3$ and $n\nmid \char K$, prove: the $n-$dimentional permutate representation of $S_n$ can be decomposed as the direct sum of a main representation and a $(n-1)-$dimentional irreducible subrepresentation
\end{problem}

\begin{solution}
 In fact, \Cref{pro:old1} of second homework has given the decomposed. easily we get $\phi|_{V_1}$ is a main representation. And since $\dim V_2=n-1$, we get $\phi|_{V_2}$ is $(n-1)-$dimentional. So we only need to prove $\phi|_{V_2}$ is irreducible. Assume $V\subset V_2$ is a invariant subspace and $V\neq \{0\}$, consider $x=\sum_{i=1}^na_ix_i\in V\minus\{0\}$. Obviously all of $a's$ can't be equal because $nk=0\to k=0$ since $\char K\nmid n$. WLOG assume $a_1\neq a_2$. Then $y=a_1x_2+a_2x_1+\sum_{i=3}^na_ix_i=\phi((1\ 2))x\in V$, and thus $x-y=(a_1-a_2)(x_1-x_2)\in V,x_1-x_2\in V$. So $x_1-x_j=\phi((2\ j))(x_1-x_2)\in V,\forall j\geq 2$. Obviously these $n-1$ vecter is linearly independent, so $\dim V\geq n-1,V=V_2$. So $\phi|_{V_2}$ is irreducible. 
\end{solution}

\begin{problem}
 Caculate the $1- $ dimentional $\mathbb{C}$ representation:
 \begin{enumerate}
  \item $(2,4)-$type of $8-$order elementary Abel group.
  \item the addition group of $\mathbb{Z}_p^n$ 
 \end{enumerate} 
\end{problem}

\begin{solution}
 \begin{enumerate}
  \item Assume this group is $\mathbb{Z}_2\times \mathbb{Z}_4$. Then $\phi(k,j)=\mathrm{e}^{\frac{(2k+j)\mathrm{\pi}\mathrm{i}}{2}}$.
  \item $$\phi(a_1,a_2,\cdots ,a_n)=\mathrm{e}^{\frac{(2 \mathrm{\pi}\mathrm{i}\sum_{k=1}^na_k)}{p}}$$
 \end{enumerate}
\end{solution}

\section{appendix}
\setcounter{problem}{0}
\begin{problem}\label{pro:old1}
 Group $G$ has an action on set $\Omega=\left\{x_1, x_2, \cdots, x_n\right\}$, let $(\phi, V)$ be the $n-$ dimensional $K$ permutation representation of $G$, where $K$ is the field of vector space $V$, and 
 $$
V=\left\{\sum_{i=1}^n a_i x_i : a_i \in K, i=1,2, \cdots, n\right\} .
$$
Let
$$
\begin{aligned}
& V_1=\left\langle\sum_{i=1}^n x_i\right\rangle, \\
& V_2=\left\{\sum_{i=1}^n a_i x_i : \sum_{i=1}^n a_i=0, a_i \in K\right\} .
\end{aligned}
$$
Prove: \begin{enumerate}
 \item $V_1$ and $V_2$ are  invariant subspaces of $G$ ;
 \item  If $\char K \nmid n$, then $\varphi=\varphi_{V_1} \oplus \varphi_{V_2}$.
\end{enumerate}
\end{problem}
\setcounter{problem}{2}
\begin{problem}\label{pro:old3}
 $\mathcal{O}(n):=\{A\in M_n(\mathbb{R}):AA^T=I_n\}$ is the set of all $n$-dimensional otheretic matrix over $\mathbb{R}$. Let: 
 \begin{equation}
\begin{aligned}
  \varphi: \mathcal{O}(n) &\rightarrow \mathrm{GL}\left(M_n(\mathbb{R})\right) \\
  A &\mapsto \varphi(A), \\
\end{aligned}
 \end{equation}
 Where,
\begin{equation}
\varphi(A) X:=A X A^{-1}: \quad \forall X \in M_n(\mathbb{R}) 
\end{equation}
$M_n^{+}(\mathbb{R}):=\{A\in M_n^0(\mathbb{R}): A=A^T\}$, $M_n^{-}(\mathrm{R}):=\{A\in M_n^0(\mathbb{R}): A^T=-A\}$.
\begin{enumerate}
\item Proof: $M_n^{+}(\mathrm{R})$ and $M_n^{-}(\mathrm{R})$ are invariant spaces of $\varphi$;
\item Let the subrepresentation of $\varphi$ on $\langle I \rangle$,$ M_n^{+}(\mathbb{R}), M_n^{-}(\mathbb{R})$ be  $\varphi_0, \varphi_1, \varphi_2$. Proof:
$$
\varphi=\varphi_0\oplus\varphi_1\oplus\varphi_2
$$
\item calculate a $\frac{1}{2} n(n-1)-$ dimensional $ \mathbb{R}$ representation of $\mathcal{O}(n)$.
\end{enumerate}
\end{problem}
\end{document}