%!Mode:: "TeX:UTF-8"
%!TEX encoding = UTF-8 Unicode
%!TEX TS-program = xelatex
\documentclass{ctexart}
\usepackage{bbm}
\usepackage{tikz}
\usepackage{amsmath,amssymb,amsthm,color,mathrsfs}
\usepackage{hyperref}
\usepackage{cleveref}
\usepackage{enumitem,anysize}%
\usepackage{expl3}
\marginsize{1in}{1in}{1in}{1in}%
\theoremstyle{remark}
\newtheorem*{claim*}{Claim}
\newtheorem{lemma}{Lemma}

\DeclareMathOperator{\ot}{ordertype}
\DeclareMathOperator{\dom}{dom}
\DeclareMathOperator{\ran}{ran}
\DeclareMathOperator{\field}{field}
\DeclareMathOperator{\Ord}{Ord}

\newcommand{\cc}{\mathfrak{c}}

\def\email#1{{\texttt{#1}}}
\newcommand{\isep}[1][0pt]{\addtolength{\itemsep}{#1}}
\newcounter{problem}
\renewcommand{\theproblem}{\Roman{problem}}
\newenvironment{problem}{\refstepcounter{problem}\noindent\color{blue}\begin{tikzpicture}[baseline]%
\node at (-0.02em,0.3em) {$\mathbb{P}$};%
\node[scale=0.7] at (0.2em,-0.0em) {R};%
\node[scale=0.7] at (0.6em,0.4em) {O};%
\node[scale=0.8] at (1.05em,0.25em) {B};%
\node at (1.55em,0.3em) {L};%
\node[scale=0.7] at (1.75em,0.45em) {E};%
\node at (2.35em,0.3em) {M};%
\end{tikzpicture}\theproblem}{}
\crefname{problem}{Problem}{Problem}
%\renewcommand\theprob{{\Roman{problem}}}
\newcommand\mysolution{\begin{tikzpicture}[baseline]%
\node at (-0.04em,0.3em) {$\mathbb{S}$};%
\node[scale=0.7] at (0.35em,0.4em) {O};%
\node at (0.7em,0.3em) {\textit{L}};%
\node[scale=0.7] at (0.95em,0.4em) {U};%
\node[scale=1.1] at (1.19em,0.32em){T};%
\node[scale=0.85] at (1.4em,0.24em){I};%
\node at (1.9em,0.32em){$\mathcal{O}$};%
\node[scale=0.75] at (2.3em,0.21em){\texttt{N}};%
\end{tikzpicture}}
\newenvironment{solution}{\begin{proof}[\mysolution]}{\end{proof}}

%%%%%%%%
\newcommand{\calL}{\mathcal{L}}

\newcommand\<{\langle}
\renewcommand\>{\rangle}
\newcommand\eneg{\mathcal{E}_{\neg}}
\newcommand\eto{\mathcal{E}_{\to}}
\newcommand\N{\mathbb{N}}
\newcommand\subini{\subsetneqq_{init}}
\def\to{\rightarrow}
\newcommand{\calA}{\mathcal{A}}
\newcommand{\xor}{\vee}
\newcommand{\bor}{\bigvee}
\newcommand{\band}{\bigwedge}
\newcommand{\xand}{\wedge}
\newcommand{\minus}{\mathbin{\backslash}}
\newcommand{\calF}{\mathcal{F}}
\newcommand\mi[1]{\mathscr{P}(#1)}
\newcommand{\calB}{\mathcal{B}}
\newcommand{\heq}{\mathop{\hat{=}}}
\newcommand{\hneq}{\mathop{\hat{\neq}}}
\newcommand{\calR}{\mathcal{R}}
\newcommand{\hle}{\mathop{\hat{<}}}
\newcommand{\R}{\mathbb{R}}
\renewcommand{\C}{\mathbb{C}}
\newcommand{\calM}{\mathcal{M}}
\newcommand{\calN}{\mathcal{N}}
\newcommand{\frA}{\mathfrak{A}}
\newcommand{\card}{\mathrm{card}}
\newcommand{\frM}{\mathfrak{M}}
\newcommand{\Th}{\mathrm{Th}}
\newcommand{\ecd}{\mathrm{EC}_\Delta}
\newcommand{\Q}{\mathbb{Q}}
\newcommand{\Z}{\mathbb{Z}}
\newcommand{\oto}{\leftrightarrow}
\newcommand{\hin}{\hat{\in}}
\newcommand{\fun}[2]{{}^{#1}#2}
\newcommand{\A}{\mathbbm{A}}
\newcommand{\e}{\mathrm{e}}
\newcommand{\gl}{\mathrm{GL}}
\newcommand{\im}{\mathrm{Im}}
\newcommand{\re }{\mathrm{Re }}
\newcommand{\rank}{\mathrm{rank}}
\newcommand{\tra}{\mathop{\mathrm{tr}}}
\renewcommand{\char}{\mathop{\mathrm{char}}}
\renewcommand{\phi }{\varphi}







\begin{document}
\title{$\mathbb{GROUP\ REPRESENTATION}$}
\author{白永乐\\ %% 姓名
SID: 202011150087\\ %% 学号
\email{202011150087"mail.bnu.edu.cn}} %% 电邮
\maketitle
\begin{problem}\label{pro:1}
Group $G$ has an action on set $\Omega=\left\{x_1, x_2, \cdots, x_n\right\}$, let $(\phi, V)$ be the $n-$ dimensional $K$ permutation representation of $G$, where $K$ is the field of vector space $V$, and 
$$
V=\left\{\sum_{i=1}^n a_i x_i : a_i \in K, i=1,2, \cdots, n\right\} .
$$
Let
$$
\begin{aligned}
& V_1=\left\langle\sum_{i=1}^n x_i\right\rangle, \\
& V_2=\left\{\sum_{i=1}^n a_i x_i : \sum_{i=1}^n a_i=0, a_i \in K\right\} .
\end{aligned}
$$
Prove: \begin{enumerate}
\item $V_1$ and $V_2$ are  invariant subspaces of $G$ ;
\item  If $\char K \nmid n$, then $\varphi=\varphi_{V_1} \oplus \varphi_{V_2}$.
\end{enumerate}
\end{problem}





\begin{problem}
Using exercise $1$, calculate a $2-$dimensional complex representation of $S_3$ and its matrix of the representation.
\end{problem}




\begin{problem}\label{pro:2}
$M_{\mathrm{n}}(K):=\{(a_{i,j})_{n\times n}:a_{ij}\in K,\forall 1\leq i,j\leq n\}$.
Let
$$
\begin{aligned}
\varphi: \mathrm{GL}_n(K) & \rightarrow \mathrm{GL}\left(M_n(K)\right) \\
A & \rightarrow \varphi(A),
\end{aligned}
$$
where
$$
\varphi(A) X:=A X A^{-1} ; \quad \forall X \in M_n(K) .
$$
\begin{enumerate}
\item Illustrate $\varphi$ is the $n^2$-dimensional $K$ representation of group  $\mathrm{GL}_n(K)$;
\item $M_n^0(K):=\{A\in M_n(K):\tra A=0\}$. Illustrate $M_n^0(K)$ and $\langle I\rangle$ are  invariant subspaces of $\varphi$;
\item Prove: If $\char K\nmid n$, then
$$
\varphi=\varphi_{\langle I\rangle} \oplus \varphi_{M^{0}_n(K)}
$$
\end{enumerate}

\end{problem}


\begin{problem}
$\mathcal{O}(n):=\{A\in M_n(\R):AA^T=I_n\}$ is the set of all $n$-dimensional otheretic matrix over $\R$. Let: 
\begin{equation}
\begin{aligned}
\varphi: \mathcal{O}(n) &\rightarrow \mathrm{GL}\left(M_n(\mathbb{R})\right) \\
A &\mapsto \varphi(A), \\
\end{aligned}
\end{equation}
Where,
\begin{equation}
\varphi(A) X:=A X A^{-1}: \quad \forall X \in M_n(\mathbb{R}) 
\end{equation}
$M_n^{+}(\mathbb{R}):=\{A\in M_n^0(\R): A=A^T\}$, $M_n^{-}(\mathrm{R}):=\{A\in M_n^0(\R): A^T=-A\}$.
\begin{enumerate}
\item Proof: $M_n^{+}(\mathrm{R})$ and $M_n^{-}(\mathrm{R})$ are invariant spaces of $\varphi$;
\item Let the subrepresentation of $\varphi$ on $\langle I \rangle$,$ M_n^{+}(\mathbb{R}), M_n^{-}(\mathbb{R})$ be  $\varphi_0, \varphi_1, \varphi_2$. Proof:
$$
\varphi=\varphi_0\oplus\varphi_1\oplus\varphi_2
$$
\item calculate a $\frac{1}{2} n(n-1)-$ dimensional $ \R$ representation of $\mathcal{O}(n)$.
\end{enumerate}
\end{problem}


\end{document}
