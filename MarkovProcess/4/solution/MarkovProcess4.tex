%!Mode:: "TeX:UTF-8"
%!TEX TS-program = xelatex
\documentclass{ctexart}
\newif\ifpreface
%\prefacetrue
\usepackage{fontspec}
\usepackage{bbm}
\usepackage{tikz}
\usepackage{amsmath,amssymb,amsthm,color,mathrsfs}
\usepackage{fixdif}
\usepackage{hyperref}
\usepackage{cleveref}
\usepackage{enumitem}%
\usepackage{expl3}
\usepackage{lipsum}
\usepackage[margin=0pt]{geometry}
\usepackage{listings}
\definecolor{mGreen}{rgb}{0,0.6,0}
\definecolor{mGray}{rgb}{0.5,0.5,0.5}
\definecolor{mPurple}{rgb}{0.58,0,0.82}
\definecolor{backgroundColour}{rgb}{0.95,0.95,0.92}

\lstdefinestyle{CStyle}{
  backgroundcolor=\color{backgroundColour},
  commentstyle=\color{mGreen},
  keywordstyle=\color{magenta},
  numberstyle=\tiny\color{mGray},
  stringstyle=\color{mPurple},
  basicstyle=\footnotesize,
  breakatwhitespace=false,
  breaklines=true,
  captionpos=b,
  keepspaces=true,
  numbers=left,
  numbersep=5pt,
  showspaces=false,
  showstringspaces=false,
  showtabs=false,
  tabsize=2,
  language=C
}
\usetikzlibrary{calc}
\theoremstyle{remark}
\newtheorem{lemma}{Lemma}
\usepackage{fontawesome5}
\usepackage{xcolor}
\newcounter{problem}
\newcommand{\Problem}{\begin{tikzpicture}[baseline]%
    \node at (-0.02em,0.3em) {$\mathbb{P}$};%
    \node[scale=0.7] at (0.2em,-0.0em) {R};%
    \node[scale=0.7] at (0.6em,0.4em) {O};%
    \node[scale=0.8] at (1.05em,0.25em) {B};%
    \node at (1.55em,0.3em) {L};%
    \node[scale=0.7] at (1.75em,0.45em) {E};%
    \node at (2.35em,0.3em) {M};%
  \end{tikzpicture}%
}
\renewcommand{\theproblem}{\Roman{problem}}
\newenvironment{problem}{\refstepcounter{problem}\noindent\color{blue}\Problem\theproblem}{}

\crefname{problem}{\protect\Problem}{Problem}
\newcommand\Solution{\begin{tikzpicture}[baseline]%
    \node at (-0.04em,0.3em) {$\mathbb{S}$};%
    \node[scale=0.7] at (0.35em,0.4em) {O};%
    \node at (0.7em,0.3em) {\textit{L}};%
    \node[scale=0.7] at (0.95em,0.4em) {U};%
    \node[scale=1.1] at (1.19em,0.32em){T};%
    \node[scale=0.85] at (1.4em,0.24em){I};%
    \node at (1.9em,0.32em){$\mathcal{O}$};%
    \node[scale=0.75] at (2.3em,0.21em){\texttt{N}};%
  \end{tikzpicture}}
\newenvironment{solution}{\begin{proof}[\Solution]}{\end{proof}}
\title{\input{../../.subject}\input{../.number}}
\makeatletter
\newcommand\email[1]{\def\@email{#1}\def\@refemail{mailto:#1}}
\newcommand\schoolid[1]{\def\@schoolid{#1}}
\ifpreface
  \def\@maketitle{
  \raggedright
  {\Huge \bfseries \sffamily \@title }\\[1cm]
  {\Huge  \bfseries \sffamily\heiti\@author}\\[1cm]
  {\Huge \@schoolid}\\[1cm]
  {\Huge\href\@refemail\@email}\\[0.5cm]
  \Huge\@date\\[1cm]}
\else
  \def\@maketitle{
    \raggedright
    \begin{center}
      {\Huge \bfseries \sffamily \@title }\\[4ex]
      {\Large  \@author}\\[4ex]
      {\large \@schoolid}\\[4ex]
      {\href\@refemail\@email}\\[4ex]
      \@date\\[8ex]
    \end{center}}
\fi
\makeatother
\ifpreface
  \usepackage[placement=bottom,scale=1,opacity=1]{background}
\fi

\author{白永乐}
\schoolid{202011150087}
\email{202011150087@mail.bnu.edu.cn}

\def\to{\rightarrow}
\newcommand{\xor}{\vee}
\newcommand{\bor}{\bigvee}
\newcommand{\band}{\bigwedge}
\newcommand{\xand}{\wedge}
\newcommand{\minus}{\mathbin{\backslash}}
\newcommand{\mi}[1]{\mathscr{P}(#1)}
\newcommand{\card}{\mathrm{card}}
\newcommand{\oto}{\leftrightarrow}
\newcommand{\hin}{\hat{\in}}
\newcommand{\gl}{\mathrm{GL}}
\newcommand{\im}{\mathrm{Im}}
\newcommand{\re }{\mathrm{Re }}
\newcommand{\rank}{\mathrm{rank}}
\newcommand{\tra}{\mathop{\mathrm{tr}}}
\renewcommand{\char}{\mathop{\mathrm{char}}}
\DeclareMathOperator{\ot}{ordertype}
\DeclareMathOperator{\dom}{dom}
\DeclareMathOperator{\ran}{ran}

\begin{document}
\large
\iffalse
  \setlength{\baselineskip}{1.2em}
  \ifpreface
    \backgroundsetup{contents={%
    \begin{tikzpicture}
      \fill [white] (current page.north west) rectangle ($(current page.north east)!.3!(current page.south east)$) coordinate (a);
      \fill [bgc] (current page.south west) rectangle (a);
\end{tikzpicture}}}
\definecolor{word}{rgb}{1,1,0}
\definecolor{bgc}{rgb}{1,0.95,0}
\setlength{\parindent}{0pt}
\thispagestyle{empty}
\begin{tikzpicture}%
  % \node[xscale=2,yscale=4] at (0cm,0cm) {\sffamily\bfseries \color{word} under};%
  \node[xscale=4.5,yscale=10] at (10cm,1cm) {\sffamily\bfseries \color{word} Graduate Homework};%
  \node[xscale=4.5,yscale=10] at (8cm,-2.5cm) {\sffamily\bfseries \color{word} In Mathematics};%
\end{tikzpicture}
\ \vspace{1cm}\\
\begin{minipage}{0.25\textwidth}
  \textcolor{bgc}{王胤雅是傻逼}
\end{minipage}
\begin{minipage}{0.75\textwidth}
  \maketitle
\end{minipage}
\vspace{4cm}\ \\
\begin{minipage}{0.2\textwidth}
  \
\end{minipage}
\begin{minipage}{0.8\textwidth}
  {\Huge
    \textinconsolatanf{}
  }General fire extinguisher
\end{minipage}
\newpage\backgroundsetup{contents={}}\setlength{\parindent}{2em}

  \else
    \maketitle
  \fi
\fi
\newgeometry{left=2cm,right=2cm,top=2cm,bottom=2cm}
%from_here_to_type
\begin{lemma}\label{lem:util}
  Assume \((N_t:t \geq 0)\) is a random process, and \(\mathbb{P}(\forall t \in [0,\infty),\lim_{s \to t+}N_s =N_t,\forall t \in (0,\infty),\lim_{s \to t-}N_s \in \mathbb{R})=1\).
  Assume \(\alpha >0\), and \(\forall t,s \geq 0\), we have \(N_{t + s}-N_s \sim Possion(\alpha t)\).
  Then \((N_t:t \geq 0)\) is a Possion process \(\iff \forall 0 \leq s \leq t,N_t-N_s \perp \mathcal{F}_s\),
  where \(\mathcal{F}_s:=\sigma(N_x:x \leq s)\).
\end{lemma}
\begin{proof}
  ``\(\implies\)'': To prove \(N_t-N_s \perp \mathcal{F}_s\), only need to prove for \(t_1 < t_2 < \cdots <t_{n-1}=s<t=t_n\), we have
  \(N_t-N_s \perp \sigma(N_{t_k}:k=1,\cdots,n-1)\).
  Easily \(N_t-N_s \perp \sigma(N_{t_{k+1}}-N_{t_k},N_{t_1}:k=1,\cdots,n-2)=\sigma(N_{t_k}:k=1,\cdots,n-1)\), so \(N_t-N_s \perp \mathcal{F}_s\).

  ``\(\impliedby\)'': For \(t_1 < \cdots < t_n\), we need to prove \(N_{t_1},N_{t_{k+1}}-N_{t_k}:1 \leq k \leq n-1\) are independent.
  Use MI to \(n\). When \(n=1\) it's obvious.
  Assume we have proved it for certain \(n \geq 1\), now consider \(n+1\).
  Since \(N_{t_{k+1}}-N_{t_k} \in \mathcal{F}_{t_n},k=1,\cdots,n-1\), we have \(N_{t_{n+1}}-N_{t_n} \perp \sigma(N_{t_1},N_{t_{k+1}}-N_{t_k}:k=1,\cdots,n-1)\).
  So \(\mathbb{P}(N_{t_1} \in A_1,N_{t_{k+1}}-N_{t_k} \in A_{k+1},k=1,\cdots,n)=\mathbb{P}(N_{t_1} \in A_1,N_{t_{k+1}}-N_{t_k} \in A_{k+1},k=1,\cdots,n-1)\mathbb{P}(N_{t_{n+1}}-N_{t_n} \in A_{n+1})\).
  By Induction assumption we get \(\mathbb{P}(N_{t_1} \in A_1,N_{t_{k+1}}-N_{t_k} \in A_{k+1},k=1,\cdots,n)=\mathbb{P}(N_{t_1}\in A_1) \prod_{k=1}^{n}\mathbb{P}(N_{t_{k+1}}-N_{t_k} \in A_{k+1})\).
  So finally we get \(N_{t_1},N_{t_{k+1}}-N_{t_k}:1 \leq k \leq n\) are independent.
\end{proof}

\begin{problem}\label{pro:1}
  Assume \((N_t:t \geq 0)\) is a Possion process with parameter \(\alpha\).
  Let \(P(t):=\mathbb{P}(2 \nmid N_t),Q(t):=\mathbb{P}(2 \mid N_t)\).
  Prove that \(P(t)=\mathrm{e}^{-\alpha t}\sinh(\alpha t),Q(t)=\mathrm{e}^{-\alpha t}\cosh(\alpha t)\).
\end{problem}
\begin{solution}
  Easily to get
  \[
    P(t)=\sum_{k=0}^{\infty}\mathbb{P}(N_t=2k+1)=\sum_{k=0}^{\infty}\frac{(\alpha t)^{2k=1}}{(2k+1)!}\mathrm{e}^{-\alpha t}
  \]
  Noting that \(\sinh(\alpha t)=\frac{1}{2}\left(\sum_{k=0}^{\infty}\frac{(\alpha t)^n}{n!}-\sum_{k=0}^{\infty}\frac{(-\alpha t)^n}{n!}\right)=\sum_{k=0}^{\infty}\frac{(\alpha t)^{2k+1}}{(2k+1)!}\),
  we easily obtain \(P(t)=\mathrm{e}^{-\alpha t}\sinh(\alpha t)\).
  Noting \(P(t)+ Q(t)=1\), we easily get \(Q(t)=1-P(t)=\mathrm{e}^{-\alpha t}\cosh(\alpha t)\).
\end{solution}
\begin{problem}\label{pro:2}
  Assume \((N_t:t \geq 0)\) is a Possion process with parameter \(\alpha\).
  Prove that \(\lim_{t \to \infty}\frac{N_t}{t}=\alpha,a.s.\).
\end{problem}
\begin{lemma}\label{lem:1}
  Assume \((N_t:t \geq 0)\) is a Possion process with parameter \(\alpha\).
  Then \(\mathbb{P}(\forall 0 \leq s \leq t,N_s \leq N_t)=1\).
\end{lemma}
\begin{proof}
  For \(s,t \in \mathbb{Q} \AND 0 \leq s \leq t\), we have \(\mathbb{P}(N_s > N_t)=0\) since \(N_t - N_s \sim Possion(\alpha(t-s))\).
  So we get \(\mathbb{P}(\exists s,t \in \mathbb{Q},0 \leq s \leq t,N_s>N_t)=0\).

  Now we will prove \(\exists s,t \in \mathbb{R},0 \leq s \leq t,N_s>N_t \implies \exists a,b \in \mathbb{Q},0 \leq a \leq b,N_a > N_b\).
  Let \(a_n= \frac{\ceil{ns}}{n},b_n=\frac{\ceil{nt}}{n}\). Then \(\lim a_n=s,\lim b_n=t\). Easily \(a_n \geq s,b_n \geq t\).
  So since \(N\) is continuous we get \(\lim N_{a_n}=N_s,\lim N_{b_n}=N_t\).
  Since \(N_s>N_t\), we get \(\exists n,N_{a_n}>N_{b_n}\). Let \(a=a_n,b=b_n\) will work.

  So \(\mathbb{P}(\forall 0 \leq s \leq t,N_s \leq N_t)=1-\mathbb{P}(\exists 0 \leq s \leq t,N_s>N_t)=1-\mathbb{P}(\exists s,t \in \mathbb{Q},0 \leq s \leq t,N_s>N_t)=1-0=1\).
\end{proof}

\begin{solution}
  Consider \(N_n:n \in \mathbb{N}\). Let \(X_n:=N_n-N_{n-1},n \geq 1\).
  Then easily \((X_n:n \in \mathbb{N}^+)\) is i.i.d sequence and \(X_1 \sim Possion(\alpha)\).
  So from the strong law of large numbers we get \(\lim_{n \to \infty}\frac{N_n}{n}=\alpha\).
  From Lemma \ref{lem:1} we get \(\frac{\floor{t}}{t} \frac{N_{\floor{t}}}{\floor{t}} \leq \frac{N_t}{t} \leq \frac{N_{\ceil{t}}}{\ceil{t}} \frac{\ceil{t}}{t}, \forall t \in \mathbb{R}\),
  let \(t \to \infty\) we get \(\floor{t},\ceil{t} \to \infty\), and \(\floor{t} \sim t \sim \ceil{t}\).
  So finally we get \(\lim_{t \to \infty} \frac{N_t}{t}=\alpha\).
\end{solution}
\begin{problem}\label{pro:3}
  Assume \((N_t:t \geq 0)\) is a Possion process with parameter \(\alpha>0\).
  Prove that \(\frac{N_t-\alpha t}{\sqrt{ \alpha t}} \overset{d}{\to} N(0,1)\).
\end{problem}
\begin{solution}
  Consider \(N_n:n \in \mathbb{N}\). Let \(X_n:=N_n-N_{n-1},n \geq 1\).
  Then easily \((X_n:n \in \mathbb{N}^+)\) is i.i.d sequence and \(X_1 \sim Possion(\alpha)\).
  Easily \(\mathbb{V}(X_n)=\alpha<\infty,\mathbb{E}(X_n)=\alpha\).
  So from the central limit theorem we get \( \frac{N_n-\alpha n}{\sqrt{ \alpha n}} \overset{d}{\to} N(0,1)\).
  Noting \(\frac{N_t - \alpha t}{\sqrt{ \alpha t}}=\frac{N_{\floor{t}}-\alpha \floor{t}}{\sqrt{ \alpha \floor{t}}} \frac{\sqrt{\floor{t}}}{\sqrt{t}} + \frac{N_t-N_{\floor{t}}-\alpha(t-\floor{t})}{\sqrt{\alpha t}}\).
  Let \(t \to \infty\) we get \(\floor{t}, \to \infty\), and \(\floor{t} \sim t\).
  Noting \(N_t-N_{\floor{t}}\overset{d}{=}N_{t-\floor{t}}\), and \(t - \floor{t} \leq 1\), we easily get \(\frac{N_t-N_{\floor{t}}}{\sqrt{ \alpha t}} \overset{d}{\to} 0\).
  Easily \(\frac{\alpha(t-\floor{t})}{\alpha t} \to 0\), so finally we get that
  \(\frac{N_t-\alpha t}{\sqrt{ \alpha t}} \overset{d}{\to} N(0,1)\)
\end{solution}
\begin{problem}\label{pro:4}
  Assume \((X_t:t \geq 0),(Y_t:t \geq 0)\) are two independent Possion processes with parameter \(\alpha,\beta\) respectively.
  Prove that \((X_t + Y_t:t \geq 0)\) is Possion process with parameter \(\alpha + \beta\).
\end{problem}
\begin{solution}
  Write \(Z_t=X_t + Y_t\). First we prove \(Z_{t+s}-Z_s \sim Possion((\alpha + \beta)t)\).
  Since \(X_{t + s}-X_s \sim Possion(\alpha t),Y_{t+s}-Y_s \sim Possion(\beta t)\), and \(X_{t + s}-X_s \perp Y_{s + t}-Y_s\), easily to get
  \(Z_{t + s}-Z_{s}=X_{t + s}-X_s+Y_{s + t}-Y_s \sim Possion((\alpha + \beta)t)\).

  Second we prove \(\forall 0 \leq s \leq t,Z_t-Z_s \perp \mathcal{H}_s\), where \(\mathcal{H}_s =\sigma(Z_x:0 \leq x \leq s)\).
  Easily \(Z_t-Z_s \in \sigma(X_t-X_s,Y_t-Y_s)\).
  Easily \(X_t-X_s \perp \mathcal{F}_s:=\sigma(X_x:0 \leq x \leq s)\) since \((X_x:x \geq 0)\) is Possion process.
  Since \((X_x:x \geq 0)\perp (Y_x:x \geq 0)\), we get \(X_t-X_s \perp \mathcal{G}_s:=\sigma(Y_x:0 \leq x \leq s)\).
  For the same reason we get \(Y_t-Y_s \perp \mathcal{F}_s,\mathcal{G}_s\).
  So we get \(Z_t-Z_s \perp \sigma(F_s,G_s) \supset \mathcal{H}_s\).

  Finally, we prove that \(\mathbb{P}(\forall t \in [0,\infty),\lim_{s \to t+}Z_s =Z_t,\forall t \in (0,\infty),\lim_{s \to t-}Z_s \in \mathbb{R})=1\)
  Since \(Z_t=X_t + Y_t\), and
  \(\mathbb{P}(\forall t \in [0,\infty),\lim_{s \to t+}Y_s =Y_t,\forall t \in (0,\infty),\lim_{s \to t-}Y_s \in \mathbb{R})=1\),
  \(\mathbb{P}(\forall t \in [0,\infty),\lim_{s \to t+}X_s =X_t,\forall t \in (0,\infty),\lim_{s \to t-}X_s \in \mathbb{R})=1\),
  obvious we get the requirement.

  All in all, \((X_t + Y_t:t \geq 0)\) is a Possion process with parameter \(\alpha + \beta\).
\end{solution}
\begin{problem}\label{pro:5}
  Assume \((\xi_n:n \in \mathbb{N}^+)\) is a sequence of i.i.d. random variable ranging in \(\mathbb{Z}^d\).
  Let \(X_n=X_0 + \sum_{k=1}^{n}\xi_k\), and \(X_0 \perp (\xi_n:n \in \mathbb{N}^+)\) ranging in \(\mathbb{Z}^d\), too.
  Assume \((N_t:t \geq 0)\) is a Possion process with parameter \(\alpha>0\).
  Discuss \(\frac{X_{N_t}}{t}\) when \(t \to \infty\).
\end{problem}
\begin{solution}
  First we prove that \(\lim_{t \to \infty}N_t=\infty,a. s.\).
  We have \(\mathbb{P}(\sup_{t}N_t \geq n)\geq \mathbb{P}(N_t \geq n),\forall t ,\forall n \in \mathbb{N}\).
  Easily \(\lim_{t \to \infty}\mathbb{P}(N_t \geq n)=1\), so \(\mathbb{P}(\sup_{t}N_t \geq n)=1,\forall n \in \mathbb{N}\).
  So \(\mathbb{P}(\sup_{t}N_t=\infty)=1\).
  Noting Lemma \ref{lem:1} we easily get \(\mathbb{P}(\lim_{t \to \infty}N_t = \infty)=1\).

  Now we can decompose \(\frac{X_{N_t}}{t}\) into \(\frac{X_{N_t}}{N_t} \frac{N_t}{t}\).
  We have proved that \(\frac{N_t}{t} \to \alpha,a. s.\) in Problem \ref{pro:2}, so we only need to find \(\frac{X_{N_t}}{N_t}\).
  Since \(N_t \to \infty,a. s.\), we only need to find \(\frac{X_n}{n}\) when \(n \to \infty\).

  If \(\mathbb{E}(\xi_1)\) exists, then easily \(\frac{X_n}{n} \to \mathbb{E}(\xi_1),a. s.\).
  Then we easily get \(\frac{X_{N_t}}{t} \to \alpha \mathbb{E}(\xi_1),a. s.\).
\end{solution}

\end{document}
