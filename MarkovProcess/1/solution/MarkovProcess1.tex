%!Mode:: "TeX:UTF-8"
%!TEX TS-program = xelatex
\documentclass{ctexart}
\newif\ifpreface
\prefacetrue
\usepackage{fontspec}
\usepackage{bbm}
\usepackage{tikz}
\usepackage{amsmath,amssymb,amsthm,color,mathrsfs}
\usepackage{fixdif}
\usepackage{hyperref}
\usepackage{cleveref}
\usepackage{enumitem}%
\usepackage{expl3}
\usepackage{lipsum}
\usepackage[margin=0pt]{geometry}
\usepackage{listings}
\definecolor{mGreen}{rgb}{0,0.6,0}
\definecolor{mGray}{rgb}{0.5,0.5,0.5}
\definecolor{mPurple}{rgb}{0.58,0,0.82}
\definecolor{backgroundColour}{rgb}{0.95,0.95,0.92}

\lstdefinestyle{CStyle}{
  backgroundcolor=\color{backgroundColour},
  commentstyle=\color{mGreen},
  keywordstyle=\color{magenta},
  numberstyle=\tiny\color{mGray},
  stringstyle=\color{mPurple},
  basicstyle=\footnotesize,
  breakatwhitespace=false,
  breaklines=true,
  captionpos=b,
  keepspaces=true,
  numbers=left,
  numbersep=5pt,
  showspaces=false,
  showstringspaces=false,
  showtabs=false,
  tabsize=2,
  language=C
}
\usetikzlibrary{calc}
\theoremstyle{remark}
\newtheorem{lemma}{Lemma}
\usepackage{fontawesome5}
\usepackage{xcolor}
\newcounter{problem}
\newcommand{\Problem}{\begin{tikzpicture}[baseline]%
    \node at (-0.02em,0.3em) {$\mathbb{P}$};%
    \node[scale=0.7] at (0.2em,-0.0em) {R};%
    \node[scale=0.7] at (0.6em,0.4em) {O};%
    \node[scale=0.8] at (1.05em,0.25em) {B};%
    \node at (1.55em,0.3em) {L};%
    \node[scale=0.7] at (1.75em,0.45em) {E};%
    \node at (2.35em,0.3em) {M};%
  \end{tikzpicture}%
}
\renewcommand{\theproblem}{\Roman{problem}}
\newenvironment{problem}{\refstepcounter{problem}\noindent\color{blue}\Problem\theproblem}{}

\crefname{problem}{\protect\Problem}{Problem}
\newcommand\Solution{\begin{tikzpicture}[baseline]%
    \node at (-0.04em,0.3em) {$\mathbb{S}$};%
    \node[scale=0.7] at (0.35em,0.4em) {O};%
    \node at (0.7em,0.3em) {\textit{L}};%
    \node[scale=0.7] at (0.95em,0.4em) {U};%
    \node[scale=1.1] at (1.19em,0.32em){T};%
    \node[scale=0.85] at (1.4em,0.24em){I};%
    \node at (1.9em,0.32em){$\mathcal{O}$};%
    \node[scale=0.75] at (2.3em,0.21em){\texttt{N}};%
  \end{tikzpicture}}
\newenvironment{solution}{\begin{proof}[\Solution]}{\end{proof}}
\title{\input{../../.subject}\input{../.number}}
\makeatletter
\newcommand\email[1]{\def\@email{#1}\def\@refemail{mailto:#1}}
\newcommand\schoolid[1]{\def\@schoolid{#1}}
\ifpreface
  \def\@maketitle{
  \raggedright
  {\Huge \bfseries \sffamily \@title }\\[1cm]
  {\Huge  \bfseries \sffamily\heiti\@author}\\[1cm]
  {\Huge \@schoolid}\\[1cm]
  {\Huge\href\@refemail\@email}\\[0.5cm]
  \Huge\@date\\[1cm]}
\else
  \def\@maketitle{
    \raggedright
    \begin{center}
      {\Huge \bfseries \sffamily \@title }\\[4ex]
      {\Large  \@author}\\[4ex]
      {\large \@schoolid}\\[4ex]
      {\href\@refemail\@email}\\[4ex]
      \@date\\[8ex]
    \end{center}}
\fi
\makeatother
\ifpreface
  \usepackage[placement=bottom,scale=1,opacity=1]{background}
\fi

\author{白永乐}
\schoolid{202011150087}
\email{202011150087@mail.bnu.edu.cn}

\def\to{\rightarrow}
\newcommand{\xor}{\vee}
\newcommand{\bor}{\bigvee}
\newcommand{\band}{\bigwedge}
\newcommand{\xand}{\wedge}
\newcommand{\minus}{\mathbin{\backslash}}
\newcommand{\mi}[1]{\mathscr{P}(#1)}
\newcommand{\card}{\mathrm{card}}
\newcommand{\oto}{\leftrightarrow}
\newcommand{\hin}{\hat{\in}}
\newcommand{\gl}{\mathrm{GL}}
\newcommand{\im}{\mathrm{Im}}
\newcommand{\re }{\mathrm{Re }}
\newcommand{\rank}{\mathrm{rank}}
\newcommand{\tra}{\mathop{\mathrm{tr}}}
\renewcommand{\char}{\mathop{\mathrm{char}}}
\DeclareMathOperator{\ot}{ordertype}
\DeclareMathOperator{\dom}{dom}
\DeclareMathOperator{\ran}{ran}

\begin{document}
\large
\setlength{\baselineskip}{1.2em}
\ifpreface
	\backgroundsetup{contents={%
    \begin{tikzpicture}
      \fill [white] (current page.north west) rectangle ($(current page.north east)!.3!(current page.south east)$) coordinate (a);
      \fill [bgc] (current page.south west) rectangle (a);
\end{tikzpicture}}}
\definecolor{word}{rgb}{1,1,0}
\definecolor{bgc}{rgb}{1,0.95,0}
\setlength{\parindent}{0pt}
\thispagestyle{empty}
\begin{tikzpicture}%
  % \node[xscale=2,yscale=4] at (0cm,0cm) {\sffamily\bfseries \color{word} under};%
  \node[xscale=4.5,yscale=10] at (10cm,1cm) {\sffamily\bfseries \color{word} Graduate Homework};%
  \node[xscale=4.5,yscale=10] at (8cm,-2.5cm) {\sffamily\bfseries \color{word} In Mathematics};%
\end{tikzpicture}
\ \vspace{1cm}\\
\begin{minipage}{0.25\textwidth}
  \textcolor{bgc}{王胤雅是傻逼}
\end{minipage}
\begin{minipage}{0.75\textwidth}
  \maketitle
\end{minipage}
\vspace{4cm}\ \\
\begin{minipage}{0.2\textwidth}
  \
\end{minipage}
\begin{minipage}{0.8\textwidth}
  {\Huge
    \textinconsolatanf{}
  }General fire extinguisher
\end{minipage}
\newpage\backgroundsetup{contents={}}\setlength{\parindent}{2em}

\else
	\maketitle
\fi
\newgeometry{left=2cm,right=2cm,top=2cm,bottom=2cm}
%from_here_to_type
\begin{problem}\label{pro:1}
Assume \((\mathscr{F}_t:t \geq 0,t \in \mathbb{R})\) is a filtration.
For \(t \geq 0\) we let \(\mathscr{F}_{t +}:=\bigcap_{s>t}\mathscr{F}_s\).
Prove that \(\mathscr{F}_t \subset \mathscr{F}_{t +}\) and \((\mathscr{F}_{t +}:t \geq 0)\) is a filtration.
\end{problem}
\begin{solution}
	To prove \(\mathscr{F}_t \subset \mathscr{F}_{t +}=\bigcap_{s>t}\mathscr{F}_s\), we only need to prove \(\forall s>t,\mathscr{F}_t \subset \mathscr{F}_s\).
	By the definition of filtration it's obvious.
	Now we will prove \((\mathscr{F}_{t +}:t \geq 0)\) is a filtration.
	Only need to prove \(\forall t,s \in \mathbb{R} \AND t \leq s,\mathscr{F}_{t +}\subset \mathscr{F}_{s +}\).
	By the definition of \(\mathscr{F}_{\dot{c} +}\) we know that
	\(\mathscr{F}_{t +}=\bigcap_{x>t}\mathscr{F}_x=\bigcap_{x>s}\mathscr{F}_x \cap \bigcap_{x:t<x \leq s}\mathscr{F}_x \subset \bigcap_{x>s}\mathscr{F}_x = \mathscr{F}_{s +}\).
	So \((\mathscr{F}_{t +}:t \geq 0)\) is a filtration.
\end{solution}
\begin{problem}\label{pro:2}
Assume \((X_t:t \geq 0,t \in \mathbb{R})\) is a stochastic process on probability space \((\Omega,\mathscr{F},\mathbb{P})\).
Prove that \(\forall s,t \geq 0,\varepsilon >0,\{\rho(X_s,X_t) \geq \varepsilon\} \in \mathscr{F}\).
\end{problem}
\begin{solution}
	Easily \(\{\rho(X_s,X_t)\geq \varepsilon\}=\bigcup_{k=1}^{\infty}\{\rho(X_s,X_t)>\varepsilon-\frac{1}{k}\varepsilon\}\).
	So we only need to prove \(\forall k \in \mathbb{N}^+,\{\rho(X_s,X_t)>\varepsilon(1-\frac{1}{k})\} \in \mathscr{F}\).
	Take \(\delta=\varepsilon(1-\frac{1}{k})\), only need to prove \(\forall \delta>0,\{\rho(X_s,X_t)>\delta\}\in \mathscr{F}\).

	\(\forall t \geq 0,X_t:\Omega \to E\) is measurable, where \(E \subset \mathbb{R}^d\).
	So we can find a countable dense set in \(\mathbb{R}^d\), write \(D\).
	We will prove that \(\{\rho(X_s,X_t)>\delta\}=\bigcup_{q \in D} \{\rho(X_s,q)-\rho(X_t,q)>\delta\}\).
	On one hand, easily \(\rho(X_s,q)-\rho(X_t,q)>\delta \implies \rho(X_s,X_t)>\delta\) from triangle inequality.
	So we easily get \(\{\rho(X_s,X_t)>\delta\}\supset\bigcup_{q \in D} \{\rho(X_s,q)-\rho(X_t,q)>\delta\}\).
	On the other hand, assume for certain \(\omega \in \Omega\) we have \(\rho(X_s(\omega),X_t(\omega))>\delta\), we will prove \(\exists q \in D,\rho(X_s(\omega),q)-\rho(X_t(\omega),q)>\delta\).
	For convenience, we omit \((\omega)\) from now on to the end of this paragraph.
	Since \(\rho(X_s,X_t)>\delta\), we know \(\gamma:=\frac{\rho(X_s,X_t)-\delta}{2}>0\).
	Since \(D\) is dense, we obtain \(\exists q \in D,\rho(X_t,q)<\gamma\).
	So from triangle inequality we get \(\rho(X_s,q)\geq\rho(X_s,X_t)-\rho(X_t,q)>2 \gamma+\delta-\gamma=\gamma+\delta\).
	So we get \(\rho(X_s,q)-\rho(X_t,q)>\gamma + \delta - \gamma =\delta\).
	Finally, we get \(\{\rho(X_s,X_t)>\delta\}= \bigcup_{q \in D}\{\rho(X_s,q)-\rho(X_t,q)>\delta\}\).

	Noting \(\bigcup_{q \in D}\{\rho(X_s,q)-\rho(X_t,q)>\delta\}=\bigcup_{q \in D}\bigcup_{p \in \mathbb{Q}^+}\{\rho(X_s,q)>\delta+p,\rho(X_t,q)<p\}\),
	and \(D,\mathbb{Q}^+\) are countable, so we only need to check \(\{\rho(X_s,q)>\delta+p,\rho(X_t,q)<p\} \in \mathscr{F},\forall q \in D,p \in \mathbb{Q}^+\).
	Noting \(\{\rho(X_s,q)>\delta+p,\rho(X_t,q)<p\}=\{\rho(X_s,q)>\delta+p\}\cap\{\rho(X_t,q)<p\}\),
	and \(X_s,X_t\) are measurable from \(\Omega\) to \(E\), we obtain \(\{\rho(X_s,q)>\delta+p\},\{\rho(X_t,q)<p\} \in \mathscr{F}\).
	So we proved \(\{\rho(X_s,X_t) >\delta\} \in \mathscr{F},\forall s,t \geq 0,\forall \delta>0\).

	Finally, we obtain \(\{\rho(X_s,X_t)\geq \varepsilon\} \in \mathscr{F},\forall s,t \geq 0,\varepsilon>0\).
\end{solution}
\begin{problem}\label{pro:3}
Let \(\mathscr{D}_X:=\{\mu_J^X:J \in S(I)\}\) be the family of finite-dimentional distributions of a stochastic process \((X_t:t \geq 0,t \in \mathbb{R})\).
\(\forall (s_1,s_2)\in S(I)\) and \(J=(t_1,\cdots,t_n)\in S(I)\), write \(K_1:=(s_1,s_2,t_1,\cdots,t_n) \in S(I),K_2:=(s_2,s_1,t_1,\cdots,t_n) \in S(I)\).
Take \(A_1,A_2 \in \mathscr{E},B \in \mathscr{E}^n\), prove that
\[
	\mu^X_{K_1}(A_1 \times A_2 \times B)=\mu^X_{K_2}(A_2 \times A_1 \times B)
\]
and
\[
	\mu^X_{K_1}(E \times E \times B)=\mu^X_{K_2}(E \times E \times B)=\mu^X_{J}(B)
\]
\end{problem}

\begin{problem}\label{pro:4}
Assume \((\tau_k:k \in \mathbb{N}^+)\) is an i.i.d sequence of r.v. with exponential distribution with parameter \(\alpha>0\).
Let \(S_n:=\sum_{k=1}^{n}\tau_k\). For \(t \geq 0,t \in \mathbb{R}\), let:
\[
	N_t:=\sum_{n=1}^{\infty}\mathbbm{1}_{\{S_n \leq t\}},
	X_t:=\sum_{n=1}^{\infty}\mathbbm{1}_{\{S_n < t\}}
\]
Prove that \(N\) and \(X\) are modifications of each other, but they are not indistinguishable.
\end{problem}

\begin{problem}\label{pro:5}
Assume \(T\) is non-negetive r.v. with distribution function \(F\) continuous on \(\mathbb{R}\).
Let \(X_t=\mathbbm{1}_{\{T \leq t\}}\).
Prove that \(X\) is stochastically continuous.
\end{problem}
\begin{problem}\label{pro:6}
Assume \(I=\mathbb{Z}^+\), then the stochastic process \(X=(X_0,X_1,\cdots)\) is a r.v. from \(\Omega\) to \(E^\infty\).
Define the distribution of \(X\), \(\mu_X\), as follows:
\[
	\mu_X(A)=\mathbb{P}(X \in A),A \in \mathscr{E}^\infty
\]
Then stochastic process \(X,Y\) are indistinguishable \(\iff \mu_X=\mu_Y\).
\end{problem}

\end{document}
