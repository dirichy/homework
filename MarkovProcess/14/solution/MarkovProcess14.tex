%!Mode:: "TeX:UTF-8"
%!TEX TS-program = xelatex
\documentclass{ctexart}
\newif\ifpreface
%\prefacetrue
\usepackage{fontspec}
\usepackage{bbm}
\usepackage{tikz}
\usepackage{amsmath,amssymb,amsthm,color,mathrsfs}
\usepackage{fixdif}
\usepackage{hyperref}
\usepackage{cleveref}
\usepackage{enumitem}%
\usepackage{expl3}
\usepackage{lipsum}
\usepackage[margin=0pt]{geometry}
\usepackage{listings}
\definecolor{mGreen}{rgb}{0,0.6,0}
\definecolor{mGray}{rgb}{0.5,0.5,0.5}
\definecolor{mPurple}{rgb}{0.58,0,0.82}
\definecolor{backgroundColour}{rgb}{0.95,0.95,0.92}

\lstdefinestyle{CStyle}{
  backgroundcolor=\color{backgroundColour},
  commentstyle=\color{mGreen},
  keywordstyle=\color{magenta},
  numberstyle=\tiny\color{mGray},
  stringstyle=\color{mPurple},
  basicstyle=\footnotesize,
  breakatwhitespace=false,
  breaklines=true,
  captionpos=b,
  keepspaces=true,
  numbers=left,
  numbersep=5pt,
  showspaces=false,
  showstringspaces=false,
  showtabs=false,
  tabsize=2,
  language=C
}
\usetikzlibrary{calc}
\theoremstyle{remark}
\newtheorem{lemma}{Lemma}
\usepackage{fontawesome5}
\usepackage{xcolor}
\newcounter{problem}
\newcommand{\Problem}{\begin{tikzpicture}[baseline]%
    \node at (-0.02em,0.3em) {$\mathbb{P}$};%
    \node[scale=0.7] at (0.2em,-0.0em) {R};%
    \node[scale=0.7] at (0.6em,0.4em) {O};%
    \node[scale=0.8] at (1.05em,0.25em) {B};%
    \node at (1.55em,0.3em) {L};%
    \node[scale=0.7] at (1.75em,0.45em) {E};%
    \node at (2.35em,0.3em) {M};%
  \end{tikzpicture}%
}
\renewcommand{\theproblem}{\Roman{problem}}
\newenvironment{problem}{\refstepcounter{problem}\noindent\color{blue}\Problem\theproblem}{}

\crefname{problem}{\protect\Problem}{Problem}
\newcommand\Solution{\begin{tikzpicture}[baseline]%
    \node at (-0.04em,0.3em) {$\mathbb{S}$};%
    \node[scale=0.7] at (0.35em,0.4em) {O};%
    \node at (0.7em,0.3em) {\textit{L}};%
    \node[scale=0.7] at (0.95em,0.4em) {U};%
    \node[scale=1.1] at (1.19em,0.32em){T};%
    \node[scale=0.85] at (1.4em,0.24em){I};%
    \node at (1.9em,0.32em){$\mathcal{O}$};%
    \node[scale=0.75] at (2.3em,0.21em){\texttt{N}};%
  \end{tikzpicture}}
\newenvironment{solution}{\begin{proof}[\Solution]}{\end{proof}}
\title{\input{../../.subject}\input{../.number}}
\makeatletter
\newcommand\email[1]{\def\@email{#1}\def\@refemail{mailto:#1}}
\newcommand\schoolid[1]{\def\@schoolid{#1}}
\ifpreface
  \def\@maketitle{
  \raggedright
  {\Huge \bfseries \sffamily \@title }\\[1cm]
  {\Huge  \bfseries \sffamily\heiti\@author}\\[1cm]
  {\Huge \@schoolid}\\[1cm]
  {\Huge\href\@refemail\@email}\\[0.5cm]
  \Huge\@date\\[1cm]}
\else
  \def\@maketitle{
    \raggedright
    \begin{center}
      {\Huge \bfseries \sffamily \@title }\\[4ex]
      {\Large  \@author}\\[4ex]
      {\large \@schoolid}\\[4ex]
      {\href\@refemail\@email}\\[4ex]
      \@date\\[8ex]
    \end{center}}
\fi
\makeatother
\ifpreface
  \usepackage[placement=bottom,scale=1,opacity=1]{background}
\fi

\author{白永乐}
\schoolid{202011150087}
\email{202011150087@mail.bnu.edu.cn}

\def\to{\rightarrow}
\newcommand{\xor}{\vee}
\newcommand{\bor}{\bigvee}
\newcommand{\band}{\bigwedge}
\newcommand{\xand}{\wedge}
\newcommand{\minus}{\mathbin{\backslash}}
\newcommand{\mi}[1]{\mathscr{P}(#1)}
\newcommand{\card}{\mathrm{card}}
\newcommand{\oto}{\leftrightarrow}
\newcommand{\hin}{\hat{\in}}
\newcommand{\gl}{\mathrm{GL}}
\newcommand{\im}{\mathrm{Im}}
\newcommand{\re }{\mathrm{Re }}
\newcommand{\rank}{\mathrm{rank}}
\newcommand{\tra}{\mathop{\mathrm{tr}}}
\renewcommand{\char}{\mathop{\mathrm{char}}}
\DeclareMathOperator{\ot}{ordertype}
\DeclareMathOperator{\dom}{dom}
\DeclareMathOperator{\ran}{ran}

\begin{document}
\large
\iffalse
  \setlength{\baselineskip}{1.2em}
  \ifpreface
    \backgroundsetup{contents={%
    \begin{tikzpicture}
      \fill [white] (current page.north west) rectangle ($(current page.north east)!.3!(current page.south east)$) coordinate (a);
      \fill [bgc] (current page.south west) rectangle (a);
\end{tikzpicture}}}
\definecolor{word}{rgb}{1,1,0}
\definecolor{bgc}{rgb}{1,0.95,0}
\setlength{\parindent}{0pt}
\thispagestyle{empty}
\begin{tikzpicture}%
  % \node[xscale=2,yscale=4] at (0cm,0cm) {\sffamily\bfseries \color{word} under};%
  \node[xscale=4.5,yscale=10] at (10cm,1cm) {\sffamily\bfseries \color{word} Graduate Homework};%
  \node[xscale=4.5,yscale=10] at (8cm,-2.5cm) {\sffamily\bfseries \color{word} In Mathematics};%
\end{tikzpicture}
\ \vspace{1cm}\\
\begin{minipage}{0.25\textwidth}
  \textcolor{bgc}{王胤雅是傻逼}
\end{minipage}
\begin{minipage}{0.75\textwidth}
  \maketitle
\end{minipage}
\vspace{4cm}\ \\
\begin{minipage}{0.2\textwidth}
  \
\end{minipage}
\begin{minipage}{0.8\textwidth}
  {\Huge
    \textinconsolatanf{}
  }General fire extinguisher
\end{minipage}
\newpage\backgroundsetup{contents={}}\setlength{\parindent}{2em}

  \else
    \maketitle
  \fi
\fi
\newgeometry{left=2cm,right=2cm,top=2cm,bottom=2cm}
%from_here_to_type
\begin{problem}\label{pro:1}
  Assume \(N(t)\) is updating process. \(X\) is the time interval distrabution of \(N(t)\).
  Assume \(\mathbb{D}(X)<\infty\).
  Let \(R(t):=S_{N(t)+1}-t\). Find:
  \[
    \lim_{T \to \infty} \frac{1}{T}\int_{0}^{T}R(t) \d t
  \]
\end{problem}
\begin{solution}
  Easily \(N(t)+1 \geq T \geq N(t)\).
  So \(\int_{0}^{T}R(t) \d t \leq \sum_{i=1}^{N(T)+1} \int_{S_{i-1}}^{S_i}(S_i-t) \d t=\frac{1}{2}\sum_{i=1}^{N(t)+1} (S_i-S_{i-1})^2=\frac{1}{2} \sum_{i=1}^{N(T)+1} \xi_i^2\).
  For the same reason, we get that \(\int_{0}^{T}R(t) \d t \geq \frac{1}{2} \sum_{i=1}^{N(T)} \xi_i^2\).
  \(\int_{0}^T R(t)\d t \geq \frac{1}{2}\sum_{i=1}^{N} \xi_i^2\)
  Easy to know that \(\lim_{T \to \infty} \frac{1}{T} \sum_{i=1}^{N(T)} \xi_i^2=\lim_{T \to \infty} \frac{1}{T} \sum_{i=1}^{N(T)+1} \xi_i^2=\frac{\mathbb{E}(X^2)}{\mathbb{E}(X)^2}\).
  So finally we get that
  \[
    \lim_{T \to \infty} \frac{1}{T}\int_{0}^{T}R(t) \d t=\frac{\mathbb{E}(X^2)}{2\mathbb{E}(X)^2}
  \]
  .
\end{solution}
\begin{problem}\label{pro:2}
  Assume the number of people arriving the cinema is distributed as a Possion process with parameter \(\lambda\).
  Assume the film begin at a fixed time \(t \geq 0\).
  Let \(A(t)\) be the sum of waiting time of all people arriving in \((0,t]\), find \(\mathbb{E}(A(t))\).
\end{problem}
\begin{solution}
  Let \(V_k\) be the arriving time of \(k\)-th people. Let \(N(t)\) be the number of people in \((0,t]\).
  Then \(A(t)=\sum_{k=1}^{N(t)} (t-V_k)\). Let \(\xi_k:=V_k-V_{k-1}\).
  Then \(\sum_{k=1}^{N(t)} V_k=\sum_{k=1}^{N(t)} (N(t)-k)\xi_k=\sum_{k=0}^{N(t)-1} k \xi_{N(t)-k}\).
  So \(\mathbb{E}(A(t))=t \mathbb{E}(N(t))-\mathbb{E}(\sum_{k=0}^{N(t)-1} k\xi_{N(t)-k})\).
  Easy to get that \(\mathbb{E}(\sum_{k=0}^{N(t)-1} k\xi_{N(t)-k} \mid N(t)=n)=\frac{nt}{2}\).
  So \(\mathbb{E}(A(t) \mid N(t)=n)=nt-\frac{nt}{2}=\frac{nt}{2}\).
  So finally we have \(\mathbb{E}(A(t))=\mathbb{E}(\mathbb{E}(A(t) \mid N(t)))=\mathbb{E}(\frac{N(t)t}{2})=\frac{\lambda t^2}{2}\).
\end{solution}
\begin{problem}\label{pro:3}
  Assume a machine has life distrabuted \(p\). When machine is broken or has been used \(T\) years, we will change a new machine.
  The price of new machine is \(C_1\), and if the machine is broken, it would cause loss \(C_2\).
  \begin{enumerate}
    \item Give the long-time average fee of this machine.
    \item Let \(C_1=10,C_2=0.5\), and \(p(x)=\mathbbm{1}_{(0,10)}(x) \frac{1}{10}\).
      Which \(T\) can let the fee be minimum.
  \end{enumerate}
\end{problem}
\begin{solution}
  \begin{enumerate}
    \item Let \(\xi\) be the time when the machine will broken. Let \(\gamma:=\xi \wedge T\).
      Then the machine will be changed at \(\gamma\).
      Obviously \(\mathbb{E}(\gamma)=T \mathbb{P}(\xi>T)+\mathbb{E}(\xi \mathbbm{1}(\xi \leq T))=T \int_{T}^{\infty}p(x) \d x + \int_{0}T xp(x) \d x\).
      Let \(\eta\) be the fee of this machine, then we have \(\eta=C_1\mathbbm{1}(\xi>T)+(C_1+C_2)\mathbbm{1}(\xi \leq T)=C_1+C_2 \mathbbm{1}(\xi \leq T)\).
      So \(\mathbb{E}(\eta)=C_1+C_2 \int_{0}^{T}p(x) \d x\).
      So the long-time average fee is
      \[
        g(T)=\frac{C_1+C_2 \int_{0}T p(x) \d x}{T \int_{T}^{\infty}p(x) \d x + \int_{0}^{T}x p(x) \d x}
      \].
    \item Easy to get that \(g(T)=\frac{200+T}{20T-T^2}\) when \(T \in (0,10)\).
      And \(g'(T)=\frac{T^2 + 400T - 4000}{(20T-T^2)^2}\).
      Let \(g'(T)=0\), then \(T^2+400T-4000=0\), then \(T=20 \sqrt{110}-200 \approx 9.76\).
      So \(T=9.76\) can make the fee get minimum.
  \end{enumerate}
\end{solution}
\begin{problem}\label{pro:4 }
  A kind of product is qualified with probability \(p(0 < p < 1)\).
  We sample these product by the following way: we check all the product at first until there appears
  \(k\) qualified product sequently. Then we check the rest of product by probability \(\alpha(0 < \alpha <1)\)
  until there appears one unqualified product, then one circle ends. Next we restart another checking
  circle. Please find out the proportion of checked product after a long time.
\end{problem}
\begin{solution}
  For sake of convenience, we call the \(k\) qualified products appearing sequently as \(k\) qualified sequence.
  Assume the proportion of checked product after a long time is \(\beta\).
  Let \(N_k\) be the amount of product when the first \(k\) qualified sequence ends. \(M_k=\mathbb{E}(N_k)\). \(G_k\) is the
  event that the next one is qualified after the first \(k-1\) qualified sequence ends. Obviously, \(\mathbb{E}(N_k-N_{k-1} \mid G_k)=1\).
  And \(\mathbb{E}(N_k-N_{k-1} \mid G_k)=\mathbb{E}(N_k) + 1\).
  Therefore, \(\mathbb{E}(N_k-N_{k-1})=p + (1-p)(\mathbb{E}(N_k) + 1)\). So \(M_{k-} M_{k-1}=p + (1-p)( 1 + M_k)\).
  Then \(pM_k=M_{k-1} + 1\). Thus, \(M_k=\frac{\frac{1}{p^k}-1}{1-p}\).
  Let \(A=\{\)The amount of product checked in one circle\(\}\), \(B=\{\)The amount of product in one circle\(\}\).
  So finding one unqualified product need average checking time \(\frac{1}{1-p}\) according to geometric distribution.
  Then we averagely need \(\frac{1}{\alpha(1-p)}\) product to find out the unqualified one.
  Then \(\mathbb{E}(A)=M_k + \frac{1}{1-p}, \mathbb{E}(B)=M_k + \frac{1}{\alpha(1-p)}\).
  So \[
    \beta = \frac{\mathbb{E}(A)}{\mathbb{E}(B)}=\frac{\frac{\frac{1}{p^k}-1}{1-p} + \frac{1}{1-p}}{\frac{\frac{1}{p^k}-1}{1-p} + \frac{1}{\alpha(1-p)}}=\frac{\alpha}{\alpha + p^k-\alpha p^k}
  \]
\end{solution}
\end{document}
