%!Mode:: "TeX:UTF-8"
%!TEX encoding = UTF-8 Unicode
%!TEX TS-program = xelatex
\documentclass{ctexart}
\newif\ifpreface
%\prefacetrue
\usepackage{fontspec}
\usepackage{bbm}
\usepackage{tikz}
\usepackage{amsmath,amssymb,amsthm,color,mathrsfs}
\usepackage{fixdif}
\usepackage{hyperref}
\usepackage{cleveref}
\usepackage{enumitem}%
\usepackage{expl3}
\usepackage{lipsum}
\usepackage[margin=0pt]{geometry}
\usepackage{listings}
\definecolor{mGreen}{rgb}{0,0.6,0}
\definecolor{mGray}{rgb}{0.5,0.5,0.5}
\definecolor{mPurple}{rgb}{0.58,0,0.82}
\definecolor{backgroundColour}{rgb}{0.95,0.95,0.92}

\lstdefinestyle{CStyle}{
  backgroundcolor=\color{backgroundColour},
  commentstyle=\color{mGreen},
  keywordstyle=\color{magenta},
  numberstyle=\tiny\color{mGray},
  stringstyle=\color{mPurple},
  basicstyle=\footnotesize,
  breakatwhitespace=false,
  breaklines=true,
  captionpos=b,
  keepspaces=true,
  numbers=left,
  numbersep=5pt,
  showspaces=false,
  showstringspaces=false,
  showtabs=false,
  tabsize=2,
  language=C
}
\usetikzlibrary{calc}
\theoremstyle{remark}
\newtheorem{lemma}{Lemma}
\usepackage{fontawesome5}
\usepackage{xcolor}
\newcounter{problem}
\newcommand{\Problem}{\begin{tikzpicture}[baseline]%
    \node at (-0.02em,0.3em) {$\mathbb{P}$};%
    \node[scale=0.7] at (0.2em,-0.0em) {R};%
    \node[scale=0.7] at (0.6em,0.4em) {O};%
    \node[scale=0.8] at (1.05em,0.25em) {B};%
    \node at (1.55em,0.3em) {L};%
    \node[scale=0.7] at (1.75em,0.45em) {E};%
    \node at (2.35em,0.3em) {M};%
  \end{tikzpicture}%
}
\renewcommand{\theproblem}{\Roman{problem}}
\newenvironment{problem}{\refstepcounter{problem}\noindent\color{blue}\Problem\theproblem}{}

\crefname{problem}{\protect\Problem}{Problem}
\newcommand\Solution{\begin{tikzpicture}[baseline]%
    \node at (-0.04em,0.3em) {$\mathbb{S}$};%
    \node[scale=0.7] at (0.35em,0.4em) {O};%
    \node at (0.7em,0.3em) {\textit{L}};%
    \node[scale=0.7] at (0.95em,0.4em) {U};%
    \node[scale=1.1] at (1.19em,0.32em){T};%
    \node[scale=0.85] at (1.4em,0.24em){I};%
    \node at (1.9em,0.32em){$\mathcal{O}$};%
    \node[scale=0.75] at (2.3em,0.21em){\texttt{N}};%
  \end{tikzpicture}}
\newenvironment{solution}{\begin{proof}[\Solution]}{\end{proof}}
\title{\input{../../.subject}\input{../.number}}
\makeatletter
\newcommand\email[1]{\def\@email{#1}\def\@refemail{mailto:#1}}
\newcommand\schoolid[1]{\def\@schoolid{#1}}
\ifpreface
  \def\@maketitle{
  \raggedright
  {\Huge \bfseries \sffamily \@title }\\[1cm]
  {\Huge  \bfseries \sffamily\heiti\@author}\\[1cm]
  {\Huge \@schoolid}\\[1cm]
  {\Huge\href\@refemail\@email}\\[0.5cm]
  \Huge\@date\\[1cm]}
\else
  \def\@maketitle{
    \raggedright
    \begin{center}
      {\Huge \bfseries \sffamily \@title }\\[4ex]
      {\Large  \@author}\\[4ex]
      {\large \@schoolid}\\[4ex]
      {\href\@refemail\@email}\\[4ex]
      \@date\\[8ex]
    \end{center}}
\fi
\makeatother
\ifpreface
  \usepackage[placement=bottom,scale=1,opacity=1]{background}
\fi

\author{白永乐}
\schoolid{25110180002}
\email{ylbai25@m.fudan.edu.cn}

\def\to{\rightarrow}
\newcommand{\xor}{\vee}
\newcommand{\AND}{\wedge}
\newcommand{\OR}{\vee}
\newcommand{\bor}{\bigvee}
\newcommand{\band}{\bigwedge}
\newcommand{\xand}{\wedge}
\newcommand{\minus}{\mathbin{\backslash}}
\newcommand{\mi}[1]{\mathscr{P}(#1)}
\newcommand{\card}{\mathrm{card}}
\newcommand{\oto}{\leftrightarrow}
\newcommand{\hin}{\hat{\in}}
\newcommand{\gl}{\mathrm{GL}}
\newcommand{\im}{\mathrm{Im}}
\newcommand{\re }{\mathrm{Re }}
\newcommand{\rank}{\mathrm{rank}}
\newcommand{\tra}{\mathop{\mathrm{tr}}}
\renewcommand{\char}{\mathop{\mathrm{char}}}
\DeclareMathOperator{\ot}{ordertype}
\DeclareMathOperator{\dom}{dom}
\DeclareMathOperator{\ran}{ran}

\begin{document}
\large
\setlength{\baselineskip}{1.2em}
\ifpreface
  \input{../../../global/preface}
\else
  %\maketitle
\fi
\newgeometry{left=2cm,right=2cm,top=2cm,bottom=2cm}
%from_here_to_type
\begin{problem}\label{pro:1}
  Let \(S=(S_n:n \geq 0)\) be the one-dimensional symmetry simple random walk with \(S_0=c \geq 0\).
  Let \(k \geq 1\) and \(\tau\) be the time of the \(k\)-th downcrossing of \(0\).
  \(X_b\) is the times of \((S_{n \AND \tau}:n \geq 0)\) downcrossing of \(b\). Prove:
  \begin{enumerate}
    \item \((X_b:b \geq c-1)\) is branch process.
      And offspring distribution is \(Geo(\frac{1}{2})\)

    \item \((X_{-a}:a \geq 1)\) is branch process.
      And offspring distribution is \(Geo(\frac{1}{2})\)

    \item \((X_b:0 \leq b \leq c-1)\) is migrating branch process.
      And offspring distribution is \(Geo(\frac{1}{2})\)

      And the migrating distribution is concentrating on \(1\).

  \end{enumerate}
\end{problem}
\begin{solution}
  For a random walk \(y\), we let \(D(n,y)\) be the number of downcrossings of \(y\) over \(n\).
  \begin{enumerate}
    \item Fix \(b \geq c-1\). Let \(\phi_0\) be the journey from start point to \(b+1\).
      Let \(e_n\) be \(n\)-th journey from \(b+1\) to \(b\).
      Let \(\varepsilon_n\) be \(n\)-th journey after \(\phi_0\) from \(b\) to \(b+1\).
      Then we know that \(e_n,\varepsilon_n\) are independent.
      Easy to get that \(D(e_n,b)=1\) and \(D(\varepsilon_n,b)=0,D(\varepsilon_n,b+1=0)\).
      Easy to get that \(D((S_{n \wedge \tau}:n \in \mathbb{N}),b+1)=\sum_{t=1}^{D((S_{n \wedge \tau}:n \in \mathbb{N}),b)} D(e_t,b+1) \).
      Noting that \(\forall d:c-1 \leq d \leq b,D((S_{n \wedge \tau}:n \in \mathbb{N}),d) \in \sigma(\varepsilon_n:n \in \mathbb{N})\).
      We easily get that \(D(e_t,b+1) \perp \sigma(\varepsilon_n:n \in \mathbb{N})\).
      So \(X_b\) is Markov process. And to prove it's branch process, we only need to prove that \(D(e_t,b+1)\) are i.i.d.
      It has been proved that \(D(e_t,b+1)\) are i.i.d and \(Geo(\frac{1}{2})\).
      So the offspring distribution is \(Geo(\frac{1}{2})\).
    \item Fix \(a \geq 1\). Let \(\phi_0\) be the journey from start point to \(-a\).
      Let \(e_n\) be \(n\)-th journey from \(-a\) to \(-a-1\), and \(\varepsilon_n\) be \(n\)-th journey from \(-a-1\) to \(-a\).
      Then easy to get that \(X_{-a-1}=\sum_{t=1}^{X_{-a}} D(\varepsilon_t,-a-1)\).
      For the same reason we easily get that \(D(\varepsilon_t,-a-1)\perp \sigma(e_n:n \in \mathbb{N})\).
      And by reflecting easy to get that \(D(\varepsilon_t,-a-1) \sim Geo(\frac{1}{2})\), too.
      So \((X_{-a}:a \geq 1)\) is branch process
      and offspring distribution is \(Geo(\frac{1}{2})\)

    \item Fix \(b<c-1\). Let \(\phi_0\) be the journey from start point to \(b+1\).
      Let \(e_n\) be the \(n\)-th journey from \(b+1\) to \(b\) and \(\varepsilon_n\) be \(n\)-th journey from \(b\) to \(b+1\).
      Then easy to prove that \(X_{b+1}=D(\phi_0,b+1)+\sum_{t=1}^{X_b} D(e_n,b+1)\).
      Noting that \(D(\phi_0,b+1)=1\). So for the same reason, we get that \((X_b:0 \leq b \leq c-1)\) is migrating branch process,
      with offspring distribution \(Geo(\frac{1}{2})\) and migrating distribution \(\delta_1\).
  \end{enumerate}
\end{solution}
\begin{problem}\label{pro:2}
  \(c < b \in \mathbb{Z}_+\). Let \(W=(W_n :n \geq 0)\) be the one-dimensional reflecting simple
  random walk with \(W_0=c \geq 0\) on \(\mathbb{Z}^{0,b}\), whose transition matrix is \(P^{0,b}\), where
  \(a = 0,p,q >0, p + q =1\).
  Let \(k \geq 1\) and \(\tau\) be the time of the \(k\)-th downcrossing over \(0\) on \((W_n)\).
  \(0 \leq a \leq b\),
  \(X_a\) is the times of \((S_{n \AND \tau}:n \geq 0)\) downcrossing over \(a\). Prove:
  \begin{enumerate}
    \item \((X_a:c-1 \leq a \leq b-1)\) is branch process.
      And offspring distribution is \(Geo(p)\).
    \item \((X_a:0 \leq a \leq c-1)\) is migrating branch process.
      And offspring distribution is \(Geo(p)\).
      And the migrating distribution is concentrating on \(1\).
  \end{enumerate}
\end{problem}
\begin{solution}
  For a random walk \(y\), we let \(D(n,y)\) be the number of downcrossings of \(y\) over \(n\).
  \begin{enumerate}
    \item Fix \(a\) such that \(c-1 \leq a < b-1\).
      Let \(\phi_0\) be the journey from start point to \(a\).
      Let \(e_n\) be the \(n\)-th journey from \(a\) to \(a+1\), and \(\varepsilon_n\) be the \(n\)-th journey from \(a+1\) to \(a\).
      For reflecting simple random walk, we can also prove that \(e_n,\varepsilon_n\) are independent.
      Noting that \(X_{a+1}=\sum_{t=1}^{X_a} D(\varepsilon_t,a+1)\), we easily get the conclusion.
    \item Fix \(a:0 \leq a<c-1\). Let \(\phi_0\) be the journey from start point to \(a+1\).
      Let \(e_n\) be the \(n\)-th journey from \(a+1\) to \(a\) and \(\varepsilon_n\) be \(n\)-th journey from \(a\) to \(a+1\).
      Then easy to prove that \(X_{a+1}=D(\phi_0,a+1)+\sum_{t=1}^{X_a} D(e_n,a+1)\).
      Noting that \(D(\phi_0,a+1)=1\). So for the same reason, we get that \((X_a:0 \leq a \leq c-1)\) is migrating branch process,
      with offspring distribution \(Geo(\frac{1}{2})\) and migrating distribution \(\delta_1\).
  \end{enumerate}
\end{solution}
\begin{problem}\label{pro:3}
  Let \(W=(W_n:n \geq 0)\) be the one-dimensional simple random walk with \(W_0=0\),
  whose transition matrix \(P\) given by equation \((4.4.3)\)
  on textbook, \(0< p<q<1\).
  \(X_a\) is the times of \((W_{n \AND \tau}:n \geq 0)\) downcrossing \(a\).
  \(r=\frac{p}{q}\).
  Prove:
  \begin{enumerate}
    \item \label{ite:3.1} \(\mathbb{P}(X_0=i)=r^i(1-r),i \geq 0\);
    \item \(a \geq 0\), \(\mathbb{P}(X_a=0)=1-r^{a +1}\), \(\mathbb{P}(X_a=i)=r^{a + 1}(1 -r),i \geq 1\).
  \end{enumerate}
\end{problem}
\begin{solution}
  \begin{enumerate}
    \item Since \(p <q\), then \(W_n \to -\infty,n \to \infty\). Let \(\tau_0=0, \forall k \geq 1\),
      \(\sigma_k=\inf \{n \geq \tau_{k-1}:W_n=1\},\tau_k=\inf \{n \geq \sigma_k:W_n=0\}\).
      \begin{enumerate}
        \item If \(i=0\), then \(\{X_0=i\}\overset{\text{a.s.}}{=}\{\sigma_1=\infty\}\). Then \(\mathbb{P}(X_0=i)=\mathbb{P}(\sigma_1=\infty)=r\).
        \item If \(i \geq 1\), then \(\{X_0=i\}\overset{\text{a.s.}}{=}\{\sigma_i <\infty,\sigma_{i + 1}=\infty\}\).
          Since \(\{\tau_{i} <\infty\} \subset \{\sigma_{i} \infty\}, \mathbb{P}(\sigma_i <\infty, \tau_i = \infty)=0\),
          then by strong markov property,
          \[
            \begin{aligned}
              \mathbb{P}(\sigma_{i + 1} <\infty \mid \sigma_{i} < \infty) & = \mathbb{P}(\sigma_{i + 1} < \infty \mid \sigma_i <\infty, \tau_i < \infty) \\
                                                                          & =\mathbb{P}(\sigma_{i + 1}<\infty \mid \tau_i < \infty)                      \\
                                                                          & =\mathbb{P}(\sigma_{i + 1}< \infty \mid \tau_i < \infty, W_{\tau_i}=0)       \\
                                                                          & =\mathbb{P}(\sigma_1 < \infty)=r
            \end{aligned}
          \]
          Therefore,
          \[
            \mathbb{P}(\sigma_{i + 1} < \infty)=\mathbb{P}(\sigma_{i + 1}<\infty \mid \sigma_i < \infty )\mathbb{P}(\sigma_i < \infty)
          \]
          Then \(\mathbb{P}(\sigma_{i }<\infty)=r^i\).
          Therefore, \(\mathbb{P}(X_0=i)=\mathbb{P}(\sigma_i < \infty,\sigma_{i + 1}=\infty)=\mathbb{P}(\sigma_i < \infty) \mathbb{P}(\sigma_{i + 1}=\infty \mid \sigma_i < \infty)=r^i(1-r)\).
      \end{enumerate}
    \item Let \(D_a=\inf (n \geq 0: W_n=a)\), then \(\mathbb{P}(D_a < \infty)=r^a\). By strong markov property,
      \((W_{D_a + n-a:n \geq 0})\) is a random walk starting from \(0\) under \(\mathbb{P}(\cdot \mid D_a < \infty)=\mathbb{P}(\cdot \mid D_a < \infty, W_{D_a}=a)\).
      By the conclusion in \ref{ite:3.1}, \(\mathbb{P}(X_a=i \mid D_a < \infty)=r^i(1-r), i \geq 0\).
      Then \[
        \begin{aligned}
          \mathbb{P}(X_a=0) & =\mathbb{P}(D_a=\infty) + \mathbb{P}(D_a < \infty, X_a=0)            \\
                            & =1-r^a + \mathbb{P}(D_a < \infty)\mathbb{P}(X_a=0 \mid D_a < \infty) \\
                            & =1-r^a + r^a(1-r)=1-r^{a + 1}
        \end{aligned}
      \]
      \(\forall i \geq 1\),
      \[
        \begin{aligned}
          \mathbb{P}(X_a=i) & =\mathbb{P}(D_a< \infty,X_a=i)                                    \\
                            & =    \mathbb{P}(D_a< \infty)\mathbb{P}(X_a = i \mid D_a < \infty) \\
                            & =r^ar^i(1-r)=r^{a + i}(1-r)
        \end{aligned}
      \]
  \end{enumerate}
\end{solution}
\begin{problem}\label{pro:4}
  Let \(W=(W_n:n \geq 0)\) be the one-dimensional simple random walk with \(W_0=0\),
  whose transition matrix \(P\) given by equation \((4.4.3)\)
  on textbook, \(0< p<q<1\).
  \(X_a\) is the times of \((W_{n \AND \tau}:n \geq 0)\) downcrossing over \(a\).
  \(r=\frac{p}{q}\).
  Prove: if \(a \leq -1\), then \(X_a \sim G(1-r)\), i.e. \(\mathbb{P}(X_a=i)=r^{i-1}(1-r),i \geq 1\).
\end{problem}
\begin{solution}

\end{solution}

\end{document}
