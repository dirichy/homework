%!Mode:: "TeX:UTF-8"
%!TEX encoding = UTF-8 Unicode
%!TEX TS-program = xelatex
\documentclass{ctexart}
\newif\ifpreface
%\prefacetrue
\usepackage{fontspec}
\usepackage{bbm}
\usepackage{tikz}
\usepackage{amsmath,amssymb,amsthm,color,mathrsfs}
\usepackage{fixdif}
\usepackage{hyperref}
\usepackage{cleveref}
\usepackage{enumitem}%
\usepackage{expl3}
\usepackage{lipsum}
\usepackage[margin=0pt]{geometry}
\usepackage{listings}
\definecolor{mGreen}{rgb}{0,0.6,0}
\definecolor{mGray}{rgb}{0.5,0.5,0.5}
\definecolor{mPurple}{rgb}{0.58,0,0.82}
\definecolor{backgroundColour}{rgb}{0.95,0.95,0.92}

\lstdefinestyle{CStyle}{
  backgroundcolor=\color{backgroundColour},
  commentstyle=\color{mGreen},
  keywordstyle=\color{magenta},
  numberstyle=\tiny\color{mGray},
  stringstyle=\color{mPurple},
  basicstyle=\footnotesize,
  breakatwhitespace=false,
  breaklines=true,
  captionpos=b,
  keepspaces=true,
  numbers=left,
  numbersep=5pt,
  showspaces=false,
  showstringspaces=false,
  showtabs=false,
  tabsize=2,
  language=C
}
\usetikzlibrary{calc}
\theoremstyle{remark}
\newtheorem{lemma}{Lemma}
\usepackage{fontawesome5}
\usepackage{xcolor}
\newcounter{problem}
\newcommand{\Problem}{\begin{tikzpicture}[baseline]%
    \node at (-0.02em,0.3em) {$\mathbb{P}$};%
    \node[scale=0.7] at (0.2em,-0.0em) {R};%
    \node[scale=0.7] at (0.6em,0.4em) {O};%
    \node[scale=0.8] at (1.05em,0.25em) {B};%
    \node at (1.55em,0.3em) {L};%
    \node[scale=0.7] at (1.75em,0.45em) {E};%
    \node at (2.35em,0.3em) {M};%
  \end{tikzpicture}%
}
\renewcommand{\theproblem}{\Roman{problem}}
\newenvironment{problem}{\refstepcounter{problem}\noindent\color{blue}\Problem\theproblem}{}

\crefname{problem}{\protect\Problem}{Problem}
\newcommand\Solution{\begin{tikzpicture}[baseline]%
    \node at (-0.04em,0.3em) {$\mathbb{S}$};%
    \node[scale=0.7] at (0.35em,0.4em) {O};%
    \node at (0.7em,0.3em) {\textit{L}};%
    \node[scale=0.7] at (0.95em,0.4em) {U};%
    \node[scale=1.1] at (1.19em,0.32em){T};%
    \node[scale=0.85] at (1.4em,0.24em){I};%
    \node at (1.9em,0.32em){$\mathcal{O}$};%
    \node[scale=0.75] at (2.3em,0.21em){\texttt{N}};%
  \end{tikzpicture}}
\newenvironment{solution}{\begin{proof}[\Solution]}{\end{proof}}
\title{\input{../../.subject}\input{../.number}}
\makeatletter
\newcommand\email[1]{\def\@email{#1}\def\@refemail{mailto:#1}}
\newcommand\schoolid[1]{\def\@schoolid{#1}}
\ifpreface
  \def\@maketitle{
  \raggedright
  {\Huge \bfseries \sffamily \@title }\\[1cm]
  {\Huge  \bfseries \sffamily\heiti\@author}\\[1cm]
  {\Huge \@schoolid}\\[1cm]
  {\Huge\href\@refemail\@email}\\[0.5cm]
  \Huge\@date\\[1cm]}
\else
  \def\@maketitle{
    \raggedright
    \begin{center}
      {\Huge \bfseries \sffamily \@title }\\[4ex]
      {\Large  \@author}\\[4ex]
      {\large \@schoolid}\\[4ex]
      {\href\@refemail\@email}\\[4ex]
      \@date\\[8ex]
    \end{center}}
\fi
\makeatother
\ifpreface
  \usepackage[placement=bottom,scale=1,opacity=1]{background}
\fi

\author{白永乐}
\schoolid{202011150087}
\email{202011150087@mail.bnu.edu.cn}

\def\to{\rightarrow}
\newcommand{\xor}{\vee}
\newcommand{\bor}{\bigvee}
\newcommand{\band}{\bigwedge}
\newcommand{\xand}{\wedge}
\newcommand{\minus}{\mathbin{\backslash}}
\newcommand{\mi}[1]{\mathscr{P}(#1)}
\newcommand{\card}{\mathrm{card}}
\newcommand{\oto}{\leftrightarrow}
\newcommand{\hin}{\hat{\in}}
\newcommand{\gl}{\mathrm{GL}}
\newcommand{\im}{\mathrm{Im}}
\newcommand{\re }{\mathrm{Re }}
\newcommand{\rank}{\mathrm{rank}}
\newcommand{\tra}{\mathop{\mathrm{tr}}}
\renewcommand{\char}{\mathop{\mathrm{char}}}
\DeclareMathOperator{\ot}{ordertype}
\DeclareMathOperator{\dom}{dom}
\DeclareMathOperator{\ran}{ran}

\begin{document}
\large
\setlength{\baselineskip}{1.2em}
\ifpreface
  \backgroundsetup{contents={%
    \begin{tikzpicture}
      \fill [white] (current page.north west) rectangle ($(current page.north east)!.3!(current page.south east)$) coordinate (a);
      \fill [bgc] (current page.south west) rectangle (a);
\end{tikzpicture}}}
\definecolor{word}{rgb}{1,1,0}
\definecolor{bgc}{rgb}{1,0.95,0}
\setlength{\parindent}{0pt}
\thispagestyle{empty}
\begin{tikzpicture}%
  % \node[xscale=2,yscale=4] at (0cm,0cm) {\sffamily\bfseries \color{word} under};%
  \node[xscale=4.5,yscale=10] at (10cm,1cm) {\sffamily\bfseries \color{word} Graduate Homework};%
  \node[xscale=4.5,yscale=10] at (8cm,-2.5cm) {\sffamily\bfseries \color{word} In Mathematics};%
\end{tikzpicture}
\ \vspace{1cm}\\
\begin{minipage}{0.25\textwidth}
  \textcolor{bgc}{王胤雅是傻逼}
\end{minipage}
\begin{minipage}{0.75\textwidth}
  \maketitle
\end{minipage}
\vspace{4cm}\ \\
\begin{minipage}{0.2\textwidth}
  \
\end{minipage}
\begin{minipage}{0.8\textwidth}
  {\Huge
    \textinconsolatanf{}
  }General fire extinguisher
\end{minipage}
\newpage\backgroundsetup{contents={}}\setlength{\parindent}{2em}

\else
  %\maketitle
\fi
\newgeometry{left=2cm,right=2cm,top=2cm,bottom=2cm}
%from_here_to_type.pr
\begin{problem}\label{pro:1}
  Assume \((X_n:n \geq 0)\) is an irreducible Markov chain on \(E\).
  Prove that \((X_n:n \geq 0)\) is recurrent (or transient) \(\iff\)
  \(\forall i \in E\),
  \[
    \mathbb{P}\left(\bigcap_{n=1}^{\infty} \bigcup_{k=n}^{\infty} \{X_k=i\}\right)=1 (\text{or } 0).
  \]
\end{problem}
\begin{solution}
  Only need to prove ``\(\implies\)''.

  First we assume \((X_n:n \in \mathbb{N})\) is recurrent, we should prove \(\mathbb{P}\left(\bigcap_{n=1}^{\infty} \bigcup_{k=n}^{\infty} \{X_k=i\}\right)=1\).
  Let \(\tau_1=\inf \{n > 0:X_n=i\}\), and for \(n \in \mathbb{N}^+\), we let \(\tau_{n+1}=\inf \{n > \tau_n:X_n=i\}\).
  Since \(i\) is recurrent and \((X_n)\) is irreducible, we know that \(\tau_1<\infty,a.s.\).
  Then \((X_{\tau_1+n}:n \in \mathbb{N})\) is a Markov chain with the same transition matrix as \((X_n)\).
  So we get that \(\tau_2-\tau_1 < \infty,a. s.\). So \(\tau_2<\infty\), a.s..
  Use MI, we can easily get that \(\forall n \in \mathbb{N}^+,\tau_n<\infty,a. s.\).
  Easy to get that \(\tau_{n+1}>\tau_n\) and \(\tau_1>0\), so \(\tau_n \geq n\).
  So \(\tau_n<\infty \implies \exists k \geq n,X_k=i\).
  So \(\forall n \in \mathbb{N},\mathbb{P}\left(  \bigcup_{k=n}^{\infty} \{X_k=i\}\right)=1\).
  Thus \(\mathbb{P}\left(\bigcap_{n=1}^{\infty} \bigcup_{k=n}^{\infty} \{X_k=i\}\right)=1\).

  Second we assume \((X_n:n \in \mathbb{N})\) is transient, we should prove that \(\mathbb{P}\left(\bigcap_{n=1}^{\infty} \bigcup_{k=n}^{\infty} \{X_k=i\}\right)=0\).
  Write \(A=\bigcap_{n=1}^{\infty} \bigcup_{k=n}^{\infty} \{X_k=i\}\).
  If not, we consider \((A,\mathcal{F} \cap A,\mathbb{P}_A:=\frac{\mathbb{P}}{\mathbb{P}(A)})\).
  We define \(\tau_n \) as above. Easy to know \(\forall \omega \in A,\forall n \in \mathbb{N}^+,\tau_n<\infty\).
  And easy to know that \(\tau_{n+1}-\tau_n \mid_{\tau_n<\infty}\) has the same distribution for every \(n\).
  And since \((X_n)\) is transient, we know \((X_{\tau_k+n})\) is transient for every \(k \in \mathbb{N}^+\).
  So we know \(\mathbb{P}(\tau_{n+1}-\tau_n<\infty \mid \tau_n<\infty)<1\).
  Then \(\mathbb{P}(A)=\mathbb{P}(\forall n,\tau_n < \infty) \leq \mathbb{P}(\forall n,\tau_{n+1}-\tau_n<\infty) \leq \prod_{n=1}^{\infty} \mathbb{P}(\tau_{n+1}-\tau_n<\infty \mid \tau_n<\infty)=\prod_{n=1}^{\infty} \mathbb{P}(\tau_2-\tau_1<\infty \mid \tau_1<\infty)=0\).
\end{solution}

\begin{problem}\label{pro:3}
  Assume \((X_n: n \geq 0)\) is Markov chain on \(E\), where \(E\) is finite. Prove that \(\exists x \in E\), \(x\) is recurrent.
\end{problem}
\begin{solution}

\end{solution}

\begin{problem}\label{pro:4}
  Assume \((X_n:n \geq 0)\) is Markov chain on \(\mathbb{Z}\). Prove it is transient \(\iff\)
  \(\forall \mu_0\) is primitive distribution, \(\lim_{n \to \infty} |X_n| \overset{\text{a.s.}}{=} \infty\).
\end{problem}
\begin{solution}
  Only need to prove that \(\forall k \in \mathbb{N},\liminf_{n \to \infty}|X_n|>k,a. s.\).
  Consider the event \(\liminf_{n \to \infty}|X_n|\leq k\), it means \(\forall n \in \mathbb{N},\exists t \geq n,X_t \in [-k,k]\).
  So we only need to prove \(\mathbb{P}\left(\bigcap_{n=1}^{\infty} \bigcup_{t=n}^{\infty} \{X_t \in [-k,k]\}\right)=0\).
  It is sufficient to prove that \(\mathbb{P}(\bigcup_{u \in [-k,k]}\bigcap_{n=1}^{\infty} \bigcup_{t=n}^{\infty} \{X_t=u\})=0\).
  Since \((X_n)\) is transient, it has been proved that \(\mathbb{P}(\bigcap_{n=1}^{\infty} \bigcup_{t=n}^{\infty} \{X_t=u\})=0\).
  So \(\mathbb{P}(\bigcup_{u \in [-k,k]}\bigcap_{n=1}^{\infty} \bigcup_{t=n}^{\infty} \{X_t=u\})=0 \leq \sum_{u \in [-k,k]}\mathbb{P}(\bigcap_{n=1}^{\infty} \bigcup_{t=n}^{\infty} \{X_t=u\})=0\).
\end{solution}

% \begin{problem}\label{pro:5}
%   Assume \(P\) is a transition matrix on \(\mathbb{Z}^+\) with first line \(\{a_0,a_1,\cdots\}\),
%   \(\forall i \geq 1\), \(p_{i,i-1}=1\), and \(\forall j \neq i-1\), \(p_{i,j}=0\).
%   Discuss the irreducibility, recurrence, ergodicity and periodicity of \(0\).
% \end{problem}
% \begin{solution}
%   If \(\exists n,\forall k \geq n,a_k=0\), then we can get that \(P_t(0,n)=0,\forall t \in \mathbb{N}\). 
%   Then the chain is reducible. 
%   Else, we get \(\forall n \in \mathbb{N}^+,\exists k \geq n,a_k>0\). 
%   Then \(P_{k-n+1}(0,n) \geq P(0,k)P(k,k-1)\cdots P(n+1,n)=P(0,k)=a_k>0\). 
%   And \(P_{n}(n,0)=1>0\). 
%   So the chain is irreducible. 
%
%   Assume \((X_n:n \in \mathbb{N})\) has transition matrix \(P\) and start at \(0\), and \(\tau=\inf\{n>0:X_n=0\}\). 
%   Then easy to get \(\tau=X_1+1,a. s.\). So \(\tau<\infty,a. s.\). So \(0\) is recurrent. 
%   It is easy to know that \(P_n(n,0)=1,\forall n \in \mathbb{N}^+\), so the class contains \(0\) is the only close class. 
%   So \(\forall n \in \mathbb{N},n\) is recurrent \(\iff n\) is in the class contains \(0\). 
%   So if \(\forall k \geq n,a_k=0\), then \(n\) is transient. 
%   And if \(\exists k \geq n,a_k>0\), then \(n\) is recurrent. 
%
% \end{solution}
\begin{problem}\label{pro:7}
  Assume \(P\) is a transition matrix on \(E\) and \(P\) is irreducible, \(j \in E\).
  Prove: \(P\) is recurrent \(\iff\) \(1\) is the minimum non negtive solution of
  \[
    y_i = \sum_{k \neq j} p_{ik}y_k + p_{ij}, i \in E\setminus\{j\}
  \]
\end{problem}
\begin{solution}

\end{solution}

\begin{problem}\label{pro:8}
  Let \(\{a_k:k \geq 0\}\) satisfies \(\sum_{k \geq 0}a_k =1, a_k \geq 1, a_0 >0\), \(\mu :=\sum_{k=1}^{\infty} ka_k >1\).
  Define \(p_{ij}=
  \begin{cases}
    a_j         & , i =0                    \\
    a_{j-i + 1} & ,i \geq 1 \AND j \geq i-1 \\
    0           & ,\text{otherwise}
  \end{cases}
  \).
  Prove: \(P\) is transient.
\end{problem}

\end{document}
