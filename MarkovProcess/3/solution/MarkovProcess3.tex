%!Mode:: "TeX:UTF-8"
%!TEX TS-program = xelatex
\documentclass{ctexart}
\newif\ifpreface
%\prefacetrue
\usepackage{fontspec}
\usepackage{bbm}
\usepackage{tikz}
\usepackage{amsmath,amssymb,amsthm,color,mathrsfs}
\usepackage{fixdif}
\usepackage{hyperref}
\usepackage{cleveref}
\usepackage{enumitem}%
\usepackage{expl3}
\usepackage{lipsum}
\usepackage[margin=0pt]{geometry}
\usepackage{listings}
\definecolor{mGreen}{rgb}{0,0.6,0}
\definecolor{mGray}{rgb}{0.5,0.5,0.5}
\definecolor{mPurple}{rgb}{0.58,0,0.82}
\definecolor{backgroundColour}{rgb}{0.95,0.95,0.92}

\lstdefinestyle{CStyle}{
  backgroundcolor=\color{backgroundColour},
  commentstyle=\color{mGreen},
  keywordstyle=\color{magenta},
  numberstyle=\tiny\color{mGray},
  stringstyle=\color{mPurple},
  basicstyle=\footnotesize,
  breakatwhitespace=false,
  breaklines=true,
  captionpos=b,
  keepspaces=true,
  numbers=left,
  numbersep=5pt,
  showspaces=false,
  showstringspaces=false,
  showtabs=false,
  tabsize=2,
  language=C
}
\usetikzlibrary{calc}
\theoremstyle{remark}
\newtheorem{lemma}{Lemma}
\usepackage{fontawesome5}
\usepackage{xcolor}
\newcounter{problem}
\newcommand{\Problem}{\begin{tikzpicture}[baseline]%
    \node at (-0.02em,0.3em) {$\mathbb{P}$};%
    \node[scale=0.7] at (0.2em,-0.0em) {R};%
    \node[scale=0.7] at (0.6em,0.4em) {O};%
    \node[scale=0.8] at (1.05em,0.25em) {B};%
    \node at (1.55em,0.3em) {L};%
    \node[scale=0.7] at (1.75em,0.45em) {E};%
    \node at (2.35em,0.3em) {M};%
  \end{tikzpicture}%
}
\renewcommand{\theproblem}{\Roman{problem}}
\newenvironment{problem}{\refstepcounter{problem}\noindent\color{blue}\Problem\theproblem}{}

\crefname{problem}{\protect\Problem}{Problem}
\newcommand\Solution{\begin{tikzpicture}[baseline]%
    \node at (-0.04em,0.3em) {$\mathbb{S}$};%
    \node[scale=0.7] at (0.35em,0.4em) {O};%
    \node at (0.7em,0.3em) {\textit{L}};%
    \node[scale=0.7] at (0.95em,0.4em) {U};%
    \node[scale=1.1] at (1.19em,0.32em){T};%
    \node[scale=0.85] at (1.4em,0.24em){I};%
    \node at (1.9em,0.32em){$\mathcal{O}$};%
    \node[scale=0.75] at (2.3em,0.21em){\texttt{N}};%
  \end{tikzpicture}}
\newenvironment{solution}{\begin{proof}[\Solution]}{\end{proof}}
\title{\input{../../.subject}\input{../.number}}
\makeatletter
\newcommand\email[1]{\def\@email{#1}\def\@refemail{mailto:#1}}
\newcommand\schoolid[1]{\def\@schoolid{#1}}
\ifpreface
  \def\@maketitle{
  \raggedright
  {\Huge \bfseries \sffamily \@title }\\[1cm]
  {\Huge  \bfseries \sffamily\heiti\@author}\\[1cm]
  {\Huge \@schoolid}\\[1cm]
  {\Huge\href\@refemail\@email}\\[0.5cm]
  \Huge\@date\\[1cm]}
\else
  \def\@maketitle{
    \raggedright
    \begin{center}
      {\Huge \bfseries \sffamily \@title }\\[4ex]
      {\Large  \@author}\\[4ex]
      {\large \@schoolid}\\[4ex]
      {\href\@refemail\@email}\\[4ex]
      \@date\\[8ex]
    \end{center}}
\fi
\makeatother
\ifpreface
  \usepackage[placement=bottom,scale=1,opacity=1]{background}
\fi

\author{白永乐}
\schoolid{202011150087}
\email{202011150087@mail.bnu.edu.cn}

\def\to{\rightarrow}
\newcommand{\xor}{\vee}
\newcommand{\bor}{\bigvee}
\newcommand{\band}{\bigwedge}
\newcommand{\xand}{\wedge}
\newcommand{\minus}{\mathbin{\backslash}}
\newcommand{\mi}[1]{\mathscr{P}(#1)}
\newcommand{\card}{\mathrm{card}}
\newcommand{\oto}{\leftrightarrow}
\newcommand{\hin}{\hat{\in}}
\newcommand{\gl}{\mathrm{GL}}
\newcommand{\im}{\mathrm{Im}}
\newcommand{\re }{\mathrm{Re }}
\newcommand{\rank}{\mathrm{rank}}
\newcommand{\tra}{\mathop{\mathrm{tr}}}
\renewcommand{\char}{\mathop{\mathrm{char}}}
\DeclareMathOperator{\ot}{ordertype}
\DeclareMathOperator{\dom}{dom}
\DeclareMathOperator{\ran}{ran}

\begin{document}
\large
\iffalse
  \setlength{\baselineskip}{1.2em}
  \ifpreface
    \backgroundsetup{contents={%
    \begin{tikzpicture}
      \fill [white] (current page.north west) rectangle ($(current page.north east)!.3!(current page.south east)$) coordinate (a);
      \fill [bgc] (current page.south west) rectangle (a);
\end{tikzpicture}}}
\definecolor{word}{rgb}{1,1,0}
\definecolor{bgc}{rgb}{1,0.95,0}
\setlength{\parindent}{0pt}
\thispagestyle{empty}
\begin{tikzpicture}%
  % \node[xscale=2,yscale=4] at (0cm,0cm) {\sffamily\bfseries \color{word} under};%
  \node[xscale=4.5,yscale=10] at (10cm,1cm) {\sffamily\bfseries \color{word} Graduate Homework};%
  \node[xscale=4.5,yscale=10] at (8cm,-2.5cm) {\sffamily\bfseries \color{word} In Mathematics};%
\end{tikzpicture}
\ \vspace{1cm}\\
\begin{minipage}{0.25\textwidth}
  \textcolor{bgc}{王胤雅是傻逼}
\end{minipage}
\begin{minipage}{0.75\textwidth}
  \maketitle
\end{minipage}
\vspace{4cm}\ \\
\begin{minipage}{0.2\textwidth}
  \
\end{minipage}
\begin{minipage}{0.8\textwidth}
  {\Huge
    \textinconsolatanf{}
  }General fire extinguisher
\end{minipage}
\newpage\backgroundsetup{contents={}}\setlength{\parindent}{2em}

  \else
    \maketitle
  \fi
\fi
\newgeometry{left=2cm,right=2cm,top=2cm,bottom=2cm}
%from_here_to_type
\begin{problem}\label{pro:1}
  Assume \((B_t:t \geq 0)\) is Brownian motion, prove that for \(r>0\), we have \((B_{t + r}-B_r:t \geq 0)\) is Brownian motion, too.
\end{problem}
\begin{solution}
  Assume \(B_t-B_s \sim N(0,a(t-s)),a >0\).
  Let \(\mathcal{F}_t:=\sigma(B_s:0 \leq s \leq t)\) and \(\mathcal{G}_t:=\sigma(B_{r+s}-B_r:0 \leq s \leq t)\).
  For \(0 \leq s \leq t\), we have \(B_{t+r}-B_r -(B_{s+r}-B_r)=B_{t+r}-B_{s+r} \sim N(0,a(t-s))\).
  And easily to know \(B_{r + s}-B_r \in \mathcal{F}_{r+s}\), so \(\mathcal{G}_t \subset \mathcal{F}_{t+r},\forall t \geq 0\).
  Since \((B_t:t \geq 0)\) is Brownian motion, easily \(\mathcal{F}_{s+r} \perp B_{t+r}-B_{s+r}\).
  Since \(\mathcal{G}_{t}\subset \mathcal{F}_{t +r}\), we obtain \(\mathcal{G}_t \perp B_{t + r}-B_{s + r}=B_{t+r}-B_r -(B_{s+r}-B_r)\).
  So \((B_{t+r}:t \geq 0)\) is Brownian motion.
\end{solution}

\begin{problem}\label{pro:2}
  Assume \((B_t:t \geq 0)\) is standrad Brownian motion start at \(0\).
  Prove that \(\forall c>0,(c B_{\frac{t}{c^2}}:t \geq 0)\) is standrad Brownian motion start at \(0\), too.
\end{problem}
\begin{solution}
  Since \(B_0=0\) we get \(c B_{\frac{0}{c^2}}=0\).
  Let \(\mathcal{F}_t:=\sigma(B_s:0 \leq s \leq t)\) and \(\mathcal{G}_t:=\sigma(c B_{\frac{s}{c^2}}:0 \leq s \leq t)\).
  Easily to know \(\mathcal{G}_t=\mathcal{F}_{\frac{t}{c^2}}\).
  For \(0 \leq s \leq t\), we have \(c B_{\frac{t}{c^2}}-c B_{\frac{s}{c^2}}= c(B_{\frac{t}{c^2}}-B_{\frac{s}{c^2}}) \sim N(0,t-s)\),
  because \(B_{\frac{t}{c^2}}-B_{\frac{s}{c^2}} \sim N(0,\frac{t-s}{c^2})\).
  And since \((B_t:t \geq 0)\) is Brownian motion, we get \(B_{\frac{t}{c^2}}-B_{\frac{s}{c^2}} \perp \mathcal{F}_{\frac{s}{c^2}}=\mathcal{G}_s\).
  So \((c B_{\frac{t}{c^2}}:t \geq 0)\) is standrad Brownian motion starts at \(0\), too.
\end{solution}
\begin{problem}\label{pro:3}
  Assume \((X_t:t \geq 0)\) and \((Y_t:t \geq 0)\) are two independent standrad Brownian motion, \(a,b \in \mathbb{R}\) and \(\sqrt{a^2 + b^2} >0\).
  Prove that \((aX_t + bY_t:t \geq 0)\) is a Brownian motion with parameter \(c=\sqrt{a^2 + b^2}\).
\end{problem}
\begin{solution}
  Let \(\mathcal{F}_t:=\sigma(X_s:0 \leq s \leq t)\) and \(\mathcal{G}_t:=\sigma(Y_s:0 \leq s \leq t)\).
  Let \(\mathcal{H}_t:=\sigma(a X_s + b Y_s:0 \leq s \leq t)\).
  Since \((X_t:t \geq 0),(Y_t:t \geq 0)\) are two independent Brownian motion, we know \(\forall 0 \leq s \leq t,X_t - X_s \perp \mathcal{F}_s,\mathcal{G}_s;Y_t-Y_s \perp \mathcal{F}_s,\mathcal{G}_s\).
  So we get \(aX_t+bY_t-aX_s-bY_s \perp \mathcal{F}_s,\mathcal{G}_s\), thus \(aX_t+bY_t-aX_s-bY_s \perp \sigma(\mathcal{F}_s,\mathcal{G}_s)\)
  Easily \(a X_s+bY_s \in \sigma(\mathcal{F}_s,\mathcal{G}_s)\), so \(\mathcal{H}_t \subset \sigma(\mathcal{F}_t,\mathcal{G},t),\forall t \geq 0\).
  So \(a X_t+bY_t-aX_s-bY_s \perp \mathcal{H}_s\).
  And easily \(a(X_t-X_s) \sim N(0,a^2(t-s)),b(Y_t-Y_s) \sin N(0,b^2(t-s))\), and since \(\mathcal{F}_t \perp \mathcal{G}_t\) we get \(a(X_t-X_s) \perp b(Y_t-Y_s)\),
  so \(a X_t+bY_t-aX_s-bY_s \sim N(0,(a^2 + b^2)(t-s))\).
  So \((a X_t + b Y_t:t \geq 0)\) is a Brownian motion with parameter \(a^2 + b^2 = c^2\).
\end{solution}
\begin{problem}\label{pro:4}
  Assume \((B_t:t \geq 0)\) is standrad Brownian motion start at \(0\).
  Let \(X_0=0\) and \(X_t:=t B_{\frac{1}{t}}\).
  Given
  \[
    \limsup_{t \to \infty}\frac{B_{t}}{\sqrt{2t \log \log t}}=1
  \]
  Prove that \((X_t:t \geq 0)\) is standrad Brownian motion start at \(0\).
\end{problem}
\begin{solution}

\end{solution}
\end{document}
