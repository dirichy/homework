%!Mode:: "TeX:UTF-8"
%!TEX encoding = UTF-8 Unicode
%!TEX TS-program = xelatex
\documentclass{ctexart}
\newif\ifpreface
%\prefacetrue
\usepackage{fontspec}
\usepackage{bbm}
\usepackage{tikz}
\usepackage{amsmath,amssymb,amsthm,color,mathrsfs}
\usepackage{fixdif}
\usepackage{hyperref}
\usepackage{cleveref}
\usepackage{enumitem}%
\usepackage{expl3}
\usepackage{lipsum}
\usepackage[margin=0pt]{geometry}
\usepackage{listings}
\definecolor{mGreen}{rgb}{0,0.6,0}
\definecolor{mGray}{rgb}{0.5,0.5,0.5}
\definecolor{mPurple}{rgb}{0.58,0,0.82}
\definecolor{backgroundColour}{rgb}{0.95,0.95,0.92}

\lstdefinestyle{CStyle}{
  backgroundcolor=\color{backgroundColour},
  commentstyle=\color{mGreen},
  keywordstyle=\color{magenta},
  numberstyle=\tiny\color{mGray},
  stringstyle=\color{mPurple},
  basicstyle=\footnotesize,
  breakatwhitespace=false,
  breaklines=true,
  captionpos=b,
  keepspaces=true,
  numbers=left,
  numbersep=5pt,
  showspaces=false,
  showstringspaces=false,
  showtabs=false,
  tabsize=2,
  language=C
}
\usetikzlibrary{calc}
\theoremstyle{remark}
\newtheorem{lemma}{Lemma}
\usepackage{fontawesome5}
\usepackage{xcolor}
\newcounter{problem}
\newcommand{\Problem}{\begin{tikzpicture}[baseline]%
    \node at (-0.02em,0.3em) {$\mathbb{P}$};%
    \node[scale=0.7] at (0.2em,-0.0em) {R};%
    \node[scale=0.7] at (0.6em,0.4em) {O};%
    \node[scale=0.8] at (1.05em,0.25em) {B};%
    \node at (1.55em,0.3em) {L};%
    \node[scale=0.7] at (1.75em,0.45em) {E};%
    \node at (2.35em,0.3em) {M};%
  \end{tikzpicture}%
}
\renewcommand{\theproblem}{\Roman{problem}}
\newenvironment{problem}{\refstepcounter{problem}\noindent\color{blue}\Problem\theproblem}{}

\crefname{problem}{\protect\Problem}{Problem}
\newcommand\Solution{\begin{tikzpicture}[baseline]%
    \node at (-0.04em,0.3em) {$\mathbb{S}$};%
    \node[scale=0.7] at (0.35em,0.4em) {O};%
    \node at (0.7em,0.3em) {\textit{L}};%
    \node[scale=0.7] at (0.95em,0.4em) {U};%
    \node[scale=1.1] at (1.19em,0.32em){T};%
    \node[scale=0.85] at (1.4em,0.24em){I};%
    \node at (1.9em,0.32em){$\mathcal{O}$};%
    \node[scale=0.75] at (2.3em,0.21em){\texttt{N}};%
  \end{tikzpicture}}
\newenvironment{solution}{\begin{proof}[\Solution]}{\end{proof}}
\title{\input{../../.subject}\input{../.number}}
\makeatletter
\newcommand\email[1]{\def\@email{#1}\def\@refemail{mailto:#1}}
\newcommand\schoolid[1]{\def\@schoolid{#1}}
\ifpreface
  \def\@maketitle{
  \raggedright
  {\Huge \bfseries \sffamily \@title }\\[1cm]
  {\Huge  \bfseries \sffamily\heiti\@author}\\[1cm]
  {\Huge \@schoolid}\\[1cm]
  {\Huge\href\@refemail\@email}\\[0.5cm]
  \Huge\@date\\[1cm]}
\else
  \def\@maketitle{
    \raggedright
    \begin{center}
      {\Huge \bfseries \sffamily \@title }\\[4ex]
      {\Large  \@author}\\[4ex]
      {\large \@schoolid}\\[4ex]
      {\href\@refemail\@email}\\[4ex]
      \@date\\[8ex]
    \end{center}}
\fi
\makeatother
\ifpreface
  \usepackage[placement=bottom,scale=1,opacity=1]{background}
\fi

\author{白永乐}
\schoolid{202011150087}
\email{202011150087@mail.bnu.edu.cn}

\def\to{\rightarrow}
\newcommand{\xor}{\vee}
\newcommand{\bor}{\bigvee}
\newcommand{\band}{\bigwedge}
\newcommand{\xand}{\wedge}
\newcommand{\minus}{\mathbin{\backslash}}
\newcommand{\mi}[1]{\mathscr{P}(#1)}
\newcommand{\card}{\mathrm{card}}
\newcommand{\oto}{\leftrightarrow}
\newcommand{\hin}{\hat{\in}}
\newcommand{\gl}{\mathrm{GL}}
\newcommand{\im}{\mathrm{Im}}
\newcommand{\re }{\mathrm{Re }}
\newcommand{\rank}{\mathrm{rank}}
\newcommand{\tra}{\mathop{\mathrm{tr}}}
\renewcommand{\char}{\mathop{\mathrm{char}}}
\DeclareMathOperator{\ot}{ordertype}
\DeclareMathOperator{\dom}{dom}
\DeclareMathOperator{\ran}{ran}

\begin{document}
\large
\setlength{\baselineskip}{1.2em}
\ifpreface
  \backgroundsetup{contents={%
    \begin{tikzpicture}
      \fill [white] (current page.north west) rectangle ($(current page.north east)!.3!(current page.south east)$) coordinate (a);
      \fill [bgc] (current page.south west) rectangle (a);
\end{tikzpicture}}}
\definecolor{word}{rgb}{1,1,0}
\definecolor{bgc}{rgb}{1,0.95,0}
\setlength{\parindent}{0pt}
\thispagestyle{empty}
\begin{tikzpicture}%
  % \node[xscale=2,yscale=4] at (0cm,0cm) {\sffamily\bfseries \color{word} under};%
  \node[xscale=4.5,yscale=10] at (10cm,1cm) {\sffamily\bfseries \color{word} Graduate Homework};%
  \node[xscale=4.5,yscale=10] at (8cm,-2.5cm) {\sffamily\bfseries \color{word} In Mathematics};%
\end{tikzpicture}
\ \vspace{1cm}\\
\begin{minipage}{0.25\textwidth}
  \textcolor{bgc}{王胤雅是傻逼}
\end{minipage}
\begin{minipage}{0.75\textwidth}
  \maketitle
\end{minipage}
\vspace{4cm}\ \\
\begin{minipage}{0.2\textwidth}
  \
\end{minipage}
\begin{minipage}{0.8\textwidth}
  {\Huge
    \textinconsolatanf{}
  }General fire extinguisher
\end{minipage}
\newpage\backgroundsetup{contents={}}\setlength{\parindent}{2em}

\else
  %\maketitle
\fi
\newgeometry{left=2cm,right=2cm,top=2cm,bottom=2cm}
%from_here_to_type
\begin{problem}\label{pro:1}
  Let \(X=\{X(n): n \geq 0\}\) be Markov chain defined on probability space \((\Omega,\mathscr{F},\mathbb{P})\), with state space \(E\) and
  transition probability matrix \(P=(p(i,j):i,j \in E)\). Let \(a, b \in E\), \(\tau_0 =0\), \(\sigma_k = \inf \{n \geq \tau_{k-1}: X(n)=b\}, \tau_k=\inf\{n \geq \sigma_{k-1}: X(n)=a\}\).
  Prove: \(\tau_n,\sigma_n, n \geq 1 \) are all stopping time on \((\mathscr{F}_n: n \geq 0)\).
\end{problem}

\begin{solution}
  We use MI to prove it. Easily \(\sigma_1=\inf \{n \geq \tau_0:X(n)=b\}=\inf \{n \geq 0:X(n)=b\}\) is stopping time.
  Assume for certain \(n \geq 1\), we have proved that \(\sigma_n,\tau_{n-1}\) are stopping times, now we need to prove \(\sigma_{n+1},\tau_n\) are stopping times.
  Since \(\sigma_n\) is stopping time, we know \(\forall k \leq m,\{k \geq \sigma_n\} \in \mathcal{F}_m\).
  And obviously \(\forall k \leq m,\{X(k)=a\}\in \mathcal{F}_m\). So we obtain that \(\{\sigma_n \leq m\}=\bigcup_{k=1}^{m} \{k \geq \sigma_n,X(k)=a\} \in \mathcal{F}_m\).
  So \(\tau_n\) is stopping time.
  Since \(\tau_n\) is stopping time, we know \(\forall k \leq m,\{k \geq \tau_n\} \in \mathcal{F}_m\).
  And obviously \(\forall k \leq m,\{X(k)=b\}\in \mathcal{F}_m\). So we obtain that \(\{\tau_n \leq m\}=\bigcup_{k=1}^{m} \{k \geq \tau_n,X(k)=b\} \in \mathcal{F}_m\).
  So \(\sigma_{n+1}\) is stopping time.
  So we finally obtain that \(\forall n \in \mathbb{N}^+,\sigma_n,\tau_n\) are stopping times.
\end{solution}

\begin{problem}\label{pro:2}
  Let \((X_{n}: n \geq 0)\) be a one-dimension simple random walk starting at \(1\).
  Let \(e(n)=\{X_{n \wedge \tau_1}: n \geq 0\}\), where \(\tau_1=\inf \{n \geq 0: X_{n}=0\}\).
  Find the distribution of \(\sup_{n \geq 0}e(n)\).
\end{problem}
\begin{solution}
  Assume \(\mathbb{P}(X_{n+1}-X_n=1)=p,\mathbb{P}(X_{n+1}-X_n=-1)=q\), where \(p+q=1\).
  Let \(E:=\sup_{n \geq 0}e(n)\).
  Let \(m \in \mathbb{N}^+\).
  Let \(\sigma:=\inf \{n \geq 0:X_n=0 \text{ or } X_n=m\}\). Then \(\sigma\) is a stopping time.

  First we assume \(p \neq  q\).
  Let \(Y_n:=\left(\frac{q}{p}\right)^{X_n}\), then it's easy to check that \(Y_n\) is martingale.
  So we know that \(Y_{n \wedge \sigma}\) is martingale, too.
  Easy to get that \(\sigma<\infty,a. s.\), so \(Y_{n \wedge \sigma} \overset{\text{a.s.}}{\to} Y_{\sigma}\).
  And \(0 \leq Y_{n \wedge \sigma} \leq m\), so \(\mathbb{E}(Y_{\sigma})=\mathbb{E}(Y_{n \wedge \sigma})=\mathbb{E}(Y_0)\).
  Noting that \(\{X_\sigma =0\}\overset{\text{a.s.}}{=}\{E<m\}\) and \(\{X_{\sigma}= m\}\overset{\text{a.s.}}{=}\{E \geq m\}\),
  we get two equations:
  \[
    \begin{cases}
      \mathbb{P}(E<m)+\mathbb{P}(E \geq m)=1 \\
      \mathbb{P}(E<m)+\mathbb{P}(E \geq m) \left(\frac{q}{p}\right)^m =\frac{q}{p}
    \end{cases}.
  \]
  Solve this equation, we get \(\mathbb{P}(E \geq m)=\frac{\frac{q}{p}-1}{(\frac{q}{p})^m-1}\).
  Then \(\mathbb{P}(E=m)=\frac{(\frac{p}{q})^m(\frac{p}{q}-1)}{((\frac{p}{q})^m-1)((\frac{p}{q})^{m+1}-1)}\).
  Furthermore, easily \(\mathbb{P}(E=\infty)=\lim_{m \to \infty}\mathbb{P}(E \geq m)=\begin{cases}
    0             & \frac{q}{p}>1 \\
    1-\frac{q}{p} & \frac{q}{p}<1
  \end{cases}\).

  Second, we consider \(p=q=\frac{1}{2}\). Then easily \(X_n\) is martingale.
  So we know that \(X_{n \wedge \sigma}\) is martingale, too.
  Easy to get that \(\sigma<\infty,a. s.\), so \(X_{n \wedge \sigma} \overset{\text{a.s.}}{\to} X_{\sigma}\).
  And \(0 \leq X_{n \wedge \sigma} \leq m\), so \(\mathbb{E}(X_{\sigma})=\mathbb{E}(X_{n \wedge \sigma})=\mathbb{E}(X_0)\).
  Noting that \(\{X_\sigma =0\}\overset{\text{a.s.}}{=}\{E<m\}\) and \(\{X_{\sigma}= m\}\overset{\text{a.s.}}{=}\{E \geq m\}\),
  we get two equations:
  \[
    \begin{cases}
      \mathbb{P}(E<m)+\mathbb{P}(E \geq m)=1 \\
      0\mathbb{P}(E<m)+m\mathbb{P}(E \geq m) =1
    \end{cases}.
  \]
  Solve this equation, we get \(\mathbb{P}(E \geq m)=\frac{1}{m}\).
  So \(\mathbb{P}(E=m)=\frac{1}{m(m+1)}\), and easily \(\mathbb{P}(E=\infty)=0\).
\end{solution}

\begin{problem}\label{pro:3}
  Prove:
  \begin{enumerate}
    \item \label{it:3.1}When \(0 < p \leq q\), the reflecting random walk with transition matrix \(Q^a_+\) is recurrent.
    \item When \(0 < q \leq p\), the reflecting random walk with transition matrix \(Q^a_-\) is recurrent.
  \end{enumerate}
\end{problem}

\begin{solution}
  By symmetry, only need to prove \ref{it:3.1}. By shifting, without loss of generality we can assume \(a=0\).
  We consider the equation
  \[
    y_0=y_1,\forall i \geq 1,y_i=qy_{i-1}+py_{i+1}
  \]
  Only need to prove its all bounded solution are all constant.
  Easy to get \(y_{i+2}=\frac{1}{p}y_{i+1}-\frac{q}{p}y_{i}\).
  Consider the charasteristic equation of this sequence, \(x^2-\frac{x}{p}+\frac{q}{p}=0\).
  We get \(x_1=1,x_2=\frac{q}{p} \geq 1\).
  If \(x_2>1\), then \(y_n=c_1 x_1^n + c_2 x_2^n\) is bounded \(\iff c_2=0\), so \(y_n=c_1 x_1^n=c_1\) is constant.
  Else, \(x_2=x_1=1\), then \(y_n=(an+b)x_1^n=an+b\) is bounded \(\iff a=0\), so \(y_n=b\) is constant.
  So the Markov chain is recurrent.
\end{solution}

\begin{problem}\label{pro:4}
  Prove colloary 4.4.3. i.e., let \(\phi_0(n:n \in \mathbb{N}^+)\) be simple random walk begin at \(\phi_0(0) \geq a+1\),
  let \(\zeta_0:=\inf \{m:\phi_0(m)=a+1\}\), let \(Y_n:n \in \mathbb{N}\) be reflecting simple random walk on \(\mathbb{Z}^a_+\), starting at \(a+1\), indenpendent with \(\phi_0\).
  Let \(X_n:=\begin{cases}
    \phi_0(n)     & n \leq \zeta_0 \\
    Y_{n-\zeta_0} & n \geq \zeta_0
  \end{cases}\).
  Prove that \(X_n:n \in \mathbb{N}\) is reflecting random walk on \(\mathbb{Z}^a_+\) begin at \(\phi_0(0)\).
\end{problem}
\begin{solution}
  Now we consider \(n \in \mathbb{N}^+\) and \(i_0,i_1,i_2,\cdots,i_{n+1}  \in \mathbb{Z}^a_+\).
  \begin{enumerate}
    \item If \(\forall k:1 \leq k \leq n,i_k \neq a+1\), then we have
      \[
        \begin{aligned}
           & \mathbb{P}(X_0=i_0,\cdots,X_{n+1} =i_{n+1} )                                                      \\
           & =\mathbb{P}(\phi_0(0)=i_0,\cdots,\phi_0(n+1)=i_{n+1} )                                            \\
           & =\mathbb{P}(\phi_0(0)=i_0,\cdots,\phi_0(n)=i_n)\mathbb{P}(\phi_0(n+1)=i_{n+1} \mid \phi_0(n)=i_n) \\
           & =\mathbb{P}(X_0=i_0,\cdots,X_n=i_n)q^a_+(i_n,i_{n+1})
        \end{aligned}
      \]
    \item Else, we let \(k:=\inf \{m:1 \leq m \leq n,i_m=a+1\}\).
      Then we have
      \[
        \begin{aligned}
           & \mathbb{P}(X_0=i_0,\cdots,X_{n+1}=i_{n+1})                                                                              \\
           & =\mathbb{P}(\phi_0(0)=i_0,\cdots,\phi_0(k)=i_k,Y_0=a+1,Y_1=i_{k+1},\cdots,Y_{n-k}=i_{n},Y_{n-k+1}=i_{n+1})              \\
           & =\mathbb{P}(\phi_0(0)=i_0,\cdots,\phi_0(k)=i_k)\mathbb{P}(Y_0=a+1,Y_1=i_{k+1},\cdots,Y_{n-k}=i_{n},Y_{n-k+1}=i_{n+1})   \\
           & =\mathbb{P}(\phi_0(0)=i_0,\cdots,\phi_0(k)=i_k)\mathbb{P}(Y_0=a+1,Y_1=i_{k+1},\cdots,Y_{n-k}=i_{n}) q^a_+(i_n,i_{n+1} ) \\
           & =\mathbb{P}(X_0=i_0,\cdots,X_n=i_n)q^a_+(i_n,i_{n+1})
        \end{aligned}
      \]
  \end{enumerate}
  So we get \((X_n:n \geq 0)\) is reflecting simple random walk on \(\mathbb{Z}^a_+\).
\end{solution}

\end{document}
