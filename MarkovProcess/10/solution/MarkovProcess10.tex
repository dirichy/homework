%!Mode:: "TeX:UTF-8"
%!TEX encoding = UTF-8 Unicode
%!TEX TS-program = xelatex
\documentclass{ctexart}
\newif\ifpreface
%\prefacetrue
\usepackage{fontspec}
\usepackage{bbm}
\usepackage{tikz}
\usepackage{amsmath,amssymb,amsthm,color,mathrsfs}
\usepackage{fixdif}
\usepackage{hyperref}
\usepackage{cleveref}
\usepackage{enumitem}%
\usepackage{expl3}
\usepackage{lipsum}
\usepackage[margin=0pt]{geometry}
\usepackage{listings}
\definecolor{mGreen}{rgb}{0,0.6,0}
\definecolor{mGray}{rgb}{0.5,0.5,0.5}
\definecolor{mPurple}{rgb}{0.58,0,0.82}
\definecolor{backgroundColour}{rgb}{0.95,0.95,0.92}

\lstdefinestyle{CStyle}{
  backgroundcolor=\color{backgroundColour},
  commentstyle=\color{mGreen},
  keywordstyle=\color{magenta},
  numberstyle=\tiny\color{mGray},
  stringstyle=\color{mPurple},
  basicstyle=\footnotesize,
  breakatwhitespace=false,
  breaklines=true,
  captionpos=b,
  keepspaces=true,
  numbers=left,
  numbersep=5pt,
  showspaces=false,
  showstringspaces=false,
  showtabs=false,
  tabsize=2,
  language=C
}
\usetikzlibrary{calc}
\theoremstyle{remark}
\newtheorem{lemma}{Lemma}
\usepackage{fontawesome5}
\usepackage{xcolor}
\newcounter{problem}
\newcommand{\Problem}{\begin{tikzpicture}[baseline]%
    \node at (-0.02em,0.3em) {$\mathbb{P}$};%
    \node[scale=0.7] at (0.2em,-0.0em) {R};%
    \node[scale=0.7] at (0.6em,0.4em) {O};%
    \node[scale=0.8] at (1.05em,0.25em) {B};%
    \node at (1.55em,0.3em) {L};%
    \node[scale=0.7] at (1.75em,0.45em) {E};%
    \node at (2.35em,0.3em) {M};%
  \end{tikzpicture}%
}
\renewcommand{\theproblem}{\Roman{problem}}
\newenvironment{problem}{\refstepcounter{problem}\noindent\color{blue}\Problem\theproblem}{}

\crefname{problem}{\protect\Problem}{Problem}
\newcommand\Solution{\begin{tikzpicture}[baseline]%
    \node at (-0.04em,0.3em) {$\mathbb{S}$};%
    \node[scale=0.7] at (0.35em,0.4em) {O};%
    \node at (0.7em,0.3em) {\textit{L}};%
    \node[scale=0.7] at (0.95em,0.4em) {U};%
    \node[scale=1.1] at (1.19em,0.32em){T};%
    \node[scale=0.85] at (1.4em,0.24em){I};%
    \node at (1.9em,0.32em){$\mathcal{O}$};%
    \node[scale=0.75] at (2.3em,0.21em){\texttt{N}};%
  \end{tikzpicture}}
\newenvironment{solution}{\begin{proof}[\Solution]}{\end{proof}}
\title{\input{../../.subject}\input{../.number}}
\makeatletter
\newcommand\email[1]{\def\@email{#1}\def\@refemail{mailto:#1}}
\newcommand\schoolid[1]{\def\@schoolid{#1}}
\ifpreface
  \def\@maketitle{
  \raggedright
  {\Huge \bfseries \sffamily \@title }\\[1cm]
  {\Huge  \bfseries \sffamily\heiti\@author}\\[1cm]
  {\Huge \@schoolid}\\[1cm]
  {\Huge\href\@refemail\@email}\\[0.5cm]
  \Huge\@date\\[1cm]}
\else
  \def\@maketitle{
    \raggedright
    \begin{center}
      {\Huge \bfseries \sffamily \@title }\\[4ex]
      {\Large  \@author}\\[4ex]
      {\large \@schoolid}\\[4ex]
      {\href\@refemail\@email}\\[4ex]
      \@date\\[8ex]
    \end{center}}
\fi
\makeatother
\ifpreface
  \usepackage[placement=bottom,scale=1,opacity=1]{background}
\fi

\author{白永乐}
\schoolid{25110180002}
\email{ylbai25@m.fudan.edu.cn}

\def\to{\rightarrow}
\newcommand{\xor}{\vee}
\newcommand{\AND}{\wedge}
\newcommand{\OR}{\vee}
\newcommand{\bor}{\bigvee}
\newcommand{\band}{\bigwedge}
\newcommand{\xand}{\wedge}
\newcommand{\minus}{\mathbin{\backslash}}
\newcommand{\mi}[1]{\mathscr{P}(#1)}
\newcommand{\card}{\mathrm{card}}
\newcommand{\oto}{\leftrightarrow}
\newcommand{\hin}{\hat{\in}}
\newcommand{\gl}{\mathrm{GL}}
\newcommand{\im}{\mathrm{Im}}
\newcommand{\re }{\mathrm{Re }}
\newcommand{\rank}{\mathrm{rank}}
\newcommand{\tra}{\mathop{\mathrm{tr}}}
\renewcommand{\char}{\mathop{\mathrm{char}}}
\DeclareMathOperator{\ot}{ordertype}
\DeclareMathOperator{\dom}{dom}
\DeclareMathOperator{\ran}{ran}

\begin{document}
\large
\setlength{\baselineskip}{1.2em}
\ifpreface
  \input{../../../global/preface}
\else
  %\maketitle
\fi
\newgeometry{left=2cm,right=2cm,top=2cm,bottom=2cm}
%from_here_to_type
\begin{problem}\label{pro:1}
  \begin{enumerate}
    \item Assume \(\{Y_1(n):n \geq 0\}, \{Y_2(n):n \geq 0\}\) are two independent migrating branch process with
      offspring distribution \((p(i):i \in \mathbb{N})\) and the migrating probobility respectively
      are \((\gamma_1(i): i \in \mathbb{Z}_{+}),(\gamma_2(i): i \in \mathbb{N})\).
      Prove: \(\{Y_1(n )+ Y_2(n): n \geq 0\}\) is migrating branching process with
      offspring distribution \(p(i): i \in \mathbb{N}\)
      and migrating probability \(\gamma_1 * \gamma_2\).
    \item Let \(\{Y(n): n \in \mathbb{N}\}\) be migrating branch process with offspring distribution \(p(j): j \in \mathbb{N}\)
      and the migrating distribution \(\gamma(i): i \in \mathbb{N}\).
      \(P_n^\gamma=(p_n^\gamma (i,j); i,j \in \mathbb{N})\) is the \(n\)-th transition matrix.
      Prove: \(\forall i,n \geq 1\)
      \[
        \sum_{j=0}^{\infty} p_n^\gamma(i,j)z^j = g_n(z)^i \prod_{k=1}^{n} h(g_{k-1}(z)), |z| \leq 1
      \]
      where \(h\) is the generating function of \((\gamma(j): j \in \mathbb{N})\).
      \(g\) is the generating function of \((p(j): j \in \mathbb{N})\).
    \item \(h,g\) are defined as above. Assume \(m :=g'(1-) < \infty, \mu:=h'(1-) < \infty\).
      Prove: \(\forall i,n \geq 1\), \[
        \mathbb{P}(Y_n \mid Y_0 =i)=im^n + \mu \sum_{k=1}^{n} m^{k-1}
      \]
  \end{enumerate}
\end{problem}
\begin{solution}
  \begin{enumerate}
    \item Since \(Y_1,Y_2\) are independent Markov chain, we easily get \(\mathbb{P}(Y_1(n+1)+Y_2(n+1)=i \mid \sigma(Y_1(j),Y_2(j):0 \leq j \leq n))=\mathbb{P}(Y_1(n+1)+Y_2(n+1)=i \mid Y_1(n),Y_2(n))\).
      So to prove \(Y_1+Y_2\) is Markov chain, we only need to prove \(\mathbb{P}(Y_1(n+1)+Y_2(n+1)=i \mid Y_1(n),Y_2(n))\mathbb{P}(Y_1(n+1)+Y_2(n+1)=i \mid Y_1(n)+Y_2(n))\).
      \[
        \begin{aligned}
            & \mathbb{P}(Y_1(n+1)+Y_2(n+1)=i \mid Y_1(n)=j,Y_2(n)=k)                                \\
          = & \sum_{x+y=i} \mathbb{P}(Y_1(n+1)=x \mid Y_1(n)=j)\mathbb{P}(Y_2(n+1)=y \mid Y_2(n)=k) \\
          = & \sum_{x+y=i} p^{*j}* \gamma_1(x) p^{*k}*\gamma_2(y)                                   \\
          = & p^{*j}*\gamma_1*p^{*k}*\gamma_2(i)                                                    \\
          = & p^{*(j+k)}*\gamma_1*\gamma_2(i)
        \end{aligned}
      \]
      So \(\mathbb{P}(Y_1(n+1)+Y_2(n+1)=i \mid Y_1(n),Y_2(n))=p^{*(Y_1(n)+Y_2(n))}*(\gamma_1*\gamma_2)(i) \in \sigma(Y_1(n)+Y_2(n))\subset \sigma(Y_1(n),Y_2(n))\).
      So \(Y_1+Y_2\) is Markov chain.
      More over, we have obtained \(\mathbb{P}(Y_1(n+1)+Y_2(n+1)=j \mid Y_1(n)+Y_2(n)=i)=p^{*i}*(\gamma_1*\gamma_2)(j)\).
      So \(\{Y_1(n )+ Y_2(n): n \geq 0\}\) is migrating branching process with
      offspring distribution \(p(i): i \in \mathbb{N}\)
      and migrating probability \(\gamma_1 * \gamma_2\).
    \item Use MI to prove it.
      Write \(G_n(i,z):=\sum_{j=0}^{\infty} p_n^{\gamma}(i,j)z^j\).
      When \(n=0\), we have \(p_0^{\gamma}(i,j)=\delta_{ij}\), so \(G_0(i,z)=z^i=g_0(z)^i\).
      When \(n=1\), we have \(p_1^\gamma(i,j)=p^{*i}*\gamma(j)\).
      So \(G_1(i,z)=g(z)^i h(z)\).
      Assume for certain \(n\) we have proved that \(G_n(i,z)=g_n(z)^i \prod_{k=1}^{n} h(g_{k-1}(z))\),
      Consider \(n+1\).
      Easily \(p_{n + 1}^\gamma(i,j)=\sum_{k \in \mathbb{N}}p_n^\gamma(k,j)p(i,\cdot)*\gamma (k)\).
      So
      \[
        \begin{aligned}
            & G_{n + 1}(i,z)=\sum_{j \in \mathbb{N}} \sum_{k \in \mathbb{N}}p_1^\gamma(i,k)  p_n^\gamma(k,j)z^j \\
          = & \sum_{k \in \mathbb{N}}p_1^\gamma(i,k)G_n(k,z)                                                    \\
          = & \prod_{k=1}^{n} h(g_{k-1} (z))\sum_{k \in \mathbb{N}}p_1^\gamma(i,k) g_n(z)^k                     \\
          = & \prod_{k=1}^{n} h(g_{k-1} (z)) G_1(i,g_n(z))                                                      \\
          = & g_{n+1}(z)\prod_{k=1}^{n + 1} h(g_{k-1} (z))
        \end{aligned}
      \]
    \item Easily \(\mathbb{P}(Y_n \mid Y_0=i)=D_z G_{n}(i,z) \mid_{z \to 1-}\).
      Noting \(g(1)=h(1)=1\), easy to get that \(\mathbb{P}(Y_n \mid Y_0=i)=im^n + \mu \sum_{k=1}^{n} m^{k-1}\).
  \end{enumerate}

\end{solution}

\begin{problem}\label{pro:2}
  Assume \(b \in (0,1), p \in (0,1)\). Let \(\mu(0)=\frac{1-b-p}{1-p},\mu(j)=bp^{j-1},j \geq 1\).
  Prove:
  \begin{enumerate}
    \item   \((\mu(j):j \in \mathbb{N})\) is probability distribution and \[
        g(z):=\sum_{j=0}^{\infty} \mu(j)z^j=\frac{1-b-p}{1-p} + \frac{bz}{1-pz}.
      \]

    \item Let \(b =(1-p)^2\).
      Then  \(g'(1)=1\) and \[
        g(z)=p + \frac{(1-p)^2 z}{1-pz}=\frac{p-(2p-1)z}{1-pz}.
      \]
      Prove:
      \(\forall n \geq 1\), \[
        g_n(z)=\frac{np-((n + 1)p-1)z}{1 + (n-1)p-npz}.
      \]
  \end{enumerate}
\end{problem}
\begin{solution}
  \begin{enumerate}
    \item Easily \(\sum_{j=1}^{\infty} \mu(j)=\frac{b}{1-p}\).
      So \(\sum_{j=0}^{\infty} \mu(j)=1\).
      Easily \(\sum_{j=1}^{\infty} \mu(j)z^j=\frac{bz}{1-pz}\).
      So \(g(z)=\mu(0)+\frac{bz}{1-pz}=\frac{1-b-p}{1-p} + \frac{bz}{1-pz}\).
    \item \(g_{n+1}(z)=g(g_n(z))=\frac{p-(2p-1)g_n(z)}{1-pg_n(z)}\).
      So \(g_{n+1}(z)-1=\frac{(g_n(z)-1)(1-p)}{1-p g_n(z)}\).
      Thus, we obtain \(\frac{1}{g_{n+1}(z)-1}=\frac{1}{g_n(z)-1}-\frac{p}{1-p}\).
      So \(\frac{1}{g_n(z)-1}=\frac{1}{z-1}-\frac{np}{1-p}\), and finally we get
      \( g_n(z)=\frac{np-((n + 1)p-1)z}{1 + (n-1)p-npz} \).
  \end{enumerate}

\end{solution}

\begin{problem}\label{pro:3}
  Let \(\{X(n): n \in \mathbb{N}\}\) be branch process with offspring distribution \(p(j): j \in \mathbb{N}\).
  And \(g\) is the generating function. Let \(m_2:=g'(1) + g''(1) < \infty\).
  Let \(m=g'(1)<\infty\).
  \(\forall k \geq 1\), \(X_n^{(k)}=k^{-1}X_n\).
  Prove: \(\forall \varepsilon >0, i,n \geq 1\), \(\mathbb{P}(|X_n^{(k)} -im^n| \geq \varepsilon \mid X_0^{(k)}=i) \to 0, k \to \infty\).
\end{problem}
\begin{solution}
  In fact, we don't need \(m_2<\infty\).
  We let \((Y(k,n):n \in \mathbb{N}),k \in \mathbb{N}\) are independent branch process with offspring distribution \(p(j):j \in \mathbb{N}\) and \(Y(k,0)=i\).
  Then \(\sum_{j=1}^{k} Y(j,n)\) is branch process with offspring distribution \(p(j):j \in \mathbb{N}\) and initial value \(ki\).
  So \(\sum_{j=1}^{k} Y(j,n) \overset{d}{=} X_n \mid X_0^{(k)}=i\).
  So \(\mathbb{P}(|X_n^{(k)}-im^n|\geq \varepsilon \mid X_0^{(k)}=i)=\mathbb{P}(|\frac{\sum_{j=1}^{k} Y(j,n)}{k}-im^n| \geq \varepsilon)\).
  By LLN we obtain \(\frac{1}{k}\sum_{j=1}^{k} Y(j,n)\overset{\text{a.s.}}{\to} im^n\).
  So finally we get \(\lim_{k \to \infty}\mathbb{P}(|X_n^{(k)}-im^n|\geq \varepsilon \mid X_0^{(k)}=i)=\lim_{k \to \infty}\mathbb{P}(|\frac{\sum_{j=1}^{k} Y(j,n)}{k}-im^n| \geq \varepsilon)=0\).
\end{solution}

\begin{problem}\label{pro:4}
  Let \(\{X(n): n \in \mathbb{N}\}\) be branch process with offspring distribution \(p(j): j \in \mathbb{N}\).
  And \(g\) is the generating function, where \(m :=g' (1) \in (1, \infty), m_2:=g'(1) + g''(1) < \infty\).
  Let \(\sigma^2:=m_2-m^2 = \mathbb{D}(X(1))\).
  It is well known that \(\exists W,\lim_{n \to \infty}\frac{X_n}{m^n} = W\).
  Prove: \[
    \lim_{n \to \infty}\mathbb{E}_1[(m^{-n}X_n -W)^2]= 0, \mathbb{D}_1(W)=\sigma^2 m^{-1}(m-1)^{-1}
  \]
\end{problem}
\begin{solution}
  For convinence we write \(\mathbb{E},\mathbb{D}\) instead of \(\mathbb{E}_1,\mathbb{D}_1\).
  Easy to get that \(\mathbb{E}(m ^{-2n} X_n^2)=\frac{\sigma^2(1-m^{-n})}{m^2-m}+1\).
  So by Fatou theorem we get that \(\mathbb{E}(W^2) \leq \lim_{n \to \infty}\mathbb{E}(m^{-2n}X_n^2)=\frac{\sigma^2}{m^2-m}+1<\infty\).
  And by Doob Stochastic Processes p317 theorem 3.4 we get that
  \(\mathbb{E}(\max_{n \in \mathbb{N}}m^{-2n}X_n^2) < \infty\).
  Thus, \(m^{-2n}X_n^2\) are integrable uniformly, and so do \((m^{-n}X_n-W)^2\).
  So by LCDT we can get \(\lim_{n \to \infty}\mathbb{E}((m^{-n}X_n-W)^2)=0\).
  Noting that
  \[
    \mathbb{E}(m^{-2n}X_n^2-W^2)=\mathbb{E}((m^{-n}X_n+W)(m^{-n}X_n-W)) \leq \sqrt{\mathbb{E}((m^{-n}X_n+W)^2) \mathbb{E}((m^{-n}X_n-W)^2)} \to 0
  \]
  ,
  we get \(\mathbb{E}(W^2)=\lim \mathbb{E}(m^{-n}X_n)=\frac{\sigma^2}{m^2-m}+1\).
  Also, \(\mathbb{E}(|m^{-n}X_n-W|)^2 \leq \mathbb{E}((m^{-n}X_n-W)^2)\),
  so \(\mathbb{E}(W)=\lim \mathbb{E}(m^{-n}X_n)=1\).
  So \(\mathbb{D}(W)=\mathbb{E}(W^2)-\mathbb{E}(W)^2=\frac{\sigma^2}{m(m-1)}\).
\end{solution}

\begin{problem}\label{pro:5}
  Let \(\{Y(n): n \in \mathbb{N}\}\) be branch process with offspring distribution \(p(j): j \in \mathbb{N}\).
  And \(g\) is the generating function, where \(m :=g' (1) \leq 1\).
  Prove \((p^\gamma(j): j \in \mathbb{N})\) is the steady-state vector of
  transition matrix \(P_n^\gamma\), that is \(\sum_{i =0}^{\infty} p^\gamma(i)p_n^\gamma(i,j)=p^{\gamma}(j), i \geq 0\).
\end{problem}
\begin{solution}
  Since \(\lim_{m \to \infty}p_m^\gamma(i,j)=p^\gamma(j)\), and fix \(k \in \mathbb{N}\), we have
  \(\sum_{j=0}^{\infty} p_m^\gamma(k,i)p_n^\gamma(i,j)=p_{n+m}^\gamma(k,j)\),
  we only need to prove that \(\lim_{m \to \infty}\sum_{i=0}^{\infty} (p_m^\gamma(k,i)-p^\gamma(i))p_n^\gamma(i,j)=0\).
  Since \(\lim_{m \to \infty}p_m^\gamma(k,i) =p^\gamma(i)\) and \(\sum_{i \in \mathbb{N}}p_m^\gamma(k,i)=1\),
  we can easily get that \(\sum_{i \in \mathbb{N}}p^\gamma(i)=1\).
  For \(\varepsilon>0\), we let \(N\) large enough such that \(\sum_{k=N}^{\infty} p^\gamma(k)<\varepsilon\).
  Then we let \(M\) large enough such that \(\forall i:0 \leq i <N,\forall m \geq M,|p_{m}^\gamma(k,i)-p^\gamma(k)|<\frac{\varepsilon}{N}\).
  Then
  \[
    \begin{aligned}
           & \left|\sum_{i=0}^{\infty} (p_m^\gamma(k,i)-p^\gamma(i))p_n^\gamma(i,j)\right|                                                                      \\
      \leq & \sum_{i=0}^{\infty} |p_m^\gamma(k,i)-p^\gamma(i)|p_n^\gamma(i,j)                                                                                   \\
      \leq & \sum_                    {i=0}^{N-1} |p_m^\gamma(k,i)-p^\gamma(i)|p_n^\gamma(i,j)+\sum_{i=N}^{\infty} (p_m^\gamma(k,i)+p^\gamma(i))p_n^\gamma(i,j) \\
      \leq & \sum_{i=0}^              {N-1} \frac{\varepsilon}{N} + \sum_{i=N}^{\infty} p_m^\gamma(k,i)+p^\gamma(i)                                             \\
      \leq & \varepsilon+\sum_        {i=N}^{\infty} p^\gamma(i) +1- \sum_{i=1}^{N-1} p_m^\gamma(k,i)                                                           \\
      \leq & \varepsilon + \varepsilon + 1 - \sum_{i=1}^{N-1} p^\gamma(i) + \sum_{i=1}^{N-1} |p_m^\gamma(k,i)-p^\gamma(i)|                                      \\
      \leq & 4\varepsilon
    \end{aligned}
  \]
  So finally we get \(\lim_{m \to \infty}\sum_{i=0}^{\infty} (p_m^\gamma(k,i)-p^\gamma(i))p_n^\gamma(i,j)=0\).
  Thus, \(\sum_{i =0}^{\infty} p^\gamma(i)p_n^\gamma(i,j)=p^{\gamma}(j), i \geq 0\).
\end{solution}

\begin{problem}\label{pro:6}
  Let \(\{Y(n): n \in \mathbb{N}\}\) be branch process with offspring distribution \(p(j): j \in \mathbb{N}\).
  And \(g\) is the generating function, where \(m :=g' (1) \leq 1\).
  Discuss \(\lim_{n \to \infty}\mathbb{E}(Y_n \mid Y_0=i)\).
\end{problem}
\begin{solution}
  Easy to get that \(\mathbb{E}(Y_n \mid Y_0=i)=im^n+\mu \sum_{k=1}^{n} m^{k-1}\).
  When \(m=1\), we know \(\mathbb{E}(Y_n \mid Y_0=i) \to \infty\).
  When \(m<1\), we know \(\mathbb{E}(Y_n \mid Y_0=i)\to \frac{\mu}{1-m}\).
\end{solution}

\end{document}
