%!Mode:: "TeX:UTF-8"
%!tex encoding = utf-8 unicode
%!tex ts-program = xelatex
\documentclass{ctexart}
\newif\ifpreface
%\prefacetrue
\usepackage{fontspec}
\usepackage{bbm}
\usepackage{tikz}
\usepackage{amsmath,amssymb,amsthm,color,mathrsfs}
\usepackage{fixdif}
\usepackage{hyperref}
\usepackage{cleveref}
\usepackage{enumitem}%
\usepackage{expl3}
\usepackage{lipsum}
\usepackage[margin=0pt]{geometry}
\usepackage{listings}
\definecolor{mGreen}{rgb}{0,0.6,0}
\definecolor{mGray}{rgb}{0.5,0.5,0.5}
\definecolor{mPurple}{rgb}{0.58,0,0.82}
\definecolor{backgroundColour}{rgb}{0.95,0.95,0.92}

\lstdefinestyle{CStyle}{
  backgroundcolor=\color{backgroundColour},
  commentstyle=\color{mGreen},
  keywordstyle=\color{magenta},
  numberstyle=\tiny\color{mGray},
  stringstyle=\color{mPurple},
  basicstyle=\footnotesize,
  breakatwhitespace=false,
  breaklines=true,
  captionpos=b,
  keepspaces=true,
  numbers=left,
  numbersep=5pt,
  showspaces=false,
  showstringspaces=false,
  showtabs=false,
  tabsize=2,
  language=C
}
\usetikzlibrary{calc}
\theoremstyle{remark}
\newtheorem{lemma}{Lemma}
\usepackage{fontawesome5}
\usepackage{xcolor}
\newcounter{problem}
\newcommand{\Problem}{\begin{tikzpicture}[baseline]%
    \node at (-0.02em,0.3em) {$\mathbb{P}$};%
    \node[scale=0.7] at (0.2em,-0.0em) {R};%
    \node[scale=0.7] at (0.6em,0.4em) {O};%
    \node[scale=0.8] at (1.05em,0.25em) {B};%
    \node at (1.55em,0.3em) {L};%
    \node[scale=0.7] at (1.75em,0.45em) {E};%
    \node at (2.35em,0.3em) {M};%
  \end{tikzpicture}%
}
\renewcommand{\theproblem}{\Roman{problem}}
\newenvironment{problem}{\refstepcounter{problem}\noindent\color{blue}\Problem\theproblem}{}

\crefname{problem}{\protect\Problem}{Problem}
\newcommand\Solution{\begin{tikzpicture}[baseline]%
    \node at (-0.04em,0.3em) {$\mathbb{S}$};%
    \node[scale=0.7] at (0.35em,0.4em) {O};%
    \node at (0.7em,0.3em) {\textit{L}};%
    \node[scale=0.7] at (0.95em,0.4em) {U};%
    \node[scale=1.1] at (1.19em,0.32em){T};%
    \node[scale=0.85] at (1.4em,0.24em){I};%
    \node at (1.9em,0.32em){$\mathcal{O}$};%
    \node[scale=0.75] at (2.3em,0.21em){\texttt{N}};%
  \end{tikzpicture}}
\newenvironment{solution}{\begin{proof}[\Solution]}{\end{proof}}
\title{\input{../../.subject}\input{../.number}}
\makeatletter
\newcommand\email[1]{\def\@email{#1}\def\@refemail{mailto:#1}}
\newcommand\schoolid[1]{\def\@schoolid{#1}}
\ifpreface
  \def\@maketitle{
  \raggedright
  {\Huge \bfseries \sffamily \@title }\\[1cm]
  {\Huge  \bfseries \sffamily\heiti\@author}\\[1cm]
  {\Huge \@schoolid}\\[1cm]
  {\Huge\href\@refemail\@email}\\[0.5cm]
  \Huge\@date\\[1cm]}
\else
  \def\@maketitle{
    \raggedright
    \begin{center}
      {\Huge \bfseries \sffamily \@title }\\[4ex]
      {\Large  \@author}\\[4ex]
      {\large \@schoolid}\\[4ex]
      {\href\@refemail\@email}\\[4ex]
      \@date\\[8ex]
    \end{center}}
\fi
\makeatother
\ifpreface
  \usepackage[placement=bottom,scale=1,opacity=1]{background}
\fi

\author{白永乐}
\schoolid{202011150087}
\email{202011150087@mail.bnu.edu.cn}

\def\to{\rightarrow}
\newcommand{\xor}{\vee}
\newcommand{\bor}{\bigvee}
\newcommand{\band}{\bigwedge}
\newcommand{\xand}{\wedge}
\newcommand{\minus}{\mathbin{\backslash}}
\newcommand{\mi}[1]{\mathscr{P}(#1)}
\newcommand{\card}{\mathrm{card}}
\newcommand{\oto}{\leftrightarrow}
\newcommand{\hin}{\hat{\in}}
\newcommand{\gl}{\mathrm{GL}}
\newcommand{\im}{\mathrm{Im}}
\newcommand{\re }{\mathrm{Re }}
\newcommand{\rank}{\mathrm{rank}}
\newcommand{\tra}{\mathop{\mathrm{tr}}}
\renewcommand{\char}{\mathop{\mathrm{char}}}
\DeclareMathOperator{\ot}{ordertype}
\DeclareMathOperator{\dom}{dom}
\DeclareMathOperator{\ran}{ran}

\begin{document}
\large
\setlength{\baselineskip}{1.2em}
\ifpreface
  \backgroundsetup{contents={%
    \begin{tikzpicture}
      \fill [white] (current page.north west) rectangle ($(current page.north east)!.3!(current page.south east)$) coordinate (a);
      \fill [bgc] (current page.south west) rectangle (a);
\end{tikzpicture}}}
\definecolor{word}{rgb}{1,1,0}
\definecolor{bgc}{rgb}{1,0.95,0}
\setlength{\parindent}{0pt}
\thispagestyle{empty}
\begin{tikzpicture}%
  % \node[xscale=2,yscale=4] at (0cm,0cm) {\sffamily\bfseries \color{word} under};%
  \node[xscale=4.5,yscale=10] at (10cm,1cm) {\sffamily\bfseries \color{word} Graduate Homework};%
  \node[xscale=4.5,yscale=10] at (8cm,-2.5cm) {\sffamily\bfseries \color{word} In Mathematics};%
\end{tikzpicture}
\ \vspace{1cm}\\
\begin{minipage}{0.25\textwidth}
  \textcolor{bgc}{王胤雅是傻逼}
\end{minipage}
\begin{minipage}{0.75\textwidth}
  \maketitle
\end{minipage}
\vspace{4cm}\ \\
\begin{minipage}{0.2\textwidth}
  \
\end{minipage}
\begin{minipage}{0.8\textwidth}
  {\Huge
    \textinconsolatanf{}
  }General fire extinguisher
\end{minipage}
\newpage\backgroundsetup{contents={}}\setlength{\parindent}{2em}

\else
  %\maketitle
\fi
\newgeometry{left=2cm,right=2cm,top=2cm,bottom=2cm}
%from_here_to_type
\allowdisplaybreaks[4]
\begin{problem}\label{pro:1}
  let \((x_n:n \geq 0)\perp (y_n: n \geq 0)\) are markov chain on \(e\) with
  transition matrix \((p_{ij}:i,j \in e),(q_{ij}:i,j \in e)\) respectively.
  prove: \(\{(x_n,y_n):n \geq 0\}\) are markov chain on \(e \times e\).
  and calculate the transition matrix of \((x_n,y_n):n \geq 0\).
\end{problem}
\begin{solution}
  easy to get that
  \[
    \begin{aligned}
       & \mathbb{p}(x_0=i_0,\cdots,x_{n+1} =i_{n+1},y_0=j_0,\cdots,y_{n+1}=j_{n+1})
      \\= & \mathbb{p}(x_0=i_0,\cdots,x_{n+1} =i_{n+1})\mathbb{p}(y_0=j_0,\cdots,y_{n+1}=j_{n+1})
      \\= & \mathbb{p}(x_0=i_0)\prod_{k=0}^{n} p_{i_k i_{k+1} } \mathbb{p}(y_0=j_0) \prod_{k=0}^{n} q_{j_k j_{k+1}}
      \\= & \mathbb{p}((x_0,y_0)=(i_0,j_0)) \prod_{k=0}^{n} p_{i_k i_{k+1} } q_{j_k j_{k+1}}
      \\= & \mathbb{p}((x_0,y_0)=(i_0,j_0)) \prod_{k=0}^{n} \mathbb{p}(x_k = i_k,x_{k + 1}=i_{k + 1})\mathbb{p}(y_k=j_k,y_{k + 1}=j_{k + 1})
      \\=& \mathbb{p}((x_0,y_0)=(i_0,j_0)) \prod_{k=0}^{n} \mathbb{p}((x_k,y_{k})=(i_k,j_{k}),(x_{k + 1},y_{k + 1})=(i_{k + 1},j_{k + 1}))
    \end{aligned}
  \]
  so we get that \(\left((x_n,y_n): n \in \mathbb{n}\right)\) is markov chain with transition matrix
  \(r_{(i,j),(m,n)}=p_{im}q_{jn}\).
\end{solution}

\begin{problem}\label{pro:2}
  let \(s_n\) is a one dimensional simple random walk. let \(a \in \mathbb{z}\).
  let \(\tau := \inf \{n \geq 0: s_n =a\}\).
  prove:
  \begin{enumerate}
    \item \((s_{\tau + n}:n \geq 0)\) is a one dimensional simple random walk.
    \item \((s_{n \wedge \tau }:n \geq 0)\) is a markov chain on \(\mathbb{z}\) and give its
      transition matrix.
    \item \((s_{n \wedge \tau}:n \geq 0) \perp (s_{\tau + n}: n \geq 0)\).
  \end{enumerate}
\end{problem}
\begin{solution}
  \begin{enumerate}
    \item easy to know that
      \[
        \begin{aligned}
           & \mathbb{p}(s_\tau=i_0,s_{\tau+1}=i_1,\cdots,s_{\tau+n}=i_n \mid \tau<\infty)
          \\= & \sum_{k \in \mathbb{n}}\mathbb{p}(\tau=k,s_\tau=i_0,s_{\tau+1}=i_1,\cdots,s_{\tau+n}=i_n \mid \tau<\infty)
          \\= & \sum_{k \in \mathbb{n}}\mathbb{p}(\tau=k,s_k=i_0,s_{k+1}=i_1,\cdots,s_{k+n}=i_n \mid \tau<\infty)
          \\= & \mathbbm{1}(a=i_0) \sum_{k \in \mathbb{n}}\mathbb{p}(s_0 \neq a,\cdots,s_{k-1}\neq a,s_k=a,s_{k+1}=i_1,\cdots,s_{k+n}=i_n \mid \tau<\infty)
          \\= & \frac{\mathbbm{1}(a=i_0) \sum_{k \in \mathbb{n}}\mathbb{p}(s_0 \neq a,\cdots,s_{k-1}\neq a,s_k=a,s_{k+1}=i_1,\cdots,s_{k+n}=i_n )}{\mathbb{p}(\tau<\infty)}
          \\= & \frac{\mathbbm{1}(a=i_0)}{\mathbb{p}(\tau<\infty)} \sum_{k \in \mathbb{n}}\mathbb{p}(s_0 \neq a,\cdots,s_{k-1}\neq a,s_k=a,s_{k+1}=i_1,\cdots,s_{k+n}=i_n)
          \\= & \frac{\mathbbm{1}(a=i_0)}{\mathbb{p}(\tau<\infty)} \sum_{k \in \mathbb{n}}\mathbb{p}(s_{k+1}=i_1,\cdots,s_{k+n}=i_n \mid s_0 \neq a,\cdots,s_{k-1}\neq a,s_k=a)
          \\ \times & \mathbb{p}(s_0 \neq a,\cdots,s_{k-1}\neq a,s_k=a)
          \\= & \frac{\mathbbm{1}(a=i_0)}{\mathbb{p}(\tau<\infty)} \sum_{k \in \mathbb{n}}\mathbb{p}(s_{k+1}=i_1,\cdots,s_{k+n}=i_n \mid s_k=a) \mathbb{p}(\tau=k)
          \\= & \frac{\mathbbm{1}(a=i_0)}{\mathbb{p}(\tau<\infty)} \sum_{k \in \mathbb{n}}\prod_{l=0}^{n-1} p_{i_l i_{l+1}} \mathbb{p}(\tau=k)=\mathbbm{1}(a=i_0) \prod_{l=0}^{n-1} p_{i_l i_{l+1}}
        \end{aligned}
      \]
      where \(p_{ij}:i,j \in \mathbb{z}\) is the transition matrix of \(s_n:n \in \mathbb{n}\).
      so \((s_{\tau+n}:n \in \mathbb{n})\) is markov chain with transition matrix same as \(s_n\).
    \item easy to know that
      \[
        \begin{aligned}
           & \mathbb{p}(s_{\tau \wedge 0}=i_0,s_{\tau\wedge 1}=i_1,\cdots,s_{\tau\wedge n}=i_n \mid \tau<\infty)
          \\= & \sum_{k \in \mathbb{n}}\mathbb{p}(\tau=k,s_{\tau \wedge 0}=i_0,s_{\tau\wedge 1}=i_1,\cdots,s_{\tau\wedge n}=i_n \mid \tau<\infty)
          \\= & \sum_{k \in \mathbb{n}}\mathbb{p}(\tau=k,s_{k \wedge 0}=i_0,s_{k\wedge 1}=i_1,\cdots,s_{k\wedge n}=i_n \mid \tau<\infty)
          \\= & \sum_{k \geq n} \mathbb{p}(\tau =k,s_0=i_0,\cdots,s_n=i_n \mid \tau<\infty)
          \\+ & \sum_{k < n} \mathbb{p}(\tau =k,s_0=i_0,\cdots,s_{k-1}=i_{k-1},s_k=i_k=i_{k+1} =\cdots=i_n \mid \tau<\infty)
          \\= & \mathbbm{1}(i_0,i_1,\cdots,i_n \neq a) \prod_{k=0}^{n-1} p_{i_k i_{k+1}} + \sum_{k=0}^{n-1} \mathbbm{1}(i_0,\cdots,i_{k-1} \neq a,i_k=i_{k+1}=\cdots=i_n=a) \prod_{l=0}^{k-1} p_{i_l i_{l+1}}
          \\=&\prod_{k=0}^{n-1} (\mathbbm{1}(i_k=i_{k+1}=a)+\mathbbm{1}(i_k \neq a)p_{i_k,i_{k+1}})
        \end{aligned}
      \]
      so \((s_{n \wedge \tau}:n \in \mathbb{n})\) is markov chain with transition matrix \(q_{i,j}=\mathbbm{1}(i=j=a)+\mathbbm{1}(i \neq a)p_{i,j}\).
    \item by the corollary 3.2.11, we only need to proof \(\tau\) is stopping time on \((\mathscr{f}_n:n \geq 0)\),
      where \(\mathscr{f}_n=\sigma(s_k:k \leq n)\).
      so we only need to prove \(\forall n \in \mathbb{n}\), \(\{\tau = n\} \in \mathscr{f}_n\).
      since \(\{\tau =n\}=\{\omega \in \omega: s_0,\cdots,s_{n + 1} \neq a, s_n = a\}=\bigcap_{0 \leq k \leq n}\{s_k \neq a\}\cap\{s_n=a\}\),
      and \(\{s_k \neq a\} \in \sigma(s_k), \forall 0 \leq k \leq n,\{s_n = a\} \in \sigma(s_n)\),
      then \(\{\tau = n\} \in \mathscr{f}_n\).
  \end{enumerate}
\end{solution}

\begin{problem}\label{pro:3}
  let \(s_n\) is a one dimensional symmetry simple random walk starting from zero.
  prove: \((|s_n|: n \geq 0)\) is a markov chain on \(\mathbb{z}^+\) and give its transition matrix.
\end{problem}
\begin{solution}
  only need to solve problem \ref{pro:4}.
\end{solution}

\begin{problem}\label{pro:4}
  let \(s_n\) is a one dimensional simple random walk starting from zero.
  prove: \((|s_n|: n \geq 0)\) is a markov chain on \(\mathbb{z}^+\) and give its transition matrix.
\end{problem}
\begin{solution}
  by the definition of \(|s_n|\), we can easily get to know
  \(\forall (i_0,\cdots,i_n) \in \mathbb{z}^{+}\), \(\mathbb{p}(|s_k|=i_k,k=0,\cdots,n)>0 \iff i_0=0,|i_k-i_{k-1}|=1, k=1,\cdots,n\).
  let \(s_n=\sum_{k=1}^n \xi_k\), where \((\xi_n:n \geq)\) are i.i.d. r.v. and \(\mathbb{p}(\xi_1=1)=p,\mathbb{p}(\xi_1=-1)=q\).
  \(a:=\{(i_0,\cdots,i_{n + 1}) \in \mathbb{z}: i_0=0,|i_k-i_{k- 1}|=1,k=1,\cdots,n + 1\}\).
  \(\forall (i_0,\cdots,i_{n + 1}) \in a\), let \(r :=\max \{k:i_k=0\}\).
  then \(i_r=0,\forall k \geq r + 1, i_k \geq 1\).
  \begin{enumerate}
    \item \(\forall (i_0,\cdots,i_{n + 1}) \notin a\), then \(\mathbb{p}(|s_k|=i_k,k=0,\cdots,n) =0\),
      then we have no need to calculate \(\mathbb{p}(|s_{n + 1}|=i_{n + 1} | |s_k|=i_k,k=0,\cdots,n)\).
    \item \(\forall(i_0,\cdots,i_{n + 1}) \in a\),
      \[
        \begin{aligned}
           & \mathbb{p}(|s_{k}|=i_k,s_n=i_n,k=0,\cdots,n| |s_k|=i_k,k=0,\cdots,r)
          \\   & =               \mathbb{p}(|s_{k}|=i_k,s_n=i_n,k=r + 1,\cdots,n| |s_k|=i_k,k=0,\cdots,r-1,s_r=0)
          \\   & =               \mathbb{p}(|s_{k}|=i_k,s_n=i_n,k=r +1,\cdots,n| s_r=0)
          \\   & =               \mathbb{p}(s_{k}=i_k,s_n=i_n,k=r +1,\cdots,n| s_r=0)
          \\&=p^{\frac{n-r + r_n}{2}}q^{\frac{n-r -r_n}{2}}
        \end{aligned}
      \]
      in the same way, we can get \(\mathbb{p}(|s_{k}|=i_k,s_n=-i_n,k=0,\cdots,n| |s_k|=i_k,k=0,\cdots,r)=p^{\frac{n-r - r_n}{2}}q^{\frac{n-r +r_n}{2}}\)
      so
      \[
        \begin{aligned}
           & \mathbb{p}(s_{n}=i_n| |s_k|=i_k,k=0,\cdots,n)
          \\   & =          \frac{\mathbb{p}(s_{n}=i_n,|s_k|=i_k,k=0,\cdots,n)}{\mathbb{p}(|s_k|=i_k,k=0,\cdots,n)}
          \\   & =          \frac{\mathbb{p}(s_{n}=i_n,|s_k|=i_k,k=0,\cdots,n||s_k|=i_k,k=0,\cdots,r)\mathbb{p}(|s_k|=r_k,k=0,\cdots,r)}{\mathbb{p}(|s_k|=i_k,k=0,\cdots,n)}
          \\   & =          \frac{\mathbb{p}(s_{n}=i_n,|s_k|=i_k,k=0,\cdots,n||s_k|=i_k,k=0,\cdots,r)}{\frac{\mathbb{p}(|s_k|=i_k,k=0,\cdots,n)}{\mathbb{p}(|s_k|=r_k,k=0,\cdots,r)}}
          \\   & =          \frac{\mathbb{p}(s_{n}=i_n,|s_k|=i_k,k=0,\cdots,n||s_k|=i_k,k=0,\cdots,r)}{\mathbb{p}(|s_k|=i_k,k=0,\cdots,n||s_k|=r_k,k=0,\cdots,r)}
          \\   & =          \frac{\mathbb{p}(s_{n}=i_n,|s_k|=i_k,k=0,\cdots,n||s_k|=i_k,k=0,\cdots,r)}{\mathbb{p}(|s_k|=i_k,k=r + 1,\cdots,n||s_k|=r_k,k=0,\cdots,r)}
          \\   & =\frac{p^{n - r + \frac{r_n}{2}}q^{n-r - \frac{r_n}{2}}}{p^{n - r + \frac{r_n}{2}}q^{n-r - \frac{r_n}{2}}+p^{n - r - \frac{r_n}{2}}q^{n-r + \frac{r_n}{2}}}
          \\&=p^{i_n}(p^{i_n}+q^{i_n})^{-1}
        \end{aligned}
      \]
      In the same way, we can get \(\mathbb{p}(s_{n}=-i_n| |s_k|=i_k,k=0,\cdots,n)=q^{r_n}(p^{r_n}+q^{r_n})^{-1}\).
      Then
      \[
        \begin{aligned}
           & \mathbb{p}(|s_{n + 1}|=i_{n + 1}| |s_k|=i_k,k=0,\cdots,n)
          \\   & =          \mathbb{p}(|s_{n + 1}|=i_{n + 1}|s_n=i_n ,|s_k|=i_k,k=0,\cdots,n)
          \\   & \times          \mathbb{p}(s_n=i_n ||s_k|=i_k,k=0,\cdots,n)
          \\   & +          \mathbb{p}(|s_{n + 1}|=i_{n + 1}|s_n=-i_n ,|s_k|=i_k,k=0,\cdots,n)
          \\   & \times          \mathbb{p}(s_n=-i_n ||s_k|=i_k,k=0,\cdots,n)
          \\   & =          \mathbb{p}(s_{n + 1}=i_{n + 1}|s_n=i_n)
          \\   & \times          \mathbb{p}(s_n=i_n ||s_k|=i_k,k=0,\cdots,n)
          \\   & +          \mathbb{p}(s_{n + 1}=-i_{n + 1}|s_n=-i_n)
          \\   & \times          \mathbb{p}(s_n=-i_n ||s_k|=i_k,k=0,\cdots,n)
          \\ &=\mathbbm{1}(i_{n + 1}=i_n +1)(p^{i_n + 1} +q^{i_n + 1})(p^{i_n} + q^{i_n})^{-1} + \mathbbm{1}(i_{n + 1}=i_n-1)(p^{r_n}q +pq^{r_n })(p^{r_n} + q^{r_n})^{-1}
        \end{aligned}
      \]
      Thus, \((|S_n|:n \geq 0)\) is markov chain on \(\mathbb{Z}^+\).

  \end{enumerate}

\end{solution}

\end{document}
