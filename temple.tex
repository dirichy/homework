%arara: xelatex
\documentclass{ctexart}
\usepackage{amsmath,amssymb,amsthm,bbm}
\newtheorem{definition}{Definition}
\newtheorem{theorem}{Theorem}
\everymath{\displaystyle}
\newlength\inlineHeight
\newlength\inlineWidth
\long\def\(#1\){%
  \settoheight{\inlineHeight}{$#1$}%
  \settowidth{\inlineWidth}{$#1$}%
  \ifdim \inlineWidth > 0.5\textwidth%
  $$#1$$%
  \else%
  \ifdim \inlineHeight > 1.5em%
  $$#1$$%
  \else%
  $#1$%
  \fi%
  \fi%
}

\begin{document}

\begin{solution}
  Since \(M\) is orientable, we can assume that \(\mathcal{B} \subset \mathcal{A}\) is an oriention of \(M\)\nolinebreak[4],where \(\mathcal{A}\) is all local coordinate of \(M\)\nolinebreak[4]。
  Now consider \(\mathcal{C}:=\{(U;-x^i):(U;x^i) \in \mathcal{B}\}\)\nolinebreak[4]。Easily to check that \(\mathcal{C}\) is an oriention of \(M\)\nolinebreak[4],too.
  And obviously \(\mathcal{B} \cap \mathcal{C} = \varnothing\)\nolinebreak[4],thus \(\mathcal{B} \neq \mathcal{C}\)\nolinebreak[4]。So there is two oriention. Now we need to prove there is no other oriention.

  Assume \(\mathcal{D}\) is an oriention of \(M\)\nolinebreak[4]。We will define a function \(\sgn:M \to \{1,-1\}\) by
  \(\sgn(p)=1 \iff \exists (U_0,x_0^i) \in \mathcal{B},\exists (V_0,y_0^i) \in \mathcal{D},p \in U_0 \cap V_0,J_{x_0,y_0}(p)>0\)\nolinebreak[4]。
  We will prove that \(\sgn(p)=1 \implies \forall (U,x^i) \in \mathcal{B},\forall (V;y^i) \in \mathcal{D},p \in U \cap V \implies J_{x,y}(p)>0\)\nolinebreak[4]。
  It's easy because \(J_{x,y}=J_{x,x_0}J_{x_0,y_0}J_{y_0,y}\)\nolinebreak[4],and \(J_{x,x_0}>0,J_{y,y_0}>0\) by definition of oriention.
  So \(\sgn(p)=-1 \iff \exists (U_0,x_0^i) \in \mathcal{C},\exists (V_0,y_0^i) \in \mathcal{D},p \in U_0 \cap V_0,J_{x_0,y_0}(p)>0 \)
  \(\iff \forall (U,x^i) \in \mathcal{B},\forall (V;y^i) \in \mathcal{D},p \in U \cap V \implies J_{x,y}(p)>0 \)\nolinebreak[4]。
  Noting that if \(\sgn(p)=1\)\nolinebreak[4],we have \(\forall q \in U_0 \cap V_0,\sgn(q)=1\)\nolinebreak[4],and so is \(\sgn(p)=-1\)\nolinebreak[4]。
  So \(\sgn\) is continuous. So \(\sgn\) is constant because \(M\) is connected.
  Easy to check that \(\sgn(p)=1 \iff \mathcal{B}=\mathcal{D}\)\nolinebreak[4],and \(\sgn(p)=-1 \iff \mathcal{C}=\mathcal{D}\)\nolinebreak[4]。
  So there is only two oriention of \(M\)\nolinebreak[4]。
\end{solution}
\end{document}
