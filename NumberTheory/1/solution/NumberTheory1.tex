%!Mode:: "TeX:UTF-8"
%!TEX TS-program = xelatex
\documentclass{ctexart}
\newif\ifpreface
\prefacetrue
\usepackage{fontspec}
\usepackage{bbm}
\usepackage{tikz}
\usepackage{amsmath,amssymb,amsthm,color,mathrsfs}
\usepackage{fixdif}
\usepackage{hyperref}
\usepackage{cleveref}
\usepackage{enumitem}%
\usepackage{expl3}
\usepackage{lipsum}
\usepackage[margin=0pt]{geometry}
\usepackage{listings}
\definecolor{mGreen}{rgb}{0,0.6,0}
\definecolor{mGray}{rgb}{0.5,0.5,0.5}
\definecolor{mPurple}{rgb}{0.58,0,0.82}
\definecolor{backgroundColour}{rgb}{0.95,0.95,0.92}

\lstdefinestyle{CStyle}{
  backgroundcolor=\color{backgroundColour},
  commentstyle=\color{mGreen},
  keywordstyle=\color{magenta},
  numberstyle=\tiny\color{mGray},
  stringstyle=\color{mPurple},
  basicstyle=\footnotesize,
  breakatwhitespace=false,
  breaklines=true,
  captionpos=b,
  keepspaces=true,
  numbers=left,
  numbersep=5pt,
  showspaces=false,
  showstringspaces=false,
  showtabs=false,
  tabsize=2,
  language=C
}
\usetikzlibrary{calc}
\theoremstyle{remark}
\newtheorem{lemma}{Lemma}
\usepackage{fontawesome5}
\usepackage{xcolor}
\newcounter{problem}
\newcommand{\Problem}{\begin{tikzpicture}[baseline]%
    \node at (-0.02em,0.3em) {$\mathbb{P}$};%
    \node[scale=0.7] at (0.2em,-0.0em) {R};%
    \node[scale=0.7] at (0.6em,0.4em) {O};%
    \node[scale=0.8] at (1.05em,0.25em) {B};%
    \node at (1.55em,0.3em) {L};%
    \node[scale=0.7] at (1.75em,0.45em) {E};%
    \node at (2.35em,0.3em) {M};%
  \end{tikzpicture}%
}
\renewcommand{\theproblem}{\Roman{problem}}
\newenvironment{problem}{\refstepcounter{problem}\noindent\color{blue}\Problem\theproblem}{}

\crefname{problem}{\protect\Problem}{Problem}
\newcommand\Solution{\begin{tikzpicture}[baseline]%
    \node at (-0.04em,0.3em) {$\mathbb{S}$};%
    \node[scale=0.7] at (0.35em,0.4em) {O};%
    \node at (0.7em,0.3em) {\textit{L}};%
    \node[scale=0.7] at (0.95em,0.4em) {U};%
    \node[scale=1.1] at (1.19em,0.32em){T};%
    \node[scale=0.85] at (1.4em,0.24em){I};%
    \node at (1.9em,0.32em){$\mathcal{O}$};%
    \node[scale=0.75] at (2.3em,0.21em){\texttt{N}};%
  \end{tikzpicture}}
\newenvironment{solution}{\begin{proof}[\Solution]}{\end{proof}}
\title{\input{../../.subject}\input{../.number}}
\makeatletter
\newcommand\email[1]{\def\@email{#1}\def\@refemail{mailto:#1}}
\newcommand\schoolid[1]{\def\@schoolid{#1}}
\ifpreface
  \def\@maketitle{
  \raggedright
  {\Huge \bfseries \sffamily \@title }\\[1cm]
  {\Huge  \bfseries \sffamily\heiti\@author}\\[1cm]
  {\Huge \@schoolid}\\[1cm]
  {\Huge\href\@refemail\@email}\\[0.5cm]
  \Huge\@date\\[1cm]}
\else
  \def\@maketitle{
    \raggedright
    \begin{center}
      {\Huge \bfseries \sffamily \@title }\\[4ex]
      {\Large  \@author}\\[4ex]
      {\large \@schoolid}\\[4ex]
      {\href\@refemail\@email}\\[4ex]
      \@date\\[8ex]
    \end{center}}
\fi
\makeatother
\ifpreface
  \usepackage[placement=bottom,scale=1,opacity=1]{background}
\fi

\author{白永乐}
\schoolid{202011150087}
\email{202011150087@mail.bnu.edu.cn}

\def\to{\rightarrow}
\newcommand{\xor}{\vee}
\newcommand{\bor}{\bigvee}
\newcommand{\band}{\bigwedge}
\newcommand{\xand}{\wedge}
\newcommand{\minus}{\mathbin{\backslash}}
\newcommand{\mi}[1]{\mathscr{P}(#1)}
\newcommand{\card}{\mathrm{card}}
\newcommand{\oto}{\leftrightarrow}
\newcommand{\hin}{\hat{\in}}
\newcommand{\gl}{\mathrm{GL}}
\newcommand{\im}{\mathrm{Im}}
\newcommand{\re }{\mathrm{Re }}
\newcommand{\rank}{\mathrm{rank}}
\newcommand{\tra}{\mathop{\mathrm{tr}}}
\renewcommand{\char}{\mathop{\mathrm{char}}}
\DeclareMathOperator{\ot}{ordertype}
\DeclareMathOperator{\dom}{dom}
\DeclareMathOperator{\ran}{ran}

\begin{document}
\large
\setlength{\baselineskip}{1.2em}
\date{北京师范大学数学科学学院}
\ifpreface
	\backgroundsetup{contents={%
    \begin{tikzpicture}
      \fill [white] (current page.north west) rectangle ($(current page.north east)!.3!(current page.south east)$) coordinate (a);
      \fill [bgc] (current page.south west) rectangle (a);
\end{tikzpicture}}}
\definecolor{word}{rgb}{1,1,0}
\definecolor{bgc}{rgb}{1,0.95,0}
\setlength{\parindent}{0pt}
\thispagestyle{empty}
\begin{tikzpicture}%
  % \node[xscale=2,yscale=4] at (0cm,0cm) {\sffamily\bfseries \color{word} under};%
  \node[xscale=4.5,yscale=10] at (10cm,1cm) {\sffamily\bfseries \color{word} Graduate Homework};%
  \node[xscale=4.5,yscale=10] at (8cm,-2.5cm) {\sffamily\bfseries \color{word} In Mathematics};%
\end{tikzpicture}
\ \vspace{1cm}\\
\begin{minipage}{0.25\textwidth}
  \textcolor{bgc}{王胤雅是傻逼}
\end{minipage}
\begin{minipage}{0.75\textwidth}
  \maketitle
\end{minipage}
\vspace{4cm}\ \\
\begin{minipage}{0.2\textwidth}
  \
\end{minipage}
\begin{minipage}{0.8\textwidth}
  {\Huge
    \textinconsolatanf{}
  }General fire extinguisher
\end{minipage}
\newpage\backgroundsetup{contents={}}\setlength{\parindent}{2em}

\else
	\maketitle
\fi
\newgeometry{left=2cm,right=2cm,top=2cm,bottom=2cm}
%from_here_to_type
%作业:第一节习题(p3): 2,4; 第二节习题(p6): 3; 第三节习题(p10): 3
\begin{problem}\label{pro:p3.2}
	Prove that \(\forall n \in \mathbb{Z},3 \mid n(n+1)(2n + 1)\).
\end{problem}
\begin{solution}
	If \(n \equiv 0 \mod 3\) then \(3 \mid n\).
	If \(n \equiv 1 \mod 3\) then \(3 \mid 2n+1\).
	So no matter what is \(n \mod 3\), we can obtain \(3 \mid n(n + 1)(2n + 1)\).
\end{solution}

\begin{problem}\label{pro:p3.4}
	Let \(a,b \in \mathbb{Z}\) and \(b \neq 0\). Prove that there exists a pair of \(s,t \in \mathbb{Z}\) such that \(a=sb + t \AND |t|\leq \frac{|b|}{2}\).
	And if \(b\) is odd, then the pair \(s,t\) is unique.
	What if \(b\) is even?
\end{problem}
\begin{solution}
	Let \(A:=\{x \in \mathbb{Z}:\exists y \in \mathbb{Z},a=yb+x\}\).
	Since \(a=0b+a\) we know \(a \in A\), so \(A \neq \emptyset\).
	By the definition of \(m\) we know \(\exists s,t \in \mathbb{Z},|t|=m,a=sb+t\).
	Then \(a=(s-1)b+(t+b),a=(s+1)b+(t-b)\). So by the definition of \(A\) we get \(t \pm b \in A\).
	Thus, by the definition of \(m\) we get \(|t \pm b| \geq |t|\).
	So we get \(||t|-|b|| \geq |t|\). Easily \(|t|-|b| < |t|\) since \(b \neq 0\), so we get \(|b|-|t| \geq |t|\), i.e., \(|t|\leq\frac{|b|}{2}\).

	Now take \(2 \nmid b\), we will prove the uniqueness.
	Assume there are two pairs \(s_1,t_1; s_2,t_2\) satisfy the given condition, then \(a=s_1 b + t_1=s_2 b + t_2\).
	Then we get \(b \mid s_1 b-s_2 b = t_2 - t_1\). Since \((s_1,t_1)\neq(s_2,t_2)\) we easily get \(t_1 \neq t_2\).
	So \(|b| \leq |t_1-t_2|\). Noting \(|t_1-t_2| \leq |t_1|+|t_2| \leq \frac{|b|}{2}+\frac{|b|}{2} = |b|\),
	we obtain \(|t_1|=|t_2|=\frac{|b|}{2}\).
	But \(2 \nmid b\), so \(\frac{|b|}{2}\notin \mathbb{Z}\), contradiction!

	Now we consider \(b \) is even.
	Take \(a \equiv \frac{b}{2}\mod b\), then \(a=kb + \frac{b}{2}\) for some \(k \in \mathbb{Z}\), and \(a=(k+1)b - \frac{b}{2}\).
	So there is exactly two pairs of \((s,t)\) satisfy the condition.
	When \(a \not \equiv \frac{b}{2} \mod b\), obviously \((s,t)\) is unique.
\end{solution}
\begin{problem}\label{pro:p6.3}
	Use Problem \ref{pro:p3.4} to prove the existence of the greatest common factor of any pair \((x,y) \in \mathbb{Z} \AND (x,y)\neq (0,0)\),
	and find an algorithm to get \(\gcd(x,y)\), and find \(\gcd(76501,9719)\) by your algorithm and Euckidean algorithm respectively.
\end{problem}
\begin{solution}
	Without loss of generality assume \(|x| \leq |y|\). If \(x=0\) then easily \(\gcd(x,y)=|y|\).
	Now assume \(|y| \geq |x| >0\). Now we prove \(\gcd(x,y)\) exists by contradiction, assume for some \(x,y \in \mathbb{Z},|x|\leq|y|\) there is \(\gcd(x,y)\) not exists.
	Let \(A:=\{x \in \mathbb{Z}:\exists y \in \mathbb{Z},|y| \geq |x|,\gcd(x,y) \text{not exists}\}\).
	Then \(A \neq \emptyset\). Let \(t = \min\{|x|:x \in A\}\). Then by the definition of \(A\) we know \(\exists s \in \mathbb{Z} \AND |s| \geq |t|\) such that \(\gcd(s,t)\) doesn't exist.
	Since we have proved \(\gcd(0,y)\) exists, we get \(t \neq 0\).
	From Problem \ref{pro:p3.4} we know there exists \(x,y \in \mathbb{Z}\) such that \(s=xt+y,|y| \leq \frac{|t|}{2}\).
	Consider the pair \((t,y)\), we know \(|y|< |t|\), so by the definition of \(t\) we get \(\gcd(t,y)\) exists.
	So \(\gcd(t,y)=\gcd(t,xt+y)=\gcd(s,t)\). Contradict to that \(\gcd(s,t)\) doesn't exist.
	So we get \(\forall(x,y) \in \mathbb{Z}^2 \AND (x,y)\neq (0,0),\gcd(x,y)\) exists.

	From above, we can get following algorithm to get \(\gcd(x,y)\):
	\lstinputlisting[style=Cstyle]{gcd.c}

	Now we use Euckidean algorithm to get \(\gcd(76501,9719)\).
	\[
		\begin{aligned}
			76501 & = 7  & \times & 9719 & + & 8468 \\
			9719  & = 1  & \times & 8468 & + & 1251 \\
			8468  & = 6  & \times & 1251 & + & 962  \\
			1251  & = 1  & \times & 962  & + & 289  \\
			962   & = 3  & \times & 289  & + & 95   \\
			289   & =3   & \times & 95   & + & 4    \\
			95    & = 23 & \times & 4    & + & 3    \\
			4     & = 1  & \times & 3    & + & 1    \\
			3     & = 3  & \times & 1    & + & 0
		\end{aligned}
	\]
	So \(\gcd(76501,9719)=1\).

	Now we use the new algorithm to get \(\gcd(76501,9719)\).
	\[
		\begin{aligned}
			76501 & = & 8   \times   & 9719    & - & 1251 \\
			9719  & = & (-8)  \times & (-1251) & - & 289  \\
			-1251 & = & 4   \times   & (-289)  & - & 95   \\
			-289  & = & 3   \times   & (-95)   & - & 4    \\
			-95   & = & 24  \times   & (-4)    & + & 1    \\
			-4    & = & (-4)  \times & 1       & + & 0
		\end{aligned}
	\]
	So \(\gcd(76501,9719)=1\).
\end{solution}
\begin{problem}\label{pro:p10.3}
	Assume \(f(x)=\sum_{k=0}^{n}a_k x^k \in \mathbb{Z}[x]\) and \(a_0,a_n \neq 0\).
	Prove that if a rational number \(\frac{p}{q},\gcd(p,q)=1\) is root of \(f\), then \(p \mid a_0,q \mid a_n\).
	Thus, \(\sqrt{2} \notin \mathbb{Q}\).
\end{problem}
\begin{solution}
	Since \(\frac{p}{q}\) is a root of \(f\), we get \(f(\frac{p}{q})=0\).
	So \(\sum_{k=0}^{n}a_k (\frac{p}{q})^k=0\).
	Multiple \(q^n\), we get \(\sum_{k=0}^{n}a_k p^k q^{n-k}=0\).
	Mod \(p\), we get \(p \mid 0 = \sum_{k=0}^{n}a_k p^k q^{n-k}\).
	For \(k>0\) we have \(p \mid a_k p^k q^{n-k}\), so \(p \mid \sum_{k=1}^{n}a_k p^k q^{n-k}\).
	So \(p \mid \sum_{k=0}^{n}a_k p^k q^{n-k}-\sum_{k=1}^{n}a_k p^k q^{n-k}=a_0q^n\).
	Since \(\gcd(p,q)=1\), easily \(p \mid a_0 q^n \iff p \mid a_0\).
	So we get \(p \mid a_0\).
	For the same reason easy to get \(q \mid a_n\).

	Consider \(f(x)=x^2-2 \in \mathbb{Z}[x]\). Easily \(f(\sqrt{2})=0\).
	So if \(\sqrt{2}=\frac{p}{q},\gcd(p,q)=1\), then we get \(p \mid 2,q \mid 1\).
	Without loss of generality assume \(q > 0\), then \(q=1\).
	Then \(\sqrt{2}=\pm 1 ,\pm 2\).
	But none of them is root of \(f\), contradiction!
\end{solution}

\end{document}
