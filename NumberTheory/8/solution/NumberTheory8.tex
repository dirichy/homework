%!Mode:: "TeX:UTF-8"
%!TEX encoding = UTF-8 Unicode
%!TEX TS-program = xelatex
\documentclass{ctexart}
\newif\ifpreface
%\prefacetrue
\usepackage{fontspec}
\usepackage{bbm}
\usepackage{tikz}
\usepackage{amsmath,amssymb,amsthm,color,mathrsfs}
\usepackage{fixdif}
\usepackage{hyperref}
\usepackage{cleveref}
\usepackage{enumitem}%
\usepackage{expl3}
\usepackage{lipsum}
\usepackage[margin=0pt]{geometry}
\usepackage{listings}
\definecolor{mGreen}{rgb}{0,0.6,0}
\definecolor{mGray}{rgb}{0.5,0.5,0.5}
\definecolor{mPurple}{rgb}{0.58,0,0.82}
\definecolor{backgroundColour}{rgb}{0.95,0.95,0.92}

\lstdefinestyle{CStyle}{
  backgroundcolor=\color{backgroundColour},
  commentstyle=\color{mGreen},
  keywordstyle=\color{magenta},
  numberstyle=\tiny\color{mGray},
  stringstyle=\color{mPurple},
  basicstyle=\footnotesize,
  breakatwhitespace=false,
  breaklines=true,
  captionpos=b,
  keepspaces=true,
  numbers=left,
  numbersep=5pt,
  showspaces=false,
  showstringspaces=false,
  showtabs=false,
  tabsize=2,
  language=C
}
\usetikzlibrary{calc}
\theoremstyle{remark}
\newtheorem{lemma}{Lemma}
\usepackage{fontawesome5}
\usepackage{xcolor}
\newcounter{problem}
\newcommand{\Problem}{\begin{tikzpicture}[baseline]%
    \node at (-0.02em,0.3em) {$\mathbb{P}$};%
    \node[scale=0.7] at (0.2em,-0.0em) {R};%
    \node[scale=0.7] at (0.6em,0.4em) {O};%
    \node[scale=0.8] at (1.05em,0.25em) {B};%
    \node at (1.55em,0.3em) {L};%
    \node[scale=0.7] at (1.75em,0.45em) {E};%
    \node at (2.35em,0.3em) {M};%
  \end{tikzpicture}%
}
\renewcommand{\theproblem}{\Roman{problem}}
\newenvironment{problem}{\refstepcounter{problem}\noindent\color{blue}\Problem\theproblem}{}

\crefname{problem}{\protect\Problem}{Problem}
\newcommand\Solution{\begin{tikzpicture}[baseline]%
    \node at (-0.04em,0.3em) {$\mathbb{S}$};%
    \node[scale=0.7] at (0.35em,0.4em) {O};%
    \node at (0.7em,0.3em) {\textit{L}};%
    \node[scale=0.7] at (0.95em,0.4em) {U};%
    \node[scale=1.1] at (1.19em,0.32em){T};%
    \node[scale=0.85] at (1.4em,0.24em){I};%
    \node at (1.9em,0.32em){$\mathcal{O}$};%
    \node[scale=0.75] at (2.3em,0.21em){\texttt{N}};%
  \end{tikzpicture}}
\newenvironment{solution}{\begin{proof}[\Solution]}{\end{proof}}
\title{\input{../../.subject}\input{../.number}}
\makeatletter
\newcommand\email[1]{\def\@email{#1}\def\@refemail{mailto:#1}}
\newcommand\schoolid[1]{\def\@schoolid{#1}}
\ifpreface
  \def\@maketitle{
  \raggedright
  {\Huge \bfseries \sffamily \@title }\\[1cm]
  {\Huge  \bfseries \sffamily\heiti\@author}\\[1cm]
  {\Huge \@schoolid}\\[1cm]
  {\Huge\href\@refemail\@email}\\[0.5cm]
  \Huge\@date\\[1cm]}
\else
  \def\@maketitle{
    \raggedright
    \begin{center}
      {\Huge \bfseries \sffamily \@title }\\[4ex]
      {\Large  \@author}\\[4ex]
      {\large \@schoolid}\\[4ex]
      {\href\@refemail\@email}\\[4ex]
      \@date\\[8ex]
    \end{center}}
\fi
\makeatother
\ifpreface
  \usepackage[placement=bottom,scale=1,opacity=1]{background}
\fi

\author{白永乐}
\schoolid{202011150087}
\email{202011150087@mail.bnu.edu.cn}

\def\to{\rightarrow}
\newcommand{\xor}{\vee}
\newcommand{\bor}{\bigvee}
\newcommand{\band}{\bigwedge}
\newcommand{\xand}{\wedge}
\newcommand{\minus}{\mathbin{\backslash}}
\newcommand{\mi}[1]{\mathscr{P}(#1)}
\newcommand{\card}{\mathrm{card}}
\newcommand{\oto}{\leftrightarrow}
\newcommand{\hin}{\hat{\in}}
\newcommand{\gl}{\mathrm{GL}}
\newcommand{\im}{\mathrm{Im}}
\newcommand{\re }{\mathrm{Re }}
\newcommand{\rank}{\mathrm{rank}}
\newcommand{\tra}{\mathop{\mathrm{tr}}}
\renewcommand{\char}{\mathop{\mathrm{char}}}
\DeclareMathOperator{\ot}{ordertype}
\DeclareMathOperator{\dom}{dom}
\DeclareMathOperator{\ran}{ran}

\begin{document}
\large
\setlength{\baselineskip}{1.2em}
\ifpreface
\backgroundsetup{contents={%
    \begin{tikzpicture}
      \fill [white] (current page.north west) rectangle ($(current page.north east)!.3!(current page.south east)$) coordinate (a);
      \fill [bgc] (current page.south west) rectangle (a);
\end{tikzpicture}}}
\definecolor{word}{rgb}{1,1,0}
\definecolor{bgc}{rgb}{1,0.95,0}
\setlength{\parindent}{0pt}
\thispagestyle{empty}
\begin{tikzpicture}%
  % \node[xscale=2,yscale=4] at (0cm,0cm) {\sffamily\bfseries \color{word} under};%
  \node[xscale=4.5,yscale=10] at (10cm,1cm) {\sffamily\bfseries \color{word} Graduate Homework};%
  \node[xscale=4.5,yscale=10] at (8cm,-2.5cm) {\sffamily\bfseries \color{word} In Mathematics};%
\end{tikzpicture}
\ \vspace{1cm}\\
\begin{minipage}{0.25\textwidth}
  \textcolor{bgc}{王胤雅是傻逼}
\end{minipage}
\begin{minipage}{0.75\textwidth}
  \maketitle
\end{minipage}
\vspace{4cm}\ \\
\begin{minipage}{0.2\textwidth}
  \
\end{minipage}
\begin{minipage}{0.8\textwidth}
  {\Huge
    \textinconsolatanf{}
  }General fire extinguisher
\end{minipage}
\newpage\backgroundsetup{contents={}}\setlength{\parindent}{2em}

\newgeometry{left=2cm,right=2cm,top=2cm,bottom=2cm}
\else
\newgeometry{left=2cm,right=2cm,top=2cm,bottom=2cm}
%\maketitle \fi
%from_here_to_type
\begin{problem}\label{pro:1}
  Use the method in the contexts of this section to judge whether these equations below have solutions.
  \begin{enumerate}
    \item \(x^2 \equiv 429 \pmod{563}\)
    \item \(x^2 \equiv 680 \pmod{769}\)
    \item \(x^2 \equiv 503 \pmod{1013}\)
  \end{enumerate}
  where \(503,563,796,1013\) are prime.
\end{problem}
\begin{solution}
  \begin{enumerate}
    \item \(\legendre{429}{563}=\legendre{3}{563}\legendre{11}{563}\legendre{13}{563}
      =(-1)^{\frac{(2+10+12)*562}{4}}\legendre{563}{3}\legendre{563}{11}\legendre{563}{13}
      =\legendre{2}{3}\legendre{2}{11}\legendre{4}{13}
      =(-1)(-1)^{\frac{11^2-1}{8}}(1)=1\).
    \item \(\legendre{680}{769}=\legendre{170}{769}=\legendre{2}{769}\legendre{5}{769}\legendre{17}{769}
      =(-1)^{\frac{769^2-1}{8}+\frac{(4+16)768}{4}}\legendre{769}{5}\legendre{769}{17}
      =\legendre{4}{5}\legendre{4}{17}=1\).
    \item \(\legendre{503}{1013}=(-1)^{\frac{502\times 1012}{4}}\legendre{1013}{503}
      =\legendre{7}{503}
      =(-1)^{\frac{6 \times 502}{4}}\legendre{503}{7}
      =-\legendre{6}{7}
      =-\legendre{-1}{7}
      =-(-1)^3=1\).
  \end{enumerate}

\end{solution}

\begin{problem}\label{pro:2}
  Find out the expression of the prime with the quadratic residue \(-2\);
  Find out the expression of the prime with the non-quadratic residue \(-2\);
\end{problem}
\begin{solution}
  Easy to get that \(\legendre{-2}{p}=\legendre{-1}{p}\legendre{2}{p}=(-1)^{\frac{p-1}{2}}(-1)^{\frac{p^2-1}{8}}=(-1)^{\frac{(p-1)(p+5)}{8}}\).
  So \(\legendre{-2}{p}=1 \iff 16 \mid (p-1)(p+5) \iff 4 \mid \frac{p-1}{2} \frac{p+5}{2}\).
  Since \(\frac{p+5}{2}-\frac{p-1}{2}=3 \equiv 1 \mod 2\), we know they can't be all even.
  So \(4 \mid \frac{p-1}{2} \OR 4 \mid \frac{p+5}{2}\), so \(p \equiv 1,3 \mod 8\).
  So \(\legendre{-2}{p}=1 \iff p \equiv 1,3 \mod 8\), and \(\legendre{-2}{p}=-1 \iff p \equiv 5,7 \mod 8\).
\end{solution}

\begin{problem}\label{pro:3}
  Assume \(n \in \mathbb{N}_+\), \(4n + 3,8n + 7\) are prime, prove:
  \[
    2^{4n + 3}\equiv 1 \pmod{8n + 7}
  \]
  Thus, prove that \(23 \mid(2^{11}-1),47 \mid (2^{23}-1),503 \mid (2^{251}-1)\).
\end{problem}
\begin{solution}
  In fact we don't need \(4n+3\) is prime.
  Since \(2^{\frac{8n+7-1}{2}}\legendre{2}{8n+7}=(-1)^{\frac{(8n+7)^2-1}{8}}=(-1)^{8n^2+14n+6}=1\),
  we easily get that \(2^{4n+3}\equiv 1 \mod 8n+7\).
  We let \(n=2,5,62\), then we get \(23 \mid(2^{11}-1),47 \mid (2^{23}-1),503 \mid (2^{251}-1)\).
\end{solution}

\begin{problem}\label{pro:4}
  Find out the expression of the prime with the quadratic residue \(\pm 3\);
  which prime has the non-quadratic residue \(\pm 3\)?
\end{problem}
\begin{solution}
  Assume \(p>3\).
  Easy to get that \(\legendre{3}{p}=(-1)^{\frac{p-1}{2}}\legendre{p}{3}\).
  And \(\legendre{-3}{p}=\legendre{p}{3}\).
  So \(\legendre{-3}{p}=1 \iff p \equiv 1 \mod 3\), and \( \legendre{-3}{p}=-1 \iff p \equiv 2 \mod 3\).
  And \(\legendre{3}{p}=1 \iff (p \equiv 1 \mod 3 \AND p \equiv 1 \mod 4)\OR(p \equiv 2 \mod 3 \AND p \equiv 3 \mod 4) \iff p \equiv 1,11 \mod 12\).
  So \(\legendre{3}{p}=1 \iff p \equiv 1,11 \mod 12\), and \(\legendre{3}{p}=-1 \iff p \equiv 5,7 \mod 12\).
\end{solution}

\begin{problem}\label{pro:5}
  Find out the expression of the prime with the minimum non-quadratic residue \( 3\).
\end{problem}
\begin{solution}
  Only need to solve \(\legendre{2}{p}=1 \AND \legendre{3}{p}=-1\).
  Easy to know that \(\legendre{2}{p}=1 \iff p \equiv 1,7 \mod 8\).
  And from \ref{pro:4} we know that \(\legendre{3}{p}=1 \iff p \equiv 1,11 \mod 12\).
  So finally we get that \(p \equiv 1,23 \mod 24\).
  So \(p \in \mathbb{P}\) with minimum non-quadratic \(3 \iff p \equiv 1,23 \mod 24\).
\end{solution}

\end{document}
