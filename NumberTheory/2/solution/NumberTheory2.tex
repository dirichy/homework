%!Mode:: "TeX:UTF-8"
%!TEX TS-program = xelatex
\documentclass{ctexart}
\newif\ifpreface
\prefacetrue
\usepackage{fontspec}
\usepackage{bbm}
\usepackage{tikz}
\usepackage{amsmath,amssymb,amsthm,color,mathrsfs}
\usepackage{fixdif}
\usepackage{hyperref}
\usepackage{cleveref}
\usepackage{enumitem}%
\usepackage{expl3}
\usepackage{lipsum}
\usepackage[margin=0pt]{geometry}
\usepackage{listings}
\definecolor{mGreen}{rgb}{0,0.6,0}
\definecolor{mGray}{rgb}{0.5,0.5,0.5}
\definecolor{mPurple}{rgb}{0.58,0,0.82}
\definecolor{backgroundColour}{rgb}{0.95,0.95,0.92}

\lstdefinestyle{CStyle}{
  backgroundcolor=\color{backgroundColour},
  commentstyle=\color{mGreen},
  keywordstyle=\color{magenta},
  numberstyle=\tiny\color{mGray},
  stringstyle=\color{mPurple},
  basicstyle=\footnotesize,
  breakatwhitespace=false,
  breaklines=true,
  captionpos=b,
  keepspaces=true,
  numbers=left,
  numbersep=5pt,
  showspaces=false,
  showstringspaces=false,
  showtabs=false,
  tabsize=2,
  language=C
}
\usetikzlibrary{calc}
\theoremstyle{remark}
\newtheorem{lemma}{Lemma}
\usepackage{fontawesome5}
\usepackage{xcolor}
\newcounter{problem}
\newcommand{\Problem}{\begin{tikzpicture}[baseline]%
    \node at (-0.02em,0.3em) {$\mathbb{P}$};%
    \node[scale=0.7] at (0.2em,-0.0em) {R};%
    \node[scale=0.7] at (0.6em,0.4em) {O};%
    \node[scale=0.8] at (1.05em,0.25em) {B};%
    \node at (1.55em,0.3em) {L};%
    \node[scale=0.7] at (1.75em,0.45em) {E};%
    \node at (2.35em,0.3em) {M};%
  \end{tikzpicture}%
}
\renewcommand{\theproblem}{\Roman{problem}}
\newenvironment{problem}{\refstepcounter{problem}\noindent\color{blue}\Problem\theproblem}{}

\crefname{problem}{\protect\Problem}{Problem}
\newcommand\Solution{\begin{tikzpicture}[baseline]%
    \node at (-0.04em,0.3em) {$\mathbb{S}$};%
    \node[scale=0.7] at (0.35em,0.4em) {O};%
    \node at (0.7em,0.3em) {\textit{L}};%
    \node[scale=0.7] at (0.95em,0.4em) {U};%
    \node[scale=1.1] at (1.19em,0.32em){T};%
    \node[scale=0.85] at (1.4em,0.24em){I};%
    \node at (1.9em,0.32em){$\mathcal{O}$};%
    \node[scale=0.75] at (2.3em,0.21em){\texttt{N}};%
  \end{tikzpicture}}
\newenvironment{solution}{\begin{proof}[\Solution]}{\end{proof}}
\title{\input{../../.subject}\input{../.number}}
\makeatletter
\newcommand\email[1]{\def\@email{#1}\def\@refemail{mailto:#1}}
\newcommand\schoolid[1]{\def\@schoolid{#1}}
\ifpreface
  \def\@maketitle{
  \raggedright
  {\Huge \bfseries \sffamily \@title }\\[1cm]
  {\Huge  \bfseries \sffamily\heiti\@author}\\[1cm]
  {\Huge \@schoolid}\\[1cm]
  {\Huge\href\@refemail\@email}\\[0.5cm]
  \Huge\@date\\[1cm]}
\else
  \def\@maketitle{
    \raggedright
    \begin{center}
      {\Huge \bfseries \sffamily \@title }\\[4ex]
      {\Large  \@author}\\[4ex]
      {\large \@schoolid}\\[4ex]
      {\href\@refemail\@email}\\[4ex]
      \@date\\[8ex]
    \end{center}}
\fi
\makeatother
\ifpreface
  \usepackage[placement=bottom,scale=1,opacity=1]{background}
\fi

\author{白永乐}
\schoolid{202011150087}
\email{202011150087@mail.bnu.edu.cn}

\def\to{\rightarrow}
\newcommand{\xor}{\vee}
\newcommand{\bor}{\bigvee}
\newcommand{\band}{\bigwedge}
\newcommand{\xand}{\wedge}
\newcommand{\minus}{\mathbin{\backslash}}
\newcommand{\mi}[1]{\mathscr{P}(#1)}
\newcommand{\card}{\mathrm{card}}
\newcommand{\oto}{\leftrightarrow}
\newcommand{\hin}{\hat{\in}}
\newcommand{\gl}{\mathrm{GL}}
\newcommand{\im}{\mathrm{Im}}
\newcommand{\re }{\mathrm{Re }}
\newcommand{\rank}{\mathrm{rank}}
\newcommand{\tra}{\mathop{\mathrm{tr}}}
\renewcommand{\char}{\mathop{\mathrm{char}}}
\DeclareMathOperator{\ot}{ordertype}
\DeclareMathOperator{\dom}{dom}
\DeclareMathOperator{\ran}{ran}

\begin{document}
\large
\iffalse
  \setlength{\baselineskip}{1.2em}
  \ifpreface
    \backgroundsetup{contents={%
    \begin{tikzpicture}
      \fill [white] (current page.north west) rectangle ($(current page.north east)!.3!(current page.south east)$) coordinate (a);
      \fill [bgc] (current page.south west) rectangle (a);
\end{tikzpicture}}}
\definecolor{word}{rgb}{1,1,0}
\definecolor{bgc}{rgb}{1,0.95,0}
\setlength{\parindent}{0pt}
\thispagestyle{empty}
\begin{tikzpicture}%
  % \node[xscale=2,yscale=4] at (0cm,0cm) {\sffamily\bfseries \color{word} under};%
  \node[xscale=4.5,yscale=10] at (10cm,1cm) {\sffamily\bfseries \color{word} Graduate Homework};%
  \node[xscale=4.5,yscale=10] at (8cm,-2.5cm) {\sffamily\bfseries \color{word} In Mathematics};%
\end{tikzpicture}
\ \vspace{1cm}\\
\begin{minipage}{0.25\textwidth}
  \textcolor{bgc}{王胤雅是傻逼}
\end{minipage}
\begin{minipage}{0.75\textwidth}
  \maketitle
\end{minipage}
\vspace{4cm}\ \\
\begin{minipage}{0.2\textwidth}
  \
\end{minipage}
\begin{minipage}{0.8\textwidth}
  {\Huge
    \textinconsolatanf{}
  }General fire extinguisher
\end{minipage}
\newpage\backgroundsetup{contents={}}\setlength{\parindent}{2em}

  \else
    \maketitle
  \fi
\fi
\newgeometry{left=2cm,right=2cm,top=2cm,bottom=2cm}
%from_here_to_type
%作业:第一章第4节习题5; 第5节习题1,2,4(iii); 第二章第1节习题1(b), 3; 第2节习题2

\begin{problem}\label{pro:p14.5}
  Assume \(n \in \mathbb{N}^+\) and \(2^n + 1\) is prime. Prive that \(\exists k \in \mathbb{N},n=2^k\).
\end{problem}
\begin{lemma}\label{lem:1}
  Assume \(b=ka\) and \(k\) is odd, then for \(x,y \in \mathbb{N}\), we have \(x^a + y^a \mid x^b + y^b\).
\end{lemma}
\begin{proof}
  Easily \(x^b + y^b \equiv (x^a)^k + y^b \equiv (x^a + y^a - y^a)^k + y^b \equiv (-y^a)^k + y^b \equiv 0 \mod x^a + y^a\).
  So \(x^a + y^a \mid x^b + y^b\).
\end{proof}

\begin{solution}
  Assume \(n\) is not power of \(2\), then \(\exists p>2\) is prime such that \(p \mid n\).
  Let \(a=\frac{n}{p}\), then from Lemma \ref{lem:1} we have \(2^a+1^a \mid 2^n+1^n\).
  Easily \(a=\frac{n}{p}<n\), so \(2^a+1<2^n+1\). And easily \(1<2^a+1\).
  So \(2^n+1\) is not prime, contradiction!
  So \(\exists k \in \mathbb{N},n=2^k\).
\end{solution}
\begin{problem}\label{pro:p16.1}
  Find the standrad decomposition of \(30!\).
\end{problem}
\begin{solution}
  There are \(2,3,5,7,11,13,17,19,23,29\), \(10\) primes, below \(30\).
  So we know \(30!\) can be break down into power and product of them.
  By calculation we can get that:
  \begin{equation}\label{equ:1}
    \begin{aligned}
      v_2(30!)    & =\sum_{k=1}^{\infty}\left[\frac{30}{2^k}\right]    & =15+7+3+1=26 \\
      v_3(30!)    & =\sum_{k=1}^{\infty}\left[\frac{30}{3^k}\right]    & =10+3+1=14   \\
      v_5(30!)    & =\sum_{k=1}^{\infty}\left[\frac{30}{5^k}\right]    & =6+1=7       \\
      v_7(30!)    & =\sum_{k=1}^{\infty}\left[\frac{30}{7^k}\right]    & =4           \\
      v_{11}(30!) & =\sum_{k=1}^{\infty}\left[\frac{30}{{11}^k}\right] & =2           \\
      v_{13}(30!) & =\sum_{k=1}^{\infty}\left[\frac{30}{{13}^k}\right] & =2           \\
      v_{17}(30!) & =\sum_{k=1}^{\infty}\left[\frac{30}{{17}^k}\right] & =1           \\
      v_{19}(30!) & =\sum_{k=1}^{\infty}\left[\frac{30}{{19}^k}\right] & =1           \\
      v_{23}(30!) & =\sum_{k=1}^{\infty}\left[\frac{30}{{23}^k}\right] & =1           \\
      v_{29}(30!) & =\sum_{k=1}^{\infty}\left[\frac{30}{{29}^k}\right] & =1
    \end{aligned}
  \end{equation}
  So finally we get \(30! = 2^{26} 3^{14} 5^7 7^4 11^2 13^2 17^1 19^1 23^1 29^1\).
\end{solution}
\begin{problem}\label{pro:p16.2}
  Assume \(n \in \mathbb{N}^+\) and \(\alpha \in \mathbb{R}\), prove that:
  \begin{enumerate}
    \item \(\left[\frac{[n \alpha]}{n}\right]=[\alpha]\).
    \item \(\sum_{k=0}^{n-1}[\alpha+\frac{k}{n}]=[n \alpha]\).
  \end{enumerate}
\end{problem}

\begin{solution}
  \begin{enumerate}
    \item \label{it:2.1} Easily \(\left[\frac{[n \alpha]}{n}\right]\leq\left[\frac{n \alpha}{n}\right] \leq[\alpha]\).
      Now we will prove \(\left[\frac{[n \alpha]}{n}\right]\geq [\alpha]\).
      By the definition of \([\cdot]\) we only need to prove \(\frac{[n \alpha]}{n}\geq [\alpha]\).
      So we only need \([n \alpha]\geq n [\alpha]\).
      By the definition of \([\cdot]\) we only need \(n \alpha \geq n[\alpha]\), which is obvious.
    \item By \ref{it:2.1} easily to know \([\alpha+\frac{k}{n}]=\left[\frac{[n(\alpha+\frac{k}{n})]}{n}\right]=
      \left[\frac{[n \alpha]+k}{n}\right]\).
      Let \(f:\mathbb{Z} \to \{0,\cdots,n-1\}\) and \(f(x)\equiv x \mod n\).
      Then easily \([\frac{x}{n}]=\frac{x}{n}-\frac{f(x)}{n}\).
      So we know \(\sum_{k=0}^{n-1}[\alpha+\frac{k}{n}]=\sum_{k=0}^{n-1}[\frac{[n \alpha]+k}{n}]=
      \sum_{k=0}^{n-1}\frac{[n \alpha]+k}{n}-\sum_{k=0}^{n-1}\frac{f([n \alpha]+k)}{n}\).
      Easily to know \((f([n \alpha]+k):k=1,\cdots,n-1)\) is a replacement of \((k:k=0,\cdots,n-1)\).
      So finally we get \(\sum_{k=0}^{n-1}\frac{[n \alpha]+k}{n}-\sum_{k=0}^{n-1}\frac{f([n \alpha]+k)}{n}=\sum_{k=0}^{n-1}\frac{[n \alpha]+k}{n}-\sum_{k=0}^{n-1}\frac{k}{n}=\sum_{k=0}^{n-1}\frac{[n \alpha]}{n}=[n \alpha]\).
  \end{enumerate}
\end{solution}
\begin{problem}\label{pro:p16.4.3}
  Assume \(r>0,r \in \mathbb{R}\). Let \(T\) be the number of integer point in  \(x^2 + y^2 \leq r^2\).
  Prove that \(T = 1 + 4[r] + 8 \sum_{0<x \leq \frac{r}{\sqrt{2}}}[\sqrt{r^2-x^2}] -4\left[\frac{r}{\sqrt{2}}\right]^2\).
\end{problem}
\begin{solution}
  \(T=\sum_{x,y \in \mathbb{Z},x^2 + y^2 \leq r^2}1=\sum_{x^2 + y^2 \leq r^2,xy=0} 1 +\sum_{x^2+y^2 \leq r^2,xy \neq 0} 1=
  1+\sum_{0<x^2 + y^2 \leq r^2,xy=0} 1 +4 \sum_{x^2 + y^2 \leq r^2,x>0,y>0} 1\).
  By symmetry, we know \(\sum_{0<x^2 + y^2 \leq r^2,xy=0} 1 =4\sum_{x^2 + y^2 \leq r^2,x >0,y=0} 1=4[r]\).
  And \(\sum_{x^2 + y^2 \leq r^2,x,,y>0} 1 =\sum_{x^2+y^2 \leq r^2,0<x \leq \frac{r}{\sqrt{2}},0<y} 1 + \sum_{x^2+y^2 \leq r^2,0<y \leq \frac{r}{\sqrt{2}},0<x} 1 -\sum_{x^2+y^2 \leq r^2,0<x \leq \frac{r}{\sqrt{2}},0<y \leq \frac{r}{\sqrt{2}}} 1\).
  Easily to know \(\sum_{x^2+y^2 \leq r^2,0<y \leq \frac{r}{\sqrt{2}},0<x} 1 =\sum_{x^2+y^2 \leq r^2,0<x \leq \frac{r}{\sqrt{2}},0<y} 1 =\sum_{0<x \leq \frac{r}{\sqrt{2}}}[\sqrt{r^2-x^2}]\),
  and \(\sum_{x^2+y^2 \leq r^2,0<x \leq \frac{r}{\sqrt{2}},0<y \leq \frac{r}{\sqrt{2}}} 1=\left[\frac{r}{\sqrt{2}}\right]^2\).
  So finally we get \(T = 1 + 4[r] + 8 \sum_{0<x \leq \frac{r}{\sqrt{2}}}[\sqrt{r^2-x^2}] -4\left[\frac{r}{\sqrt{2}}\right]^2\).
\end{solution}

\begin{problem}\label{pro:p23.1.b}
  Find all integer solution of \(306x-360y=630\).
\end{problem}
\begin{solution}
  The origin equation is equavilent to \(17x-20y=35\).
  Consider \(\mod 5\), we get \(5 \mid 17 x\). So \(5 \mid x\).
  Assume \(x=5k\), then \(17k-4y=7\).
  Then \(17(k+1)-4y=24\), consider \(\mod 4\), we get \(4 \mid k+1\), so \(k+1=4s\) and \(17s-y=6\).
  So \(y=17s-6\) and easily \(x=5s=5(4s-1)=20s-5\).
  So \(\begin{cases}
    x=20s-5 \\
    y=172-6
  \end{cases}\) is all of solutions of the equation.
\end{solution}
\begin{problem}\label{pro:p23.3}
  Assume \(N,a,b \in \mathbb{N},a,b>0,\gcd(a,b)=1\).
  Prove that the number of positive integer solutions of the equation \(ax+by=N\)
  is \(\left[\frac{N}{ab}\right]\) or \(\left[\frac{N}{ab}\right]+1\).
\end{problem}

\begin{solution}
  Since \(\gcd(a,b)=1\), we know \(\exists s,t \in \mathbb{Z},as+bt=N\).
  So we know \(x=s+kb,y=t-ka\). Let \(x,y>0\), we get \(k> -\frac{s}{b},k<\frac{t}{a}\).
  So we know the number of solution is \(\left[\frac{t}{a}\right]+\left[\frac{s}{b}\right]+1\).
  Now we only need \(\left[\frac{N}{ab}\right]\leq\left[\frac{t}{a}\right]+\left[\frac{s}{b}\right]+1 \leq \left[\frac{N}{ab}\right]+1\).

  To prove \(\left[\frac{N}{ab}\right]\leq\left[\frac{t}{a}\right]+\left[\frac{s}{b}\right]+1\) we only need
  \(\left[\frac{N}{ab}\right]\leq\left[\frac{t}{a}\right]+\frac{s}{b}+1\).
  Only need \(\left[\frac{N}{ab}\right]\leq\frac{t}{a}+\frac{s}{b}\).
  Noting \(ab \left[\frac{N}{ab}\right] \leq ab \frac{N}{ab}=N=as+bt=ab(\frac{t}{a}+\frac{s}{b})\) it's obvious.

  To prove \(\left[\frac{t}{a}\right]+\left[\frac{s}{b}\right]+1 \leq \left[\frac{N}{ab}\right]+1\),
  we only need \(\left[\frac{t}{a}\right]+\left[\frac{s}{b}\right] \leq \frac{N}{ab}\).
  Noting \(ab(\left[\frac{t}{a}\right]+\left[\frac{s}{b}\right]) \leq ab(\frac{t}{a}+\frac{s}{b})=as+bt=N=ab(\frac{N}{ab})\) it's obvious.
\end{solution}
\begin{problem}\label{pro:p24.2}
  Write \(\frac{17}{60}\) as sum of three reduced fraction whose denominators are coprime to each other.
\end{problem}
\begin{solution}
  Consider \(\frac{17}{60}=\frac{x}{4}+\frac{y}{3}+\frac{z}{5}\), i.e., \(17=15x+20y+12z\).
  Since \(\gcd(15,20,12)=1\), we know this equation has some solution.
  Easy to know \(x=-1,y=1,z=1\) is a solution.
  So \(\frac{17}{60}=-\frac{1}{4}+\frac{1}{3}+\frac{1}{5}\) satisfy the condition.
\end{solution}

\end{document}
