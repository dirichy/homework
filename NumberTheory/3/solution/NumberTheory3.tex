%!Mode:: "TeX:UTF-8"
%!TEX TS-program = xelatex
\documentclass{ctexart}
\newif\ifpreface
%\prefacetrue
\usepackage{fontspec}
\usepackage{bbm}
\usepackage{tikz}
\usepackage{amsmath,amssymb,amsthm,color,mathrsfs}
\usepackage{fixdif}
\usepackage{hyperref}
\usepackage{cleveref}
\usepackage{enumitem}%
\usepackage{expl3}
\usepackage{lipsum}
\usepackage[margin=0pt]{geometry}
\usepackage{listings}
\definecolor{mGreen}{rgb}{0,0.6,0}
\definecolor{mGray}{rgb}{0.5,0.5,0.5}
\definecolor{mPurple}{rgb}{0.58,0,0.82}
\definecolor{backgroundColour}{rgb}{0.95,0.95,0.92}

\lstdefinestyle{CStyle}{
  backgroundcolor=\color{backgroundColour},
  commentstyle=\color{mGreen},
  keywordstyle=\color{magenta},
  numberstyle=\tiny\color{mGray},
  stringstyle=\color{mPurple},
  basicstyle=\footnotesize,
  breakatwhitespace=false,
  breaklines=true,
  captionpos=b,
  keepspaces=true,
  numbers=left,
  numbersep=5pt,
  showspaces=false,
  showstringspaces=false,
  showtabs=false,
  tabsize=2,
  language=C
}
\usetikzlibrary{calc}
\theoremstyle{remark}
\newtheorem{lemma}{Lemma}
\usepackage{fontawesome5}
\usepackage{xcolor}
\newcounter{problem}
\newcommand{\Problem}{\begin{tikzpicture}[baseline]%
    \node at (-0.02em,0.3em) {$\mathbb{P}$};%
    \node[scale=0.7] at (0.2em,-0.0em) {R};%
    \node[scale=0.7] at (0.6em,0.4em) {O};%
    \node[scale=0.8] at (1.05em,0.25em) {B};%
    \node at (1.55em,0.3em) {L};%
    \node[scale=0.7] at (1.75em,0.45em) {E};%
    \node at (2.35em,0.3em) {M};%
  \end{tikzpicture}%
}
\renewcommand{\theproblem}{\Roman{problem}}
\newenvironment{problem}{\refstepcounter{problem}\noindent\color{blue}\Problem\theproblem}{}

\crefname{problem}{\protect\Problem}{Problem}
\newcommand\Solution{\begin{tikzpicture}[baseline]%
    \node at (-0.04em,0.3em) {$\mathbb{S}$};%
    \node[scale=0.7] at (0.35em,0.4em) {O};%
    \node at (0.7em,0.3em) {\textit{L}};%
    \node[scale=0.7] at (0.95em,0.4em) {U};%
    \node[scale=1.1] at (1.19em,0.32em){T};%
    \node[scale=0.85] at (1.4em,0.24em){I};%
    \node at (1.9em,0.32em){$\mathcal{O}$};%
    \node[scale=0.75] at (2.3em,0.21em){\texttt{N}};%
  \end{tikzpicture}}
\newenvironment{solution}{\begin{proof}[\Solution]}{\end{proof}}
\title{\input{../../.subject}\input{../.number}}
\makeatletter
\newcommand\email[1]{\def\@email{#1}\def\@refemail{mailto:#1}}
\newcommand\schoolid[1]{\def\@schoolid{#1}}
\ifpreface
  \def\@maketitle{
  \raggedright
  {\Huge \bfseries \sffamily \@title }\\[1cm]
  {\Huge  \bfseries \sffamily\heiti\@author}\\[1cm]
  {\Huge \@schoolid}\\[1cm]
  {\Huge\href\@refemail\@email}\\[0.5cm]
  \Huge\@date\\[1cm]}
\else
  \def\@maketitle{
    \raggedright
    \begin{center}
      {\Huge \bfseries \sffamily \@title }\\[4ex]
      {\Large  \@author}\\[4ex]
      {\large \@schoolid}\\[4ex]
      {\href\@refemail\@email}\\[4ex]
      \@date\\[8ex]
    \end{center}}
\fi
\makeatother
\ifpreface
  \usepackage[placement=bottom,scale=1,opacity=1]{background}
\fi

\author{白永乐}
\schoolid{25110180002}
\email{ylbai25@m.fudan.edu.cn}

\def\to{\rightarrow}
\newcommand{\xor}{\vee}
\newcommand{\AND}{\wedge}
\newcommand{\OR}{\vee}
\newcommand{\bor}{\bigvee}
\newcommand{\band}{\bigwedge}
\newcommand{\xand}{\wedge}
\newcommand{\minus}{\mathbin{\backslash}}
\newcommand{\mi}[1]{\mathscr{P}(#1)}
\newcommand{\card}{\mathrm{card}}
\newcommand{\oto}{\leftrightarrow}
\newcommand{\hin}{\hat{\in}}
\newcommand{\gl}{\mathrm{GL}}
\newcommand{\im}{\mathrm{Im}}
\newcommand{\re }{\mathrm{Re }}
\newcommand{\rank}{\mathrm{rank}}
\newcommand{\tra}{\mathop{\mathrm{tr}}}
\renewcommand{\char}{\mathop{\mathrm{char}}}
\DeclareMathOperator{\ot}{ordertype}
\DeclareMathOperator{\dom}{dom}
\DeclareMathOperator{\ran}{ran}

\begin{document}
\large
\iffalse
  \setlength{\baselineskip}{1.2em}
  \ifpreface
    \input{../../../global/preface}
  \else
    \maketitle
  \fi
\fi
\newgeometry{left=2cm,right=2cm,top=2cm,bottom=2cm}
%from_here_to_type
% p26.习题3;p38.习题3,5;p40.1,2,3,4
\begin{problem}\label{pro:p26.3}
  Prove that solution of equation
  \[
    x^2 + y^2 = z^4,\gcd(x,y)=1,x>0,y>0,z>0,2 \mid x
  \]
  is
  \[
    \begin{cases}
      x=4ab(a^2-b^2)      \\
      y=|a^4+b^4-6a^2b^2| \\
      z=a^2 + b^2
    \end{cases}
  \]
  where \(a>0,b>0,\gcd(a,b)=1,a \not \equiv b \mod 2\).
\end{problem}
\begin{solution}
  On one hand,
  we know \(x,y,z^2\) is solution of Pythagorean equation, so there exists \(s,t \in \mathbb{N}^+,s>t,\gcd(s,t)=1,s \not \equiv t \mod 2\), such that
  \[
    \begin{cases}
      x=2st     \\
      y=s^2-t^2 \\
      z^2=s^2+t^2
    \end{cases}
  \]
  For convinence we dismiss the condition \(s>t\) and let \(y=|s^2-t^2|\) instead.
  Now \(s,t\) are symmetry, so without loss of generality assume \(2 \mid s\), then \((s,t,z)\) is solution of Pythagorean equation, so \(\exists a,b \in \mathbb{N}^+,\gcd(a,b)=1,a \not\equiv b \mod 2,a>b\), such that
  \[
    \begin{cases}
      s=2ab     \\
      t=a^2-b^2 \\
      z=a^2 + b^2
    \end{cases}
  \]
  So \(x=2st=2 \times 2ab \times (a^2-b^2)=4ab(a^2-b^2)\), and \(y=|s^2-t^2|=|4a^2b^2-a^4-b^4+2a^2b^2|=|a^4+b^4-6a^2b^2|\), and \(z=a^2+b^2\).

  On the other hand, it is easy to check that for \(x=4ab(a^2-b^2),    y=|a^4+b^4-6a^2b^2|,    z=a^2 + b^2\) we have \(x^2 + y^2 = z^4\).
  And since \(\gcd(a,b)=1\) we get \(\gcd(x,y)\mid \gcd(2,y)^2 \gcd(a,y) \gcd(b,y)\gcd(a+b,y) \gcd(a-b,y)=\gcd(2,a^4+b^4)^2 \gcd(a,b^4)\gcd(b,a^4) \gcd(a+b,4a^4)\gcd(a-b,4a^4)\).
  Since \(a \not \equiv b \mod 2\) we get \(\gcd(2,a^4+b^4)=1\).
  Easily to know \(\gcd(a,b^4)=\gcd(b,a^4)=1\).
  And \(\gcd(a \pm b,4a^4) \mid \gcd(a \pm b,2)^2 \gcd(a \pm b,a)^4=1\).
  Finally we get \(\gcd(x,y)=1\).
  Easy to check \(x,y,z>0\) and \(2 \mid x\).

  All in all, they are all solution of the given equation.
\end{solution}
\begin{problem}\label{pro:p38.3}
  Find a method to judge whether a number can be divided by \(37,101\).
\end{problem}
\begin{solution}
  Noting \(1000 \equiv 1 \mod 37\), so we can use \(1000\)-binary.
  Assume \(n=\sum_{k=0}^{m}a_k 1000^k\) and \(0 \leq a_k<1000\).
  Then \(n \equiv \sum_{k=0}^{m}a_k \mod 37\). So \(37 \mid n \iff 37 \mid \sum_{k=0}^{m}a_k\).

  For \(10\)-binary, we can combine them as group of three.
  Assume \(n=\sum_{k=0}^{t}b_k 10^k,0 \leq b_k <10\). Let \(s=\ceil{\frac{t}{3}}\) and let \(b_k=0\) for \(k>t\).
  Then \(n=\sum_{k=0}^{s}(b_{3k}+10b_{3k+1}+100b_{3k+2}) 1000^k\).
  Then \(37 \mid n \iff 37 \mid \sum_{k=0}^{s}b_{3k}+10b_{3k+1}+100b_{3k+2}\).

  Noting \(100 \equiv -1 \mod 101\), so we can consider \(100\)-binary for \(101\).
  Assume \(n=\sum_{k=0}^{m}a_k 100^k\) and \(0 \leq a_k<100\).
  Then \(n \equiv \sum_{k=0}^{m}(-1)^ka_k \mod 101\). So \(101 \mid n \iff 101 \mid \sum_{k=0}^{m}(-1)^ka_k\).

  For \(10\)-binary, we can combine them as group of two.
  Assume \(n=\sum_{k=0}^{t}b_k 10^k,0 \leq b_k <10\). Let \(s=\ceil{\frac{t}{2}}\) and let \(b_k=0\) for \(k>t\).
  Then \(n=\sum_{k=0}^{s}(b_{2k}+10b_{2k+1}) 100^k\).
  Then \(101 \mid n \iff 101 \mid \sum_{k=0}^{s}(-1)^k(b_{2k}+10b_{2k+1})\).
\end{solution}
\begin{problem}\label{pro:p38.5}
  Assume \(2 \nmid a\), then \(a^{2^n} \equiv 1 \mod 2^{n + 2}\).
\end{problem}
\begin{solution}
  We will prove \(a^{2^n}-1=(a^2-1)\prod_{k=1}^{n-1}(a^{2^k}+1)\) first.
  Prove it by MI to \(n\). When \(n=1\) there is nothing to do.
  Assume it holds for certain \(n\), consider \(n+1\), we get
  \(a^{2^{n + 1}}-1=(a^{2^n}+1)(a^{2^n}-1)=(a^{2^n}+1)(a^2-1)\prod_{k=1}^{n-1}(a^{2^k}+1)=(a^2-1)\prod_{k=1}^{n}(a^{2^k}+1)\).
  So we get it holds for every \(n \in \mathbb{N}^+\).

  Since \(2 \nmid a\), assume \(a=2b+1\), then \(a^2-1=4b^2+4b=4b(b+1)\). Noting \(2 \mid b(b+1)\), we get \(8 \mid a^2-1\).
  And easily \(2 \mid a^{2^k}+1,\forall k \in \mathbb{N}\).
  So we get \(8 \times \prod_{k=1}^{n-1}2 \mid a^{2^n}-1\), i.e., \(2^{n+2}\mid a^{2^n}-1\).
  Finally we get \(a^{2^n}\equiv 1 \mod 2^{n+2}\).
\end{solution}
\begin{problem}\label{pro:p40.1}
  Let \(p\) be a prime and \(s,t\) be integers and \(t \leq s\).
  Prove that \((u+p^{s-t}v:0 \leq u \leq p^{s-t}-1,0 \leq v \leq p^t-1)\) is a Complete residue system of \(p^s\).
\end{problem}
\begin{solution}
  Since there are \(p^{s-t}\) different \(u\) and \(p^t-1\) different \(v\), there are \(p^s\) different elements in total.
  So we only need to prove any two of them are not equal \(\mod p^s\).

  Assume \(u_1+p^{s-t}v_1 \equiv u_2 + p^{s-t}v_2 \mod p^s\), we need to prove \((u_1,v_1)=(u_2,v_2)\).
  Consider \(\mod p^{s-t}\), we get \(u_1 \equiv u_2 \mod p^{s-t}\).
  Since \(|u_1-u_2|\leq p^{s-t}-1\), we easily get \(u_1=u_2\).
  Then \(p^{s-t}v_1 \equiv p^{s-t}v_2 \mod p^s\).
  So \(v_1 \equiv v_2 \mod p^t\).
  For the same reason since \(|v_1-v_2|\leq p^t-1\) we get \(v_1=v_2\).
  So we proved any two of them are not equal \(\mod p^s\).

  So \((u+p^{s-t}v:0 \leq u \leq p^{s-t}-1,0 \leq v \leq p^t-1)\) is a Complete residue system of \(p^s\).
\end{solution}
\begin{problem}\label{pro:p40.2}
  Assume \(m_1,\cdots,m_k\) is \(k\) integers coprime to each other.
  Assume \(A_1,A_2,\cdots,A_k\) is Complete residue of \(m_1,\cdots,m_k\) respectively.
  Let \(m=\prod_{t=1}^{k}m_t\) and \(M_t:=\frac{m}{m_t},t=1,\cdots,k\).
  Prove that \(A:=\{\sum_{t=1}^{k}M_t x_t:x_t \in A_t,t=1,\cdots,k\}\) is a Complete residue of \(m\).
\end{problem}
\begin{solution}
  Easily \(|A_t|=m_t\), so there is \(m\) different \((x_1,\cdots,x_k)\).
  So we only need to prove for different \((x_1,\cdots,x_k)\) the value of \(\sum_{t=1}^{k}M_t x_t \mod m\) is different.

  Assume \(\sum_{t=1}^{k}M_t x_t \equiv \sum_{t=1}^{k}M_t y_t\) and \(x_t,y_t \in A_t,t=1,\cdots,k\).
  Now we need to prove \((x_1,\cdots,x_k)=(y_1,\cdots,y_k)\).
  Noting for \(i \neq j\) we have \(m_i \mid M_j\).
  So we consider \(\mod m_i\), we get \(M_ix_i \equiv M_iy_i \mod m_i\).
  Then \(m_i \mid M_i(x_i-y_i)\).
  Since \(\gcd(m_i,m_j)=1\) for \(i \neq j\), we get \(\gcd(m_i,M_i)=1\).
  So \(m_i \mid x_i-y_i\).
  Since \(x_i,y_i \in A_i\) and \(A_i\) is Complete residue of \(m_i\), we get \(x_i=y_i\).
  So \((x_1,\cdots,x_k)=(y_1,\cdots,y_k)\).

  So finally we get \(A\) is Complete residue of \(m\).
\end{solution}
\begin{problem}\label{pro:p40.3}
  Let \(H=\frac{3^{n+1}-1}{3-1}\). Let \(I=\{(x_0,\cdots,x_n):x_k \in \{-1,0,1\},k=0,\cdots,n\}\).
  Let \(f:I \to N:=[-H,H]\cap \mathbb{Z}\), and \(f(x_0,\cdots,x_n)=\sum_{k=0}^{n}x_k 3^k\).
  Prove that \(f\) is bijection.
  Thus, we can use \(n+1\) weights and a balance to weigh all integer weights between \(1\) and \(H\).
\end{problem}
\begin{solution}
  First we prove \(f\) is well-defined. i.e., \(\forall (x_0,\cdots,x_n) \in I,-H \leq \sum_{k=0}^{n}x_k 3^k \leq H\).
  Since \(x_k=-1,0,1\), we get \(\sum_{k=0}^{n}x_k 3^k \leq \sum_{k=0}^{n}1 \times 3^k = \frac{3^{n + 1}-1}{3-1}=H\).
  For the same reason, we get \(\sum_{k=0}^{n}x_k 3^k \geq -H\).

  Second we will prove \(f\) is injection.
  Assume \(x,y \in I\) and \(f(x)=f(y)\), i.e., \(\sum_{k=0}^{n}x_k 3^k=\sum_{k=0}^{n}y_k 3^k\), we need to prove \(x=y\).
  If \(x \neq y\), then assume \(m=\min\{t:x_t \neq y_t\}\).
  Then \(x_t=y_t,\forall t<m\). So \(\sum_{k=m}^{n}x_k 3^k=\sum_{k=m}^{n}y_k3^k\).
  Consider \(\mod 3^{m+1}\), we get \(x_m 3^m \equiv y_m 3^m \mod 3^{m+1}\), i.e., \(x_m \equiv y_m \mod 3\).
  But \(x_m,y_m \in \{-1,0,1\}\) and \(x_m \neq y_m\), contradiction!
  So \(x=y\) and thus \(f\) is injection.

  Finally we prove \(f\) is surjection.
  Since \(|I|=3^{n+1}\), and \(|N|=2H+1=3^{n+1}\), we get \(|I|=|N|<\infty\).
  Noting we have proved \(f\) is injection, so \(f\) is surjection.

  All in all, \(f\) is bijection.

  Now we use \(n+1\) weights, \(3^0,3^1,\cdots,3^n\). For every integer \(n:1 \leq n \leq H\),
  we know there is a \(x \in I\) such that \(f(x)=n\).
  Let \(L:=\{t:x_t=1\}\) and \(R=\{t:x_t=-1\}\), put the thing to weigh on right,
  and put weights in \(R\) on right, then put weights in \(L\) on left, we can weigh this thing out if it's weight is \(n\).
\end{solution}
\begin{problem}\label{pro:p40.4}
  Assume \(m_1,\cdots,m_k\) is \(k\) integers coprime to each other.
  Assume \(A_1,A_2,\cdots,A_k\) is Complete residue of \(m_1,\cdots,m_k\) respectively.
  Let \(m=\prod_{t=1}^{k}m_t\) and \(M_t:=\prod_{i=1}^{t-1}m_i,t=1,\cdots,k\).
  Prove that \(A:=\{\sum_{t=1}^{k}M_t x_t:x_t \in A_t,t=1,\cdots,k\}\) is a Complete residue of \(m\).
\end{problem}
\begin{solution}
  Easily \(|A_t|=m_t\), so there is \(m\) different \((x_1,\cdots,x_k)\).
  So we only need to prove for different \((x_1,\cdots,x_k)\) the value of \(\sum_{t=1}^{k}M_t x_t \mod m\) is different.

  Assume \(\sum_{t=1}^{k}M_t x_t \equiv \sum_{t=1}^{k}M_t y_t\) and \(x_t,y_t \in A_t,t=1,\cdots,k\).
  Now we need to prove \((x_1,\cdots,x_k)=(y_1,\cdots,y_k)\).
  Noting for \(i < j\) we have \(m_i \mid M_j\).
  If \((x_1,\cdots,x_k) \neq (y_1,\cdots,y_k)\), then let \(i=\min\{t:x_t \neq y_t\}\).
  Then \(\sum_{t=i}^{k}M_t x_t=\sum_{t=i}^{k}M_t y_t\).
  Consider \(\mod m_i\), we get \(M_ix_i \equiv M_iy_i \mod m_i\).
  Then \(m_i \mid M_i(x_i-y_i)\).
  Since \(\gcd(m_i,m_j)=1\) for \(i > j\), we get \(\gcd(m_i,M_i)=1\).
  So \(m_i \mid x_i-y_i\).
  Since \(x_i,y_i \in A_i\) and \(A_i\) is Complete residue of \(m_i\), we get \(x_i=y_i\).
  So \((x_1,\cdots,x_k)=(y_1,\cdots,y_k)\).

  So finally we get \(A\) is Complete residue of \(m\).
\end{solution}

\end{document}

