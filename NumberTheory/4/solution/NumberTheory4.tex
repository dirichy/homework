%!Mode:: "TeX:UTF-8"
%!TEX TS-program = xelatex
\documentclass{ctexart}
\newif\ifpreface
%\prefacetrue
\usepackage{fontspec}
\usepackage{bbm}
\usepackage{tikz}
\usepackage{amsmath,amssymb,amsthm,color,mathrsfs}
\usepackage{fixdif}
\usepackage{hyperref}
\usepackage{cleveref}
\usepackage{enumitem}%
\usepackage{expl3}
\usepackage{lipsum}
\usepackage[margin=0pt]{geometry}
\usepackage{listings}
\definecolor{mGreen}{rgb}{0,0.6,0}
\definecolor{mGray}{rgb}{0.5,0.5,0.5}
\definecolor{mPurple}{rgb}{0.58,0,0.82}
\definecolor{backgroundColour}{rgb}{0.95,0.95,0.92}

\lstdefinestyle{CStyle}{
  backgroundcolor=\color{backgroundColour},
  commentstyle=\color{mGreen},
  keywordstyle=\color{magenta},
  numberstyle=\tiny\color{mGray},
  stringstyle=\color{mPurple},
  basicstyle=\footnotesize,
  breakatwhitespace=false,
  breaklines=true,
  captionpos=b,
  keepspaces=true,
  numbers=left,
  numbersep=5pt,
  showspaces=false,
  showstringspaces=false,
  showtabs=false,
  tabsize=2,
  language=C
}
\usetikzlibrary{calc}
\theoremstyle{remark}
\newtheorem{lemma}{Lemma}
\usepackage{fontawesome5}
\usepackage{xcolor}
\newcounter{problem}
\newcommand{\Problem}{\begin{tikzpicture}[baseline]%
    \node at (-0.02em,0.3em) {$\mathbb{P}$};%
    \node[scale=0.7] at (0.2em,-0.0em) {R};%
    \node[scale=0.7] at (0.6em,0.4em) {O};%
    \node[scale=0.8] at (1.05em,0.25em) {B};%
    \node at (1.55em,0.3em) {L};%
    \node[scale=0.7] at (1.75em,0.45em) {E};%
    \node at (2.35em,0.3em) {M};%
  \end{tikzpicture}%
}
\renewcommand{\theproblem}{\Roman{problem}}
\newenvironment{problem}{\refstepcounter{problem}\noindent\color{blue}\Problem\theproblem}{}

\crefname{problem}{\protect\Problem}{Problem}
\newcommand\Solution{\begin{tikzpicture}[baseline]%
    \node at (-0.04em,0.3em) {$\mathbb{S}$};%
    \node[scale=0.7] at (0.35em,0.4em) {O};%
    \node at (0.7em,0.3em) {\textit{L}};%
    \node[scale=0.7] at (0.95em,0.4em) {U};%
    \node[scale=1.1] at (1.19em,0.32em){T};%
    \node[scale=0.85] at (1.4em,0.24em){I};%
    \node at (1.9em,0.32em){$\mathcal{O}$};%
    \node[scale=0.75] at (2.3em,0.21em){\texttt{N}};%
  \end{tikzpicture}}
\newenvironment{solution}{\begin{proof}[\Solution]}{\end{proof}}
\title{\input{../../.subject}\input{../.number}}
\makeatletter
\newcommand\email[1]{\def\@email{#1}\def\@refemail{mailto:#1}}
\newcommand\schoolid[1]{\def\@schoolid{#1}}
\ifpreface
  \def\@maketitle{
  \raggedright
  {\Huge \bfseries \sffamily \@title }\\[1cm]
  {\Huge  \bfseries \sffamily\heiti\@author}\\[1cm]
  {\Huge \@schoolid}\\[1cm]
  {\Huge\href\@refemail\@email}\\[0.5cm]
  \Huge\@date\\[1cm]}
\else
  \def\@maketitle{
    \raggedright
    \begin{center}
      {\Huge \bfseries \sffamily \@title }\\[4ex]
      {\Large  \@author}\\[4ex]
      {\large \@schoolid}\\[4ex]
      {\href\@refemail\@email}\\[4ex]
      \@date\\[8ex]
    \end{center}}
\fi
\makeatother
\ifpreface
  \usepackage[placement=bottom,scale=1,opacity=1]{background}
\fi

\author{白永乐}
\schoolid{202011150087}
\email{202011150087@mail.bnu.edu.cn}

\def\to{\rightarrow}
\newcommand{\xor}{\vee}
\newcommand{\bor}{\bigvee}
\newcommand{\band}{\bigwedge}
\newcommand{\xand}{\wedge}
\newcommand{\minus}{\mathbin{\backslash}}
\newcommand{\mi}[1]{\mathscr{P}(#1)}
\newcommand{\card}{\mathrm{card}}
\newcommand{\oto}{\leftrightarrow}
\newcommand{\hin}{\hat{\in}}
\newcommand{\gl}{\mathrm{GL}}
\newcommand{\im}{\mathrm{Im}}
\newcommand{\re }{\mathrm{Re }}
\newcommand{\rank}{\mathrm{rank}}
\newcommand{\tra}{\mathop{\mathrm{tr}}}
\renewcommand{\char}{\mathop{\mathrm{char}}}
\DeclareMathOperator{\ot}{ordertype}
\DeclareMathOperator{\dom}{dom}
\DeclareMathOperator{\ran}{ran}

\renewcommand{\phi}{\varphi}
\begin{document}
\large
\iffalse
  \setlength{\baselineskip}{1.2em}
  \ifpreface
    \backgroundsetup{contents={%
    \begin{tikzpicture}
      \fill [white] (current page.north west) rectangle ($(current page.north east)!.3!(current page.south east)$) coordinate (a);
      \fill [bgc] (current page.south west) rectangle (a);
\end{tikzpicture}}}
\definecolor{word}{rgb}{1,1,0}
\definecolor{bgc}{rgb}{1,0.95,0}
\setlength{\parindent}{0pt}
\thispagestyle{empty}
\begin{tikzpicture}%
  % \node[xscale=2,yscale=4] at (0cm,0cm) {\sffamily\bfseries \color{word} under};%
  \node[xscale=4.5,yscale=10] at (10cm,1cm) {\sffamily\bfseries \color{word} Graduate Homework};%
  \node[xscale=4.5,yscale=10] at (8cm,-2.5cm) {\sffamily\bfseries \color{word} In Mathematics};%
\end{tikzpicture}
\ \vspace{1cm}\\
\begin{minipage}{0.25\textwidth}
  \textcolor{bgc}{王胤雅是傻逼}
\end{minipage}
\begin{minipage}{0.75\textwidth}
  \maketitle
\end{minipage}
\vspace{4cm}\ \\
\begin{minipage}{0.2\textwidth}
  \
\end{minipage}
\begin{minipage}{0.8\textwidth}
  {\Huge
    \textinconsolatanf{}
  }General fire extinguisher
\end{minipage}
\newpage\backgroundsetup{contents={}}\setlength{\parindent}{2em}

  \else
    \maketitle
  \fi
\fi
\newgeometry{left=2cm,right=2cm,top=2cm,bottom=2cm}
%from_here_to_type
%作业:p42: 2, 3;    p45: 1, 2, 4.  额外的好玩作业:尝试写一个RSA小程序,包括给出若干密钥对(e,d),加密,解密。下次课用噢!
\begin{problem}\label{pro:p42.2}
  Prove: If \(m \in \mathbb{Z}^{+},a \in \mathbb{Z}, \gcd(a,m)=1\), \(A\) is reduced residue system of \(m\), then
  \[
    \sum_{i \in A}\left\{\frac{ai}{m}\right\} = \frac{1}{2} \phi(m)
  \]
\end{problem}
\begin{solution}
  Let \(f:\mathbb{Z} \to \{0,\cdots,m-1\}\) and \(f(x)\equiv x \mod m\).
  Then easily \(\{\frac{x}{m}\} = \{\frac{f(x)}{m}\}\).
  So we get \(\sum_{i \in A}\left\{\frac{ai}{m}\right\}=\sum_{i \in A}\left\{\frac{f(ai)}{m}\right\}\).
  Let \(B=\{f(ai):i \in A\}\), since \(\gcd(m,a)=1\), we obtain \(B\) is reduced residue system of \(m\), too.
  Easily to know \(\left\{\frac{f(ai)}{m}\right\}=\frac{f(ai)}{m}\),
  then \(\sum_{i \in A} \left\{\frac{ai}{m}\right\}=\sum_{j \in B}\frac{j}{m}\).
  Noting that \(g:B \to B,j \mapsto m-j\) is bijection, so
  \(\sum_{j \in B}\frac{j}{m}=\sum_{j \in B}\frac{m-j}{m}=\frac{1}{2}\sum_{j \in B}\frac{j + m-j}{m}=\frac{1}{2}|B|\).
  Obvious since \(B\) is reduced residue system of \(m\), we easily get \(|B|=\phi(m)\).
  So finally we get \(\sum_{i \in A}\left\{\frac{i}{m}\right\}=\frac{1}{2}\phi(m)\).
\end{solution}
\begin{problem}\label{pro:42.3}
  \begin{enumerate}
    \item Prove: \(\sum_{i=0}^{a}\phi(p^i)=p^a\), where \(p\) is prime.
    \item Prove: \(\sum_{d \in \mathbb{N}: d \mid a}\phi(d)=a\).
  \end{enumerate}
\end{problem}
\begin{solution}
  \begin{enumerate}
    \item \label{it:2.1} Easily to know \(\phi(p^k)=p^k \times \frac{p-1}{p}=(p-1)p^{k-1} )\).
      So we get \(\sum_{i=1}^{a}\phi(p^i)=\sum_{i=1}^{a}(p-1)p^{i-1}=(p-1) \frac{p^a-1}{p-1}=p^a-1\).
      And \(\phi(1)=1\), so finally we get \(\sum_{i=0}^{a}\phi(p^i)=p^a-1+1=p^a\).
    \item Let \(A:=\{n \in \mathbb{N}: \sum_{d \in \mathbb{N}:d \mid n}\phi(d)=n\}\), then from \ref{it:2.1} we get \(\{p^k:p \in \mathbb{P},k \in \mathbb{N}\} \subset A\),
      where \(\mathbb{P}\) is the set of primes.
      Now to prove \(A=\mathbb{N}\), we only need to prove that for \(m,n \in A \AND \gcd(m,n)=1\) we have \(nm \in A\).

      Let \(M:=\{d \in \mathbb{N}:d \mid m\},N:=\{d \in \mathbb{N}:d \mid n\},D:=\{d \in \mathbb{N}:d \mid nm\}\),
      \(f:M \times N \to D,f(x,y):=xy\), we will prove that \(f\) is bijection,
      and et \(g:D \to M \times N,g(z)=(\gcd(z,m),\gcd(z,n))\), we will prove \(g=f^{-1}\).

      For \((x,y)\in M \times N\), we need to prove \(g \circ f (x,y)=(x,y)\). i.e., \(\gcd(xy,m)=x,\gcd(xy,n)=y\).
      Since \(y \mid n\) and \(\gcd(n,m)=1\), we easily get \(\gcd(y,m)=1\), so \(\gcd(xy,m)=\gcd(x,m)\).
      Noting \(x \mid m\), we get \(\gcd(x,m)=x\). So \(\gcd(xy,m)=x\).
      For the same reason we get \(\gcd(xy,n)=y\).

      For \(z \in D\), write \(x=\gcd(z,m),y=\gcd(z,n)\), then \(g(z)=(x,y)\).
      We need to prove \(f(x,y)=z\), i.e., \(xy=z\).
      Since \(\gcd(m,n)=1\), easily \(z = \gcd(z,nm) = \gcd(z,m)\gcd(z,n)=xy\).

      So \(g=f^{-1}\) and thus \(f\) is bijection.
      So we know \(\sum_{d \in D}\phi(d)=\sum_{(x,y)\in M \times N}\phi(xy)\).
      Noting \(\gcd(x,y)\mid \gcd(m,n)=1\), we know \(\phi(x,y)=\phi(x)\phi(y)\).
      So \(\sum_{d \in D}\phi(d)=\sum_{x \in M}\sum_{y \in N}\phi(x)\phi(y)=\sum_{x \in M}\phi(x)\sum_{y \in N}\phi(y)\).
      Recalling \(m,n \in A\), we know \(\sum_{x \in M}\phi(x)=m,\sum_{y \in N}\phi(y)=n\), so finally \(\sum_{d \in D}\phi(d)=nm\).
      So \(nm \in A\).

      Now for any \(a \in \mathbb{N}^+\), we know \(a=\prod_{k=1}^{t}p_k^{\alpha_k}\), where \(p_k:k=1,\cdots,t\) are different primes.
      Then \(p_k^{\alpha_k}\in A\).
      So \(\forall a \in \mathbb{N}^+,a \in A\), thus \(\sum_{d \in \mathbb{N}:d \mid a}\phi(d)=a\).
  \end{enumerate}
\end{solution}

\begin{problem}\label{pro:45.1}
  If today is Monday, then what day is it \(10^{10^{10}}\) days after today?
\end{problem}
\begin{solution}
  Only need to find the remainder of \(10^{10^{10} } \mod 7\).
  Noting that \(\phi(7)=6\) and \(\gcd(10,7)=1\), so \(10^6 \equiv 1 \mod 7\).
  So we only need to find \(10^{10} \mod 6\).
  Since \(6 = 2 \times 3\), we only need to find \(10^{10} \mod 2,10^{10} \mod 3\).
  Easy to know \(10^{10} \equiv 0 \mod 2\).
  Noting \(10 \equiv 1 \mod 3\), we get \(10^{10} \equiv 1 \mod 3\).
  So from the Chinese remainder theorem we get \(10^{10} \equiv 4 \mod 6\).
  So \(10^{10^{10} }\equiv 10^4 \equiv 3^4\mod 7\).
  By calculation easy to know \(3^4 \equiv 9^2 \equiv 2^2 \equiv 4 \mod 7\).
  So it is Friday \(10^{10^{10} }\) days after today.
\end{solution}

\begin{problem}\label{pro:45.2}
  Find the remainder of \((12371^{56} + 34)^{28} \mod 111 \).
\end{problem}
\begin{solution}
  Easily \(111=3 \times 37\), so we only need to find the remainder \(\mod 3,\mod 37\) respectively.
  Easily \((12371^{56} + 34)^{28} \equiv ((-1)^{56} + 1)^{28} \equiv (-1)^{28} \equiv 1 \mod 3\).
  Easily \(\phi(37)=36\), and \(\gcd(12371,37)=1\), so
  \((12371^{56} + 34)^{28} \equiv (13^{20} -3)^{28} \mod 37\).
  Noting \(13^4 \equiv -3 \mod 37\), we get
  \(13^{20} \equiv (-3)^5 \equiv -243 \equiv 16 \mod 37\).
  Thus, \((13^{20} -3)^{28} \equiv (16-3)^{28} \equiv 13^{28} \equiv 13^{20} \times 13^8 \equiv 16 \times (-3)^2 \equiv 144 \equiv 33 \mod 37\).

  Now by Chinese remainder theorem we get \((12371^{56} +34)^{28} \equiv 70 \mod 111\).
\end{solution}

\begin{problem}\label{pro:45.4}
  Assume \(\frac{a}{b} \in \mathbb{Q}\), where \(0<a<b, \gcd(a,b)=1\). Then \(\frac{a}{b}\) is pure recurring decimal \(\iff\)
  \(\exists t \in \mathbb{N}^+\) s.t. \(10^t \equiv 1\mod b\).
  Besides, \(\min\{t \in \mathbb{N}^+: 10^t \equiv 1\mod b\}\) is the length of cycle section.
\end{problem}
\begin{solution}
  Let \(l\) be the length of cycle section of \(\frac{a}{b}\).

  ``\(\implies\)'': Assume \(\frac{a}{b}=\sum_{k=1}^{\infty} 10^{-kl} x\), where \(x \in \mathbb{N},0 < x <10^l\).
  Then we get \(\frac{a}{b}=\frac{1}{10^l} \frac{1}{1-10^{-l}}=\frac{x}{10^l-1}\).
  Then \(a(10^l-1)=bx\). Since \(\gcd(a,b)=1\), we get \(b \mid 10^l-1\). Let \(t=l\) will work.

  ``\(\impliedby\)'': Assume \(10^t \equiv 1 \mod b\), where \(t \in \mathbb{N}^+\).
  Let \(10^t-1=bk\), where \(k \in \mathbb{N}^+\). Let \(x=ak\), we will prove \(\frac{a}{b}=\sum_{k=1}^{\infty} 10^{-kt} x\).
  Easily \(\sum_{k=1}^{\infty} 10^{-kt}x = \frac{x}{10^t-1}=\frac{ak}{bk}=\frac{a}{b}\).
  So \(\frac{a}{b}\) is pure recurring decimal and \(l \mid t\).

  In the first part we have proved that \(l \in \{t \in \mathbb{N}^+:10^t \equiv 1 \mod b\}\).
  In the second part we have proved that \(\forall t \in \mathbb{N}^+ \AND 10^t \equiv 1 \mod b, l \mid t\).
  So obviously \(l=\min\{t \in \mathbb{N}^+:10^t \equiv 1 \mod b\}\).
\end{solution}

\end{document}
