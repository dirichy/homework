%!Mode:: "TeX:UTF-8"
%!TEX encoding = UTF-8 Unicode
%!TEX TS-program = xelatex
\documentclass{ctexart}
\newif\ifpreface
%\prefacetrue
\usepackage{fontspec}
\usepackage{bbm}
\usepackage{tikz}
\usepackage{amsmath,amssymb,amsthm,color,mathrsfs}
\usepackage{fixdif}
\usepackage{hyperref}
\usepackage{cleveref}
\usepackage{enumitem}%
\usepackage{expl3}
\usepackage{lipsum}
\usepackage[margin=0pt]{geometry}
\usepackage{listings}
\definecolor{mGreen}{rgb}{0,0.6,0}
\definecolor{mGray}{rgb}{0.5,0.5,0.5}
\definecolor{mPurple}{rgb}{0.58,0,0.82}
\definecolor{backgroundColour}{rgb}{0.95,0.95,0.92}

\lstdefinestyle{CStyle}{
  backgroundcolor=\color{backgroundColour},
  commentstyle=\color{mGreen},
  keywordstyle=\color{magenta},
  numberstyle=\tiny\color{mGray},
  stringstyle=\color{mPurple},
  basicstyle=\footnotesize,
  breakatwhitespace=false,
  breaklines=true,
  captionpos=b,
  keepspaces=true,
  numbers=left,
  numbersep=5pt,
  showspaces=false,
  showstringspaces=false,
  showtabs=false,
  tabsize=2,
  language=C
}
\usetikzlibrary{calc}
\theoremstyle{remark}
\newtheorem{lemma}{Lemma}
\usepackage{fontawesome5}
\usepackage{xcolor}
\newcounter{problem}
\newcommand{\Problem}{\begin{tikzpicture}[baseline]%
    \node at (-0.02em,0.3em) {$\mathbb{P}$};%
    \node[scale=0.7] at (0.2em,-0.0em) {R};%
    \node[scale=0.7] at (0.6em,0.4em) {O};%
    \node[scale=0.8] at (1.05em,0.25em) {B};%
    \node at (1.55em,0.3em) {L};%
    \node[scale=0.7] at (1.75em,0.45em) {E};%
    \node at (2.35em,0.3em) {M};%
  \end{tikzpicture}%
}
\renewcommand{\theproblem}{\Roman{problem}}
\newenvironment{problem}{\refstepcounter{problem}\noindent\color{blue}\Problem\theproblem}{}

\crefname{problem}{\protect\Problem}{Problem}
\newcommand\Solution{\begin{tikzpicture}[baseline]%
    \node at (-0.04em,0.3em) {$\mathbb{S}$};%
    \node[scale=0.7] at (0.35em,0.4em) {O};%
    \node at (0.7em,0.3em) {\textit{L}};%
    \node[scale=0.7] at (0.95em,0.4em) {U};%
    \node[scale=1.1] at (1.19em,0.32em){T};%
    \node[scale=0.85] at (1.4em,0.24em){I};%
    \node at (1.9em,0.32em){$\mathcal{O}$};%
    \node[scale=0.75] at (2.3em,0.21em){\texttt{N}};%
  \end{tikzpicture}}
\newenvironment{solution}{\begin{proof}[\Solution]}{\end{proof}}
\title{\input{../../.subject}\input{../.number}}
\makeatletter
\newcommand\email[1]{\def\@email{#1}\def\@refemail{mailto:#1}}
\newcommand\schoolid[1]{\def\@schoolid{#1}}
\ifpreface
  \def\@maketitle{
  \raggedright
  {\Huge \bfseries \sffamily \@title }\\[1cm]
  {\Huge  \bfseries \sffamily\heiti\@author}\\[1cm]
  {\Huge \@schoolid}\\[1cm]
  {\Huge\href\@refemail\@email}\\[0.5cm]
  \Huge\@date\\[1cm]}
\else
  \def\@maketitle{
    \raggedright
    \begin{center}
      {\Huge \bfseries \sffamily \@title }\\[4ex]
      {\Large  \@author}\\[4ex]
      {\large \@schoolid}\\[4ex]
      {\href\@refemail\@email}\\[4ex]
      \@date\\[8ex]
    \end{center}}
\fi
\makeatother
\ifpreface
  \usepackage[placement=bottom,scale=1,opacity=1]{background}
\fi

\author{白永乐}
\schoolid{202011150087}
\email{202011150087@mail.bnu.edu.cn}

\def\to{\rightarrow}
\newcommand{\xor}{\vee}
\newcommand{\bor}{\bigvee}
\newcommand{\band}{\bigwedge}
\newcommand{\xand}{\wedge}
\newcommand{\minus}{\mathbin{\backslash}}
\newcommand{\mi}[1]{\mathscr{P}(#1)}
\newcommand{\card}{\mathrm{card}}
\newcommand{\oto}{\leftrightarrow}
\newcommand{\hin}{\hat{\in}}
\newcommand{\gl}{\mathrm{GL}}
\newcommand{\im}{\mathrm{Im}}
\newcommand{\re }{\mathrm{Re }}
\newcommand{\rank}{\mathrm{rank}}
\newcommand{\tra}{\mathop{\mathrm{tr}}}
\renewcommand{\char}{\mathop{\mathrm{char}}}
\DeclareMathOperator{\ot}{ordertype}
\DeclareMathOperator{\dom}{dom}
\DeclareMathOperator{\ran}{ran}

\begin{document}
\large
\setlength{\baselineskip}{1.2em}
\ifpreface
\backgroundsetup{contents={%
    \begin{tikzpicture}
      \fill [white] (current page.north west) rectangle ($(current page.north east)!.3!(current page.south east)$) coordinate (a);
      \fill [bgc] (current page.south west) rectangle (a);
\end{tikzpicture}}}
\definecolor{word}{rgb}{1,1,0}
\definecolor{bgc}{rgb}{1,0.95,0}
\setlength{\parindent}{0pt}
\thispagestyle{empty}
\begin{tikzpicture}%
  % \node[xscale=2,yscale=4] at (0cm,0cm) {\sffamily\bfseries \color{word} under};%
  \node[xscale=4.5,yscale=10] at (10cm,1cm) {\sffamily\bfseries \color{word} Graduate Homework};%
  \node[xscale=4.5,yscale=10] at (8cm,-2.5cm) {\sffamily\bfseries \color{word} In Mathematics};%
\end{tikzpicture}
\ \vspace{1cm}\\
\begin{minipage}{0.25\textwidth}
  \textcolor{bgc}{王胤雅是傻逼}
\end{minipage}
\begin{minipage}{0.75\textwidth}
  \maketitle
\end{minipage}
\vspace{4cm}\ \\
\begin{minipage}{0.2\textwidth}
  \
\end{minipage}
\begin{minipage}{0.8\textwidth}
  {\Huge
    \textinconsolatanf{}
  }General fire extinguisher
\end{minipage}
\newpage\backgroundsetup{contents={}}\setlength{\parindent}{2em}

\newgeometry{left=2cm,right=2cm,top=2cm,bottom=2cm}
\else
\newgeometry{left=2cm,right=2cm,top=2cm,bottom=2cm}
%\maketitle \fi
%from_here_to_type
%p59: 1, 2, 4; p61: 1, 2
\begin{problem}\label{pro:1}
  Find the solution of \(6x^3 + 27 x^2 + 17x + 20 \equiv 0 \pmod{ 30}\).
\end{problem}
\begin{solution}
  Easily \(30=2\times3\times5\), so we consider three equations:
  \[
    \begin{cases}
      6x^3 + 27 x^2 + 17x + 20 \equiv 0 \mod 2 \\
      6x^3 + 27 x^2 + 17x + 20 \equiv 0 \mod 3 \\
      6x^3 + 27 x^2 + 17x + 20 \equiv 0 \mod 5
    \end{cases}
  \]
  The first equation is always true, because \(6x^3 + 27 x^2 + 17x + 20 \equiv x^2+x \equiv 0 \mod 2\).
  The second equation is equivalent to \(2x+2 \equiv 0 \mod 3\). Solve it and get \(x \equiv 2 \mod 3\).
  The last equation is equivalent to \(x^3+2x^2+2x =x(x^2+2x+2)\equiv 0 \mod 5\).
  Easy to get that \(x \equiv 0,1,2 \mod 5\).
  So finally we get that \(x \equiv 2,20,26 \mod 30\).
\end{solution}

\begin{problem}\label{pro:2}
  Find the solution of \(31x^4 + 57x^3 + 96x + 191 \equiv 0 \pmod{ 225}\).
\end{problem}
\begin{solution}
  Easy to get that \(225 = 3^2 \times 5^2\).
  So the equation is equivalent to
  \[
    \begin{cases}
      4 x^4+3x^3+6x+2 \equiv 0 \mod 9 \\
      6 x^4 + 7 x^3 -4x-9 \equiv 0 \mod 25
    \end{cases}
  \]
  First we consider \(4x^4+3x^3+6x+2 \equiv 0 \mod 3\).
  Easy to get \(x \equiv 1,2 \mod 3\).
  If \(x \equiv 1 \mod 3\), assume \(x \equiv 1+3k \mod 9\), then \((4+3+6+2)+16 \times 3k + 9 \times 3k + 6 \times 3k \equiv 0 \mod 9\),
  then \(2+k \equiv 0 \mod 3\), so \(x \equiv 4 \mod 9\).
  If \(x \equiv 2 \mod 3\), then for the same reason we get \((4\times 2^4+3\times 2^3+6\times 2+2)+(-16+9+6)\times 3k \equiv 0 \mod 9\),
  thus \(x \equiv 5 \mod 9\). So \(x \equiv 4,5 \mod 9\).

  Second we consider \(6 x^4 + 7 x^3 -4x-9 \equiv 0 \mod 5\). Obviously \(x \not \equiv 0 \mod 5\), so \(x^4 \equiv 1 \mod 5\).
  So \(2x^3+x-3 \equiv 0 \mod 5\), i.e., \((x-1)(2x^2+2x+3) \equiv 0 \mod 5\). Then \(x \equiv 1 \mod 5\) or \(2x^2+2x+3 \equiv 0 \mod 5\).
  Noting \(2x^2+2x+3 \equiv (x-2)(2x+6) \mod 5\), so we finally get \(x \equiv 1,2 \mod 5\).
  Use the same method as above, we can get that \(x \equiv 1,22 \mod 25\)

  Finally, we consider
  \[
    \begin{cases}
      x \equiv 4,5 \mod 9 \\
      x \equiv 1,22 \mod 25
    \end{cases}
  \]
  We obtain that \(x \equiv 22,122,76,176 \mod 225\).
\end{solution}
\begin{problem}\label{pro:3}
  Prove: \(\forall m \in \mathbb{N}^+,\exists x,y \in \mathbb{Z},5x^2 + 11y^2 \equiv 1 \pmod{ m}\).
\end{problem}
\begin{solution}
  Let \(s=3^{16} \times \prod_{p \in \mathbb{P},5<p \leq m}p^{16}\), then easily \(s \equiv 1 \mod 32\).
  Let \(t=5\), and let
  \[
    \begin{cases}
      a=11s^2-22st-5t^2  \\
      b=-11s^2-10st+5t^2 \\
      c=20t^2+44s^2
    \end{cases}
  \]
  Then easy to check that \(5a^2+11b^2=c^2\).
  Easily \(a \equiv 11-110-125 \equiv 0\mod 32\).
  Then since \(32 \mid b+a = -32st\) we get \(32 \mid b\).
  Then \(32^2 \mid 5a^2+11b^2 = c^2\), thus \(32 \mid c\).
  Let \(a_1=\frac{a}{32},b_1=\frac{b}{32},c_1=\frac{c}{32}\), then \(5 a_1^2+11 b_1^2=c_1^2\).
  Now we will prove that \(\gcd(c_1,m)=1\). If not, assume \(p \in \mathbb{P}\) and \(p \mid \gcd(c_1,m)\).
  If \(p>5 \OR p=3\), then since \(p \mid m\) we get \(p \leq m\), so \(p \mid s\).
  Since \(p \mid c=20t^2+44s^2\), we get \(p \mid t\), then \(p=5\), contradiction!
  If \(p=5\), then \(p \mid 20 t^2\), then \(p \mid 44 s^2\), but \(5 \nmid s\), contradiction!
  If \(p=2\), then \(2 \mid \frac{c}{32}\), then \(16 \mid 5t^2+11s^2\).
  But easily \(5 t^2 + 11 s^2 \equiv 125+11 \equiv 8 \mod p\), contradiction!
  So we get \(\gcd(c_1,m)=1\).
  So \(\exists d,c_1 d \equiv 1 \mod m\).
  Let \(x=a_1 d,y=b_1 d\), then \(5x^2+11y^2 =c^2d^2 \equiv 1 \mod m\).
\end{solution}

\begin{problem}\label{pro:4}
  If \(n \mid p-1, n > 1,\gcd(a,p)=1\), prove :
  \begin{enumerate}
    \item \(x^n \equiv a \pmod{ p}\) has solution \(\iff\) \(a^{\frac{p-1}{n}} \equiv 1 \pmod{ p}\).
    \item If \(x^n \equiv a \pmod{ p}\) has solution, then it has \(n\) solutions.
  \end{enumerate}
\end{problem}
\begin{solution}
  \begin{enumerate}
    \item ``\(\implies\)'':Since \(\gcd(a,p)=1\) easily \(\gcd(x,p)=1\). So \(a^{\frac{p-1}{n}} \equiv x^{p-1} \equiv 1 \mod p\).
      ``\(\impliedby\)'': Easy to know that there is at most \(\frac{p-1}{n}\) different \(a\) satisfy \(\exists x,x^n \equiv a \mod p\).
      For every \(a\), there is at most \(n\) different \(x\) satisfy \(x^n \equiv a \mod p\).
      And for every \(x\) satisfy \(\gcd(x,p)=1\), there is a unique \(a\) satisfy \(x^n \equiv a \mod p\).
      So \(\sum_{x,a \in \mathbb{Z} / p \mathbb{Z},x^n \equiv a,x \neq 0}1=\sum_{a \in \mathbb{Z} / p \mathbb{Z},a^{\frac{p-1}{n}} \equiv 1}\sum_{x \in \mathbb{Z} / p \mathbb{Z},x^n \equiv a}1 \leq p-1\).
      But \(\sum_{x,a \in \mathbb{Z} / p \mathbb{Z},x^n \equiv a,x \neq 0}1=\sum_{x \in \mathbb{Z} / p \mathbb{Z},x \neq 0} 1 = p-1\)
      So we get \(\forall a:a^{\frac{p-1}{n}}\equiv 1 \mod p\), there is \(n\) different \(x\) satisfy \(x^n \equiv a \mod p\).
    \item Have been proved above.
  \end{enumerate}
\end{solution}

\begin{problem}\label{pro:5}
  \(n \in \mathbb{N}^+\), \(\gcd(a,m)=1\), \(x^n \equiv a \pmod{ m}\) has a solution \(x \equiv x_0 \pmod{ m}\).
  Prove all the solution of \(x^n \equiv a \pmod{ m}\) have the form of \(x \equiv yx_0 \pmod{ m}\),
  where \(y\) is the solution of \(y^n \equiv 1\pmod{ m}\).
\end{problem}
\begin{solution}
  Easy to know \(x \equiv y x_0 \mod m\) is solution of \(x^n \equiv a \mod m\).
  Now only need to check every solution has this form.
  Assume \(x^n \equiv a \mod m\). Easily \(\gcd(x,m)=\gcd(x_0,m)=1\).
  Then \(\exists b,bx_0 \equiv 1 \mod m\).
  Then \(x^n b^n \equiv x_0^n b^n \equiv 1 \mod m\).
  Let \(y = xb\), then \(y^n \equiv 1 \mod m\).
  Then \(y x_0 \equiv x b x_0 \equiv x \mod m\).
\end{solution}
\end{document}
