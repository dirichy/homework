%!Mode:: "TeX:UTF-8"
%!TEX encoding = UTF-8 Unicode
%!TEX TS-program = xelatex
\documentclass{ctexart}
\newif\ifpreface
%\prefacetrue
\usepackage{fontspec}
\usepackage{bbm}
\usepackage{tikz}
\usepackage{amsmath,amssymb,amsthm,color,mathrsfs}
\usepackage{fixdif}
\usepackage{hyperref}
\usepackage{cleveref}
\usepackage{enumitem}%
\usepackage{expl3}
\usepackage{lipsum}
\usepackage[margin=0pt]{geometry}
\usepackage{listings}
\definecolor{mGreen}{rgb}{0,0.6,0}
\definecolor{mGray}{rgb}{0.5,0.5,0.5}
\definecolor{mPurple}{rgb}{0.58,0,0.82}
\definecolor{backgroundColour}{rgb}{0.95,0.95,0.92}

\lstdefinestyle{CStyle}{
  backgroundcolor=\color{backgroundColour},
  commentstyle=\color{mGreen},
  keywordstyle=\color{magenta},
  numberstyle=\tiny\color{mGray},
  stringstyle=\color{mPurple},
  basicstyle=\footnotesize,
  breakatwhitespace=false,
  breaklines=true,
  captionpos=b,
  keepspaces=true,
  numbers=left,
  numbersep=5pt,
  showspaces=false,
  showstringspaces=false,
  showtabs=false,
  tabsize=2,
  language=C
}
\usetikzlibrary{calc}
\theoremstyle{remark}
\newtheorem{lemma}{Lemma}
\usepackage{fontawesome5}
\usepackage{xcolor}
\newcounter{problem}
\newcommand{\Problem}{\begin{tikzpicture}[baseline]%
    \node at (-0.02em,0.3em) {$\mathbb{P}$};%
    \node[scale=0.7] at (0.2em,-0.0em) {R};%
    \node[scale=0.7] at (0.6em,0.4em) {O};%
    \node[scale=0.8] at (1.05em,0.25em) {B};%
    \node at (1.55em,0.3em) {L};%
    \node[scale=0.7] at (1.75em,0.45em) {E};%
    \node at (2.35em,0.3em) {M};%
  \end{tikzpicture}%
}
\renewcommand{\theproblem}{\Roman{problem}}
\newenvironment{problem}{\refstepcounter{problem}\noindent\color{blue}\Problem\theproblem}{}

\crefname{problem}{\protect\Problem}{Problem}
\newcommand\Solution{\begin{tikzpicture}[baseline]%
    \node at (-0.04em,0.3em) {$\mathbb{S}$};%
    \node[scale=0.7] at (0.35em,0.4em) {O};%
    \node at (0.7em,0.3em) {\textit{L}};%
    \node[scale=0.7] at (0.95em,0.4em) {U};%
    \node[scale=1.1] at (1.19em,0.32em){T};%
    \node[scale=0.85] at (1.4em,0.24em){I};%
    \node at (1.9em,0.32em){$\mathcal{O}$};%
    \node[scale=0.75] at (2.3em,0.21em){\texttt{N}};%
  \end{tikzpicture}}
\newenvironment{solution}{\begin{proof}[\Solution]}{\end{proof}}
\title{\input{../../.subject}\input{../.number}}
\makeatletter
\newcommand\email[1]{\def\@email{#1}\def\@refemail{mailto:#1}}
\newcommand\schoolid[1]{\def\@schoolid{#1}}
\ifpreface
  \def\@maketitle{
  \raggedright
  {\Huge \bfseries \sffamily \@title }\\[1cm]
  {\Huge  \bfseries \sffamily\heiti\@author}\\[1cm]
  {\Huge \@schoolid}\\[1cm]
  {\Huge\href\@refemail\@email}\\[0.5cm]
  \Huge\@date\\[1cm]}
\else
  \def\@maketitle{
    \raggedright
    \begin{center}
      {\Huge \bfseries \sffamily \@title }\\[4ex]
      {\Large  \@author}\\[4ex]
      {\large \@schoolid}\\[4ex]
      {\href\@refemail\@email}\\[4ex]
      \@date\\[8ex]
    \end{center}}
\fi
\makeatother
\ifpreface
  \usepackage[placement=bottom,scale=1,opacity=1]{background}
\fi

\author{白永乐}
\schoolid{202011150087}
\email{202011150087@mail.bnu.edu.cn}

\def\to{\rightarrow}
\newcommand{\xor}{\vee}
\newcommand{\bor}{\bigvee}
\newcommand{\band}{\bigwedge}
\newcommand{\xand}{\wedge}
\newcommand{\minus}{\mathbin{\backslash}}
\newcommand{\mi}[1]{\mathscr{P}(#1)}
\newcommand{\card}{\mathrm{card}}
\newcommand{\oto}{\leftrightarrow}
\newcommand{\hin}{\hat{\in}}
\newcommand{\gl}{\mathrm{GL}}
\newcommand{\im}{\mathrm{Im}}
\newcommand{\re }{\mathrm{Re }}
\newcommand{\rank}{\mathrm{rank}}
\newcommand{\tra}{\mathop{\mathrm{tr}}}
\renewcommand{\char}{\mathop{\mathrm{char}}}
\DeclareMathOperator{\ot}{ordertype}
\DeclareMathOperator{\dom}{dom}
\DeclareMathOperator{\ran}{ran}

\begin{document}
\large
\setlength{\baselineskip}{1.2em}
\ifpreface
\backgroundsetup{contents={%
    \begin{tikzpicture}
      \fill [white] (current page.north west) rectangle ($(current page.north east)!.3!(current page.south east)$) coordinate (a);
      \fill [bgc] (current page.south west) rectangle (a);
\end{tikzpicture}}}
\definecolor{word}{rgb}{1,1,0}
\definecolor{bgc}{rgb}{1,0.95,0}
\setlength{\parindent}{0pt}
\thispagestyle{empty}
\begin{tikzpicture}%
  % \node[xscale=2,yscale=4] at (0cm,0cm) {\sffamily\bfseries \color{word} under};%
  \node[xscale=4.5,yscale=10] at (10cm,1cm) {\sffamily\bfseries \color{word} Graduate Homework};%
  \node[xscale=4.5,yscale=10] at (8cm,-2.5cm) {\sffamily\bfseries \color{word} In Mathematics};%
\end{tikzpicture}
\ \vspace{1cm}\\
\begin{minipage}{0.25\textwidth}
  \textcolor{bgc}{王胤雅是傻逼}
\end{minipage}
\begin{minipage}{0.75\textwidth}
  \maketitle
\end{minipage}
\vspace{4cm}\ \\
\begin{minipage}{0.2\textwidth}
  \
\end{minipage}
\begin{minipage}{0.8\textwidth}
  {\Huge
    \textinconsolatanf{}
  }General fire extinguisher
\end{minipage}
\newpage\backgroundsetup{contents={}}\setlength{\parindent}{2em}

\newgeometry{left=2cm,right=2cm,top=2cm,bottom=2cm}
\else
\newgeometry{left=2cm,right=2cm,top=2cm,bottom=2cm}
%\maketitle \fi
%from_here_to_type

\begin{problem}\label{pro:1}
  When \(p\) is prime, \(p > 2,  p^\alpha\mid A\), find all the solution of \(y^2 \equiv A \pmod{p^\alpha}\).
\end{problem}
\begin{solution}
  Since \( p^\alpha\mid A\), then it is equal to find the solution of \(y^2 \equiv 0 \pmod{p^\alpha}\).
  Easy to prove that \(p^\alpha \mid y^2 \iff p^{\ceil{\frac{\alpha}{2}}} \mid y\).
  So all the solutions are \(y=k p^{\ceil{\frac{\alpha}{2}}},k \in \mathbb{Z}\).
\end{solution}
\begin{problem}\label{pro:2}
  Prove:
  \[
    \exists x,ax^2 + bx + c \equiv 0 \pmod{m},\gcd(2a,m)=1
  \]
  \(\iff\)
  \[
    \exists x,x^2 \equiv q \pmod{m},q=b^2 -4ac
  \]
\end{problem}
\begin{solution}
  Since \(\gcd(2a,m)=1\), we can get \(\gcd(4a,m)=1\).
  So \(ax^2 +bx +c \equiv 0 \mod m \iff (2ax+b)^2 \equiv b^2 -4ac \mod m\).
  Let \(y=2ax+b\), and let \(t\) satisfy \(2at \equiv 1 \mod m\), then \(x \equiv t(y-b)\mod m\).
  So the two equation has solution at same condition, and \(ax^2 +bx+c \equiv 0 \mod m \iff x \equiv t(y-b) \mod m \AND y^2 \equiv b^2 -4ac \mod m\) .
\end{solution}

\begin{problem}\label{pro:3}
  Find out all the squared remainder and non-squared remainder of \(37\).
\end{problem}
\begin{solution}
  Only need to calculate \(\{m^2 \mod 37:m \in \mathbb{Z},1 \leq m \leq 18\}\).\\
  They are \(\{1,4,9,16,25,36,12,27,7,26,10,33,21,11,3,34,30,28\}\).
  So squared remainder of \(37\) are \(\{1,4,9,16,25,36,12,27,7,26,10,33,21,11,3,34,30,28\}\), and
  non-squared remainder of \(37\) are \(\{2,5,6,8,13,14,15,17,18,19,20,22,23,24,29,31,32,35\}\).
\end{solution}

\begin{problem}\label{pro:4}
  \begin{enumerate}
    \item Use the conclusion in the former chapters, prove:
      there must exist quadratic residue and
      non-quadratic residue in the reduced residue system of \(p\), where \(p \in \mathbb{P} \AND p \neq 2\).
    \item Assume \(x_1,x_2\) are quadratic residues, \(x_3\) is non-quadratic residue:
      prove \(x_1x_2\) is quadratic residue, \(x_1x_3\) is non-quadratic residue.
    \item Apply the conclusions above, prove that both the quadratic residue and the non-quadratic residue
      in the reduced residue system of \(p\) have
      \(\frac{p-1}{2}\) elements.
  \end{enumerate}
\end{problem}
\begin{solution}
  \begin{enumerate}
    \item \label{it:4.1}Obviously \(1 \equiv 1^2 \mod p\), so \(1\) is quadratic residue.
      Consider the function \(f:\mathbb{Z}_p\setminus\{0\} \to \mathbb{Z}_p\setminus\{0\},i \mapsto i^2\).
      If every element is quadratic residue, then \(f\) is surjective. Then \(f\) is bijective.
      But since \(p>2\), we know \(1 \not \equiv -1 \mod p\) and \(f(1)\equiv 1 \equiv f(-1)\mod p\), contradiction!
      So there must exist non-quadratic residue of \(p\).
    \item Assume \(x_1 \equiv y_1^2,x_2 \equiv y_2^2 \mod p\), then \(x_1 x_2 \equiv y_1^2 y_2^2 \mod p\), so \(x_1 x_2\) is quadratic residue.
      Since \(y_1 \not \equiv 0 \mod p\), we know there exists \(z\) such that \(y_1 z \equiv 1 \mod p\).
      So if \(x_1 x_3 \equiv t^2 \mod p\) for some \(t\), then \(x_3 \equiv z^2 x_1 x_3 \equiv (zt)^2 \mod p\), contradict to \(x_3\) is non-quadratic.
    \item Recall \(f\) in \ref{it:4.1}, we only need to prove that \(|f(\mathbb{Z}_p\setminus\{0\})|=\frac{p-1}{2}\).
      For every \(x \in f(\mathbb{Z}_p\setminus\{0\})\), consider the equation \(x \equiv y^2 \mod p\).
      By defination we know there exists \(y\) such that \(x \equiv y^2 \mod p\).
      If \(y_1^2 \equiv y_2^2 \equiv x \mod p\), then \(p \mid (y_1+y_2)(y_1-y_2)\), then \(y_2 \equiv \pm y_1 \mod p\).
      So \(|f^{-1}(x)|\leq 2\).
      On the other hand, easy to prove that \(y \not \equiv 0 \mod p \implies y \not \equiv -y \mod p\), and \(x \equiv y^2 \mod p \implies x \equiv (-y)^2 \mod p\).
      So \(|f^{-1}(x)|=2\).
      So \(\sum_{x \in f(\mathbb{Z}_p\setminus\{0\})} 2=\sum_{x \in f(\mathbb{Z}_p\setminus\{0\})} \sum_{y \in \mathbb{Z}_p,x \equiv y^2}1=\sum_{y \in \mathbb{Z}_p\setminus\{0\}} \sum_{x \equiv y^2}1 =\sum_{y \in \mathbb{Z}_p\setminus\{0\}}1=p-1\).
      So \(|f(\mathbb{Z}_p\setminus\{0\})|\frac{p-1}{2}\).
  \end{enumerate}

\end{solution}

\begin{problem}\label{pro:5}
  Prove: the solution of \(x^2 \equiv a\pmod{p^\alpha},\gcd(a,p)=1\) is \(x \equiv \pm PQ'\pmod{p^\alpha}\), where \[
    P=\frac{(z + \sqrt{\alpha})^\alpha + (z - \sqrt{\alpha})^\alpha}{2}, Q=\frac{(z + \sqrt{\alpha})^\alpha - (z-\sqrt{\alpha})^\alpha}{\sqrt{\alpha}},
  \]
  \[
    z^2\equiv \alpha\pmod{p}, QQ' \equiv 1 \pmod{p^\alpha}.
  \]
\end{problem}
\begin{solution}
  First, if \(x^2 \equiv a \mod p^\alpha\) has solution, then \(z^2 \equiv a \mod p\) has solution.
  So we only need to prove that if \(z^2 \equiv a \mod p\) has solution, then \(\pm PQ'\) is solution of \(x^2 \equiv a \mod p^\alpha\).
  Easy to get that \(P+\sqrt{a}Q=(z+\sqrt{a})^\alpha\) and \(P-\sqrt{a}Q=(z-\sqrt{a})^\alpha\).
  So \(P^2-aQ^2=((z+\sqrt{a})(z-\sqrt{a}))^\alpha =(z^2-a)^\alpha\).
  Since \(z^2 \equiv a \mod p\), we know \(p \mid z^2-a\), so \(p^\alpha \mid P^2-aQ^2\).
  So \(P^2 \equiv aQ^2 \mod p\).
  So \(x^2 \equiv P^2 Q'^2 \equiv aQ^2 Q'^2 \equiv a \mod p\).
\end{solution}

\begin{problem}\label{pro:6}
  Prove the solution of \(x^2 + 1 \equiv 0 \pmod{p},p=4m + 1\) is \(x \equiv \pm 1 \cdot 2 \cdot \cdots \cdot (2m)\pmod{p}\).
\end{problem}
\begin{solution}
  Easy to know that \(x^2 \equiv \prod_{i=1}^{2m} i \prod_{i=1}^{2m} i \equiv \prod_{i=1}^{2m} i (-1)^{2m} \prod_{i=1}^{2m} -i \equiv \prod_{i=1}^{4m} i \mod p\).
  So we only need to prove that for \(p \in \mathbb{P} \AND p \neq 2,(p-1)! \equiv -1 \mod p\).
  It is obvious by Wilson's theorem.
\end{solution}

\end{document}
