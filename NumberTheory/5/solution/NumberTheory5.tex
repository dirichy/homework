%!Mode:: "TeX:UTF-8"
%!TEX TS-program = xelatex
\documentclass{ctexart}
\newif\ifpreface
%\prefacetrue
\usepackage{fontspec}
\usepackage{bbm}
\usepackage{tikz}
\usepackage{amsmath,amssymb,amsthm,color,mathrsfs}
\usepackage{fixdif}
\usepackage{hyperref}
\usepackage{cleveref}
\usepackage{enumitem}%
\usepackage{expl3}
\usepackage{lipsum}
\usepackage[margin=0pt]{geometry}
\usepackage{listings}
\definecolor{mGreen}{rgb}{0,0.6,0}
\definecolor{mGray}{rgb}{0.5,0.5,0.5}
\definecolor{mPurple}{rgb}{0.58,0,0.82}
\definecolor{backgroundColour}{rgb}{0.95,0.95,0.92}

\lstdefinestyle{CStyle}{
  backgroundcolor=\color{backgroundColour},
  commentstyle=\color{mGreen},
  keywordstyle=\color{magenta},
  numberstyle=\tiny\color{mGray},
  stringstyle=\color{mPurple},
  basicstyle=\footnotesize,
  breakatwhitespace=false,
  breaklines=true,
  captionpos=b,
  keepspaces=true,
  numbers=left,
  numbersep=5pt,
  showspaces=false,
  showstringspaces=false,
  showtabs=false,
  tabsize=2,
  language=C
}
\usetikzlibrary{calc}
\theoremstyle{remark}
\newtheorem{lemma}{Lemma}
\usepackage{fontawesome5}
\usepackage{xcolor}
\newcounter{problem}
\newcommand{\Problem}{\begin{tikzpicture}[baseline]%
    \node at (-0.02em,0.3em) {$\mathbb{P}$};%
    \node[scale=0.7] at (0.2em,-0.0em) {R};%
    \node[scale=0.7] at (0.6em,0.4em) {O};%
    \node[scale=0.8] at (1.05em,0.25em) {B};%
    \node at (1.55em,0.3em) {L};%
    \node[scale=0.7] at (1.75em,0.45em) {E};%
    \node at (2.35em,0.3em) {M};%
  \end{tikzpicture}%
}
\renewcommand{\theproblem}{\Roman{problem}}
\newenvironment{problem}{\refstepcounter{problem}\noindent\color{blue}\Problem\theproblem}{}

\crefname{problem}{\protect\Problem}{Problem}
\newcommand\Solution{\begin{tikzpicture}[baseline]%
    \node at (-0.04em,0.3em) {$\mathbb{S}$};%
    \node[scale=0.7] at (0.35em,0.4em) {O};%
    \node at (0.7em,0.3em) {\textit{L}};%
    \node[scale=0.7] at (0.95em,0.4em) {U};%
    \node[scale=1.1] at (1.19em,0.32em){T};%
    \node[scale=0.85] at (1.4em,0.24em){I};%
    \node at (1.9em,0.32em){$\mathcal{O}$};%
    \node[scale=0.75] at (2.3em,0.21em){\texttt{N}};%
  \end{tikzpicture}}
\newenvironment{solution}{\begin{proof}[\Solution]}{\end{proof}}
\title{\input{../../.subject}\input{../.number}}
\makeatletter
\newcommand\email[1]{\def\@email{#1}\def\@refemail{mailto:#1}}
\newcommand\schoolid[1]{\def\@schoolid{#1}}
\ifpreface
  \def\@maketitle{
  \raggedright
  {\Huge \bfseries \sffamily \@title }\\[1cm]
  {\Huge  \bfseries \sffamily\heiti\@author}\\[1cm]
  {\Huge \@schoolid}\\[1cm]
  {\Huge\href\@refemail\@email}\\[0.5cm]
  \Huge\@date\\[1cm]}
\else
  \def\@maketitle{
    \raggedright
    \begin{center}
      {\Huge \bfseries \sffamily \@title }\\[4ex]
      {\Large  \@author}\\[4ex]
      {\large \@schoolid}\\[4ex]
      {\href\@refemail\@email}\\[4ex]
      \@date\\[8ex]
    \end{center}}
\fi
\makeatother
\ifpreface
  \usepackage[placement=bottom,scale=1,opacity=1]{background}
\fi

\author{白永乐}
\schoolid{25110180002}
\email{ylbai25@m.fudan.edu.cn}

\def\to{\rightarrow}
\newcommand{\xor}{\vee}
\newcommand{\AND}{\wedge}
\newcommand{\OR}{\vee}
\newcommand{\bor}{\bigvee}
\newcommand{\band}{\bigwedge}
\newcommand{\xand}{\wedge}
\newcommand{\minus}{\mathbin{\backslash}}
\newcommand{\mi}[1]{\mathscr{P}(#1)}
\newcommand{\card}{\mathrm{card}}
\newcommand{\oto}{\leftrightarrow}
\newcommand{\hin}{\hat{\in}}
\newcommand{\gl}{\mathrm{GL}}
\newcommand{\im}{\mathrm{Im}}
\newcommand{\re }{\mathrm{Re }}
\newcommand{\rank}{\mathrm{rank}}
\newcommand{\tra}{\mathop{\mathrm{tr}}}
\renewcommand{\char}{\mathop{\mathrm{char}}}
\DeclareMathOperator{\ot}{ordertype}
\DeclareMathOperator{\dom}{dom}
\DeclareMathOperator{\ran}{ran}

\begin{document}
\large
\iffalse
  \setlength{\baselineskip}{1.2em}
  \ifpreface
    \input{../../../global/preface}
  \else
    \maketitle
  \fi
\fi
\newgeometry{left=2cm,right=2cm,top=2cm,bottom=2cm}
%from_here_to_type
%作业:p53.1(ii);2,3.p55.1(ii),3.(一些同学习惯             于直接“参考”网上流传的答案,但是,友情提醒,那些现成的解答有些是有错的哦!并且考试的时候是不能参考的……所以尽量还是自己动手做啊,要不然考试的时候会卡壳……)
\begin{problem}\label{pro:p53.1ii}
  Solve this equation: \(1215x \equiv 560 \mod 2755\).
\end{problem}
\begin{solution}
  Easily \(1215x \equiv 560 \mod 2755 \iff 243 x \equiv 112 \mod 551 \).
  Obviously \(x=200\) is a solution, and \(\gcd(243,551)=1\), so all the solutions are
  \(x=200+551t,t \in \mathbb{Z}\).
\end{solution}
\begin{problem}\label{pro:p53.2}
  Find the solution of \(\begin{cases}
    x+4y-29\equiv 0 \mod 143 \\
    2x-9y+84 \equiv 0 \mod 143
  \end{cases}\)
\end{problem}
\begin{solution}
  Double the first equation then minus the second, we get \(17 y -142 \equiv 0 \mod 143\).
  Then \(y \equiv 42 \mod 143\).
  Subsititute it in the first equation, we get \(x \equiv 4 \mod 143\).
\end{solution}
\begin{problem}\label{pro:p53.3}
  \begin{enumerate}
    \item Assume \(m \in \mathbb{N}^+,\gcd(a,m)=1\), prove that
      \(x \equiv b a^{\phi(n)-1} \mod m\) is the solution of \(ax \equiv b \mod m\).
    \item Assume \(p\) is prime and \(0<a<p\). Prove that \(x=b(-1)^{a-1} \frac{\binom{p}{a}}{p} \mod p\) is solution of \(ax \equiv b \mod p\).
  \end{enumerate}
\end{problem}
\begin{solution}
  \begin{enumerate}
    \item Only need to check \(aba^{\phi(m)-1}\equiv b \mod m\).
      Since \(\gcd(a,m)=1\), easily \(a^{\phi(m)} \equiv 1 \mod m\), so it's obvious.
    \item
      We multiply \(a!\) to the equation, we get \(a! x \equiv b(-1)^{a-1} \prod_{k=1}^{a-1} (-k) \equiv b (a-1)! \mod p\).
      Since \(0<a<p\), we get \(\gcd((a-1)!,p)=1\), so \(a x \equiv b \mod p\).
  \end{enumerate}
\end{solution}
\begin{problem}\label{pro:p55.1ii}
  Solve the equation:
  \[
    \begin{cases}
      x \equiv 1 \mod 2 \\
      x \equiv 2 \mod 5 \\
      x \equiv 3 \mod 7 \\
      x \equiv 4 \mod 9
    \end{cases}
  \]
\end{problem}
\begin{solution}
  Let \(m_1=2,m_2=5,m_3=7,m_4=9\), and \(M_1=315,M_2=126,M_3=90,M_4=70\).
  Then \(M_1'=1,M_2'=1,M_3'=-1,M_4'=4\).
  So \(x \equiv 1 \times 315 \times 1 + 1 \times 126 \times 2 - 1 \times 90 \times 3 + 4 \times 70 \times 4 \equiv 1417\equiv 157\mod 630\)
\end{solution}
I think this question should be as follows, but may be I make a mistake.
\begin{problem}\label{pro:3}
  \begin{enumerate}
    \item
      Assume \(m_1,\cdots,m_k \in \mathbb{N}^+,b_1,\cdots,b_k \in \mathbb{Z}\), and \(\forall i,j,\gcd(m_i,m_j) \mid b_i-b_j\).
      Let \(m_i':=\prod_{p \in \mathbb{P},\forall j <i,v_p(m_j)<v_p(m_i) \AND \forall j,v_p(m_j) \leq v_p(m_i)} p^{v_p(m_i)}\), where
      \(\mathbb{P}\) is the set of primes, and \(v_p(x)\) is the biggest integer \(t\) such that \(p^t \mid x\).
      Then following two equation has same solution:
      \begin{equation}\label{equ:1}
        x \equiv b_i \mod m_i,\forall i
      \end{equation}
      \begin{equation}\label{equ:2}
        x \equiv b_i \mod m_i',\forall i
      \end{equation}
    \item find the solution of
      \[
        \begin{cases}
          x \equiv 0 \mod 5     \\
          x \equiv 10 \mod 715  \\
          x \equiv 140 \mod 247 \\
          x \equiv 245 \mod 391 \\
          x \equiv 109 \mod 187
        \end{cases}
      \]
  \end{enumerate}
\end{problem}
\begin{solution}
  \begin{enumerate}
    \item
      Easily solution of Equation \eqref{equ:1} must be solution of Equation \eqref{equ:2}, now we will prove the reverse.
      Assume \(x \equiv b_i \mod m_i',\forall i\). Now we will prove \(x \equiv b_i \mod m_i\).
      Only need to prove \(\forall p \in \mathbb{P},x \equiv b_i \mod p^{v_p(m_i)}\).
      Assume \(j=\min \{t:v_p(m_t)=\max_{r}v_p(m_r)\}\), then by the defination of \(m_i'\), we know that \(p^{v_p(m_j)} \mid m_j'\).
      And easily \(p^{v_p(m_i)} \mid p^{v_p(m_j)}\), so we get \(x \equiv b_j \mod p^{v_p(m_i)}\).
      More over, easily to know \(p^{v_p(m_i)} \mid \gcd(m_i,m_j)\), so \(b_j \equiv b_i \mod p^{v_p(m_i)}\).
      So finally we get the result.

      It is easy to prove that \(\forall i \neq j,\gcd(m_i',m_j')=1\), so we can solve the second equation.
    \item
      From above we get the given equation is equvilate to
      \[
        \begin{cases}
          x \equiv 0 \mod 5     \\
          x \equiv 10 \mod 143  \\
          x \equiv 7 \mod 19    \\
          x \equiv 245 \mod 391 \\
          x \equiv 0 \mod 1
        \end{cases}
      \]
      Assume \(x=5y\), then we get \(\begin{cases}
        y \equiv 2 \mod 143 \\
        y \equiv 9 \mod 19  \\
        y \equiv 49 \mod 391
      \end{cases}\).
      Solve this equation, we get \(y \equiv 2004 \mod 1062347\).
      So finally we get \(x \equiv 10020 \mod 5311735\).
  \end{enumerate}

\end{solution}

\end{document}
