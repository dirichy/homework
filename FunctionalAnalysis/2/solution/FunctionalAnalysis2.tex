%!Mode:: "TeX:UTF-8"
%!TEX TS-program = xelatex
%arara: xelatex
\documentclass{ctexart}
\newif\ifpreface
\prefacetrue
\usepackage{fontspec}
\usepackage{bbm}
\usepackage{tikz}
\usepackage{amsmath,amssymb,amsthm,color,mathrsfs}
\usepackage{fixdif}
\usepackage{hyperref}
\usepackage{cleveref}
\usepackage{enumitem}%
\usepackage{expl3}
\usepackage{lipsum}
\usepackage[margin=0pt]{geometry}
\usepackage{listings}
\definecolor{mGreen}{rgb}{0,0.6,0}
\definecolor{mGray}{rgb}{0.5,0.5,0.5}
\definecolor{mPurple}{rgb}{0.58,0,0.82}
\definecolor{backgroundColour}{rgb}{0.95,0.95,0.92}

\lstdefinestyle{CStyle}{
  backgroundcolor=\color{backgroundColour},
  commentstyle=\color{mGreen},
  keywordstyle=\color{magenta},
  numberstyle=\tiny\color{mGray},
  stringstyle=\color{mPurple},
  basicstyle=\footnotesize,
  breakatwhitespace=false,
  breaklines=true,
  captionpos=b,
  keepspaces=true,
  numbers=left,
  numbersep=5pt,
  showspaces=false,
  showstringspaces=false,
  showtabs=false,
  tabsize=2,
  language=C
}
\usetikzlibrary{calc}
\theoremstyle{remark}
\newtheorem{lemma}{Lemma}
\usepackage{fontawesome5}
\usepackage{xcolor}
\newcounter{problem}
\newcommand{\Problem}{\begin{tikzpicture}[baseline]%
    \node at (-0.02em,0.3em) {$\mathbb{P}$};%
    \node[scale=0.7] at (0.2em,-0.0em) {R};%
    \node[scale=0.7] at (0.6em,0.4em) {O};%
    \node[scale=0.8] at (1.05em,0.25em) {B};%
    \node at (1.55em,0.3em) {L};%
    \node[scale=0.7] at (1.75em,0.45em) {E};%
    \node at (2.35em,0.3em) {M};%
  \end{tikzpicture}%
}
\renewcommand{\theproblem}{\Roman{problem}}
\newenvironment{problem}{\refstepcounter{problem}\noindent\color{blue}\Problem\theproblem}{}

\crefname{problem}{\protect\Problem}{Problem}
\newcommand\Solution{\begin{tikzpicture}[baseline]%
    \node at (-0.04em,0.3em) {$\mathbb{S}$};%
    \node[scale=0.7] at (0.35em,0.4em) {O};%
    \node at (0.7em,0.3em) {\textit{L}};%
    \node[scale=0.7] at (0.95em,0.4em) {U};%
    \node[scale=1.1] at (1.19em,0.32em){T};%
    \node[scale=0.85] at (1.4em,0.24em){I};%
    \node at (1.9em,0.32em){$\mathcal{O}$};%
    \node[scale=0.75] at (2.3em,0.21em){\texttt{N}};%
  \end{tikzpicture}}
\newenvironment{solution}{\begin{proof}[\Solution]}{\end{proof}}
\title{\input{../../.subject}\input{../.number}}
\makeatletter
\newcommand\email[1]{\def\@email{#1}\def\@refemail{mailto:#1}}
\newcommand\schoolid[1]{\def\@schoolid{#1}}
\ifpreface
  \def\@maketitle{
  \raggedright
  {\Huge \bfseries \sffamily \@title }\\[1cm]
  {\Huge  \bfseries \sffamily\heiti\@author}\\[1cm]
  {\Huge \@schoolid}\\[1cm]
  {\Huge\href\@refemail\@email}\\[0.5cm]
  \Huge\@date\\[1cm]}
\else
  \def\@maketitle{
    \raggedright
    \begin{center}
      {\Huge \bfseries \sffamily \@title }\\[4ex]
      {\Large  \@author}\\[4ex]
      {\large \@schoolid}\\[4ex]
      {\href\@refemail\@email}\\[4ex]
      \@date\\[8ex]
    \end{center}}
\fi
\makeatother
\ifpreface
  \usepackage[placement=bottom,scale=1,opacity=1]{background}
\fi

\author{白永乐}
\schoolid{25110180002}
\email{ylbai25@m.fudan.edu.cn}

\def\to{\rightarrow}
\newcommand{\xor}{\vee}
\newcommand{\AND}{\wedge}
\newcommand{\OR}{\vee}
\newcommand{\bor}{\bigvee}
\newcommand{\band}{\bigwedge}
\newcommand{\xand}{\wedge}
\newcommand{\minus}{\mathbin{\backslash}}
\newcommand{\mi}[1]{\mathscr{P}(#1)}
\newcommand{\card}{\mathrm{card}}
\newcommand{\oto}{\leftrightarrow}
\newcommand{\hin}{\hat{\in}}
\newcommand{\gl}{\mathrm{GL}}
\newcommand{\im}{\mathrm{Im}}
\newcommand{\re }{\mathrm{Re }}
\newcommand{\rank}{\mathrm{rank}}
\newcommand{\tra}{\mathop{\mathrm{tr}}}
\renewcommand{\char}{\mathop{\mathrm{char}}}
\DeclareMathOperator{\ot}{ordertype}
\DeclareMathOperator{\dom}{dom}
\DeclareMathOperator{\ran}{ran}

\newtheorem{definition}{Def}
\newtheorem{example}{Example}
\newtheorem{theorem}{Theorem}
\newcommand{\norm}[1]{\left\lVert #1 \right\rVert}
\begin{document}
\large
\setlength{\baselineskip}{1.2em}
\ifpreface
\input{../../../global/preface}
\else
\maketitle
\fi
\newgeometry{left=2cm,right=2cm,top=2cm,bottom=2cm}
%from_here_to_type
\section{谱}
\begin{definition}\label{def:banach_algebra}
  设\(\mathcal{A} \)是复 Banach 空间,同时是代数。若\(\norm{x}\norm{y} \geq \norm{xy} \),则称\(\mathcal{A} \)是 Banach 代数。
\end{definition}

\begin{example}\label{exa:banach_algebra_1}
  设\(X \)为 Banach 空间,则\(X \)上的有界线性算子全体 \(B(X) \)是一个 Banach 代数。
\end{example}

\begin{example}\label{exa:banach_algebra_2}
  用\(A(\mathbb{D}) \)记圆盘代数,即在开单位圆盘\(\mathbb{D} \)中解析,在\(\overline{\mathbb{D}} \)上连续的函数全体。作为\(C(\overline{\mathbb{D}}) \)的一个子代数,它是 Banach 代数。
\end{example}

\begin{example}\label{exa:banach_algebra_3}
  在\(l^1(\mathbb{Z}) \)上定义乘法:
  \((x {*} y)_n := \sum_{k = - \infty}^{\infty}x_k y_{n-k} \),
  可验证\(l^1(\mathbb{Z}) \)是有单位元的交换 Banach 代数。
\end{example}

\begin{example}\label{exa:banach_algrbra_4}
  在\(L^1(\mathbb{R}) \)中定义乘法:
  \(f {*} g := \frac{1}{\sqrt{2 \pi}} \int_{\mathbb{R}} f(x - t)g(t) \d t \),则\(L^1(\mathbb{R}) \)是有单位元的交换 Banach 代数。
  反设\(e \)是单位元,则\(\hat{e {*} e} = \hat{e} {*} \hat{e} \)。又\(\lim_{x \to \infty} \hat{e}(x)=0 \),故\(e(x)\equiv 0 \),矛盾!
\end{example}

\begin{theorem}\label{the:1}
  \begin{enumerate}
    \item \label{it:11} 若\(x \in \mathcal{A},\norm{x} < 1 \),则\(\mathbf{1} -x \)可逆,其逆为\((\mathbf{1}-x)^{-1} = \sum_{n = 0}^{\infty}x^n \),且
      \(\norm{(\mathbf{1}- x)^{-1}} \leq (\norm{\mathbf{1} - x})^{-1} \);
    \item \label{it:12} 若\(x \in \mathcal{A}^{-1} \)且\(h \in \mathcal{A} \)满足\(\norm{h}<\frac{1}{2} \norm{x^{-1}}^{-1} \),则\(x + h \in \mathcal{A}^{-1} \),且
      \(\norm{(x + h)^{-1} - x^{-1}} \leq 2\norm{x^{-1}}^2 \norm{h} \)。
    \item \(\mathcal{A}^{-1} \)是开集,且映射\(x \mapsto x^{-1} \)是\(\mathcal{A}^{-1} \) 的自同胚。
  \end{enumerate}
\end{theorem}
\begin{proof}
  \begin{enumerate}
    \item 计算\((\mathbf{1}- x) \sum_{n = 0}^{\infty}x^n=0 \)即可,\(\norm{x} < 1 \)保证了收敛性。
      显然有\(\norm{( \mathbf{1}-x )^{-1}} \leq \sum_{n = 0}^{\infty}\norm{x^n} \leq \frac{1}{1-\norm{x}} \)。
    \item 由\(x+h = x(1+hx^{-1}) \)结合\ref{it:11} 即得。
    \item 由\ref{it:12}即得。
  \end{enumerate}

\end{proof}

\begin{definition}\label{def:shc}
  设\(\mathcal{A} \)是 Banach 代数,\(x \in \mathcal{A} \),定义\(\sigma(x):= \{\lambda \in \mathbb{C}:x - \lambda \mathbf{1} \notin \mathcal{A}^{-1}\} \)。
  若\(A \in B(X)\),则\(\sigma(A):=\{\lambda:\lambda I - A \text{不可逆}\} \)为\(A \)的谱。
  \(A \)的点谱\(\sigma_p(A):= \{\lambda \in \mathbb{C}: \exists x \in X \setminus \{0\},(A\setminus \lambda \mathbf{1})x = 0\} \)。
  \(\sigma(A)\setminus \sigma_p(A) \)称为\(A \)的连续谱(剩余谱)。
\end{definition}

\begin{theorem}\label{the:shc_nonempty}
  对每个\(x \in \mathcal{A},\sigma(x)\neq \varnothing \)。
\end{theorem}
\begin{proof}
  若\(\sigma(x)= \emptyset \),那么\(f(\lambda) = (x - \lambda \mathbf{1})^{-1} \)是向量值的整函数。又\(\lim_{|\lambda| \to \infty}\norm{f(\lambda)} = 0 \),
  故\(f \)是有界整函数,于是\(f \equiv 0 \),矛盾!故\(\sigma(x) \neq \emptyset \)。
\end{proof}

\begin{definition}\label{def:shc_riduas}
  对\(x \in \mathcal{A}\),定义\(r(x):= \sup_{\lambda \in \sigma(x)}|\lambda| \)为\(x \)的谱半径。
\end{definition}

\begin{theorem}\label{the:shc_riduas}
  当\(x \in \mathcal{A} \)时,\(r(x) = \lim_{n \to \infty}\norm{x^r}^{\frac{1}{n}} \)。
\end{theorem}
\begin{proof}
  一方面,\(\lambda \in \sigma(x) \implies \lambda^n \in \sigma(x^n) \),于是\(|\lambda| \leq \norm{x^n}^{\frac{1}{n}} \)。

  另一方面,令\(R_x(\lambda):= (x - \lambda \mathbf{1})^{-1} \),它有 Laurent 展开式\(R_x(\lambda) = \sum_{n = 0}^{\infty}\frac{y_n}{\lambda^n},y_n \in \mathcal{A} \)。
  因为\(|\lambda| > \norm{x} \)时\(R_x(\lambda) = -\frac{1}{\lambda} \sum_{n = 0}^{\infty} \)
  \(|\lambda| > r(x) \)时,\(R_x(\lambda) = -\sum_{n = 1}^{\infty}\frac{x^{n - 1}}{\lambda^n} \)收敛,于是
  \(\limsup \norm{\frac{x^n}{\lambda^{n + 1}}}^{\frac{1}{n}} \leq 1 \)。
\end{proof}

\begin{theorem}\label{the:shc_subset}
  设\(\mathbf{1} \in \mathcal{A} \subset \mathcal{B},x \in \mathcal{A} \),则
  \begin{enumerate}
    \item \(\partial \sigma_{\mathcal{A}}(x) \subset \sigma_{\mathcal{B}}(x) \subset \sigma_{\mathcal{A}}(x)\);
    \item \label{it:1} 如果\(\Omega \) 是\(\mathbb{C} \setminus \sigma_{\mathcal{B}}(x) \)的一个有界连通分支,那么\(\Omega \cap \sigma_{\mathcal{A}}(x)=\emptyset \text{或} \Omega \)。
    \item \(\sigma_{\mathcal{A}}(x) = \sigma_{\mathcal{B}}(x) \cup \bigcup_{k = 1}^{\infty} \Omega_k \),其中\(\Omega_n \)是\(\mathbb{C} \setminus \sigma_{\mathcal{B}}(x) \)的某些有界分支。
  \end{enumerate}
\end{theorem}
\begin{proof}
  \begin{enumerate}
    \item 只需证\(0 \in \partial \sigma_{\mathcal{A}}(x) \implies 0 \in \sigma_{\mathcal{B}}(x) \)即可。
      由\(0 \in \sigma_{\mathcal{A}}(x) \),取\(\lambda_n \to 0,\lambda_n \notin \sigma_{\mathcal{A}}(x) \)。
      若\(0 \notin \sigma_{\mathcal{B}}(x) \),则\((x - \lambda_n \mathbf{1})^{-1} \to x^{-1} \) 在\(\mathcal{B} \)中成立。
      又\((x - \lambda_n \mathbf{1})^{-1} \in \mathcal{A} \),且\(\mathcal{A} \)为Banach 空间,故\(x^{-1} \in \mathcal{A} \),矛盾!
    \item 令\(X : = \Omega \cap \sigma_{\mathcal{A}}(x) \),则\(X \)是\(\Omega \)的闭子集。
      \(\partial_{\Omega} X \subset \partial \sigma_{\mathcal{A}}(x) \subset \sigma_{\mathcal{B}}(x) \subset \mathbb{C} \setminus \Omega \),故\(\partial_{\Omega} X = \emptyset \)。
      于是\(X = \Omega \)或\(X = \emptyset \)。
    \item 由\ref{it:1}显然。
  \end{enumerate}
\end{proof}
\begin{example}\label{exa:shc_subset}
  设\(K:= \{z \in \mathbb{C}:1 \leq |z| \leq 2\},\mathcal{A},\mathcal{B} \)分别的由\((1,z) \)和\((1,z,z^{-1}) \)生成的\(C(K) \)的闭子代数,则\(\mathcal{A} \subsetneqq \mathcal{B} \)。
  令\(f(z) = z \),那么\(\sigma_{\mathcal{B}}(f) = K,\sigma_{\mathcal{A}} \)
\end{example}
\begin{definition}\label{def:l2_t}
  \(\mathbb{T} \)是单位圆周,\(\{\mathrm{e}^{\mathrm{i} n \theta}:n \in \mathbb{Z}\} \)是\(L^2(\mathbb{T}) \)的标准正交基。
  \(H^2(\mathbb{T}) := \{f \in L^2(\mathbb{T}): \hat{f}(n)= 0,n <0\} \)。
\end{definition}

\end{document}
