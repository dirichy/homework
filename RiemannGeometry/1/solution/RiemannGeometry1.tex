%arara: xelatex
%!Mode:: "TeX:UTF-8"
%!TEX TS-program = xelatex
\documentclass{ctexart}
\newif\ifpreface
\prefacetrue
\usepackage{fontspec}
\usepackage{bbm}
\usepackage{tikz}
\usepackage{amsmath,amssymb,amsthm,color,mathrsfs}
\usepackage{fixdif}
\usepackage{hyperref}
\usepackage{cleveref}
\usepackage{enumitem}%
\usepackage{expl3}
\usepackage{lipsum}
\usepackage[margin=0pt]{geometry}
\usepackage{listings}
\definecolor{mGreen}{rgb}{0,0.6,0}
\definecolor{mGray}{rgb}{0.5,0.5,0.5}
\definecolor{mPurple}{rgb}{0.58,0,0.82}
\definecolor{backgroundColour}{rgb}{0.95,0.95,0.92}

\lstdefinestyle{CStyle}{
  backgroundcolor=\color{backgroundColour},
  commentstyle=\color{mGreen},
  keywordstyle=\color{magenta},
  numberstyle=\tiny\color{mGray},
  stringstyle=\color{mPurple},
  basicstyle=\footnotesize,
  breakatwhitespace=false,
  breaklines=true,
  captionpos=b,
  keepspaces=true,
  numbers=left,
  numbersep=5pt,
  showspaces=false,
  showstringspaces=false,
  showtabs=false,
  tabsize=2,
  language=C
}
\usetikzlibrary{calc}
\theoremstyle{remark}
\newtheorem{lemma}{Lemma}
\usepackage{fontawesome5}
\usepackage{xcolor}
\newcounter{problem}
\newcommand{\Problem}{\begin{tikzpicture}[baseline]%
    \node at (-0.02em,0.3em) {$\mathbb{P}$};%
    \node[scale=0.7] at (0.2em,-0.0em) {R};%
    \node[scale=0.7] at (0.6em,0.4em) {O};%
    \node[scale=0.8] at (1.05em,0.25em) {B};%
    \node at (1.55em,0.3em) {L};%
    \node[scale=0.7] at (1.75em,0.45em) {E};%
    \node at (2.35em,0.3em) {M};%
  \end{tikzpicture}%
}
\renewcommand{\theproblem}{\Roman{problem}}
\newenvironment{problem}{\refstepcounter{problem}\noindent\color{blue}\Problem\theproblem}{}

\crefname{problem}{\protect\Problem}{Problem}
\newcommand\Solution{\begin{tikzpicture}[baseline]%
    \node at (-0.04em,0.3em) {$\mathbb{S}$};%
    \node[scale=0.7] at (0.35em,0.4em) {O};%
    \node at (0.7em,0.3em) {\textit{L}};%
    \node[scale=0.7] at (0.95em,0.4em) {U};%
    \node[scale=1.1] at (1.19em,0.32em){T};%
    \node[scale=0.85] at (1.4em,0.24em){I};%
    \node at (1.9em,0.32em){$\mathcal{O}$};%
    \node[scale=0.75] at (2.3em,0.21em){\texttt{N}};%
  \end{tikzpicture}}
\newenvironment{solution}{\begin{proof}[\Solution]}{\end{proof}}
\title{\input{../../.subject}\input{../.number}}
\makeatletter
\newcommand\email[1]{\def\@email{#1}\def\@refemail{mailto:#1}}
\newcommand\schoolid[1]{\def\@schoolid{#1}}
\ifpreface
  \def\@maketitle{
  \raggedright
  {\Huge \bfseries \sffamily \@title }\\[1cm]
  {\Huge  \bfseries \sffamily\heiti\@author}\\[1cm]
  {\Huge \@schoolid}\\[1cm]
  {\Huge\href\@refemail\@email}\\[0.5cm]
  \Huge\@date\\[1cm]}
\else
  \def\@maketitle{
    \raggedright
    \begin{center}
      {\Huge \bfseries \sffamily \@title }\\[4ex]
      {\Large  \@author}\\[4ex]
      {\large \@schoolid}\\[4ex]
      {\href\@refemail\@email}\\[4ex]
      \@date\\[8ex]
    \end{center}}
\fi
\makeatother
\ifpreface
  \usepackage[placement=bottom,scale=1,opacity=1]{background}
\fi

\author{白永乐}
\schoolid{202011150087}
\email{202011150087@mail.bnu.edu.cn}

\def\to{\rightarrow}
\newcommand{\xor}{\vee}
\newcommand{\bor}{\bigvee}
\newcommand{\band}{\bigwedge}
\newcommand{\xand}{\wedge}
\newcommand{\minus}{\mathbin{\backslash}}
\newcommand{\mi}[1]{\mathscr{P}(#1)}
\newcommand{\card}{\mathrm{card}}
\newcommand{\oto}{\leftrightarrow}
\newcommand{\hin}{\hat{\in}}
\newcommand{\gl}{\mathrm{GL}}
\newcommand{\im}{\mathrm{Im}}
\newcommand{\re }{\mathrm{Re }}
\newcommand{\rank}{\mathrm{rank}}
\newcommand{\tra}{\mathop{\mathrm{tr}}}
\renewcommand{\char}{\mathop{\mathrm{char}}}
\DeclareMathOperator{\ot}{ordertype}
\DeclareMathOperator{\dom}{dom}
\DeclareMathOperator{\ran}{ran}

\DeclareMathOperator{\sgn}{sgn}
\begin{document}
\large
\setlength{\baselineskip}{1.2em}
\ifpreface
\backgroundsetup{contents={%
    \begin{tikzpicture}
      \fill [white] (current page.north west) rectangle ($(current page.north east)!.3!(current page.south east)$) coordinate (a);
      \fill [bgc] (current page.south west) rectangle (a);
\end{tikzpicture}}}
\definecolor{word}{rgb}{1,1,0}
\definecolor{bgc}{rgb}{1,0.95,0}
\setlength{\parindent}{0pt}
\thispagestyle{empty}
\begin{tikzpicture}%
  % \node[xscale=2,yscale=4] at (0cm,0cm) {\sffamily\bfseries \color{word} under};%
  \node[xscale=4.5,yscale=10] at (10cm,1cm) {\sffamily\bfseries \color{word} Graduate Homework};%
  \node[xscale=4.5,yscale=10] at (8cm,-2.5cm) {\sffamily\bfseries \color{word} In Mathematics};%
\end{tikzpicture}
\ \vspace{1cm}\\
\begin{minipage}{0.25\textwidth}
  \textcolor{bgc}{王胤雅是傻逼}
\end{minipage}
\begin{minipage}{0.75\textwidth}
  \maketitle
\end{minipage}
\vspace{4cm}\ \\
\begin{minipage}{0.2\textwidth}
  \
\end{minipage}
\begin{minipage}{0.8\textwidth}
  {\Huge
    \textinconsolatanf{}
  }General fire extinguisher
\end{minipage}
\newpage\backgroundsetup{contents={}}\setlength{\parindent}{2em}

\else
\maketitle
\fi
\newgeometry{left=2cm,right=2cm,top=2cm,bottom=2cm}
\everymath{\displaystyle}
\newlength\inlineHeight
\newlength\inlineWidth
\long\def\(#1\){%
  \settoheight{\inlineHeight}{$#1$}%
  \settowidth{\inlineWidth}{$#1$}%
  \ifdim \inlineWidth > 0.5\textwidth%
  $$#1$$%
  \else%
  \ifdim \inlineHeight > 1.5em%
  $$#1$$%
  \else%
  $#1$%
  \fi%
  \fi%
}
%from_here_to_type
\begin{problem}\label{pro:1.1}
  Assume \(\mathcal{A}_0=\{(U_\alpha,\phi_\alpha):\alpha \in I\}\) is a \(C^r\)-compatible coordinate covery of a \(m\)-dimensional manifold \(M\), let
  \(\mathcal{A}:= \{(U,\phi): (U,\phi) \text{ is chart of } M, \AND \forall (V,\psi)\in \mathcal{A}_0,(U,\phi)\text{ is compatible with }(V,\psi)\}\).
  Then \(\mathcal{A}\) is unique \(C^r\)-differential structure on \(M\) contains \(\mathcal{A}_0\).
\end{problem}
\begin{solution}

  First, easily \(\mathcal{A}_0 \subset \mathcal{A}\) by definition of \(\mathcal{A}_0\).
  Now we should prove \(\mathcal{A}\) is differential structure on \(M\). Let \((U,\phi),(V,\psi) \in \mathcal{A}\).
  If \((U,\phi)\in \mathcal{A}_0\), then by definition of \(\mathcal{A}_0\) we know \((U,\phi)\) is compatible to \((V,\psi)\).
  If \((U,\phi),(V,\psi) \notin \mathcal{A}_0\), then consider \(U\cap V\). If \(U\cap V=\varnothing\), then \((U,\phi)\) is compatible with \((V,\psi)\).
  Now assume \(W=U \cap V\neq \varnothing\). Consider \(\gamma:=\psi \circ \phi^{-1} :\phi(W) \to \psi(W)\).
  For any \(x \in W\), since \(\mathcal{A}_0\) is covery of \(M\), we know \(\exists (T,\tau) \in \mathcal{A}_0,x \in T\).
  Then by the definition of \(\mathcal{A}\), we know \((T,\tau)\) is compatible locally on \(x\) with both \((U,\phi)\) and \((V,\psi)\).
  So \((U,\phi)\) is locally compatible with \((V,\psi)\) on \(x\). Since \(x\) is arbitrary, we know \((U,\phi)\) is compatible with \((V,\psi)\).
  So \(\mathcal{A}\) is differential structure of \(M\).

  Now we assume \(\mathcal{B}\) is another differential structure of \(M\) contains \(\mathcal{A}_0\).
  Since \(\mathcal{B}\) is compatible, we get \(\mathcal{B} \subset \mathcal{A}\).
  Since \(\mathcal{B}\) is maximal, we get \(\mathcal{B}=\mathcal{A}\).
  So \(\mathcal{A}\) is unique.
\end{solution}
\begin{problem}\label{pro:1.2}
  Assume \((U,\phi;x^i),(V,\psi;y^i),(W,\chi;z^i)\) are three local coordinate on an \(m\)-dimensional smooth manifold \(M\), and \(W \cap V \cap U \neq \varnothing\).
  Prove that on \(\phi(U \cap V \cap W)\), we have:
  \(\left( \frac{\partial z^i}{\partial x^j}\right)=\left(\frac{\partial z^i}{\partial y^k}\right)\left(\frac{\partial y^k}{\partial x^j}\right)\).
\end{problem}
\begin{solution}
  For fixed \(1 \leq i,j \leq m\), we have \(\frac{\partial z^i}{\partial x^j}=\sum_{k=1}^{m} \frac{\partial z^i}{\partial y^k} \frac{\partial y^k}{\partial x^j}\).
  So easily to get that \(\left( \frac{\partial z^i}{\partial x^j}\right)=\left(\frac{\partial z^i}{\partial y^k}\right)\left(\frac{\partial y^k}{\partial x^j}\right)\).

  We let \((W,\chi;z^i)=(U,\phi;x^i)\), then we get:
  \(I_m=\left(\frac{\partial x^i}{\partial y^k}\right)\left(\frac{\partial y^k}{\partial x^j}\right)\).
  So both terms on the right side are invertible, thus non-singular.
\end{solution}
\begin{problem}\label{pro:1.3}
  Assume \(M\) is orientable and connected, prove that \(M\) has exactly two different oriention.
\end{problem}
% \begin{lemma}\label{lem:connected_manifold_is_path_connected}
%   Assume \(M\) is a connected manifold, then it's path-connected.
% \end{lemma}
% \begin{proof}
%   For \(x,y \in M\), we define \(x \sim y\) if and only if there is a path from \(x\) to \(y\).
%   Easily \(\sim\) is equivalent relation.
%   For \(x \in M\), we define \(U:=\{y \in M: x \sim y\}\). Obviously \(x \in U \neq \varnothing\),
%   so we only need to prove \(U\) is both open and closed.
%   Assume \(y \in U\), since \(M\) is manifold, we obtain \(\exists V\) is homeomorphic to \(\mathbb{R}^n\), thus path-connected, and \(y \in V\).
%   So \(\forall z \in V,z \sim y\), so \(z \sim x\). So \(V \subset U\). So \(U\) is open.
%   For \(y \notin U\), samely \(\exists V\) is open and path-connected and \(y \in V\).
%   If \(z \in V\) and \(z \sim x\) then we will get \(y \sim x\), contradiction!
%   So \(\forall z \in V,z \not \sim x\). So \(V \subset U^c\), so \(U^c\) is open, thus \(U\) is closed.
%   Since \(M\) is connected and \(U\) is both open and closed and \(U \neq \varnothing\), we obtain \(U = M\).
%   So \(M\) is path-connected.
% \end{proof}

% \begin{lemma}\label{lem:1.3}
%   Assume \(\mathcal{A}\) is an oriention of \(M\). Then for every local corrdinate \((V;y^i)\), one of \((V;y^i)\) and \((V;z^i)\) is in \(\mathcal{A}\),
%   where \(z^i=
%     \begin{cases}
%       y^i & i>1\\
%       -y^i & i=1
%   \end{cases}\).
% \end{lemma}
% \begin{proof}
%   Since Jacobi determinant is continious, and Jacobi determinant of transition map is non-zero, so it's always positive or negative.
%   Let \(\mathcal{B}:=\{(U;x^i) \in \mathcal{A}:U \cap V \neq \varnothing, J_{x,y}>0\}\) and \(\mathcal{C}:=\{(U;x^i) \in \mathcal{A}:U \cap V \neq \varnothing, J_{x,y}<0\}\).
%   If \(\mathcal{C}\) is empty then \((V;y^i)\in \mathcal{A}\) by maximalism of \(\mathcal{A}\).
%   If \(\mathcal{B}\) is empty then \((V;z^i)\in \mathcal{A}\) because \(J_{x,y}=-J_{x,z}\).
%   So we only need to prove it's impossible that \(\mathcal{B}\) and \(\mathcal{C}\) are both non-empty.
%
%   Assume \((U;x^i) \in \mathcal{B}\) and \((T;z^i) \in \mathcal{C}\). By definition of \(\mathcal{B},\mathcal{C}\) we get \(U \cap V \neq \varnothing,T \cap V \neq \varnothing\).
%   Let \(p \in U \cap V,q \in T \cap V\). Since \((V;y^i)\) is homeomorphic to a subset of \(\mathbb{R}^n\), we know there is a path from \(p\) to \(q\).
%   Easily this path is compact, so \(\exists p_0=p,p_1,p_2,\cdots,p_k=q\) on this path and \(\{(U_i;y_i^j):i=0,\cdots,k\} \subset \mathcal{A}\) is local coordinate of \(p_i\), and this path is in \(\bigcup_{t=0}^{k} U_t\).
% \end{proof}

\begin{solution}
  Since \(M\) is orientable, we can assume that \(\mathcal{B} \subset \mathcal{A}\) is an oriention of \(M\), where \(\mathcal{A}\) is all local coordinate of \(M\).
  Now consider \(\mathcal{C}:=\{(U;-x^i):(U;x^i) \in \mathcal{B}\}\). Easily to check that \(\mathcal{C}\) is an oriention of \(M\), too.
  And obviously \(\mathcal{B} \cap \mathcal{C} = \varnothing\), thus \(\mathcal{B} \neq \mathcal{C}\). So there is two oriention. Now we need to prove there is no other oriention.

  Assume \(\mathcal{D}\) is an oriention of \(M\). We will define a function \(\sgn:M \to \{1,-1\}\) by
  \(\sgn(p)=1 \iff \exists (U_0,x_0^i) \in \mathcal{B},\exists (V_0,y_0^i) \in \mathcal{D},p \in U_0 \cap V_0,J_{x_0,y_0}(p)>0\).
  We will prove that \(\sgn(p)=1 \implies \forall (U,x^i) \in \mathcal{B},\forall (V;y^i) \in \mathcal{D},p \in U \cap V \implies J_{x,y}(p)>0\).
  It's easy because \(J_{x,y}=J_{x,x_0}J_{x_0,y_0}J_{y_0,y}\), and \(J_{x,x_0}>0,J_{y,y_0}>0\) by definition of oriention.
  So \(\sgn(p)=-1 \iff \exists (U_0,x_0^i) \in \mathcal{C},\exists (V_0,y_0^i) \in \mathcal{D},p \in U_0 \cap V_0,J_{x_0,y_0}(p)>0 \)
  \(\iff \forall (U,x^i) \in \mathcal{B},\forall (V;y^i) \in \mathcal{D},p \in U \cap V \implies J_{x,y}(p)>0 \).
  Noting that if \(\sgn(p)=1\), we have \(\forall q \in U_0 \cap V_0,\sgn(q)=1\), and so is \(\sgn(p)=-1\).
  So \(\sgn\) is continuous. So \(\sgn\) is constant because \(M\) is connected.
  Easy to check that \(\sgn(p)=1 \iff \mathcal{B}=\mathcal{D}\), and \(\sgn(p)=-1 \iff \mathcal{C}=\mathcal{D}\).
  So there is only two oriention of \(M\).
\end{solution}

\begin{problem}\label{pro:1.4}
  Let \(S^n(a):=\{(x^{(1)},\cdots,x^{(n + 1)} )\in \mathbb{R}_1^{n+1} :\sum_{k=1}^{n+1} (x^{(k)})^2=a^2\}\) be the ball in \(\mathbb{R}^{n+1}\) with radius \(a>0\).
  Let \(S:=(0,\cdots,-a),N:=(0,\cdots,a)\) be the South Pole and North Pole respectively.
  Let \(U_+=S^n(a)\setminus\{S\},U_-=S^n(a)\setminus\{N\}\).
  Let \(\phi_+:U_+ \to \mathbb{R}^n,\phi_-:U_- \to \mathbb{R}^n\).
  \((\xi^{(1)},\cdots,\xi^{(n)})=\phi_+(x^{(1)},\cdots,x^{(n+1)}):=\left(\frac{ax^{(1)}}{a+x^{(n+1)}},\cdots,\frac{ax^{(n)}}{a+x^{(n+1)}}\right)\).
  \((\eta^{(1)},\cdots,\eta^{(n)})=\phi_+(x^{(1)},\cdots,x^{(n+1)}):=\left(\frac{ax^{(1)}}{a-x^{(n+1)}},\cdots,\frac{ax^{(n)}}{a-x^{(n+1)}}\right)\).
  Calculate the inverses of \(\phi_+\) and \(\phi_-\), thus prove \(\{(U_+,\phi_+),(U_-,\phi_-)\}\) gives a smooth structuction of \(S^n(a)\).
\end{problem}
%TODO:Calculate this out.
\begin{solution}
  Noting \(\xi^{(i)}=\frac{a x^{(i)}}{a+x^{(n+1)}}\), so \(x^{(i)}=\frac{\xi^{(i)}(a+x^{(n+1)})}{a}\).
  And \(\sum_{k=1}^{n+1} (x^{(k)})^2=a^2\), so \(\sum_{k=1}^{n} \frac{(\xi^{(i)})^2 (a+x^{(n+1)})^2}{a^2} + (x^{(n+1)})^2 = a^2\).
\end{solution}
\begin{problem}\label{pro:1.5}
\end{problem}
\end{document}

